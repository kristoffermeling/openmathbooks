\documentclass[english, 11 pt, class=article, crop=false]{standalone}

\newcommand{\note}{Merk}

% Geometry
\newcommand{\hlikb}{Midtnormalen i en likebeint trekant}
\newcommand{\arealsetn}{Arealsetningen}
\newcommand{\trkmedian}{Medianer i trekanter}
\newcommand{\midtrk}{Midtnormaler i trekanter}
\newcommand{\innskrsirk}{Halveringslinjer og innskrevet sirkel i trekanter}
\newcommand{\cossetn}{Cosinussetningen}
\newcommand{\perfvink}{Sentral- og periferivinkel}
\newcommand{\tang}{Tangent}


% Vectors
\newcommand{\detar}{Arealformler med determinanter}
\usepackage[T1]{fontenc}
%\renewcommand*\familydefault{\sfdefault} % For dyslexia-friendly text
\usepackage{lmodern} % load a font with all the characters
\usepackage{geometry}
\geometry{verbose,paperwidth=16.1 cm, paperheight=24 cm, inner=2.3cm, outer=1.8 cm, bmargin=2cm, tmargin=1.8cm}
\setlength{\parindent}{0bp}
\usepackage{import}
\usepackage[subpreambles=false]{standalone}
\usepackage{amsmath}
\usepackage{amssymb}
\usepackage{esint}
\usepackage{babel}
\usepackage{tabu}
\makeatother
\makeatletter

\usepackage{titlesec}
\usepackage{ragged2e}
\RaggedRight
\raggedbottom
\frenchspacing

% Norwegian names of figures, chapters, parts and content
\addto\captionsenglish{\renewcommand{\figurename}{Figur}}
\makeatletter
\addto\captionsenglish{\renewcommand{\chaptername}{Kapittel}}
\addto\captionsenglish{\renewcommand{\partname}{Del}}


\usepackage{graphicx}
\usepackage{float}
\usepackage{subfig}
\usepackage{placeins}
\usepackage{cancel}
\usepackage{framed}
\usepackage{wrapfig}
\usepackage[subfigure]{tocloft}
\usepackage[font=footnotesize,labelfont=sl]{caption} % Figure caption
\usepackage{bm}
\usepackage[dvipsnames, table]{xcolor}
\definecolor{shadecolor}{rgb}{0.105469, 0.613281, 1}
\colorlet{shadecolor}{Emerald!15} 
\usepackage{icomma}
\makeatother
\usepackage[many]{tcolorbox}
\usepackage{multicol}
\usepackage{stackengine}

\usepackage{esvect} %For vectors with capital letters

% For tabular
\usepackage{array}
\usepackage{multirow}
\usepackage{longtable} %breakable table

% Ligningsreferanser
\usepackage{mathtools}
\mathtoolsset{showonlyrefs}

% index
\usepackage{imakeidx}
\makeindex[title=Indeks]

%Footnote:
\usepackage[bottom, hang, flushmargin]{footmisc}
\usepackage{perpage} 
\MakePerPage{footnote}
\addtolength{\footnotesep}{2mm}
\renewcommand{\thefootnote}{\arabic{footnote}}
\renewcommand\footnoterule{\rule{\linewidth}{0.4pt}}
\renewcommand{\thempfootnote}{\arabic{mpfootnote}}

%colors
\definecolor{c1}{cmyk}{0,0.5,1,0}
\definecolor{c2}{cmyk}{1,0.25,1,0}
\definecolor{n3}{cmyk}{1,0.,1,0}
\definecolor{neg}{cmyk}{1,0.,0.,0}

% Lister med bokstavar
\usepackage[inline]{enumitem}

\newcounter{rg}
\numberwithin{rg}{chapter}
\newcommand{\reg}[2][]{\begin{tcolorbox}[boxrule=0.3 mm,arc=0mm,colback=blue!3] {\refstepcounter{rg}\phantomsection \large \textbf{\therg \;#1} \vspace{5 pt}}\newline #2  \end{tcolorbox}\vspace{-5pt}}

\newcommand\alg[1]{\begin{align} #1 \end{align}}

\newcommand\eks[2][]{\begin{tcolorbox}[boxrule=0.3 mm,arc=0mm,enhanced jigsaw,breakable,colback=green!3] {\large \textbf{Eksempel #1} \vspace{5 pt}\\} #2 \end{tcolorbox}\vspace{-5pt} }

\newcommand{\st}[1]{\begin{tcolorbox}[boxrule=0.0 mm,arc=0mm,enhanced jigsaw,breakable,colback=yellow!12]{ #1} \end{tcolorbox}}

\newcommand{\spr}[1]{\begin{tcolorbox}[boxrule=0.3 mm,arc=0mm,enhanced jigsaw,breakable,colback=yellow!7] {\large \textbf{Språkboksen} \vspace{5 pt}\\} #1 \end{tcolorbox}\vspace{-5pt} }

\newcommand{\sym}[1]{\colorbox{blue!15}{#1}}

\newcommand{\info}[2]{\begin{tcolorbox}[boxrule=0.3 mm,arc=0mm,enhanced jigsaw,breakable,colback=cyan!6] {\large \textbf{#1} \vspace{5 pt}\\} #2 \end{tcolorbox}\vspace{-5pt} }

\newcommand\algv[1]{\vspace{-11 pt}\begin{align*} #1 \end{align*}}

\newcommand{\regv}{\vspace{5pt}}
\newcommand{\mer}{\textsl{Merk}: }
\newcommand{\mers}[1]{{\footnotesize \mer #1}}
\newcommand\vsk{\vspace{11pt}}
\newcommand\vs{\vspace{-11pt}}
\newcommand\vsb{\vspace{-16pt}}
\newcommand\sv{\vsk \textbf{Svar} \vspace{4 pt}\\}
\newcommand\br{\\[5 pt]}
\newcommand{\figp}[1]{../fig/#1}
\newcommand\algvv[1]{\vs\vs\begin{align*} #1 \end{align*}}
\newcommand{\y}[1]{$ {#1} $}
\newcommand{\os}{\\[5 pt]}
\newcommand{\prbxl}[2]{
\parbox[l][][l]{#1\linewidth}{#2
	}}
\newcommand{\prbxr}[2]{\parbox[r][][l]{#1\linewidth}{
		\setlength{\abovedisplayskip}{5pt}
		\setlength{\belowdisplayskip}{5pt}	
		\setlength{\abovedisplayshortskip}{0pt}
		\setlength{\belowdisplayshortskip}{0pt} 
		\begin{shaded}
			\footnotesize	#2 \end{shaded}}}

\renewcommand{\cfttoctitlefont}{\Large\bfseries}
\setlength{\cftaftertoctitleskip}{0 pt}
\setlength{\cftbeforetoctitleskip}{0 pt}

\newcommand{\bs}{\\[3pt]}
\newcommand{\vn}{\\[6pt]}
\newcommand{\fig}[1]{\begin{figure}
		\centering
		\includegraphics[]{\figp{#1}}
\end{figure}}

\newcommand{\figc}[2]{\begin{figure}
		\centering
		\includegraphics[]{\figp{#1}}
		\caption{#2}
\end{figure}}

\newcommand{\sectionbreak}{\clearpage} % New page on each section

\newcommand{\nn}[1]{
\begin{equation}
	#1
\end{equation}
}

% Equation comments
\newcommand{\cm}[1]{\llap{\color{blue} #1}}

\newcommand\fork[2]{\begin{tcolorbox}[boxrule=0.3 mm,arc=0mm,enhanced jigsaw,breakable,colback=yellow!7] {\large \textbf{#1 (forklaring)} \vspace{5 pt}\\} #2 \end{tcolorbox}\vspace{-5pt} }
 
%colors
\newcommand{\colr}[1]{{\color{red} #1}}
\newcommand{\colb}[1]{{\color{blue} #1}}
\newcommand{\colo}[1]{{\color{orange} #1}}
\newcommand{\colc}[1]{{\color{cyan} #1}}
\definecolor{projectgreen}{cmyk}{100,0,100,0}
\newcommand{\colg}[1]{{\color{projectgreen} #1}}

% Methods
\newcommand{\metode}[2]{
	\textsl{#1} \\[-8pt]
	\rule{#2}{0.75pt}
}

%Opg
\newcommand{\abc}[1]{
	\begin{enumerate}[label=\alph*),leftmargin=18pt]
		#1
	\end{enumerate}
}
\newcommand{\abcs}[2]{
	\begin{enumerate}[label=\alph*),start=#1,leftmargin=18pt]
		#2
	\end{enumerate}
}
\newcommand{\abcn}[1]{
	\begin{enumerate}[label=\arabic*),leftmargin=18pt]
		#1
	\end{enumerate}
}
\newcommand{\abch}[1]{
	\hspace{-2pt}	\begin{enumerate*}[label=\alph*), itemjoin=\hspace{1cm}]
		#1
	\end{enumerate*}
}
\newcommand{\abchs}[2]{
	\hspace{-2pt}	\begin{enumerate*}[label=\alph*), itemjoin=\hspace{1cm}, start=#1]
		#2
	\end{enumerate*}
}

% Oppgaver
\newcommand{\opgt}{\phantomsection \addcontentsline{toc}{section}{Oppgaver} \section*{Oppgaver for kapittel \thechapter}\vs \setcounter{section}{1}}
\newcounter{opg}
\numberwithin{opg}{section}
\newcommand{\op}[1]{\vspace{15pt} \refstepcounter{opg}\large \textbf{\color{blue}\theopg} \vspace{2 pt} \label{#1} \\}
\newcommand{\ekspop}[1]{\vsk\textbf{Gruble \thechapter.#1}\vspace{2 pt} \\}
\newcommand{\nes}{\stepcounter{section}
	\setcounter{opg}{0}}
\newcommand{\opr}[1]{\vspace{3pt}\textbf{\ref{#1}}}
\newcommand{\oeks}[1]{\begin{tcolorbox}[boxrule=0.3 mm,arc=0mm,colback=white]
		\textit{Eksempel: } #1	  
\end{tcolorbox}}
\newcommand\opgeks[2][]{\begin{tcolorbox}[boxrule=0.1 mm,arc=0mm,enhanced jigsaw,breakable,colback=white] {\footnotesize \textbf{Eksempel #1} \\} \footnotesize #2 \end{tcolorbox}\vspace{-5pt} }
\newcommand{\rknut}{
Rekn ut.
}

%License
\newcommand{\lic}{\textit{Matematikken sine byggesteinar by Sindre Sogge Heggen is licensed under CC BY-NC-SA 4.0. To view a copy of this license, visit\\ 
		\net{http://creativecommons.org/licenses/by-nc-sa/4.0/}{http://creativecommons.org/licenses/by-nc-sa/4.0/}}}

%referances
\newcommand{\net}[2]{{\color{blue}\href{#1}{#2}}}
\newcommand{\hrs}[2]{\hyperref[#1]{\color{blue}\textsl{#2 \ref*{#1}}}}
\newcommand{\rref}[1]{\hrs{#1}{regel}}
\newcommand{\refkap}[1]{\hrs{#1}{kapittel}}
\newcommand{\refsec}[1]{\hrs{#1}{seksjon}}

\newcommand{\mb}{\net{https://sindrsh.github.io/FirstPrinciplesOfMath/}{MB}}


%line to seperate examples
\newcommand{\linje}{\rule{\linewidth}{1pt} }

\usepackage{datetime2}
%%\usepackage{sansmathfonts} for dyslexia-friendly math
\usepackage[]{hyperref}



\begin{document}
\section{Kvadratsetningene og fullstendige kvadrat}
\reg[Kvadratsetningene]{
For to reelle tall $ a $ og $ b $ er
\alg{
&&(a+b)^2 &= a^2+2ab+b^2 && \text{(1. kvadratsetning)}\vn
&&(a-b)^2 &= a^2-2ab+b^2 && \text{(2. kvadratsetning)} \vn
&&(a+b)(a-b)&= a^2-b^2 && \text{(3. kvadratsetning)}
}
}
\spr{
$ (a+b)^2 $ og $ (a-b)^2 $ kalles \textit{fullstendige kvadrat}.\vsk	
	
3. kvadratsetning kalles også \textit{konjugatsetningen}.
}
\eks[1]{
Skriv om $ a^2+8a+16 $ til et fullstendig kvadrat.

\sv \vsb \vs
\alg{
a^2+8a+16 &= a^2+2\cdot4a+4^2 \\
&= (a+4)^2
}
}
\eks[??]{
Skriv om $ k^2+6k +7 $ til et uttrykk der alle instanser av $ k $ er i et fullstendig kvadrat.

\sv \vsb \vs
\alg{
k^2+6k +7 &= k^2+2\cdot3k+7  \\
&= k^2+2\cdot3k+ 3^2-3^2+7 \\
&= (k+3)^2-2
}
}
\newpage
\eks[??]{
Faktoriser $ x^2-10x+16 $.

\sv
Vi starter med å lage et fullstendig kvadrat:
\alg{
x^2-10x+16 &= x^2-2\cdot5x+5^2-5^2+16 \\
&= (x-5)^2-9
}
Vi legger merke til at $ 9=3^2 $, og bruker 3. kvadratsetning:
\alg{
(x-5)^2-3^2 &= (x-5+3)(x-5-3) \\
&= (x-2)(x-8)
}
Altså er 
\[ x^2-10x+16=(x-2)(x-8) \]
} \vsk

\reg[$\bm {a_1 a_2}$-metoden \label{a1a2}]{
Gitt $ x, b, c \in \mathbb{R} $. Hvis $ a_1a_2=c$ og $ a_1+a_2=b $, er 
\begin{equation}
x^2+bx+c=(x+a_1)(x+a_2)	\label{a1a2eq}
\end{equation}
}
\eks[1]{
Faktoriser uttrykket $ x^2-x-6 $.

\sv
Siden $ 2(-3)=-6 $ og $ 2+(-3)=-1 $, er
\[ x^2-1x-6=(x+2)(x-3) \]
}
\eks[2]{
	Faktoriser uttrykket $ b^2-5b+4 $.
	
	\sv
	Siden ${(-4)( -1)=4} $ og ${ (-4)+(-1)=-5} $, er 
	\[ b^2-5b+4 =(b-4)(b-1) \] 
}
\newpage
\section{Andregradslikninger}
\reg[Andregradslikning uten konstantledd]{
	Gitt likningen
	\nn{
		ax^2+bx=0
	}
	Likningen kan da faktoriseres til
	\nn{
		x(ax+b)=0
	}
	Altså er 
	\nn{x=0\qquad \vee \qquad x=-\frac{b}{a}}
}
\eks[1]{
	Løs likningen 
	\nn{2x^2-4x=0}
	\sv \vs \vs
	
	\algv{
		2x^2-4x&=0 \\
		x(2x-4)&=0
	}
	Altså er $ x=0 $, eller 
	\algv{
		2x-4&=0 \\
		2x&=4 \\
		x&=2
	}
} \vsk

\reg[Andregradslikning med konstantledd]{
	Gitt likningen
	\begin{equation}\label{abc}
		ax^2+bx+c=0
	\end{equation}
	hvor $ a, b $ og $ c $ er konstanter. Da er $ x $ gitt ved \textit{abc}-formelen:
	\begin{align}
		&&\qquad\qquad\qquad	x&= \frac{-b\pm\sqrt{b^2-4ac}}{2a} &&\qquad(\textit{abc}\text{\,-\,formelen})
	\end{align}
	Hvis $ {x=x_1} $ og $ {x_2} $ er løsninger gitt av \textit{abc}-formelen, kan vi skrive
	\begin{equation}\label{abcfakt}
		ax^2+bx+x=0
	\end{equation}
}
\eks[1]{
	Løs likningen	
	\[ 2x^2-7x+5=0 \]
	\sv
	Vi bruker \textit{abc}-formelen. Da er $ a=2 $, $ b=-7 $ og $ c=5 $. Nå får vi at
	\alg{
		x &=\frac{-(-7)\pm\sqrt{(-7)^2-4\cdot2\cdot5}}{2\cdot2} \br
		&=\frac{7\pm\sqrt{49-40}}{4} \br
		&=\frac{7\pm\sqrt{9}}{4}\br
		&=\frac{7\pm3}{4}	
	}
	Enten er
	\algv{
		x&=\frac{7+3}{4} \br
		&=\frac{5}{2}	
	}	
	Eller så er
	\alg{
		x&=\frac{7-3}{4} \br
		&=1
	}	 	
}
\eks[2]{
	Løs likningen 
	\[ x^2+3-10=0 \]
	\sv
	Vi bruker \textit{abc}-formelen. Da er $ a=1 $, $ b=3 $ og $ c=-10 $. Nå får vi at
	\alg{
		x&=\frac{-3\pm\sqrt{3^2-4\cdot1\cdot(-10)}}{2\cdot1}\br
		&=\frac{-3\pm\sqrt{9+40}}{2} \br
		&=\frac{-3\pm7}{2}
	}
	Altså er
	\[ x=-5\qquad\vee\qquad x=2 \]
}
\eks[3]{
Løs likningen	
\[ 4x^2-8x+1 \]
\sv
Av \textit{abc}-formelen har vi at
\alg{
x&=\frac{8\pm\sqrt{(-8)^2-4\cdot4\cdot1}}{2\cdot4} \\
&=\frac{8\pm 4\sqrt{4-1}}{8} \br
&=\frac{2\pm\sqrt{3}}{2}
}
Altså er 
\[ x=\frac{2+\sqrt{3}}{2}\quad\vee\quad \frac{2-\sqrt{3}}{2} \]
}

\newpage
\section{Navn på funksjoner}
\reg[Potensfunksjoner]{
Gitt $ k, b\in \mathbb{R} $. En funksjon på formen
\begin{equation}\label{powrfunc}
	f(x)=k x^m	
\end{equation}
er da en \textit{potensfunksjon} med \textit{koeffisient} $ k $.
}
\reg[Polynomfunksjoner]{
En polynomfunksjon er én av følgende:
\begin{itemize}
	\item en potensfunksjon med heltalls eksponent større eller lik 0.
	\item summen av flere potensfunksjoner med heltalls eksponent større eller lik 0.
\end{itemize}
Polynomfunksjoner kategoriseres etter den største eksponenten i funksjonsuttrykket. For konstantene $ a $, $ b $, $ c $ og $ d$, og en variabel $ x $, har vi at \vs
\begin{center}
	\begin{tabular}{|r|l|} \hline
		\textbf{funksjonsuttyrykk} & \textbf{funksjonsnavn} \\ \hline
		$ ax+b $ &1. grads funksjon/polynom (lineær) \\
		$ ax^2+bx+c $& 2. grads funksjon/polynom (kvadratisk) \\
		$ ax^3+bx^2+cx+d $& 3. grads funksjon/polynom (kubisk) \\ \hline
	\end{tabular}
\end{center}
}
\eks[1]{ \vs
\abc{
\item Skriv uttrykket til et 7. grads polynom med utelukkende heltalls koeffisienter.
\item Skriv uttrykket til et 5. grads polynom med minst én koeffisient uttrykt som et rasjonalt tall.
} \vs
\sv \vs
\abc{
\item  $ 4x^7-5x^2+4 $
\item $ \frac{2}{7}x^5-3 $
}

}
\section{Polynomdivisjon}
Når to gitte tall ikke er delelige med hverandre, kan vi bruke brøker for å uttrykke kvotienten. For eksempel er
\begin{equation}\label{17div3}
	\frac{17}{3}=5+\frac{2}{3}
\end{equation}
At likheten over stemmer er også lett å bekrefte ved utregningen
\[ \left(5+\frac{2}{3}\right)\cdot3=5\cdot3+\frac{2}{3}\cdot3=17 \]
Tanken bak \eqref{17div3} er at vi skriver om telleren slik at den delen av 17 som er delelig med 5 framkommer:
\[ \frac{17}{3}=\frac{5\cdot3+2}{3}=5+\frac{2}{3} 
\]
Den samme tankegangen om divisjon kan brukes for brøker med polynomer, og da kalles det \textit{polynomdivisjon}: 
\regv
\eks[1]{
Utfør polynomdivisjon på uttrykket
\[ \frac{2x^2+3x-4}{x+5} \] 
\sv
\metode{Metode 1}{0.4\linewidth} \\
Vi gjør følgende trinnvis; med den største potensen av $ x $ i telleren som utgangspunkt, lager vi uttrykk som er delelige med telleren.
\alg{
	\frac{2x^2+3x-4}{x+5}&=\frac{2x(x+5)-10x+3x-4}{x+5} \br
	&= 2x+\frac{-7x-4}{x+5} \br
	&= 2x+\frac{-7(x+5)+35-4}{x+5} \br
	&= 2x-7 +\frac{31}{x+5}
}
\newpage
\metode{Metode 2}{0.4\linewidth}
(Se utregningen under punktene)
\begin{enumerate}[label=\roman*)]
	\item  Vi observerer at leddet med den høyste ordenen av $ x $ i dividenden er $ 2x^2 $. Dette uttrykket kan vi framkalle ved å multiplisere dividenden med $ 2x $. Vi skriver $ 2x $ til høgre for likhetstegnet, og subtraherer $ 2x(x+5)=2x^2+10x $.
	\item Differansen fra punkt ii) er ${-7x-4}$. Vi kan framkalle leddet med den høyeste ordenen av $ x $ ved å multiplisere dividenden med $ -7 $. Vi skriver $ -7 $ til høgre for likhetstegnet, og subtraherer $ -7(x+5)=-7x-5 $. 
	\item Differansen fra punkt iii) er $ 31 $. Dette er et uttrykk som har lavere orden av $ x $ enn dividenden, og dermed skriver vi $ \frac{31}{x+5} $ til høgre for likhetstegnet.
\end{enumerate}
\alg{
	\phantom{-}&(2x^2+3x-4):(x+5) =2x-7+\frac{31}{x+5} \\ 
	-&\underline{(2x^2+10x)} \\
	&\phantom{02x-}-7x-4  \\
	&\phantom{xx}-\underline{(-7x-35)}\\
	&\phantom{aaaaaaaaaa''\,}31 
}
}
\newpage
\eks[2]{
Utfør polynomdivisjon på uttrykket
\[ \frac{x^3-4x^2+9}{x^2-2} \]
\sv
\metode{Metode 1}{0.4\linewidth}\\ \vs
\alg{
\frac{x^3-4x^2+9}{x^2-2}&= \frac{x(x^2-2)+2x-4x^2+9}{x^2-2} \br
&=x+\frac{-4x^2+2x+9}{x^2-2}\br
&=x+\frac{-4(x^2-2)-8+2x+9}{x^2-2}\br
&= x-4+\frac{2x+1}{x^2-2}
}
\metode{Metode 2}{0.4\linewidth} \\ \vs
\alg{
	\phantom{-}&(x^3-4x^2+9):(x^2-2) =x-4+\frac{2x+1}{x^2-2} \\ 
	-&\underline{(x^3-2x)} \\
	&\phantom{02x}-4x^2+2x+9  \\
	&\phantom{'}-\underline{(-\,4x^2+8)}\\
	&\phantom{aaaaaaaaaa'}2x+1 
}
}\newpage

\eks[3]{
Utfør polynomdivisjon på uttrykket
\[ \frac{x^3-3x^2-6x+8 }{x-4}\]

\sv
\metode{Metode 1}{0.4\linewidth} \\ \vs
\alg{
\frac{x^3-3x^2-6x+8 }{x-4} &= \frac{x^2(x-4)+4x^2-3x^2-6x+8}{x-4} \br
&= x^2 +\frac{x^2-6x+8}{x-4} \br
&= x^2+\frac{x(x-4)+4x-6x+8}{x-4}\br
&=x^2+x+\frac{-2x+8}{x-4} \br
&= x^2+x-2
}
\metode{Metode 2}{0.4\linewidth} \\ \vs
\alg{
	\phantom{-}&(x^3-3x^2-6x+8):(x-4) =x^2+x-2 \\ 
	-&\underline{(x^3-4x^2)} \\
	&\phantom{02xx''',}x^2-6x+8  \\
	&\phantom{''}-\underline{(-\,x^2-4x)}\\
	&\phantom{aaaaaa''''}-2x+8 \\
	&\phantom{aaaa,,,}-\underline{(-2x+8)}\\
	&\phantom{aaaaaaaaaaaaaa,}0
}
%\mer Dette betyr at
%\[ x^3-3x^2-6x+8 = (x^2+x-2)(x-4) \]
}
\newpage
\section{Polynomers egenskaper}
Eksemplene på side ??-?? peker på noen viktige sammenhenger som gjelder for generelle tilfeller:
\reg[Polinomdivisjon \label{polydiv}]{
La $ A_k $ betegne et polynom $ A $ med grad $ k $. Gitt polynomet $ P_m $, da fins polynomene $ Q_n $, $ S_{m-n} $ og $ R_{n-1} $, hvor $ m\geq n>0 $, slik at
\begin{equation}\label{polydiveq}
\frac{P_m}{Q_n}=S_{m-n}+\frac{R_{n-1}}{Q_n}	
\end{equation}
}
\spr{
Hvis $ R_{n-1}=0 $, sier vi at $ P_m $ er delelig med $ Q_n $.
} \vsk


\newpage
\reg[Faktorer i polynomer \label{polyfakt}]{
	Gitt et polynom $ P(x) $ og en konstant $ a $. Da har vi at
	\begin{equation}
		P \text{ er delelig med } x-a  \Longleftrightarrow P(a)=0 
	\end{equation}
	Hvis dette stemmer, fins det et polynom $ S(x) $ slik at
	\begin{equation}\label{PfaktS}
		P=(a-x)S
	\end{equation}
}
\eks[1]{
Gitt polynomet
\[ P(x)= x^3-3x^2-6x+8\]
\abc{
\item Vis at $ x=1 $ løser likningen $ P=0 $.
\item Faktoriser $ P $. 
}
\sv

\abc{
\item Vi undersøker $ P(1) $:
\alg{
P(1)&=1^3-3\cdot1^3-6\cdot1+8 \\
&= 0
}
Altså er $ P=0 $ når $ x=1 $.
\item Siden $ P(1)=0 $, er $ x-1 $ en faktor i $ P $. Ved polynomdivisjon (se s.??-??) finner vi at
\[ P=(x-1)(x^2-2x-8) \]
Da $ 2(-4)=-8 $ og $ -4+2=-2 $, er
\[ x^2-2x-8=(x+2)(x-4) \]
Dette betyr at
\[ P=(x-1)(x+2)(x-4) \]
}
}
\section{Logaritmer}
I \mb\, så vi på potenstall, som består av et grunntall og en eksponent. En \textit{logaritme} er en matematisk operasjon relativ til et tall. Hvis en logaritme er relativ til grunntallet til en potens, vil operasjonen resultere i ekspontenten.\vsk

La oss skrive logaritmen relativ til 10 som $ \log_{10} $, da er for eksempel
\[ \log_{10} 10^2 = 2 \]
Videre er
\[ \log_{10} 1000= \log_{10} 10^3 = 3\]
Følelig kan vi også skrive
\[ 10^3 = 10^{\log_{10} 1000} \]
Med potensreglene som ugangspunkt (se \mb), kan man utlede mange regler for logartimer.\regv
\reg[Definisjon av logaritmen]{
La $ \log_a $ betegne logaritmen relativ til $ a\in\{\mathbb{R} | a\neq0 \} $. For $ m\in\mathbb{R} $ er da
\begin{equation}
	\log_a a^m = m
\end{equation}
Alternativt kan vi skrive
\begin{equation}
	m=a^{\log_{a} m}
\end{equation}
} \regv

\reg[Logaritmeregler]{
Gitt de reelle tallene $a, x$ og $y$, alle forskjellige fra 0. Da er	
\begin{align}
	\log_a a&= 1\label{loga}\vn	
	\log_a 1 &= 0 \label{log1}\vn
	\log_a (xy)&=\log_a x + \log_a y \label{logxy} \vn
	\log_a \left(\frac{x}{y}\right)&= \log_{a} x - \log_{a} y \label{logxdivy} \vn
	\log_{a} x^y &= y \log_{a} x \label{logxexpy}
\end{align}
}
\spr{
\sym{$ \log_{10}$} skrives ofte bare som \sym{$ \log $}.\vsk

\sym{$ \log_e $} skrives ofte som \sym{$ \ln $}. (Se ?? for mer om tallet $ e $.)
}
\eks[?]{
Løs likningen 
\[4 e^x-8=16 \]
\sv \vsb \vsb \vs
\alg{
4e^x &= 24\\
e^x &= 6 \\
\ln e^x &= \ln 6 \\
x &= \ln 6
}
}

\section{Å løse likninger ved bytte av variabel}
La oss løse likningen 
\begin{equation}\label{eqsqrtxu}
	x-11\sqrt{x}+28=0
\end{equation}
Hvis vi ser nøye etter, innser vi at dette er en andregradslikning for $ \sqrt{x} $. Enda tydeligere blir dette hvis vi definerer variabelen $ u=\sqrt{x} $, da kan vi skrive \eqref{eqsqrtxu} som
\[ u^2-11u+28=0 \]
Siden $ (-7)\cdot(-4)=28 $ og $ -7-4=-11 $, har vi av \eqref{a1a2eq} at
\[ (u-4)(u-7)=0 \]
Altså er 
\[ u=4 \qquad \vee \qquad u=7 \]
Dette betyr at
\[ \sqrt{x}=4 \qquad \vee \qquad \sqrt{x}=7 \]
Dermed er
\[ x=16 \qquad \vee \qquad x=49 \]
\newpage
\section{Forklaringer}
\fork{Andregradslikninger}{
	Gitt likningen
	\[ ax^2+bx+c=0 \]
	Vi starter med å omskrive likningen:
	\[ x^2+\frac{b}{a}x+\frac{c}{a}=0 \]
	Så lager vi et fullstendig kvadrat, og anvdender konjugatsetningen til å faktorisere uttrykket:
	\alg{
		x^2+\frac{b}{a}x+\frac{c}{a}&=x^2+2\cdot\frac{b}{2a}x+\frac{c}{a}\\
		&=\left(x+\frac{b}{2a}\right)^2-\frac{b^2}{4a^2}+\frac{c}{a} \\
		&=\left(x+\frac{b}{2a}\right)^2-\frac{b^2-4ac}{4a^2} \\
		&=\left(x+\frac{b}{2a}\right)^2-\left(\sqrt{\frac{b^2-4ac}{4a^2}}\,\right)^2 \br
		&=\left(x+\frac{b}{2a}\right)^2-\left(\frac{\sqrt{b^2-4ac}}{2a}\,\right)^2 \br
		&=\left(x+\frac{b}{2a}+\frac{\sqrt{b^2-4ac}}{2a}\,\right)\left(x+\frac{b}{2a}-\frac{\sqrt{b^2-4ac}}{2a}\,\right)
	}	
	Uttrykket over er lik 0 når
	\[ x=\frac{-b\pm\sqrt{b^2-4ac}}{2a}\qquad\vee\qquad x=\frac{-b\pm\sqrt{b^2-4ac}}{2a} \]
}

\fork{Logaritmerregler}{
\textbf{Likning} \textbf{\eqref{loga}}		
\[ \log_a a= \log_a a^1=1  \]
\textbf{Likning} \eqref{log1}
\[ \log_a 1 = \log_{a} a^0 =0  \]
\textbf{Likning} \eqref{logxy}\\
For $ m, n\in \mathbb{R} $, har vi at
\alg{
\log_a a^{m+n}&=m+n \\
&= \log_{a} a^m + \log_a a^n
}
Vi setter\footnote{Vi tar det her for gitt at ethvert reelt tall forskjellig fra 0 kan uttrykkes som et potenstall.} $ {x=a^m} $ og $ {y=a^n} $. Siden $ {\log_a a^{m+n}=\log_{a} (a^m\cdot a^n)}$, her da
\[ \log_{a} (x y)=\log_{a} x+\log_{a} y \]
\textbf{Likning} \eqref{logxdivy}\\
Ved å undersøke $ \log_a a^{m-n} $, og ved å sette $ {y=a^{-n}} $, blir\\ forklaringen tilsvarende den gitt for likning \eqref{logxy}.\vsk

\textbf{Likning} \eqref{logxexpy} \os
Siden $ x=a^{\log_a x} $ og $ \left(a^{\log_a x}\right)^y = a^{y \log_{a} x} $ (se potensregler i \mb), har vi at
\algv{
\log_{a} x^y &= \log_{a} a^{y\log_a x}  \\
&= y\log_{a} x
}
} \vsk
\newpage
\fork{Polynomdivisjon (\ref{polydiv})}{
Gitt polynomene
\alg{
&P_m\text{\quad hvor }ax^m\text{ er leddet med høyest grad}\vn
&Q_{n}\text{\quad hvor } bx^n\text{ er leddet med høyest grad}	
}
Da kan vi skrive
\begin{equation}\label{polydivforkla}
	P_m=\frac{a}{b}x^{m-n}Q_n-\frac{a}{b}x^{m-n}Q_n+P_m
\end{equation}
Polynomet $ {-\frac{a}{b}x^{m-n}Q_n+P_m} $ må nødvendigvis ha grad lavere eller lik $ m-1 $. Vi kaller dette polynomet $ U$, og får at
\begin{equation}\label{polydivforklb}
	 P_m = \frac{a}{b}x^{m-n}Q_n+U
\end{equation}
Dermed er
\begin{equation}\label{polydivu}
\frac{P_m}{Q_n}=\frac{a}{b}x^{m-n}+\frac{U}{Q_n}
\end{equation}
Vi kan nå stadig gjenta prosedyren fra \eqref{polydivforkla} og \eqref{polydivforklb}, hvor høgresiden i \eqref{polydivu} får ledd med grad stadig mindre enn $ {m-n} $, fram til polynomet i telleren på høgresiden får grad $ n-1 $.
}
\newpage
\fork{Faktorisering av polynom}{
(i) Vi starter med å vise at
\begin{center}
	Hvis $ P $ er delelig med $ x-a $ 
	er $ x=a $ en løsning for $ P=0 $.
\end{center}
For et polynom $ S $ har vi av \eqref{polydiveq} at
\alg{
\frac{P}{x-a}&= S \\
P &= (x-a)S
}
Da er åpenbart $ x=a $ en løsning for likningen $ P=0 $.\vsk

(ii) Vi går over til å vise at
\begin{center}
	Hvis $ x=a $ er en løsning for $ P=0 $, er $ P $ delelig med $ x-a $.
\end{center}
For polynomene $ S $ og $ R $
\alg{
\frac{P}{x-a}&=S+\frac{R}{x-a} \br
P &= (x-a)S+R
}
Siden $ {x-a} $ har grad 1, må $ R $ ha grad 0, og er dermed en konstant. Hvis $ P(a)=0 $, er 
\[ 0=R \]
Altså er $ P $ delelig med $ x-a $.
}
\end{document}