\documentclass[english, 11 pt, class=article, crop=false]{standalone}
\usepackage[T1]{fontenc}
%\renewcommand*\familydefault{\sfdefault} % For dyslexia-friendly text
\usepackage{lmodern} % load a font with all the characters
\usepackage{geometry}
\geometry{verbose,paperwidth=16.1 cm, paperheight=24 cm, inner=2.3cm, outer=1.8 cm, bmargin=2cm, tmargin=1.8cm}
\setlength{\parindent}{0bp}
\usepackage{import}
\usepackage[subpreambles=false]{standalone}
\usepackage{amsmath}
\usepackage{amssymb}
\usepackage{esint}
\usepackage{babel}
\usepackage{tabu}
\makeatother
\makeatletter

\usepackage{titlesec}
\usepackage{ragged2e}
\RaggedRight
\raggedbottom
\frenchspacing

% Norwegian names of figures, chapters, parts and content
\addto\captionsenglish{\renewcommand{\figurename}{Figur}}
\makeatletter
\addto\captionsenglish{\renewcommand{\chaptername}{Kapittel}}
\addto\captionsenglish{\renewcommand{\partname}{Del}}


\usepackage{graphicx}
\usepackage{float}
\usepackage{subfig}
\usepackage{placeins}
\usepackage{cancel}
\usepackage{framed}
\usepackage{wrapfig}
\usepackage[subfigure]{tocloft}
\usepackage[font=footnotesize,labelfont=sl]{caption} % Figure caption
\usepackage{bm}
\usepackage[dvipsnames, table]{xcolor}
\definecolor{shadecolor}{rgb}{0.105469, 0.613281, 1}
\colorlet{shadecolor}{Emerald!15} 
\usepackage{icomma}
\makeatother
\usepackage[many]{tcolorbox}
\usepackage{multicol}
\usepackage{stackengine}

\usepackage{esvect} %For vectors with capital letters

% For tabular
\usepackage{array}
\usepackage{multirow}
\usepackage{longtable} %breakable table

% Ligningsreferanser
\usepackage{mathtools}
\mathtoolsset{showonlyrefs}

% index
\usepackage{imakeidx}
\makeindex[title=Indeks]

%Footnote:
\usepackage[bottom, hang, flushmargin]{footmisc}
\usepackage{perpage} 
\MakePerPage{footnote}
\addtolength{\footnotesep}{2mm}
\renewcommand{\thefootnote}{\arabic{footnote}}
\renewcommand\footnoterule{\rule{\linewidth}{0.4pt}}
\renewcommand{\thempfootnote}{\arabic{mpfootnote}}

%colors
\definecolor{c1}{cmyk}{0,0.5,1,0}
\definecolor{c2}{cmyk}{1,0.25,1,0}
\definecolor{n3}{cmyk}{1,0.,1,0}
\definecolor{neg}{cmyk}{1,0.,0.,0}

% Lister med bokstavar
\usepackage[inline]{enumitem}

\newcounter{rg}
\numberwithin{rg}{chapter}
\newcommand{\reg}[2][]{\begin{tcolorbox}[boxrule=0.3 mm,arc=0mm,colback=blue!3] {\refstepcounter{rg}\phantomsection \large \textbf{\therg \;#1} \vspace{5 pt}}\newline #2  \end{tcolorbox}\vspace{-5pt}}

\newcommand\alg[1]{\begin{align} #1 \end{align}}

\newcommand\eks[2][]{\begin{tcolorbox}[boxrule=0.3 mm,arc=0mm,enhanced jigsaw,breakable,colback=green!3] {\large \textbf{Eksempel #1} \vspace{5 pt}\\} #2 \end{tcolorbox}\vspace{-5pt} }

\newcommand{\st}[1]{\begin{tcolorbox}[boxrule=0.0 mm,arc=0mm,enhanced jigsaw,breakable,colback=yellow!12]{ #1} \end{tcolorbox}}

\newcommand{\spr}[1]{\begin{tcolorbox}[boxrule=0.3 mm,arc=0mm,enhanced jigsaw,breakable,colback=yellow!7] {\large \textbf{Språkboksen} \vspace{5 pt}\\} #1 \end{tcolorbox}\vspace{-5pt} }

\newcommand{\sym}[1]{\colorbox{blue!15}{#1}}

\newcommand{\info}[2]{\begin{tcolorbox}[boxrule=0.3 mm,arc=0mm,enhanced jigsaw,breakable,colback=cyan!6] {\large \textbf{#1} \vspace{5 pt}\\} #2 \end{tcolorbox}\vspace{-5pt} }

\newcommand\algv[1]{\vspace{-11 pt}\begin{align*} #1 \end{align*}}

\newcommand{\regv}{\vspace{5pt}}
\newcommand{\mer}{\textsl{Merk}: }
\newcommand{\mers}[1]{{\footnotesize \mer #1}}
\newcommand\vsk{\vspace{11pt}}
\newcommand\vs{\vspace{-11pt}}
\newcommand\vsb{\vspace{-16pt}}
\newcommand\sv{\vsk \textbf{Svar} \vspace{4 pt}\\}
\newcommand\br{\\[5 pt]}
\newcommand{\figp}[1]{../fig/#1}
\newcommand\algvv[1]{\vs\vs\begin{align*} #1 \end{align*}}
\newcommand{\y}[1]{$ {#1} $}
\newcommand{\os}{\\[5 pt]}
\newcommand{\prbxl}[2]{
\parbox[l][][l]{#1\linewidth}{#2
	}}
\newcommand{\prbxr}[2]{\parbox[r][][l]{#1\linewidth}{
		\setlength{\abovedisplayskip}{5pt}
		\setlength{\belowdisplayskip}{5pt}	
		\setlength{\abovedisplayshortskip}{0pt}
		\setlength{\belowdisplayshortskip}{0pt} 
		\begin{shaded}
			\footnotesize	#2 \end{shaded}}}

\renewcommand{\cfttoctitlefont}{\Large\bfseries}
\setlength{\cftaftertoctitleskip}{0 pt}
\setlength{\cftbeforetoctitleskip}{0 pt}

\newcommand{\bs}{\\[3pt]}
\newcommand{\vn}{\\[6pt]}
\newcommand{\fig}[1]{\begin{figure}
		\centering
		\includegraphics[]{\figp{#1}}
\end{figure}}

\newcommand{\figc}[2]{\begin{figure}
		\centering
		\includegraphics[]{\figp{#1}}
		\caption{#2}
\end{figure}}

\newcommand{\sectionbreak}{\clearpage} % New page on each section

\newcommand{\nn}[1]{
\begin{equation}
	#1
\end{equation}
}

% Equation comments
\newcommand{\cm}[1]{\llap{\color{blue} #1}}

\newcommand\fork[2]{\begin{tcolorbox}[boxrule=0.3 mm,arc=0mm,enhanced jigsaw,breakable,colback=yellow!7] {\large \textbf{#1 (forklaring)} \vspace{5 pt}\\} #2 \end{tcolorbox}\vspace{-5pt} }
 
%colors
\newcommand{\colr}[1]{{\color{red} #1}}
\newcommand{\colb}[1]{{\color{blue} #1}}
\newcommand{\colo}[1]{{\color{orange} #1}}
\newcommand{\colc}[1]{{\color{cyan} #1}}
\definecolor{projectgreen}{cmyk}{100,0,100,0}
\newcommand{\colg}[1]{{\color{projectgreen} #1}}

% Methods
\newcommand{\metode}[2]{
	\textsl{#1} \\[-8pt]
	\rule{#2}{0.75pt}
}

%Opg
\newcommand{\abc}[1]{
	\begin{enumerate}[label=\alph*),leftmargin=18pt]
		#1
	\end{enumerate}
}
\newcommand{\abcs}[2]{
	\begin{enumerate}[label=\alph*),start=#1,leftmargin=18pt]
		#2
	\end{enumerate}
}
\newcommand{\abcn}[1]{
	\begin{enumerate}[label=\arabic*),leftmargin=18pt]
		#1
	\end{enumerate}
}
\newcommand{\abch}[1]{
	\hspace{-2pt}	\begin{enumerate*}[label=\alph*), itemjoin=\hspace{1cm}]
		#1
	\end{enumerate*}
}
\newcommand{\abchs}[2]{
	\hspace{-2pt}	\begin{enumerate*}[label=\alph*), itemjoin=\hspace{1cm}, start=#1]
		#2
	\end{enumerate*}
}

% Oppgaver
\newcommand{\opgt}{\phantomsection \addcontentsline{toc}{section}{Oppgaver} \section*{Oppgaver for kapittel \thechapter}\vs \setcounter{section}{1}}
\newcounter{opg}
\numberwithin{opg}{section}
\newcommand{\op}[1]{\vspace{15pt} \refstepcounter{opg}\large \textbf{\color{blue}\theopg} \vspace{2 pt} \label{#1} \\}
\newcommand{\ekspop}[1]{\vsk\textbf{Gruble \thechapter.#1}\vspace{2 pt} \\}
\newcommand{\nes}{\stepcounter{section}
	\setcounter{opg}{0}}
\newcommand{\opr}[1]{\vspace{3pt}\textbf{\ref{#1}}}
\newcommand{\oeks}[1]{\begin{tcolorbox}[boxrule=0.3 mm,arc=0mm,colback=white]
		\textit{Eksempel: } #1	  
\end{tcolorbox}}
\newcommand\opgeks[2][]{\begin{tcolorbox}[boxrule=0.1 mm,arc=0mm,enhanced jigsaw,breakable,colback=white] {\footnotesize \textbf{Eksempel #1} \\} \footnotesize #2 \end{tcolorbox}\vspace{-5pt} }
\newcommand{\rknut}{
Rekn ut.
}

%License
\newcommand{\lic}{\textit{Matematikken sine byggesteinar by Sindre Sogge Heggen is licensed under CC BY-NC-SA 4.0. To view a copy of this license, visit\\ 
		\net{http://creativecommons.org/licenses/by-nc-sa/4.0/}{http://creativecommons.org/licenses/by-nc-sa/4.0/}}}

%referances
\newcommand{\net}[2]{{\color{blue}\href{#1}{#2}}}
\newcommand{\hrs}[2]{\hyperref[#1]{\color{blue}\textsl{#2 \ref*{#1}}}}
\newcommand{\rref}[1]{\hrs{#1}{regel}}
\newcommand{\refkap}[1]{\hrs{#1}{kapittel}}
\newcommand{\refsec}[1]{\hrs{#1}{seksjon}}

\newcommand{\mb}{\net{https://sindrsh.github.io/FirstPrinciplesOfMath/}{MB}}


%line to seperate examples
\newcommand{\linje}{\rule{\linewidth}{1pt} }

\usepackage{datetime2}
%%\usepackage{sansmathfonts} for dyslexia-friendly math
\usepackage[]{hyperref}


\newcommand{\note}{Merk}
\newcommand{\notesm}[1]{{\footnotesize \textsl{\note:} #1}}
\newcommand{\ekstitle}{Eksempel }
\newcommand{\sprtitle}{Språkboksen}
\newcommand{\expl}{forklaring}

\newcommand{\vedlegg}[1]{\refstepcounter{vedl}\section*{Vedlegg \thevedl: #1}  \setcounter{vedleq}{0}}

\newcommand\sv{\vsk \textbf{Svar} \vspace{4 pt}\\}

%references
\newcommand{\reftab}[1]{\hrs{#1}{tabell}}
\newcommand{\rref}[1]{\hrs{#1}{regel}}
\newcommand{\dref}[1]{\hrs{#1}{definisjon}}
\newcommand{\refkap}[1]{\hrs{#1}{kapittel}}
\newcommand{\refsec}[1]{\hrs{#1}{seksjon}}
\newcommand{\refdsec}[1]{\hrs{#1}{delseksjon}}
\newcommand{\refved}[1]{\hrs{#1}{vedlegg}}
\newcommand{\eksref}[1]{\textsl{#1}}
\newcommand\fref[2][]{\hyperref[#2]{\textsl{figur \ref*{#2}#1}}}
\newcommand{\refop}[1]{{\color{blue}Oppgave \ref{#1}}}
\newcommand{\refops}[1]{{\color{blue}oppgave \ref{#1}}}
\newcommand{\refgrubs}[1]{{\color{blue}gruble \ref{#1}}}

\newcommand{\openmathser}{\openmath\,-\,serien}

% Exercises
\newcommand{\opgt}{\newpage \phantomsection \addcontentsline{toc}{section}{Oppgaver} \section*{Oppgaver for kapittel \thechapter}\vs \setcounter{section}{1}}


% Sequences and series
\newcommand{\sumarrek}{Summen av en aritmetisk rekke}
\newcommand{\sumgerek}{Summen av en geometrisk rekke}
\newcommand{\regnregsum}{Regneregler for summetegnet}

% Trigonometry
\newcommand{\sincoskomb}{Sinus og cosinus kombinert}
\newcommand{\cosfunk}{Cosinusfunksjonen}
\newcommand{\trid}{Trigonometriske identiteter}
\newcommand{\deravtri}{Den deriverte av de trigonometriske funksjonene}
% Solutions manual
\newcommand{\selos}{Se løsningsforslag.}
\newcommand{\se}[1]{Se eksempel på side \pageref{#1}}

%Vectors
\newcommand{\parvek}{Parallelle vektorer}
\newcommand{\vekpro}{Vektorproduktet}
\newcommand{\vekproarvol}{Vektorproduktet som areal og volum}


% 3D geometries
\newcommand{\linrom}{Linje i rommet}
\newcommand{\avstplnpkt}{Avstand mellom punkt og plan}


% Integral
\newcommand{\bestminten}{Bestemt integral I}
\newcommand{\anfundteo}{Analysens fundamentalteorem}
\newcommand{\intuf}{Integralet av utvalge funksjoner}
\newcommand{\bytvar}{Bytte av variabel}
\newcommand{\intvol}{Integral som volum}
\newcommand{\andordlindif}{Andre ordens lineære differensialligninger}



\begin{document}

\opgt	

\op{opgalgfullkvad}
Skriv som fullstendige kvadrat.\os
\abch{
\item $ x^2+6x+9 $
\item $ b^2+14b+49 $
\item $ a^2-2a+1 $
} \os
\abchs{4}{
\item $ k^2-\frac{2}{3}k+\frac{1}{9} $
\item $ c^2-\frac{1}{2}c+\frac{1}{16} $
\,\item $ y^2+\frac{6}{7}y+\frac{9}{49} $
}

\op{opgalgfullkvad2}
Skriv som fullstendige kvadrat.\os
\abch{
	\item $ 25a^2+90a+81 $
	\item $ 9b^2+12a+4 $
	\item $ 64c^2-16c+1 $
} \os
\abchs{4}{
\item $ \frac{1}{4}d^2+\frac{3}{4}d+\frac{9}{16} $
\item $ \frac{1}{25}e^2+\frac{4}{35}e+\frac{4}{49} $
\item $ \frac{81}{64}f^2-\frac{15}{4}f+\frac{25}{9} $
}

\op{opgalgbrokforenkl} \vs
\abc{
\item Gitt to heltall $ a $ og $ b $. Forklar hvorfor
$ (a+\sqrt{b})(a-\sqrt{b}) $ er et heltall.
\item Skriv om brøken $ \frac{5}{2-\sqrt{3}} $ til en brøk med heltalls nevner.
}

\op{opgalgledd}
Skriv om til et uttrykk der $ x $ er et ledd i et fullstendig kvadrat. \os
\abch{
\item $ x^2+6x-7 $
\item $ x^2-8x-20 $
\item $ x^2+12-45 $
}

\op{algopga1a2}
Hvorfor er det ved bruk av \refunnbr{a1a2}{sum-produkt -metoden} lurte å starte med å finne tall som oppfyller kravet $ a_1a_2=c $ (i motsetning til å finne tall som oppfyller kravet $ a_1+a_2=b $)?

\op{opgalgfakt}
Faktoriser uttrykkene fra \refop{opgalgledd}.

\op{opgalgfullkvad3}
Faktoriser uttrykkene.\os
\abch{
	\item $ x^2-10kx+25k^2$
	\item $ y^2+8yz+16z^2 $
	\item $ a^2-20aq+100q^2 $
} \os
\abchs{4}{
	\item $ x^2 + x y - 20y^2 $
	\item $ a^2-9ab+14b^2 $
	\item $ y^2-9k^5y-k^2y+9k^7 $
}
\newpage

\op{opgalgulikgraf}
Gitt ulikheten 
\[ x^2-9x+20>x-1 \]
\abc{
\item Bruk figuren under til å løse ulikheten.
\item Løs ulikheten ved hjelp av faktorisering.
}
\fig{opgalgulik}

\eksop{1TH21D1}{1TH21D1opg3}
Skriv så enkelt som mulig
\[ \frac{2x^2-2}{x2-2x +1} \]

\eksop{1TV21D1}{1TV21D1opg3}
Skriv så enkelt som mulig
\[ \frac{x}{x-3}+\frac{x-6}{x+3}-\frac{18}{x^2-9} \]

\eksop{1TH21D1}{1TH21D1opg2}
Løs ulikheten.
\[ x^2+2x-8 < 0 \]

\op{opgalgulikbrok}
Gitt ulikheten
\[ \frac{10}{x+3}-\frac{2}{x+5}>0 \]
\abc{
\item Forklar hvorfor det er problematisk å gange begge sider av ulikheten med en fellesnevner.
\item Løs ulikheten.
}

\nes
\op{opgalgligukonst} 
Gitt likningen
\nn{
	ax^2+bx=0
}
Vis, uten å bruke \textit{abc}-formelen, at
\nn{x=0\qquad \vee \qquad x=-\frac{b}{a}}

\op{opgalgligukonst2}
Løs likningene.\os
\abch{
\item $ 2x^2-4x=0 $
\item $ 3x^2+27x=0 $ 
}\\[12pt]
\abchs{3}{
\item $ 7x^2+2x=0 $
\item $8x-9x^2=0 $
}

\op{opgalgabc}
Løs likningene.\os
\abch{
	\item $ x^2-4x-4=0 $ 	
	\item $ x^2+2x-15 $
	\item $ x^2+3x-70=0 $ 
	
}
\\[12pt]
\abchs{4}{
	\item $ x^2+5x-7=0 $
	\item $x^2-x-1=0 $
	\item $x^2-2x-9=0 $	
}
\\[12pt]
\abchs{7}{
	\item $ 5x^2+2x-7=0 $
	\item $8x^2-2x^2-9=0 $
	\item $3x^2-12x+1=0 $ 
}

\eksop{1TH21D1}{1TH21D1opg4}
Grafen til en andregradsfunksjon $ f $ går gjennom punktene $ (0, 12) $, $ (-3, 0) $ og $ (2, 0) $. Bestem $ f(x) $.

\op{opgalgsym}
Grafen til $ {f(x)=x^2+2x-8} $ er symmetrisk om vertikallinja som går gjennom bunnpunktet. Finn $ x $-verdien til dette punktet.
\fig{opgalgsym}
\newpage
\eksop{1TH21D1}{1TH21D1opg5} \vs \vs
\alg{
x^2+2x-y =-1 \tag{I}\vn
x+y = -2 \tag{II}
}
Vis at ligningssystemet ikke har løsning
\abc{
\item grafisk
\item ved regning
}
\nes
\newpage
\op{algopgpoldiv}
Utfør polynomdivisjon på uttrykkene \os
\abch{
\item $ \frac{x^4-3x^2+5}{x^3+x} $
\item $ \frac{-7x^3-9x^2+x}{-4x^2+3} $
\item $ \frac{2x^3-6x^2+9x-27}{2x^2+9} $
}

\nes

\op{algopgpolfakt}
$ P(x)=0 $ for én av $ x\in\{-1, 2, 3\} $.
Faktoriser $ P $ når
\abc{
\item $ P=x^3-37x+84 $
\item $ P=x^3+10x^2+17x+18 $
\item $ P=2x^3+21x^2+61x+42 $
}

\nes

\op{opgalgpot1}
Løs likningen. \os
\abch{
\item $ 7\cdot5^x=14 $
\item $ 3\cdot8^x=27 $
\item $ 10\cdot2^x=19 $
}

\op{opgalgpot2}
Vis at likningen 
\[ b\cdot a^x =c \]
har løsningen
\[ x=\log_a\frac{c}{b} \]

\op{opgalgsub}
Løs likningen. (Hint; se \refved{Bytvar})\os

\abch{
\item $ (\ln x)^2-5\ln x+ 6=0$
\item $ (\log x)^2-3\ln x- 70=0$
} \os
\abchs{3}{
\item $ e^{2x}-2x-3=0 $ \hspace{1cm}
\item $ e^{2x}+7x-18=0 $
}

\eksop{1TH21D1}{1TH21D1opg7}
Løs ligningene
\abc{
\item $ \lg(2x-6)=2 $
\item $ \dfrac{3^{2x}+3^{2x}+4}{2}=29 $
}

\newpage
\grubop{opggeoher}
For en trekant med sidelengder $ a $, $ b $ og $ c $ er arealet $ A $ gitt ved \outl{Herons formel}:
\[ A=\frac{1}{4}\sqrt{(a+b+c)(a+b-c)(a-b+c)(b+c-a)} \]
Bevis formelen.

\grubop{opggeokvadsym}
Gitt funksjonen  ${f(x)=a x^2+bx +c} $. 
\abc{
\item Vis at grafen til $ f $ er symmetrisk om vertikallinja som går gjennom punktet $ \left(-\frac{b}{2a}, 0\right) $.
\item Vis at $ -\frac{b}{2a} $ er $ x $-verdien til toppunktet/bunnpunktet til $ f $.
}
\end{document}