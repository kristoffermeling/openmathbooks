\documentclass[english, 11 pt, class=article, crop=false]{standalone}
\usepackage[T1]{fontenc}
%\renewcommand*\familydefault{\sfdefault} % For dyslexia-friendly text
\usepackage{lmodern} % load a font with all the characters
\usepackage{geometry}
\geometry{verbose,paperwidth=16.1 cm, paperheight=24 cm, inner=2.3cm, outer=1.8 cm, bmargin=2cm, tmargin=1.8cm}
\setlength{\parindent}{0bp}
\usepackage{import}
\usepackage[subpreambles=false]{standalone}
\usepackage{amsmath}
\usepackage{amssymb}
\usepackage{esint}
\usepackage{babel}
\usepackage{tabu}
\makeatother
\makeatletter

\usepackage{titlesec}
\usepackage{ragged2e}
\RaggedRight
\raggedbottom
\frenchspacing

% Norwegian names of figures, chapters, parts and content
\addto\captionsenglish{\renewcommand{\figurename}{Figur}}
\makeatletter
\addto\captionsenglish{\renewcommand{\chaptername}{Kapittel}}
\addto\captionsenglish{\renewcommand{\partname}{Del}}


\usepackage{graphicx}
\usepackage{float}
\usepackage{subfig}
\usepackage{placeins}
\usepackage{cancel}
\usepackage{framed}
\usepackage{wrapfig}
\usepackage[subfigure]{tocloft}
\usepackage[font=footnotesize,labelfont=sl]{caption} % Figure caption
\usepackage{bm}
\usepackage[dvipsnames, table]{xcolor}
\definecolor{shadecolor}{rgb}{0.105469, 0.613281, 1}
\colorlet{shadecolor}{Emerald!15} 
\usepackage{icomma}
\makeatother
\usepackage[many]{tcolorbox}
\usepackage{multicol}
\usepackage{stackengine}

\usepackage{esvect} %For vectors with capital letters

% For tabular
\usepackage{array}
\usepackage{multirow}
\usepackage{longtable} %breakable table

% Ligningsreferanser
\usepackage{mathtools}
\mathtoolsset{showonlyrefs}

% index
\usepackage{imakeidx}
\makeindex[title=Indeks]

%Footnote:
\usepackage[bottom, hang, flushmargin]{footmisc}
\usepackage{perpage} 
\MakePerPage{footnote}
\addtolength{\footnotesep}{2mm}
\renewcommand{\thefootnote}{\arabic{footnote}}
\renewcommand\footnoterule{\rule{\linewidth}{0.4pt}}
\renewcommand{\thempfootnote}{\arabic{mpfootnote}}

%colors
\definecolor{c1}{cmyk}{0,0.5,1,0}
\definecolor{c2}{cmyk}{1,0.25,1,0}
\definecolor{n3}{cmyk}{1,0.,1,0}
\definecolor{neg}{cmyk}{1,0.,0.,0}

% Lister med bokstavar
\usepackage[inline]{enumitem}

\newcounter{rg}
\numberwithin{rg}{chapter}
\newcommand{\reg}[2][]{\begin{tcolorbox}[boxrule=0.3 mm,arc=0mm,colback=blue!3] {\refstepcounter{rg}\phantomsection \large \textbf{\therg \;#1} \vspace{5 pt}}\newline #2  \end{tcolorbox}\vspace{-5pt}}

\newcommand\alg[1]{\begin{align} #1 \end{align}}

\newcommand\eks[2][]{\begin{tcolorbox}[boxrule=0.3 mm,arc=0mm,enhanced jigsaw,breakable,colback=green!3] {\large \textbf{Eksempel #1} \vspace{5 pt}\\} #2 \end{tcolorbox}\vspace{-5pt} }

\newcommand{\st}[1]{\begin{tcolorbox}[boxrule=0.0 mm,arc=0mm,enhanced jigsaw,breakable,colback=yellow!12]{ #1} \end{tcolorbox}}

\newcommand{\spr}[1]{\begin{tcolorbox}[boxrule=0.3 mm,arc=0mm,enhanced jigsaw,breakable,colback=yellow!7] {\large \textbf{Språkboksen} \vspace{5 pt}\\} #1 \end{tcolorbox}\vspace{-5pt} }

\newcommand{\sym}[1]{\colorbox{blue!15}{#1}}

\newcommand{\info}[2]{\begin{tcolorbox}[boxrule=0.3 mm,arc=0mm,enhanced jigsaw,breakable,colback=cyan!6] {\large \textbf{#1} \vspace{5 pt}\\} #2 \end{tcolorbox}\vspace{-5pt} }

\newcommand\algv[1]{\vspace{-11 pt}\begin{align*} #1 \end{align*}}

\newcommand{\regv}{\vspace{5pt}}
\newcommand{\mer}{\textsl{Merk}: }
\newcommand{\mers}[1]{{\footnotesize \mer #1}}
\newcommand\vsk{\vspace{11pt}}
\newcommand\vs{\vspace{-11pt}}
\newcommand\vsb{\vspace{-16pt}}
\newcommand\sv{\vsk \textbf{Svar} \vspace{4 pt}\\}
\newcommand\br{\\[5 pt]}
\newcommand{\figp}[1]{../fig/#1}
\newcommand\algvv[1]{\vs\vs\begin{align*} #1 \end{align*}}
\newcommand{\y}[1]{$ {#1} $}
\newcommand{\os}{\\[5 pt]}
\newcommand{\prbxl}[2]{
\parbox[l][][l]{#1\linewidth}{#2
	}}
\newcommand{\prbxr}[2]{\parbox[r][][l]{#1\linewidth}{
		\setlength{\abovedisplayskip}{5pt}
		\setlength{\belowdisplayskip}{5pt}	
		\setlength{\abovedisplayshortskip}{0pt}
		\setlength{\belowdisplayshortskip}{0pt} 
		\begin{shaded}
			\footnotesize	#2 \end{shaded}}}

\renewcommand{\cfttoctitlefont}{\Large\bfseries}
\setlength{\cftaftertoctitleskip}{0 pt}
\setlength{\cftbeforetoctitleskip}{0 pt}

\newcommand{\bs}{\\[3pt]}
\newcommand{\vn}{\\[6pt]}
\newcommand{\fig}[1]{\begin{figure}
		\centering
		\includegraphics[]{\figp{#1}}
\end{figure}}

\newcommand{\figc}[2]{\begin{figure}
		\centering
		\includegraphics[]{\figp{#1}}
		\caption{#2}
\end{figure}}

\newcommand{\sectionbreak}{\clearpage} % New page on each section

\newcommand{\nn}[1]{
\begin{equation}
	#1
\end{equation}
}

% Equation comments
\newcommand{\cm}[1]{\llap{\color{blue} #1}}

\newcommand\fork[2]{\begin{tcolorbox}[boxrule=0.3 mm,arc=0mm,enhanced jigsaw,breakable,colback=yellow!7] {\large \textbf{#1 (forklaring)} \vspace{5 pt}\\} #2 \end{tcolorbox}\vspace{-5pt} }
 
%colors
\newcommand{\colr}[1]{{\color{red} #1}}
\newcommand{\colb}[1]{{\color{blue} #1}}
\newcommand{\colo}[1]{{\color{orange} #1}}
\newcommand{\colc}[1]{{\color{cyan} #1}}
\definecolor{projectgreen}{cmyk}{100,0,100,0}
\newcommand{\colg}[1]{{\color{projectgreen} #1}}

% Methods
\newcommand{\metode}[2]{
	\textsl{#1} \\[-8pt]
	\rule{#2}{0.75pt}
}

%Opg
\newcommand{\abc}[1]{
	\begin{enumerate}[label=\alph*),leftmargin=18pt]
		#1
	\end{enumerate}
}
\newcommand{\abcs}[2]{
	\begin{enumerate}[label=\alph*),start=#1,leftmargin=18pt]
		#2
	\end{enumerate}
}
\newcommand{\abcn}[1]{
	\begin{enumerate}[label=\arabic*),leftmargin=18pt]
		#1
	\end{enumerate}
}
\newcommand{\abch}[1]{
	\hspace{-2pt}	\begin{enumerate*}[label=\alph*), itemjoin=\hspace{1cm}]
		#1
	\end{enumerate*}
}
\newcommand{\abchs}[2]{
	\hspace{-2pt}	\begin{enumerate*}[label=\alph*), itemjoin=\hspace{1cm}, start=#1]
		#2
	\end{enumerate*}
}

% Oppgaver
\newcommand{\opgt}{\phantomsection \addcontentsline{toc}{section}{Oppgaver} \section*{Oppgaver for kapittel \thechapter}\vs \setcounter{section}{1}}
\newcounter{opg}
\numberwithin{opg}{section}
\newcommand{\op}[1]{\vspace{15pt} \refstepcounter{opg}\large \textbf{\color{blue}\theopg} \vspace{2 pt} \label{#1} \\}
\newcommand{\ekspop}[1]{\vsk\textbf{Gruble \thechapter.#1}\vspace{2 pt} \\}
\newcommand{\nes}{\stepcounter{section}
	\setcounter{opg}{0}}
\newcommand{\opr}[1]{\vspace{3pt}\textbf{\ref{#1}}}
\newcommand{\oeks}[1]{\begin{tcolorbox}[boxrule=0.3 mm,arc=0mm,colback=white]
		\textit{Eksempel: } #1	  
\end{tcolorbox}}
\newcommand\opgeks[2][]{\begin{tcolorbox}[boxrule=0.1 mm,arc=0mm,enhanced jigsaw,breakable,colback=white] {\footnotesize \textbf{Eksempel #1} \\} \footnotesize #2 \end{tcolorbox}\vspace{-5pt} }
\newcommand{\rknut}{
Rekn ut.
}

%License
\newcommand{\lic}{\textit{Matematikken sine byggesteinar by Sindre Sogge Heggen is licensed under CC BY-NC-SA 4.0. To view a copy of this license, visit\\ 
		\net{http://creativecommons.org/licenses/by-nc-sa/4.0/}{http://creativecommons.org/licenses/by-nc-sa/4.0/}}}

%referances
\newcommand{\net}[2]{{\color{blue}\href{#1}{#2}}}
\newcommand{\hrs}[2]{\hyperref[#1]{\color{blue}\textsl{#2 \ref*{#1}}}}
\newcommand{\rref}[1]{\hrs{#1}{regel}}
\newcommand{\refkap}[1]{\hrs{#1}{kapittel}}
\newcommand{\refsec}[1]{\hrs{#1}{seksjon}}

\newcommand{\mb}{\net{https://sindrsh.github.io/FirstPrinciplesOfMath/}{MB}}


%line to seperate examples
\newcommand{\linje}{\rule{\linewidth}{1pt} }

\usepackage{datetime2}
%%\usepackage{sansmathfonts} for dyslexia-friendly math
\usepackage[]{hyperref}


\newcommand{\note}{Merk}
\newcommand{\notesm}[1]{{\footnotesize \textsl{\note:} #1}}
\newcommand{\ekstitle}{Eksempel }
\newcommand{\sprtitle}{Språkboksen}
\newcommand{\expl}{forklaring}

\newcommand{\vedlegg}[1]{\refstepcounter{vedl}\section*{Vedlegg \thevedl: #1}  \setcounter{vedleq}{0}}

\newcommand\sv{\vsk \textbf{Svar} \vspace{4 pt}\\}

%references
\newcommand{\reftab}[1]{\hrs{#1}{tabell}}
\newcommand{\rref}[1]{\hrs{#1}{regel}}
\newcommand{\dref}[1]{\hrs{#1}{definisjon}}
\newcommand{\refkap}[1]{\hrs{#1}{kapittel}}
\newcommand{\refsec}[1]{\hrs{#1}{seksjon}}
\newcommand{\refdsec}[1]{\hrs{#1}{delseksjon}}
\newcommand{\refved}[1]{\hrs{#1}{vedlegg}}
\newcommand{\eksref}[1]{\textsl{#1}}
\newcommand\fref[2][]{\hyperref[#2]{\textsl{figur \ref*{#2}#1}}}
\newcommand{\refop}[1]{{\color{blue}Oppgave \ref{#1}}}
\newcommand{\refops}[1]{{\color{blue}oppgave \ref{#1}}}
\newcommand{\refgrubs}[1]{{\color{blue}gruble \ref{#1}}}

\newcommand{\openmathser}{\openmath\,-\,serien}

% Exercises
\newcommand{\opgt}{\newpage \phantomsection \addcontentsline{toc}{section}{Oppgaver} \section*{Oppgaver for kapittel \thechapter}\vs \setcounter{section}{1}}


% Sequences and series
\newcommand{\sumarrek}{Summen av en aritmetisk rekke}
\newcommand{\sumgerek}{Summen av en geometrisk rekke}
\newcommand{\regnregsum}{Regneregler for summetegnet}

% Trigonometry
\newcommand{\sincoskomb}{Sinus og cosinus kombinert}
\newcommand{\cosfunk}{Cosinusfunksjonen}
\newcommand{\trid}{Trigonometriske identiteter}
\newcommand{\deravtri}{Den deriverte av de trigonometriske funksjonene}
% Solutions manual
\newcommand{\selos}{Se løsningsforslag.}
\newcommand{\se}[1]{Se eksempel på side \pageref{#1}}

%Vectors
\newcommand{\parvek}{Parallelle vektorer}
\newcommand{\vekpro}{Vektorproduktet}
\newcommand{\vekproarvol}{Vektorproduktet som areal og volum}


% 3D geometries
\newcommand{\linrom}{Linje i rommet}
\newcommand{\avstplnpkt}{Avstand mellom punkt og plan}


% Integral
\newcommand{\bestminten}{Bestemt integral I}
\newcommand{\anfundteo}{Analysens fundamentalteorem}
\newcommand{\intuf}{Integralet av utvalge funksjoner}
\newcommand{\bytvar}{Bytte av variabel}
\newcommand{\intvol}{Integral som volum}
\newcommand{\andordlindif}{Andre ordens lineære differensialligninger}



\begin{document}	
\section{Introduksjon}\index{vektor}
En \outl{todimensjonal vektor} angir en forflytning i et koordinatsystem med en $ x $-akse og en $ y $-akse. En vektor tegner vi som et linjestykke mellom to punkt, i tillegg til at vi lar en pil vise til hva som er endepunktet. Det betyr at forflytningen starter i punktet uten pil, og ender i punktet med pil.
\begin{figure}
	\centering
\subfloat[]{\includegraphics{\figp{vekdefa}}}\qquad
\subfloat[]{\includegraphics{\figp{vekdefb}}}
\end{figure}
I figur \textsl{(a)} er vektoren $ \vec{v} $ vist med startpunkt $ (0, 0) $ og endepunkt $ (3,1) $. Når en vektor har startpunkt $ (0, 0) $, sier vi at den er vist i \outl{grunn-stillingen}. I figur \textsl{(b)} er $ \vec{v} $ vist med startpunkt $ (1, -2) $ og endepunkt $ (3, 1) $. Forflytningen $ \vec{v} $ viser til er å vandre 2 mot høgre langs $ x $-aksen og $ 3 $ opp langs $ y $-aksen. Dette skriver vi som $ \vec{u}=[2, 3] $, som kalles $ \vec{u} $ skrevet på \outl{komponentform}. \vsk

\spr{En todimensjonal vektor kalles også en \outl{vektor i rommet}.}
\newpage
\eks[1]{\vs
\alg{
\vec{a}&= [1, 3] & \vec{b} &= [0, -2] \\[15pt]
\vec{c}&= [-3, -4] & \vec{d}&= [5, 0]
}
\fig{vek1}
}
\newpage
\regdef[Vektoren mellom to punkt]{
En vektor $ \vec{v} $ med startpunkt $ {(x_1, y_1)} $ og endepunkt $ (x_2, y_2) $ er gitt som 
\begin{equation}
	\vec{v}= [x_2-x_1, y_2-y_1]
\end{equation}
}
\eks[1]{
Skriv vektorene på komponentform.
\begin{itemize}
	\item $ \vec{a} $ har startpunkt $ (1, 3) $ og endepunkt $ (7, 5) $
	\item $ \vec{b} $ har startpunkt $ (0, 9) $ og endepunkt $ (-3, 2) $
	\item $ \vec{c} $ har startpunkt $ (-3, 7) $ og endepunkt $ (2, -4) $
	\item $ \vec{d} $ har startpunkt $ (-7, -5) $ og endepunkt $ (3, 0) $
\end{itemize}
\sv \vs

\algv{
\vec{a}&=[7-1, 5-3]=[6, 2] \br
\vec{b}&=[-3-0, 2-9]=[-3, -7] \br
\vec{c}&=[2-(-3), -4-7]=[5, -11]\br
\vec{d}&=[3-(-7),0-(-5)]=[10, 5]
}
}
\info{Punkt eller vektor}{
Rent matematisk er det ingen forskjell på et punkt og en vektor; punktet $ (a, b) $ viser til akkurat samme plassering som vektoren $ [a, b] $, og begge kan vise til den samme forflytningen. Ofte kan det likevel være greit å skille mellom når vi snakker om en plassering og når vi snakker om en forlytning, og til det bruker vi begrepene punkt (plassering) og vektorer (forflytning).
}
\section{Regneregler for vektorer}
\regdef[Addisjon og subtraksjon av vektorer]{
	Gitt vektorene $ {\vec{u}=[x_1, y_1]} $ og $ {\vec{v}=[x_2, y_2]} $, punktet ${ A=(x_0, y_0)} $. Da er \vs
	\begin{align}
		A+\vec{u} &= (x_0+x_1, y_0+y_1) \label{Aplusu}	\br
		\vec{u}+\vec{v} &= [x_1+x_2, y_1+y_2] \br
		\vec{u}-\vec{v} &= [x_1-x_2, y_1-y_2] 
	\end{align}
	Summen eller differansen av $ \vec{u} $ og $ \vec{v} $ kan vi tegne slik:
	\begin{figure}
		\centering
		\subfloat[]{\includegraphics[scale=1]{\figp{uplusv}}}\qquad\quad
		\subfloat[]{\includegraphics[scale=1]{\figp{uminusv}}}
	\end{figure}
}
\reg[Regneregler for vektorer]{
For vektorene $\vec{u}$, $ \vec{v} $ og $\vec{w} $, og et tall $ t $, har vi at
\begin{align}
	t\vec{u} &= [tx_1, ty_1, tz_1] \br
	t(\vec{u}+\vec{v})&= t\vec{u}+t\vec{v}\br
	(\vec{u}+\vec{v})+\vec{w} &= \vec{u}+(\vec{v}+\vec{w}) \br
	\vec{u}-(\vec{v}+\vec{w}) &= \vec{u}-\vec{v}-\vec{w}
\end{align}
}
\newpage
\regdef[Vinkelen mellom to vektorer]{
Vinkelen mellom to vektorer er (den minste) vinkelen som blir dannet når vektorene plasseres i samme startpunkt. For to vektorer $ \vec{u} $ og $ \vec{v} $ skriver vi denne vinkelen som $ \angle(\vec{u}, \vec{v}) $.
\fig{vekvink}}
\info{Vinkelmål}{I vektorregning er det vanlig å oppgi vinkler i grader, altså på\\ intervallet $ [0^\circ, 180^\circ] $.}
\section{Lengden til en vektor}
Gitt en vektor $ \vec{v}=[x_1, y_1] $. \outl{Lengden} til $ \vec{v} $ er avstanden mellom startpunktet og endepunktet. 
\begin{figure}
\centering
\subfloat[]{\includegraphics{\figp{veklena}}}\qquad\quad
\subfloat[]{\includegraphics{\figp{veklenb}}}
\end{figure}
Av enhver vektor kan vi danne en rettvinklet trekant hvor $ |\vec{v}| $ er lengden til hypotenusen, og $ |x_1| $ og $ |y_1| $ er de respektive lengdene til katetene. Dermed er $ |\vec{v}| $ gitt av Pytagoras' setning.\regv

\reg[Lengden til en vektor ]{
Gitt en vektor $ \vec{v}=[x_1, y_1] $. Lengden $ |\vec{v}| $ er da 
\begin{equation}
	|\vec{v}|=\sqrt{x_1^2+y_1^2} \label{vekleneq}
\end{equation}
}
\eks[1]{
Finn lengden til vektorene $ \vec{a}= [7, 4] $ og $ \vec{b}= [-3, 2] $.

\sv \vs
\algv{
	|\vec{a}|&=\sqrt{7^2+4^2}=\sqrt{65} \br
	|\vec{b}|&=\sqrt{(-3)^2+2^2}=\sqrt{13}
}
}

\section{Skalarproduktet I}
\regdef[Skalarproduktet I]{
For to vektorer $ {\vec{u}=[x_1, y_1]} $ og $ {\vec{v}=[x_2, y_2]} $, er \outl{skalarproduktet} gitt som
\begin{equation}
	\vec{u}\cdot\vec{v}=x_1x_2+y_1y_2 \label{skaldef1}
\end{equation}
}
\spr{
Skalarproduktet kalles også \outl{prikkproduktet} eller \\ \outl{indreproduktet}.\vsk

Ordet \textit{skalar} viser til en éndimensjonal størrelse.
}
\eks[1]{
Gitt vektorene $ \vec{a}=[3, 2] $, $ \vec{b}=[4,7] $ og $ \vec{c}=[1,-9] $. Regn ut $ \vec{a}\cdot\vec{b} $ og $ \vec{a}\cdot\vec{c} $.

\sv \vsb
\algv{
\vec{a}\cdot\vec{b}&=3\cdot4+2\cdot7=26 \br
\vec{a}\cdot\vec{c}&=3\cdot1+2(-9)=-15
}
}
\newpage
\reg[Regneregler for skalarproduktet]{
	For vektorene $ \vec{u} $, $ \vec{v} $ og $ \vec{w} $ har vi at
\begin{align}
	\vec{u}\cdot \vec{u}&=\vec{u}^{\,2} \label{veku2}\br
	\vec{u}\cdot\vec{v} &= \vec{v}\cdot\vec{u} \br
	\vec{u}\cdot\left(\vec{v}+\vec{w}\right) &= \vec{u}\cdot\vec{v}+\vec{u}\cdot\vec{w} \br
	\left(\vec{u}+\vec{v}\right)^2 &= \vec{u}^{\,2} + 2\vec{u}\cdot\vec{v}+\vec{v}^{\,2}
\end{align}
}
\eks[]{Forkort uttrykket
	\[   \vec{b}\cdot(\vec{a}+\vec{c}) + \vec{a}\cdot(\vec{a}+\vec{b})+\vec{b}^{\,2} \]
	
	når du vet at $ \vec{b}\cdot\vec{c}=0 $. \\
	
	\sv
	\algv{\vec{b}\cdot(\vec{a}+\vec{c}) + \vec{a}\cdot(\vec{a}+\vec{b})+\vec{b}^{\,2} &= \vec{b}\cdot\vec{a}+\vec{b}\cdot\vec{c} + \vec{a}\cdot\vec{a}+\vec{a}\cdot\vec{b}+\vec{b}^2 \\
		&= \vec{a}^{\,2} + 2 \vec{a}\cdot \vec{b} + \vec{b}^{\,2} \\
		&= \left(\vec{a}+ \vec{b}\right)^2
	}
} 

\section{Skalarproduktet II}
Gitt vektoren $ \vec{u}-\vec{v} $, hvor 
$ \vec{u}=[x_1, y_1] $ og $ \vec{v}=[x_2, y_2] $. Da er
\[ \vec{u}-\vec{v}=[x_1-x_2, y_1-y_2] \]
Av \eqref{vekleneq} har vi at
\begin{align}
	|\vec{u}-\vec{v}|&= \sqrt{(x_1-x_2)^2 + (y_1-y_2)^2} \nonumber \\
	&= \sqrt{x_1^2 - 2x_1 x_2 + x_2^2 + y_1^2 - 2y_1 y_2 + y_2^2 }\label{preskal}
\end{align}

Ved hjelp av \eqref{skaldef1} og \eqref{veku2} kan vi skrive \eqref{preskal} som
\begin{equation}
	|\vec{u}-\vec{v}|= \sqrt{\vec{u}^{\,2} - 2\vec{u}\cdot\vec{v} + \vec{v}^{\,2}} \label{skl1}
\end{equation}
Videre merker vi oss følgende figur:
\fig{skaldef}
Av \refunnbr{cossetn}{cosinussetningen} og \eqref{skl1} er
\alg{
	|(\vec{v}-\vec{u})|^2&= |\vec{v}|^2+|\vec{u}|^2-2|\vec{u}||\vec{v}|\cos \angle(\vec{u}, \vec{v}) \\
	\vec{v}^{\,2}- 2\vec{u}\cdot \vec{v} + \vec{u}^{\,2} &= \vec{v}^{\,2}+\vec{u}^{\,2}-2|\vec{u}||\vec{v}|\cos \angle(\vec{u}, \vec{v}) \\
	\vec{u}\cdot \vec{v} &= |\vec{u}||\vec{v}|\cos \angle(\vec{u}, \vec{v})
}
\reg[Skalarproduktet II]{
For to vektorer $ \vec{u} $ og $ \vec{v} $ er
\begin{equation}\label{skal2}
	\vec{u}\cdot \vec{v} = |\vec{u}||\vec{v}|\cos \angle(\vec{u}, \vec{v})
\end{equation}
}
\section{Vektorer vinkelrett på hverandre}
Fra (\ref{skal2}) kan vi gjøre en viktig observasjon; hvis $\angle (\vec{u},\vec{v})=90^\circ $, er $ {\cos\angle (\vec{u},\vec{v})=0} $, og da blir
\nn{
	\vec{u}\cdot\vec{v}=0
}
\reg[Vinkelrette vektorer \label{vinkrett} ]{
For to vektorer $ \vec{u} $ og $ \vec{v} $ har vi at
	\begin{equation}\label{vinkretteq}
		\vec{u} \cdot \vec{v} = 0\iff \vec{u}\perp \vec{v}
	\end{equation}
}
\spr{
Det er mange måter å uttrykke at $ {\vec{u}\perp \vec{v}} $ på. Blant annet kan vi si at
\begin{itemize}
	\item $ \vec{u} $ og $ \vec{v} $ står vinkelrett på hverandre.
	\item $ \vec{u} $ og $ \vec{v} $ står normalt på hverandre. 
	\item $\vec{u} $ er en normalvektor til $ \vec{v}$ (og omvendt).
	\item $\vec{u} $ og $ \vec{v}$ er ortogonale.
\end{itemize}
}
\eks[1]{
	Sjekk om vektorene $ {\vec{a}=[5, -3]} $, $ {\vec{b}=[6, -10]} $ og $ {\vec{c}=[2, 7] }$ er ortogonale.
	
	\sv 
	Vi har at
	\algv{
		\vec{a}\cdot\vec{b}&=5\cdot6+(-3)10 \\
		&=0
	}
	Altså er $ \vec{a}\perp\vec{b} $. Videre er
	\alg{
	\vec{a}\cdot\vec{c}&=5\cdot2+(-3)7\cdot \\
	&= 11
	}
Altså er $ \vec{a} $ og $ \vec{c} $ \textsl{ikke} ortogonale. Da $ {\vec{a}\perp\vec{b}} $, kan heller ikke $ \vec{b} $ og $ \vec{c} $ være ortogonale.
}
\newpage
\info{Nullvektoren}{
I forkant av \rref{vinkrett} har vi bare argumentert for at\\ ${\vec{u}\perp\vec{v}\Rightarrow \vec{u}\cdot\vec{v}=0} $. For å rettferdiggjøre vilkåret som går begge veier i \eqref{vinkretteq}, må vi spørre: Kan vi få ${ \vec{u}\cdot\vec{v} =0}$ om vinkelen mellom $ \vec{u} $ og $ \vec{v} $ \textsl{ikke} er $ 90^\circ $? \vsk

På intervallet $ [0^\circ, 180^\circ] $ er det bare vinkelverdien $ 90^\circ $ som resulterer i cosinusverdi 0. Skal skalarproduktet bli 0 for andre vinkler, må derfor lengden av $ \vec{u} $ eller $ \vec{v} $ være 0. Den eneste vektoren med denne lengden er \outl{nullvektoren} ${\vec{0}=[0, 0] }$, som rett og slett ikke har noen retning\footnote{Eventuelt kan man hevde at den peker i alle retninger!}. Det er likevel vanlig å definere at nullvektoren står vinkelrett på \textit{alle} vektorer.
}
\section{Parallelle vektorer}
\regdef[Parallelle vektorer]{Hvis vinkelen mellom to vektorer er $ 0^\circ $ eller $ 180^\circ $, er de parallelle.}


\begin{figure}
\centering
\subfloat[]{\includegraphics{\figp{vekpara}}}\qquad
\subfloat[]{\includegraphics{\figp{vekparb}}}
\end{figure}
Gitt vektorene $ {\vec{u}=[x_1, y_1]} $ og $ {\vec{v}=[x_2, y_2]} $. La $ \theta $ og $ \alpha $ være vinkelen mellom $ x $-aksen og henholdsvis $ \vec{u} $ og $ \vec{v} $, med $ x $-aksen som høgre vinkelbein. Da er $ {\tan \theta =\frac{y_1}{x_1}} $ og $ {\tan \alpha =\frac{y_2}{x_2}} $. Hvis $ {\frac{y_1}{x_1}=\frac{y_2}{x_2}} $, er det to muligheter:
\begin{enumerate}[label=(\roman*)]
\item $ \theta=0^\circ $ og $ \alpha=180^\circ $, eller omvendt.
\item $ \theta = \alpha $
\end{enumerate}
I begge tilfeller er $ \angle(\vec{u}, \vec{v}) $ enten $ 0^\circ $ eller $ 180^\circ $, og da er $ \vec{u} $ og $ \vec{v} $ parallelle. Det omvendte gjelder også: Hvis punkt (i) eller (ii) gjelder, er $ {\frac{y_1}{x_1}=\frac{y_2}{x_2}} $. Det er ofte praktisk å omskrive denne sammehengen til forholdet mellom samsvarende komponenter\footnote{
For vektorene $ [x_1, y_1] $ og $ [x_2, y_2] $ er disse samsvarende komponenter: \vspace{-6pt}
\begin{itemize}
	\item $ x_1 $ og $ x_2 $
	\item $ y_1 $ og $ y_2 $
\end{itemize}
}: \regv
\reg[Parallelle vektorer]{
	For to vektorer $ {\vec{u}=[x_1, y_1]} \text{ og } {\vec{v}=[x_2, y_2]} $ har vi at
	\begin{equation}
		\frac{x_1}{x_2}=\frac{y_1}{y_2} \iff \vec{u}\parallel\vec{v}
	\end{equation}
Alternativt, for et tall $ t $ har vi at
\begin{equation}
 \vec{u}=t\vec{v}\iff \vec{u}\parallel\vec{v}
\end{equation}
}
\spr{
Når $ {\vec{u}=t \vec{v}} $, sier vi at $ \vec{u} $ er et \outl{multiplum} av $ \vec{v} $ (og omvendt). Vi sier også at $ \vec{u} $ og $ \vec{v} $ er \outl{lineært avhengige}. \vsk

Om to vektorer ikke er parallelle, sier vi at de er \outl{lineært uavhengige}.
}
\eks[]{
Undersøk hvorvidt $ {\vec{a}=[2, -3]} $ og $ {\vec{b}=[20, -45]} $ er parallelle med $ {\vec{c}=[10, -15]} $.

\sv
Vi har at
\nn{
\vec{c}=5[2, -4]=5\vec{a}
}
Dermed er $ \vec{a}\parallel \vec{c} $. Da $ \frac{20}{10}\neq \frac{-45}{15} $, er $ \vec{b} $ og $ \vec{c} $ \textsl{ikke} parallelle.
}
\section{Vektorfunksjoner; parameterisering}
\regdef{
Gitt to funksjoner $ f(t) $ og $ g(t) $. En vektor $ \vec{v} $ på formen
\[ \vec{v}(t)=[f(t), g(t)] \]
er da en \outl{vektorfunksjon}.\vsk

$ \vec{v} $ kan skrives på \outl{parameterisert form} som
\begin{equation}\vec{v}(t): \left\lbrace{
		\begin{array}{lll}
			x= f(t)   \\
			y= g(t)   
		\end{array}
	}\right. 
\end{equation}
\fig{vek0}
}
\info{\note}{
Til forskjell fra grafen til en skalarfunksjon, kan grafen til en vektorfunksjon ''begevege seg fritt'' i koordinatsystemet.
}

\subsection{Vektorfunksjonen til en linje}
Gitt en linje $ \vec{l}(t) $, som vist i figuren under
\fig{linje}
Hvis en vektor $ \vec{r} $ er parallell med $ \vec{l} $, kalles den en \outl{retningsvektor}\index{retningsvektor!for linje}  for linja. Si at $ {\vec{r}=[a, b]} $ er en retningsvektor for $ \vec{l} $, og at $ A=(x_0, y_0) $ er et punkt på $ \vec{l} $. Om vi starter i $ A $ og vandrer parallellt med $ \vec{r} $, kan vi være sikre på at vi fortsatt befinner oss på linja. Dette må bety at vi for en variabel $ t $ kan nå et vilkårlig punkt $ {B=(x, y)} $ på linja ved følgende utregning:
\[B= A+t\vec{r} \]
På koordinatform kan vi skrive dette som\footnote{Se \eqref{Aplusu}.}
\[(x, y)= (x_0 +at, y_0+bt) \]
Altså kan linja skrives som en vektorfunksjon:\regv
\reg[Linje som vektorfunksjon]{
	En linje $ \vec{l}(t) $ som går gjennom punktet $ {A=(x_0, y_0,)} $ og har retningsvektor $ {\vec{r}=[a, b] }$ er gitt som
	\[ \vec{l}=[x_0 + at, y_0+ bt] \]
}
\section{Sirkellikningen}
Gitt en sirkel med sentrum $ S=(x_0, y_0) $ og et punkt $ A=(x, y) $, som ligger på buen til sirkelen. 
\fig{sirklig1}
Da er
\[ \vv{SA}=[x-x_0, y-y_0] \]
Av \eqref{vekleneq} er da
\[ \left|\vv{SA}\right|^2=(x-x_0)^2+(y-y_0)^2 \]
Hvis vi lar $ r $ være radien til sirkelen, er $ \left|\vv{SA}\right|=r $, og dermed kan vi uttrykke $ r $ ved koordinatene til $ S $ og $ A $.\regv
\reg[Sirkelligningen]{
Gitt en sirkel radius $ r $ og sentrum $ {S=(x_0, y_0)} $. Hvis punktet $ A=(x, y) $ ligger på buen til sirkelen, er
\nn{
(x-x_0)^2+(y-y_0)^2=r^2
}
}
\eks[]{
Finn sentrum og radien til sirkelen gitt av likningen
\begin{equation}\label{sirkligeks1}
	x^2 + y^2 - 4x + 10y - 20=0
\end{equation} \vs
\sv
Vi starter med å lage fullstendige kvadrat:
\alg{
x^2-4x &= (x-2)^2-2^2 \vn
y^2+10y &= (y+5)^2-5^2 
}
Altså kan vi skrive \eqref{sirkligeks1} som
\alg{
(x-2)^2+(y+5)^2-2^2-5^2-20&=0 \\
(x-2)^2+(y+5)^2&=7^2
}
Dermed har sirkelen sentrum $ (2, -5) $ og radius 7.
}

\section{Determinanter} \index{determinant}
\reg[\boldmath$ 2\times 2 $ determinanter]{
\outl{Determinanten} $ \det(\vec{u}, \vec{v}) $ av to vektorer $ {\vec{u}=[a, b] }$ og $ {\vec{v}=[b, c]} $ er gitt som
\begin{equation}\label{determinant2eq}
	\det(\vec{u}, \vec{v}) = \left|\begin{matrix}
		a & b \\
		c & d
	\end{matrix}\right| = ad-bc
\end{equation}
}
\eks{
	Gitt vektorene $ {\vec{u}=[-1, 3] }$ og $ {\vec{v}=[-2, 4]} $. Regn ut $ \det(\vec{u}, \vec{v})$.
	
	\sv
	\vs	\algv{
		\det(\vec{u}, \vec{v}) &= \left|\begin{matrix}
			-1 & 3 \\
			-2 & 4
		\end{matrix}\right| \\
		&= (-1)4-3(-2)	\\
		&= 2
	}
}
\reg[\detar \label{det2area}]{
Arealet $ A $ til et parallellogram formet av to vektorer $ \vec{u} $ og $ \vec{v} $ er gitt ved
\begin{equation}
	A= |\det(\vec{u}, \vec{v})| \label{det2arparl}
\end{equation}
\fig{utspprl}
Arealet $ A $ til en trekant formet av to vektorer $ \vec{u} $ og $ \vec{v} $ er gitt ved
\begin{equation}
	A= \frac{1}{2}|\det(\vec{u}, \vec{v})| \label{det2artrk}
\end{equation}
\fig{utsptrk}
}
\reg[\avstpunktlin \label{avstpunktlin}]{
	Avstanden $ h $ mellom et punkt $ B $ og en linje gitt av punktet $ A $ og retningsvektoren $ \vec{r} $ er gitt som
	\begin{equation}
		h = \frac{\left|\det\left(\vv{AB}, \vec r\,\right)\right| }{|\vec{r}|}
	\end{equation}
\fig{plin}
}
\newpage
\fork{\ref{avstpunktlin} \avstpunktlin}{
La en linje $ \vec{l}(t) $ i rommet være gitt  av et punkt $ A $ og en retningsvektor $ \vec{r} $. I tillegg ligger et punkt \textit{B} utenfor linja, som vist i figuren under
\fig{plin}
Den korteste avstanden fra \textit{B} til linja er høyden \textit{h} i trekanten utspent av $ \vec{r} $ og $ \vv{AB} $. Arealet til denne trekanten er gitt ved \eqref{det2artrk}:
\[ \frac{1}{2}\left|\det\left(\vv{AB}, \vec r\right)\right| \]
Av den klassiske arealformelen for en trekant (se \mb) har vi nå at
\alg{\frac{1}{2}|\vec{r}\,|h &=\frac{1}{2}\left|\det\left(\vv{AB}, \vec r\right)\right|  \br
	h &= \frac{\left|\det\left(\vv{AB}, \vec r\right)\right| }{|\vec{r}|}
}
}
\section*{Forklaringer}
\fork{\ref{det2area} \detar }{
Vi lar $ A_N $ betegne arealet til en geometrisk form $ N $.
\begin{figure}
\subfloat[]{\includegraphics{\figp{utspprl}}}\qquad
\subfloat[]{\includegraphics{\figp{utspprlforkl}}}\qquad
\end{figure}
Gitt to vektorer $ {\vec{u}=[a, b]} $ og $ {\vec{v}=[c, d]} $, hvor $ {a, b, c, d >0} $, som vist i figur \textsl{(a)}. Plasserer vi vektorene i grunnstillingen er punktene vist i figur \textsl{(b)} gitt som
\alg{
O &= (0, 0) & B&=(a, b) & C &= (a+b, c+d) \br
  D&= (c, d) & E &= (a+c, 0) & F &= (0, b+d)
}
Med $ OE $ som grunnlinje har $ \triangle OEB $ høgde $ b $, altså er
\[  2A_{\triangle OEB}=(a+c)b \]
Tilsvarende er
\[  2A_{\triangle FDO}=(b+d)c \]
Da $ A_{\triangle OEB}=A_{\triangle CDF}$ og $ A_{\triangle FDO}=A_{\triangle EBC} $, har vi at
\alg{
A_{\square ABCD}&=A_{\square OECF}-2A_{\triangle OBE}-2A_{\triangle FDO} \\
&=(a+c)(b+d)-(a+c)b-(b+d)c \\
&=(a+c)d-(b+d)c\\
&= ad-bc
}
I figurene har vi antatt at (den minste) vinkelen mellom $ \vec{v} $ og $ x $-aksen er mindre enn vinkelen mellom $ \vec{u} $ og $ x $-aksen. I omvendt tilfelle ville vi fått at 
\[ A_{\square OECF}=bc-ad \]
Altså er
\[  A_{\square OECF}=|ac-bd| \]
På lignende måte kan det vises at \eqref{det2arparl} gjelder for alle $a, b, c, d\in\mathbb{R} $, se \refops{det2arbevis}. \vsk

\mers{\eqref{det2arparl} kan også vises veldig kortfattet ved å anvende trigonometri. Se oppgave ?? i \tmto\ for dette.}
}
\end{document}