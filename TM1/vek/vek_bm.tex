\documentclass[english, 11 pt, class=article, crop=false]{standalone}
\usepackage[T1]{fontenc}
%\renewcommand*\familydefault{\sfdefault} % For dyslexia-friendly text
\usepackage{lmodern} % load a font with all the characters
\usepackage{geometry}
\geometry{verbose,paperwidth=16.1 cm, paperheight=24 cm, inner=2.3cm, outer=1.8 cm, bmargin=2cm, tmargin=1.8cm}
\setlength{\parindent}{0bp}
\usepackage{import}
\usepackage[subpreambles=false]{standalone}
\usepackage{amsmath}
\usepackage{amssymb}
\usepackage{esint}
\usepackage{babel}
\usepackage{tabu}
\makeatother
\makeatletter

\usepackage{titlesec}
\usepackage{ragged2e}
\RaggedRight
\raggedbottom
\frenchspacing

% Norwegian names of figures, chapters, parts and content
\addto\captionsenglish{\renewcommand{\figurename}{Figur}}
\makeatletter
\addto\captionsenglish{\renewcommand{\chaptername}{Kapittel}}
\addto\captionsenglish{\renewcommand{\partname}{Del}}


\usepackage{graphicx}
\usepackage{float}
\usepackage{subfig}
\usepackage{placeins}
\usepackage{cancel}
\usepackage{framed}
\usepackage{wrapfig}
\usepackage[subfigure]{tocloft}
\usepackage[font=footnotesize,labelfont=sl]{caption} % Figure caption
\usepackage{bm}
\usepackage[dvipsnames, table]{xcolor}
\definecolor{shadecolor}{rgb}{0.105469, 0.613281, 1}
\colorlet{shadecolor}{Emerald!15} 
\usepackage{icomma}
\makeatother
\usepackage[many]{tcolorbox}
\usepackage{multicol}
\usepackage{stackengine}

\usepackage{esvect} %For vectors with capital letters

% For tabular
\usepackage{array}
\usepackage{multirow}
\usepackage{longtable} %breakable table

% Ligningsreferanser
\usepackage{mathtools}
\mathtoolsset{showonlyrefs}

% index
\usepackage{imakeidx}
\makeindex[title=Indeks]

%Footnote:
\usepackage[bottom, hang, flushmargin]{footmisc}
\usepackage{perpage} 
\MakePerPage{footnote}
\addtolength{\footnotesep}{2mm}
\renewcommand{\thefootnote}{\arabic{footnote}}
\renewcommand\footnoterule{\rule{\linewidth}{0.4pt}}
\renewcommand{\thempfootnote}{\arabic{mpfootnote}}

%colors
\definecolor{c1}{cmyk}{0,0.5,1,0}
\definecolor{c2}{cmyk}{1,0.25,1,0}
\definecolor{n3}{cmyk}{1,0.,1,0}
\definecolor{neg}{cmyk}{1,0.,0.,0}

% Lister med bokstavar
\usepackage[inline]{enumitem}

\newcounter{rg}
\numberwithin{rg}{chapter}
\newcommand{\reg}[2][]{\begin{tcolorbox}[boxrule=0.3 mm,arc=0mm,colback=blue!3] {\refstepcounter{rg}\phantomsection \large \textbf{\therg \;#1} \vspace{5 pt}}\newline #2  \end{tcolorbox}\vspace{-5pt}}

\newcommand\alg[1]{\begin{align} #1 \end{align}}

\newcommand\eks[2][]{\begin{tcolorbox}[boxrule=0.3 mm,arc=0mm,enhanced jigsaw,breakable,colback=green!3] {\large \textbf{Eksempel #1} \vspace{5 pt}\\} #2 \end{tcolorbox}\vspace{-5pt} }

\newcommand{\st}[1]{\begin{tcolorbox}[boxrule=0.0 mm,arc=0mm,enhanced jigsaw,breakable,colback=yellow!12]{ #1} \end{tcolorbox}}

\newcommand{\spr}[1]{\begin{tcolorbox}[boxrule=0.3 mm,arc=0mm,enhanced jigsaw,breakable,colback=yellow!7] {\large \textbf{Språkboksen} \vspace{5 pt}\\} #1 \end{tcolorbox}\vspace{-5pt} }

\newcommand{\sym}[1]{\colorbox{blue!15}{#1}}

\newcommand{\info}[2]{\begin{tcolorbox}[boxrule=0.3 mm,arc=0mm,enhanced jigsaw,breakable,colback=cyan!6] {\large \textbf{#1} \vspace{5 pt}\\} #2 \end{tcolorbox}\vspace{-5pt} }

\newcommand\algv[1]{\vspace{-11 pt}\begin{align*} #1 \end{align*}}

\newcommand{\regv}{\vspace{5pt}}
\newcommand{\mer}{\textsl{Merk}: }
\newcommand{\mers}[1]{{\footnotesize \mer #1}}
\newcommand\vsk{\vspace{11pt}}
\newcommand\vs{\vspace{-11pt}}
\newcommand\vsb{\vspace{-16pt}}
\newcommand\sv{\vsk \textbf{Svar} \vspace{4 pt}\\}
\newcommand\br{\\[5 pt]}
\newcommand{\figp}[1]{../fig/#1}
\newcommand\algvv[1]{\vs\vs\begin{align*} #1 \end{align*}}
\newcommand{\y}[1]{$ {#1} $}
\newcommand{\os}{\\[5 pt]}
\newcommand{\prbxl}[2]{
\parbox[l][][l]{#1\linewidth}{#2
	}}
\newcommand{\prbxr}[2]{\parbox[r][][l]{#1\linewidth}{
		\setlength{\abovedisplayskip}{5pt}
		\setlength{\belowdisplayskip}{5pt}	
		\setlength{\abovedisplayshortskip}{0pt}
		\setlength{\belowdisplayshortskip}{0pt} 
		\begin{shaded}
			\footnotesize	#2 \end{shaded}}}

\renewcommand{\cfttoctitlefont}{\Large\bfseries}
\setlength{\cftaftertoctitleskip}{0 pt}
\setlength{\cftbeforetoctitleskip}{0 pt}

\newcommand{\bs}{\\[3pt]}
\newcommand{\vn}{\\[6pt]}
\newcommand{\fig}[1]{\begin{figure}
		\centering
		\includegraphics[]{\figp{#1}}
\end{figure}}

\newcommand{\figc}[2]{\begin{figure}
		\centering
		\includegraphics[]{\figp{#1}}
		\caption{#2}
\end{figure}}

\newcommand{\sectionbreak}{\clearpage} % New page on each section

\newcommand{\nn}[1]{
\begin{equation}
	#1
\end{equation}
}

% Equation comments
\newcommand{\cm}[1]{\llap{\color{blue} #1}}

\newcommand\fork[2]{\begin{tcolorbox}[boxrule=0.3 mm,arc=0mm,enhanced jigsaw,breakable,colback=yellow!7] {\large \textbf{#1 (forklaring)} \vspace{5 pt}\\} #2 \end{tcolorbox}\vspace{-5pt} }
 
%colors
\newcommand{\colr}[1]{{\color{red} #1}}
\newcommand{\colb}[1]{{\color{blue} #1}}
\newcommand{\colo}[1]{{\color{orange} #1}}
\newcommand{\colc}[1]{{\color{cyan} #1}}
\definecolor{projectgreen}{cmyk}{100,0,100,0}
\newcommand{\colg}[1]{{\color{projectgreen} #1}}

% Methods
\newcommand{\metode}[2]{
	\textsl{#1} \\[-8pt]
	\rule{#2}{0.75pt}
}

%Opg
\newcommand{\abc}[1]{
	\begin{enumerate}[label=\alph*),leftmargin=18pt]
		#1
	\end{enumerate}
}
\newcommand{\abcs}[2]{
	\begin{enumerate}[label=\alph*),start=#1,leftmargin=18pt]
		#2
	\end{enumerate}
}
\newcommand{\abcn}[1]{
	\begin{enumerate}[label=\arabic*),leftmargin=18pt]
		#1
	\end{enumerate}
}
\newcommand{\abch}[1]{
	\hspace{-2pt}	\begin{enumerate*}[label=\alph*), itemjoin=\hspace{1cm}]
		#1
	\end{enumerate*}
}
\newcommand{\abchs}[2]{
	\hspace{-2pt}	\begin{enumerate*}[label=\alph*), itemjoin=\hspace{1cm}, start=#1]
		#2
	\end{enumerate*}
}

% Oppgaver
\newcommand{\opgt}{\phantomsection \addcontentsline{toc}{section}{Oppgaver} \section*{Oppgaver for kapittel \thechapter}\vs \setcounter{section}{1}}
\newcounter{opg}
\numberwithin{opg}{section}
\newcommand{\op}[1]{\vspace{15pt} \refstepcounter{opg}\large \textbf{\color{blue}\theopg} \vspace{2 pt} \label{#1} \\}
\newcommand{\ekspop}[1]{\vsk\textbf{Gruble \thechapter.#1}\vspace{2 pt} \\}
\newcommand{\nes}{\stepcounter{section}
	\setcounter{opg}{0}}
\newcommand{\opr}[1]{\vspace{3pt}\textbf{\ref{#1}}}
\newcommand{\oeks}[1]{\begin{tcolorbox}[boxrule=0.3 mm,arc=0mm,colback=white]
		\textit{Eksempel: } #1	  
\end{tcolorbox}}
\newcommand\opgeks[2][]{\begin{tcolorbox}[boxrule=0.1 mm,arc=0mm,enhanced jigsaw,breakable,colback=white] {\footnotesize \textbf{Eksempel #1} \\} \footnotesize #2 \end{tcolorbox}\vspace{-5pt} }
\newcommand{\rknut}{
Rekn ut.
}

%License
\newcommand{\lic}{\textit{Matematikken sine byggesteinar by Sindre Sogge Heggen is licensed under CC BY-NC-SA 4.0. To view a copy of this license, visit\\ 
		\net{http://creativecommons.org/licenses/by-nc-sa/4.0/}{http://creativecommons.org/licenses/by-nc-sa/4.0/}}}

%referances
\newcommand{\net}[2]{{\color{blue}\href{#1}{#2}}}
\newcommand{\hrs}[2]{\hyperref[#1]{\color{blue}\textsl{#2 \ref*{#1}}}}
\newcommand{\rref}[1]{\hrs{#1}{regel}}
\newcommand{\refkap}[1]{\hrs{#1}{kapittel}}
\newcommand{\refsec}[1]{\hrs{#1}{seksjon}}

\newcommand{\mb}{\net{https://sindrsh.github.io/FirstPrinciplesOfMath/}{MB}}


%line to seperate examples
\newcommand{\linje}{\rule{\linewidth}{1pt} }

\usepackage{datetime2}
%%\usepackage{sansmathfonts} for dyslexia-friendly math
\usepackage[]{hyperref}


\newcommand{\note}{Merk}

% Geometry
\newcommand{\hlikb}{Midtnormalen i en likebeint trekant}
\newcommand{\arealsetn}{Arealsetningen}
\newcommand{\trkmedian}{Medianer i trekanter}
\newcommand{\midtrk}{Midtnormaler i trekanter}
\newcommand{\innskrsirk}{Halveringslinjer og innskrevet sirkel i trekanter}
\newcommand{\cossetn}{Cosinussetningen}
\newcommand{\perfvink}{Sentral- og periferivinkel}
\newcommand{\tang}{Tangent}


% Vectors
\newcommand{\detar}{Arealformler med determinanter}

\begin{document}	
\section{Introduksjon}\index{vektor}
En todimensjonal vektor angir en forflytning i et koordinatsystem med en $ x $-akse og en $ y $-akse. En vektor tegner vi som et linjestykke mellom to punkt, i tillegg til at vi lar en pil vise til hva som er endepunktet. Det betyr at forflytningen starter i punktet uten pil, og ender i punktet med pil.
\begin{figure}
	\centering
\subfloat[]{\includegraphics{\figp{vekdefa}}}\qquad
\subfloat[]{\includegraphics{\figp{vekdefb}}}
\end{figure}
I figur \textsl{(a)} er vektoren $ \vec{u} $ vist med startpunkt $ (0, 0) $ og endepunkt $ (3,1) $. Når en vektor har startpunkt $ (0, 0) $, sier vi at den er vist i \textit{grunnstill-ingen}. I figur \textsl{(b)} er $ \vec{u} $ vist med startpunkt $ (1, -2) $ og endepunkt $ (3, 1) $. Forflytningen $ \vec{u} $ viser til er å vandre 2 mot høgre langs $ x $-aksen og $ 3 $ opp langs $ y $-aksen. Dette skriver vi som $ \vec{u}=[2, 3] $, som kalles $ \vec{u} $ skrevet på \textit{komponentform}. 
\newpage
\eks[1]{
\alg{
\vec{a}&= [1, 3] & \vec{b} &= [0, -2] \\[15pt]
\vec{c}&= [-3, -4] & \vec{d}&= [5, 0]
}
\fig{vek1}
}
\newpage
\reg[Vektoren mellom to punkt]{
En vektor $ \vec{v} $ med startpunkt $ {(x_1, y_1)} $ og endepunkt $ (x_2, y_2) $ er gitt som 
\begin{equation}
	\vec{v}= [x_2-x_1, y_2-y_1]
\end{equation}
}
\eks[1]{
Skriv vektorene på komponentform.
\begin{itemize}
	\item $ \vec{a} $ har startpunkt $ (1, 3) $ og endepunkt $ (7, 5) $
	\item $ \vec{b} $ har startpunkt $ (0, 9) $ og endepunkt $ (-3, 2) $
	\item $ \vec{c} $ har startpunkt $ (-3, 7) $ og endepunkt $ (2, -4) $
	\item $ \vec{d} $ har startpunkt $ (-7, -5) $ og endepunkt $ (3, 0) $
\end{itemize}
\sv \vs

\algv{
\vec{a}&=[7-1, 5-3]=[6, 2] \br
\vec{b}&=[-3-0, 2-9]=[-3, -7] \br
\vec{c}&=[2-(-3), -4-7]=[5, -11]\br
\vec{d}&=[3-(-7),0-(-5)]=[10, 5]
}
}
\section{Regneregler}
\reg[Regneregler for vektorer]{
	Gitt vektorene $ {\vec{u}=[x_1, y_1]} $ og $ {\vec{v}=[x_2, y_2]} $, punktet ${ A=(x_0, y_0)} $ og en konstant $ t $. Da er
	\begin{align}
		A+\vec{u} &= (x_0+x_1, y_0+y_1) \label{Aplusu}	\br
		\vec{u}+\vec{v} &= [x_1+x_2, y_1+y_2] \br
		\vec{u}-\vec{v} &= [x_1-x_2, y_1-y_2] \br
		t\vec{u} &= [tx_1, ty_1, tz_1] \br
		t(\vec{u}+\vec{v})&= t\vec{u}+t\vec{v}
	\end{align}
	Summen eller differansen av $ \vec{u} $ og $ \vec{v} $ kan vi tegne slik:
	\begin{figure}
		\centering
		\subfloat[]{\includegraphics[scale=1]{\figp{uplusv}}}\qquad\quad
		\subfloat[]{\includegraphics[scale=1]{\figp{uminusv}}}
	\end{figure}
	For en vektor $ \vec{w} $ har vi videre at
	\begin{align}
		(\vec{u}+\vec{v})+\vec{w} &= \vec{u}+(\vec{v}+\vec{w}) \br
		\vec{u}-(\vec{v}+\vec{w}) &= \vec{u}-\vec{v}-\vec{w}
	\end{align}
}
\section{Lengden til en vektor}
Gitt en vektor $ \vec{v}=[x_1, y_1] $. \textit{Lengden} til $ \vec{v} $ er avstanden mellom startpunktet og endepunktet. 
\begin{figure}
\centering
\subfloat[]{\includegraphics{\figp{veklena}}}\qquad\quad
\subfloat[]{\includegraphics{\figp{veklenb}}}
\end{figure}
Av enhver vektor kan vi danne en rettvinklet trekant hvor $ |\vec{v}| $ er lengden til hypotenusen og $ x_1 $ og $ y_1 $ er de respektive lengdene til katetene. Dermed er $ |\vec{v}| $ gitt av Pytagoras' setning.\regv

\reg[Lengden til en vektor ]{
Gitt en vektor $ \vec{v}=[x_1, y_1] $. Lengden $ |\vec{v}| $ er da 
\begin{equation}
	|\vec{v}|=\sqrt{x_1^2+y_1^2} \label{veklen}
\end{equation}
}
\eks[1]{
Finn lengden til vektorene $ \vec{a}= [7, 4] $ og $ \vec{b}= [-3, 2] $.

\sv \vs
\algv{
	|\vec{a}|&=\sqrt{7^2+4^2}=\sqrt{65} \br
	|\vec{b}|&=\sqrt{(-3)^2+2^2}=\sqrt{13}
}
}
\section{Skalarproduktet I}
\reg[Skalarproduktet I]{
For to vektorer $ {\vec{u}=[x_1, y_1]} $ og $ {\vec{v}=[x_2, y_2]} $, er \textit{skalarproduktet} gitt som
\nn{
\vec{u}\cdot\vec{v}=x_1x_2+y_1y_2 \label{skaldef1}
}
}
\spr{
Skalarproduktet kalles også \textit{prikkproduktet} eller \textit{indreproduktet}.
}
\eks[1]{
Gitt vektorene $ \vec{a}=[3, 2] $, $ \vec{b}=[4,7] $ og $ \vec{c}=[1,-9] $. Regn ut $ \vec{a}\cdot\vec{b} $ og $ \vec{a}\cdot\vec{c} $.

\sv \vsb
\algv{
\vec{a}\cdot\vec{b}&=3\cdot4+2\cdot7=26 \br
\vec{a}\cdot\vec{c}&=3\cdot1+2(-9)=-15
}
} \vsk

\reg[Regneregler for skalarproduktet]{
	For vektorene $ \vec{u} $, $ \vec{v} $ og $ \vec{w} $ har vi at
\begin{align}
	\vec{u}\cdot \vec{u}&=\vec{u}^{\,2} \label{veku2}\br
	\vec{u}\cdot\vec{v} &= \vec{v}\cdot\vec{u} \br
	\vec{u}\cdot\left(\vec{v}+\vec{w}\right) &= \vec{u}\cdot\vec{v}+\vec{u}\cdot\vec{w} \br
	\left(\vec{u}+\vec{v}\right)^2 &= \vec{u}^{\,2} + 2\vec{u}\cdot\vec{v}+\vec{v}^{\,2}
\end{align}
}
\eks[]{Forkort uttrykket
	\[   \vec{b}\cdot(\vec{a}+\vec{c}) + \vec{a}\cdot(\vec{a}+\vec{b})+\vec{b}^{\,2} \]
	
	når du vet at $ \vec{b}\cdot\vec{c}=0 $. \\
	
	\sv
	\algv{\vec{b}\cdot(\vec{a}+\vec{c}) + \vec{a}\cdot(\vec{a}+\vec{b})+\vec{b}^{\,2} &= \vec{b}\cdot\vec{a}+\vec{b}\cdot\vec{c} + \vec{a}\cdot\vec{a}+\vec{a}\cdot\vec{b}+\vec{b}^2 \\
		&= \vec{a}^{\,2} + 2 \vec{a}\cdot \vec{b} + \vec{b}^{\,2} \\
		&= \left(\vec{a}+ \vec{b}\right)^2
	}
}
\section{Skalarproduktet II}
Vinkelen mellom to vektorer er (den minste) vinkelen som blir dannet når vektorene plasseres i samme startpunkt. For to vektorer $ \vec{u} $ og $ \vec{v} $ skriver vi denne vinkelen som $ \angle(\vec{u}, \vec{v}) $.
\fig{vekvink}
I vektorregning er det vanlig å oppgi vinkler i grader, altså fra intervallet $ [0^\circ, 180^\circ] $.\vsk

Gitt vektoren $ \vec{u}-\vec{v} $, hvor 
$ \vec{u}=[x_1, y_1] $ og $ \vec{v}=[x_2, y_2] $. Da er
\[ \vec{u}-\vec{v}=[x_1-x_2, y_1-y_2] \]
Av \eqref{veklen} har vi at
\begin{align}
	|\vec{u}-\vec{v}|&= \sqrt{(x_1-x_2)^2 + (y_1-y_2)^2}  \\
	&= \sqrt{x_1^2 - 2x_1 x_2 + x_2^2 + y_1^2 - 2y_1 y_2 + y_2^2 }\label{preskal}
\end{align}

Ved hjelp av \eqref{skaldef1} og \eqref{veku2} kan vi skrive \eqref{preskal} som
\begin{equation}
	|\vec{u}-\vec{v}|= \sqrt{\vec{u}^{\,2} - 2\vec{u}\cdot\vec{v} + \vec{v}^{\,2}} \label{skl1}
\end{equation}
Videre merker vi oss følgende figur:
\fig{skaldef}.
Av cosinussetningen\footnote{Se ??} og \eqref{skl1} er
\alg{
	|(\vec{v}-\vec{u})|^2&= |\vec{v}|^2+|\vec{u}|^2-2\vec{u}||\vec{v}|\cos \angle(\vec{u}, \vec{v}) \\
	\vec{v}^{\,2}- 2\vec{u}\cdot \vec{v} + \vec{u}^{\,2} &= \vec{v}^{\,2}+\vec{u}^{\,2}-2|\vec{u}||\vec{v}|\cos \angle(\vec{u}, \vec{v}) \\
	\vec{u}\cdot \vec{v} &= |\vec{u}||\vec{v}|\cos \angle(\vec{u}, \vec{v})
}
\reg[Skalarproduktet II]{
For to vektorer $ \vec{u} $ og $ \vec{v} $ er
\begin{equation}\label{skal2}
	\vec{u}\cdot \vec{v} = |\vec{u}||\vec{v}|\cos \angle(\vec{u}, \vec{v})
\end{equation}
}
\section{Vektorer vinkelrett på hverandre}
Fra (\ref{skal2}) kan vi gjøre en viktig observasjon; Hvis $\angle (\vec{u},\vec{v})=90^\circ $, er $ {\cos\angle (\vec{u},\vec{v})=0} $, og da blir
\nn{
	\vec{u}\cdot\vec{v}=0
}
\reg[Vinkelrette vektorer \label{vinkrett} ]{
	\begin{equation}\label{vinkretteq}
		\vec{u} \cdot \vec{v} = 0\iff \vec{u}\perp \vec{v}
	\end{equation}
}
\spr{
Det er mange måter å uttrykke at $ {\vec{u}\perp \vec{v}} $ på. Blant annet kan vi si at
\begin{itemize}
	\item $ \vec{u} $ og $ \vec{v} $ står vinkelrett på hverandre.
	\item $ \vec{u} $ og $ \vec{v} $ står normalt på hverandre. 
	\item $\vec{u} $ er en normalvektor til $ \vec{v}$ (og omvendt).
	\item $\vec{u} $ og $ \vec{v}$ er ortogonale.
\end{itemize}
}
\eks[1]{
	Sjekk om vektorene $ {\vec{a}=[5, -3]} $, $ {\vec{b}=[6, -10]} $ og $ {\vec{c}=[2, 7] }$ er ortogonale.
	
	\sv 
	Vi har at
	\algv{
		\vec{a}\cdot\vec{b}&=5\cdot6+(-3)10 \\
		&=0
	}
	Altså er $ \vec{a}\perp\vec{b} $. Videre er
	\alg{
	\vec{a}\cdot\vec{c}&=5\cdot2+(-3)7\cdot \\
	&= 11
	}
Altså er $ \vec{a} $ og $ \vec{c} $ \textsl{ikke} ortogonale. Da $ {\vec{a}\perp\vec{b}} $, kan heller ikke $ \vec{b} $ og $ \vec{c} $ være ortogonale.
}
\newpage
\info{Nullvektoren}{
I forkant av \rref{vinkrett} har vi bare argumentert for at\\ ${\vec{u}\perp\vec{v}\Rightarrow \vec{u}\cdot\vec{v}=0} $. For å rettferdiggjøre betingelsen som går begge veier i \eqref{vinkretteq}, må vi spørre: Kan vi få ${ \vec{u}\cdot\vec{v} =0}$ om vinkelen mellom $ \vec{u} $ og $ \vec{v} $ \textsl{ikke} er $ 90^\circ $? \vsk

På intervallet $ [0^\circ, 180^\circ) $ er det bare en vinkel lik $ 90^\circ $ som resulterer i cosinusverdi 0. Skal skalarproduktet bli 0 for andre vinkler, må derfor lengden av $ \vec{u} $ eller $ \vec{v} $ være 0. Den eneste vektoren med denne lengden er \textit{nullvektoren} ${\vec{0}=[0, 0] }$, som rett og slett ikke har noen retning\footnote{Eventuelt kan man hevde at den peker i alle retninger!}. Det er likevel vanlig å definere at nullvektoren står vinkelrett på \textit{alle} vektorer.
}
\section{Parallelle vektorer}
Vi har tidligere sett hvordan finne lengden til en vektor, men en vektor har også en retning. Retningen til en vektor kan uttrykkes ved (den minste) vinkelen vektoren danner med $ x $-aksen når den er plassert i grunnstillingen. Hvis to vektorer danner den samme vinkelen med $ x $-aksen er de parallelle.
\begin{figure}
\centering
\subfloat[]{\includegraphics{\figp{vekpara}}}\qquad
\subfloat[]{\includegraphics{\figp{vekparb}}}
\end{figure}
Gitt to vektorer $ \vec{u}=[x_1, y_1] $ og $ \vec{v}=[x_2, y_2] $. La $ \theta $ og $ \alpha $ være vinkelen mellom $ x $-aksen og henholdsvis $ \vec{u} $ og $ \vec{v} $. Alle tangensverdier til vinkler på intervallet $ [0^\circ, 180^\circ) $ er unike, og dette betyr at hvis $ \theta=\alpha $, må vi ha at
\[ \frac{y_1}{x_1}=\frac{y_2}{x_2} \]
Sammenhengen blir enda mer tydelig hvis vi omskriver likningen over til å gjelde de samsvarende komponentene\footnote{
For vektorene $ [x_1, y_1] $ og $ [x_2, y_2] $ er disse samsvarende komponenter: \vspace{-6pt}
\begin{itemize}
	\item $ x_1 $ og $ x_2 $
	\item $ y_1 $ og $ y_2 $
\end{itemize}
}: \regv
\reg[Parallelle vektorer]{
	To vektorer $ {\vec{u}=[x_1, y_1]} \text{ og } {\vec{v}=[x_2, y_2]} $ er parallelle hvis forholdet mellom samsvarende komponenter er likt.
	\begin{equation}
		\frac{x_1}{x_2}=\frac{y_1}{y_2} \iff \vec{u}\parallel\vec{v}
	\end{equation}
Alternativt, for et tall $ t $ har vi at
\begin{equation}
 \vec{u}=t\vec{v}\iff \vec{u}\parallel\vec{v}
\end{equation}
}
\spr{
Når $ {\vec{u}=t \vec{v}} $, sier vi at $ \vec{u} $ er et \textit{multiplum} av $ \vec{v} $ (og omvendt). Vi sier også at $ \vec{u} $ og $ \vec{v} $ er \textit{lineært uavhengige}. 
}
\eks[]{
Undersøk hvorvidt $ {\vec{a}=[2, -3]} $ og $ {\vec{b}=[20, -45]} $ er parallelle med $ {\vec{c}=[10, -15]} $.

\sv
Vi har at
\nn{
\vec{c}=5[2, -4]=5\vec{a}
}
Dermed er $ \vec{a}\parallel \vec{c} $. Da $ \frac{20}{10}\neq \frac{-45}{15} $, er $ \vec{b} $ og $ \vec{c} $ \textsl{ikke} parallelle.

}
\end{document}