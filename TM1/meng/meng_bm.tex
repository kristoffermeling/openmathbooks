\documentclass[english, 11 pt, class=article, crop=false]{standalone}
\usepackage[T1]{fontenc}
%\renewcommand*\familydefault{\sfdefault} % For dyslexia-friendly text
\usepackage{lmodern} % load a font with all the characters
\usepackage{geometry}
\geometry{verbose,paperwidth=16.1 cm, paperheight=24 cm, inner=2.3cm, outer=1.8 cm, bmargin=2cm, tmargin=1.8cm}
\setlength{\parindent}{0bp}
\usepackage{import}
\usepackage[subpreambles=false]{standalone}
\usepackage{amsmath}
\usepackage{amssymb}
\usepackage{esint}
\usepackage{babel}
\usepackage{tabu}
\makeatother
\makeatletter

\usepackage{titlesec}
\usepackage{ragged2e}
\RaggedRight
\raggedbottom
\frenchspacing

% Norwegian names of figures, chapters, parts and content
\addto\captionsenglish{\renewcommand{\figurename}{Figur}}
\makeatletter
\addto\captionsenglish{\renewcommand{\chaptername}{Kapittel}}
\addto\captionsenglish{\renewcommand{\partname}{Del}}


\usepackage{graphicx}
\usepackage{float}
\usepackage{subfig}
\usepackage{placeins}
\usepackage{cancel}
\usepackage{framed}
\usepackage{wrapfig}
\usepackage[subfigure]{tocloft}
\usepackage[font=footnotesize,labelfont=sl]{caption} % Figure caption
\usepackage{bm}
\usepackage[dvipsnames, table]{xcolor}
\definecolor{shadecolor}{rgb}{0.105469, 0.613281, 1}
\colorlet{shadecolor}{Emerald!15} 
\usepackage{icomma}
\makeatother
\usepackage[many]{tcolorbox}
\usepackage{multicol}
\usepackage{stackengine}

\usepackage{esvect} %For vectors with capital letters

% For tabular
\usepackage{array}
\usepackage{multirow}
\usepackage{longtable} %breakable table

% Ligningsreferanser
\usepackage{mathtools}
\mathtoolsset{showonlyrefs}

% index
\usepackage{imakeidx}
\makeindex[title=Indeks]

%Footnote:
\usepackage[bottom, hang, flushmargin]{footmisc}
\usepackage{perpage} 
\MakePerPage{footnote}
\addtolength{\footnotesep}{2mm}
\renewcommand{\thefootnote}{\arabic{footnote}}
\renewcommand\footnoterule{\rule{\linewidth}{0.4pt}}
\renewcommand{\thempfootnote}{\arabic{mpfootnote}}

%colors
\definecolor{c1}{cmyk}{0,0.5,1,0}
\definecolor{c2}{cmyk}{1,0.25,1,0}
\definecolor{n3}{cmyk}{1,0.,1,0}
\definecolor{neg}{cmyk}{1,0.,0.,0}

% Lister med bokstavar
\usepackage[inline]{enumitem}

\newcounter{rg}
\numberwithin{rg}{chapter}
\newcommand{\reg}[2][]{\begin{tcolorbox}[boxrule=0.3 mm,arc=0mm,colback=blue!3] {\refstepcounter{rg}\phantomsection \large \textbf{\therg \;#1} \vspace{5 pt}}\newline #2  \end{tcolorbox}\vspace{-5pt}}

\newcommand\alg[1]{\begin{align} #1 \end{align}}

\newcommand\eks[2][]{\begin{tcolorbox}[boxrule=0.3 mm,arc=0mm,enhanced jigsaw,breakable,colback=green!3] {\large \textbf{Eksempel #1} \vspace{5 pt}\\} #2 \end{tcolorbox}\vspace{-5pt} }

\newcommand{\st}[1]{\begin{tcolorbox}[boxrule=0.0 mm,arc=0mm,enhanced jigsaw,breakable,colback=yellow!12]{ #1} \end{tcolorbox}}

\newcommand{\spr}[1]{\begin{tcolorbox}[boxrule=0.3 mm,arc=0mm,enhanced jigsaw,breakable,colback=yellow!7] {\large \textbf{Språkboksen} \vspace{5 pt}\\} #1 \end{tcolorbox}\vspace{-5pt} }

\newcommand{\sym}[1]{\colorbox{blue!15}{#1}}

\newcommand{\info}[2]{\begin{tcolorbox}[boxrule=0.3 mm,arc=0mm,enhanced jigsaw,breakable,colback=cyan!6] {\large \textbf{#1} \vspace{5 pt}\\} #2 \end{tcolorbox}\vspace{-5pt} }

\newcommand\algv[1]{\vspace{-11 pt}\begin{align*} #1 \end{align*}}

\newcommand{\regv}{\vspace{5pt}}
\newcommand{\mer}{\textsl{Merk}: }
\newcommand{\mers}[1]{{\footnotesize \mer #1}}
\newcommand\vsk{\vspace{11pt}}
\newcommand\vs{\vspace{-11pt}}
\newcommand\vsb{\vspace{-16pt}}
\newcommand\sv{\vsk \textbf{Svar} \vspace{4 pt}\\}
\newcommand\br{\\[5 pt]}
\newcommand{\figp}[1]{../fig/#1}
\newcommand\algvv[1]{\vs\vs\begin{align*} #1 \end{align*}}
\newcommand{\y}[1]{$ {#1} $}
\newcommand{\os}{\\[5 pt]}
\newcommand{\prbxl}[2]{
\parbox[l][][l]{#1\linewidth}{#2
	}}
\newcommand{\prbxr}[2]{\parbox[r][][l]{#1\linewidth}{
		\setlength{\abovedisplayskip}{5pt}
		\setlength{\belowdisplayskip}{5pt}	
		\setlength{\abovedisplayshortskip}{0pt}
		\setlength{\belowdisplayshortskip}{0pt} 
		\begin{shaded}
			\footnotesize	#2 \end{shaded}}}

\renewcommand{\cfttoctitlefont}{\Large\bfseries}
\setlength{\cftaftertoctitleskip}{0 pt}
\setlength{\cftbeforetoctitleskip}{0 pt}

\newcommand{\bs}{\\[3pt]}
\newcommand{\vn}{\\[6pt]}
\newcommand{\fig}[1]{\begin{figure}
		\centering
		\includegraphics[]{\figp{#1}}
\end{figure}}

\newcommand{\figc}[2]{\begin{figure}
		\centering
		\includegraphics[]{\figp{#1}}
		\caption{#2}
\end{figure}}

\newcommand{\sectionbreak}{\clearpage} % New page on each section

\newcommand{\nn}[1]{
\begin{equation}
	#1
\end{equation}
}

% Equation comments
\newcommand{\cm}[1]{\llap{\color{blue} #1}}

\newcommand\fork[2]{\begin{tcolorbox}[boxrule=0.3 mm,arc=0mm,enhanced jigsaw,breakable,colback=yellow!7] {\large \textbf{#1 (forklaring)} \vspace{5 pt}\\} #2 \end{tcolorbox}\vspace{-5pt} }
 
%colors
\newcommand{\colr}[1]{{\color{red} #1}}
\newcommand{\colb}[1]{{\color{blue} #1}}
\newcommand{\colo}[1]{{\color{orange} #1}}
\newcommand{\colc}[1]{{\color{cyan} #1}}
\definecolor{projectgreen}{cmyk}{100,0,100,0}
\newcommand{\colg}[1]{{\color{projectgreen} #1}}

% Methods
\newcommand{\metode}[2]{
	\textsl{#1} \\[-8pt]
	\rule{#2}{0.75pt}
}

%Opg
\newcommand{\abc}[1]{
	\begin{enumerate}[label=\alph*),leftmargin=18pt]
		#1
	\end{enumerate}
}
\newcommand{\abcs}[2]{
	\begin{enumerate}[label=\alph*),start=#1,leftmargin=18pt]
		#2
	\end{enumerate}
}
\newcommand{\abcn}[1]{
	\begin{enumerate}[label=\arabic*),leftmargin=18pt]
		#1
	\end{enumerate}
}
\newcommand{\abch}[1]{
	\hspace{-2pt}	\begin{enumerate*}[label=\alph*), itemjoin=\hspace{1cm}]
		#1
	\end{enumerate*}
}
\newcommand{\abchs}[2]{
	\hspace{-2pt}	\begin{enumerate*}[label=\alph*), itemjoin=\hspace{1cm}, start=#1]
		#2
	\end{enumerate*}
}

% Oppgaver
\newcommand{\opgt}{\phantomsection \addcontentsline{toc}{section}{Oppgaver} \section*{Oppgaver for kapittel \thechapter}\vs \setcounter{section}{1}}
\newcounter{opg}
\numberwithin{opg}{section}
\newcommand{\op}[1]{\vspace{15pt} \refstepcounter{opg}\large \textbf{\color{blue}\theopg} \vspace{2 pt} \label{#1} \\}
\newcommand{\ekspop}[1]{\vsk\textbf{Gruble \thechapter.#1}\vspace{2 pt} \\}
\newcommand{\nes}{\stepcounter{section}
	\setcounter{opg}{0}}
\newcommand{\opr}[1]{\vspace{3pt}\textbf{\ref{#1}}}
\newcommand{\oeks}[1]{\begin{tcolorbox}[boxrule=0.3 mm,arc=0mm,colback=white]
		\textit{Eksempel: } #1	  
\end{tcolorbox}}
\newcommand\opgeks[2][]{\begin{tcolorbox}[boxrule=0.1 mm,arc=0mm,enhanced jigsaw,breakable,colback=white] {\footnotesize \textbf{Eksempel #1} \\} \footnotesize #2 \end{tcolorbox}\vspace{-5pt} }
\newcommand{\rknut}{
Rekn ut.
}

%License
\newcommand{\lic}{\textit{Matematikken sine byggesteinar by Sindre Sogge Heggen is licensed under CC BY-NC-SA 4.0. To view a copy of this license, visit\\ 
		\net{http://creativecommons.org/licenses/by-nc-sa/4.0/}{http://creativecommons.org/licenses/by-nc-sa/4.0/}}}

%referances
\newcommand{\net}[2]{{\color{blue}\href{#1}{#2}}}
\newcommand{\hrs}[2]{\hyperref[#1]{\color{blue}\textsl{#2 \ref*{#1}}}}
\newcommand{\rref}[1]{\hrs{#1}{regel}}
\newcommand{\refkap}[1]{\hrs{#1}{kapittel}}
\newcommand{\refsec}[1]{\hrs{#1}{seksjon}}

\newcommand{\mb}{\net{https://sindrsh.github.io/FirstPrinciplesOfMath/}{MB}}


%line to seperate examples
\newcommand{\linje}{\rule{\linewidth}{1pt} }

\usepackage{datetime2}
%%\usepackage{sansmathfonts} for dyslexia-friendly math
\usepackage[]{hyperref}


\newcommand{\note}{Merk}

% Geometry
\newcommand{\hlikb}{Midtnormalen i en likebeint trekant}
\newcommand{\arealsetn}{Arealsetningen}
\newcommand{\trkmedian}{Medianer i trekanter}
\newcommand{\midtrk}{Midtnormaler i trekanter}
\newcommand{\innskrsirk}{Halveringslinjer og innskrevet sirkel i trekanter}
\newcommand{\cossetn}{Cosinussetningen}
\newcommand{\perfvink}{Sentral- og periferivinkel}
\newcommand{\tang}{Tangent}


% Vectors
\newcommand{\detar}{Arealformler med determinanter}

\begin{document}
\section{Mengder}
En samling av tall kalles en \textit{mengde}\footnote{En mengde kan også være en samling av andre matematiske objekter, som for eksempel funksjoner, men i denne boka holder det å se på mengder av tall.
}, og et tall som er en del av en mengde kalles et \textit{element}. Mengder kan inneholde et endelig\\ antall elementer og de kan inneholde uendelig mange elementer. \regv
\reg[Mengder]{
For to reelle tall $ a $ og $ b $, hvor $ a\leq b $, har vi at
\begin{center}
	\begin{tabular}{c l}
		$ [a, b] $ & er mengden av alle reelle tall større eller lik $ a $ \\
		&og mindre eller lik $ b $. \\
		$ (a, b] $ & er mengden av alle reelle tall større enn $ a $ \\
		&og mindre eller lik $ b $.\\ 
		$ [a, b) $ & er mengden av alle reelle tall større eller lik $ a $ \\
		&og mindre enn $ b $.
	\end{tabular}
\end{center}
$ [a, b] $ kalles et lukket intervall, $ (a, b) $ kalles et åpent intervall, og både $ (a, b] $ og $ [a, b) $ kalles halvåpne intervall.\vsk

Mengden som inneholder bare $ a $ og $ b $ skrives som $ \{a, b\} $. \vsk

At $ x $ er et element i en mengde $ M $, skrives som $ x\in M $.\vsk

At $ x $ ikke er et element i en mengde $ M $, skrives som $ x\not \in M $. \vsk

At $ x $ er et element i både en mengde $ M_1 $ og en mengde $ M_2 $, skrives som $ x \in M_1 \cup M_2 $. 
}
\spr{
$ x \in M$ uttales ''$ x $ inneholdt i $ M $''.\vsk

Mange tekster bruker \sym{$ \langle $} istedenfor \sym{$ ( $} for å indikere åpne (eller halvåpne) intervall.
}
\info{\note}{
Når vi heretter i boka definerer et intervall beskrevet av $ a $ og $ b $, tar vi det for gitt at $ a $ og $ b $ er to reelle tall og at $ a\leq b $.
}
\newpage
\eks[1]{
Mengden av alle heltall større enn 0 og mindre enn 10 skriver vi som 
\[ \{1, 2, 3, 4, 5, 6, 7, 8, 9\} \]
Denne mengden inneholder 9 elementer. 3 er et element i denne mengden, og da kan vi skrive $ 3\in\{1, 2, 3, 4, 5, 6, 7, 8, 9\}  $\vsk 

10 er ikke et element i denne mengden, og da kan vi skrive \\$ 10 \not\in  \{1, 2, 3, 4, 5, 6, 7, 8, 9\} $.
} 
\eks[2]{
Skriv opp ulikhetene som gjelder for alle $ {x\in M} $, og om 1 er inneholdt i $ M $.
\abc{
\item $ M = [0, 1] $
\item $ M = (0, 1] $
\item $ M = [0, 1) $
}
\sv \vs
\abc{
\item $ 0\leq x \leq 1 $. Videre er $ 1\in M $.
\item $ 0< x \leq 1 $. Videre er $ 1\in M $.
\item $ 0\leq x < 1 $. Videre er $ 1\not \in M $.
}
}\vsk

\regdef[Navn på mengder \label{mengder}]{\vs
\begin{center}
	\begin{tabular}{c l}
		$ \mathbb{N} $ & Mengden av alle positive heltall\footnote{Inneholder \textit{ikke} 0.}\\
		$ \mathbb{Z} $ & Mengden av alle heltall\footnote{Inneholder 0.}\\
		$ \mathbb{Q} $ & Mengden av alle rasjonale tall\\
		$ \mathbb{R} $ & Mengden av alle reelle tall\\
		$ \mathbb{C} $ & Mengden av alle komplekse tall\\
	\end{tabular}
\end{center}
}
\subsubsection{Symbolet for uendelig}
Mengdene i \dref{mengder} inneholder uendelig mange elementer. Noen ganger ønsker vi å avgrense deler av en uendelig mengde, og da melder det seg et behov for et symbol som  er med på å symbolisere dette. \sym{$ \infty $} er symbolet for en uendelig stor, positiv verdi.\regv 

\eks[]{
Et vilkår om at $ \geq>2 $ kan vi skrive som $ x\in[2 ,\, \infty) $.\vsk

Et vilkår om at $ x<-7 $ kan vi skrive som $ x\in(-\infty ,-7) $.\vsk
}
\spr{
De to intervallene i eksempelet over kan også skrives som $ [2, \rightarrow] $ og $ (\leftarrow, -7) $.
}
\info{\note}{
\sym{$ \infty $} er ikke noe bestemt tall. Å bruke de fire grunnleggende regneartene alene med dette symbolet gir derfor ingen mening.
}
\section{Verdi- og definisjonsmengder}
\regdef[Verdi- og definisjonsmengder]{
Gitt en funksjon $ f(x) $. Mengden som utelukkende inneholder alle verdier $ x $ kan ha, er definisjonsmengden til $ f $. Denne mengden skrives som $ D_f $. Mengden som utelukkende inneholder alle verdier $ f $ kan ha når $ {x\in D_f} $, er verdimengden til $ f $. Denne mengden skrives som $ V_f $. 
}
\eks[1]
{ \label{mengeks1}
Figuren under viser$ {f(x)=2x+1} $, hvor $ {D_f=[1, 3]} $. Da er ${ V_f=[1, 5]} $.
\fig{defmengeks1}
}
\eks[2]{
Figuren under viser $ f(x)={\frac{1}{x}} $, hvor $ {D_f}=[-3, -1] \cup [2, 5] $. Da er $ {V_f=\left[-1, -\frac{1}{3}\right]\cup \left[\frac{1}{2},\frac{1}{5}\right]} $
\fig{defmengeks2}
}
\newpage
\info{\note}{
Definisjonsmengden til en funksjon bestemmes av to ting; hvilken sammenheng funksjonen skal brukes i og eventuelle verdier som gir et udefinert funksjonsuttrykk. I \textsl{Eksempel 1} på side \pageref{mengeks1} er definisjonsmengden helt vilkårlig valgt, siden funksjonen er definert for alle $ x $. I \textsl{Eksempel 2} derimot er ikke funksjonen definert for $ {x=0} $, så en definisjonsmengde som inneholdt denne verdien for $ x $ ville ikke gitt mening.
}
\section{Betingelser}
Symbolet \sym{$ \Rightarrow $} bruker vi for å vise til at hvis et vilkår er oppfylt, så er en annen (eller flere) vilkår også oppfylt. For eksempel; i \mb\;så vi at hvis en trekant er rettvinklet, er Pytagoras' setning gyldig. Dette kan vi skrive slik:
\[ \text{trekanten er rettvinklet} \Rightarrow \text{Pytagoras' setning er gyldig}\]
Men vi så også at det omvendte gjelder; hvis Pytagoras' setning er gyldig, må trekanten være rettvinklet. Da kan vi skrive
\[ \text{trekanten er rettvinklet} \iff \text{Pytagoras' setning er gyldig}\]
Det er veldig viktig å være bevisst forskjellen på \sym{$ \Rightarrow $} og \sym{$ \iff $}; at vilkår A oppfylt gir B oppfylt, trenger ikke å bety at vilkår B oppfylt gir vilkår A oppfylt! \regv

\eks[1]{ \vs \vs
\[ \text{firkanten er et kvadrat}\Rightarrow\text{firkanten har fire like lange sider} \]
}
\eks[2]{ \vs 
	\[ \text{tallet er et primtall større enn 2}\Rightarrow\text{tallet er et oddetall} \]
}
\eks[3]{\vs
\[ \text{tallet er et partall}\iff\text{tallet er delelig med 2} \]
}
\newpage
\subsubsection{Funksjoner med betingelser}
Funksjoner kan gjerne ha flere uttrykk som gjelder for forskjellige vilkår. La oss for eksempel definere en funksjon $ f(x) $ slik:
\alg{
&\text{For }x<1 \text{ er funksjonsuttrykket } {-2x+1}\\
&\text{For }x\geq 1\text{ er funksjonsuttrykket } x^2-2x
}
\fig{funkvilk}
Dette kan vi skrive som
\begin{equation}f(x)= \left\lbrace{
		\begin{array}{rcr}
			-2x+1 &,&x<1 \\
			x^2-2x   &,& x\geq 1
		\end{array}
	}\right. 
\end{equation}

\end{document}