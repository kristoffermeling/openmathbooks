\documentclass[english, 11 pt, class=article, crop=false]{standalone}
\usepackage[T1]{fontenc}
%\renewcommand*\familydefault{\sfdefault} % For dyslexia-friendly text
\usepackage{lmodern} % load a font with all the characters
\usepackage{geometry}
\geometry{verbose,paperwidth=16.1 cm, paperheight=24 cm, inner=2.3cm, outer=1.8 cm, bmargin=2cm, tmargin=1.8cm}
\setlength{\parindent}{0bp}
\usepackage{import}
\usepackage[subpreambles=false]{standalone}
\usepackage{amsmath}
\usepackage{amssymb}
\usepackage{esint}
\usepackage{babel}
\usepackage{tabu}
\makeatother
\makeatletter

\usepackage{titlesec}
\usepackage{ragged2e}
\RaggedRight
\raggedbottom
\frenchspacing

% Norwegian names of figures, chapters, parts and content
\addto\captionsenglish{\renewcommand{\figurename}{Figur}}
\makeatletter
\addto\captionsenglish{\renewcommand{\chaptername}{Kapittel}}
\addto\captionsenglish{\renewcommand{\partname}{Del}}


\usepackage{graphicx}
\usepackage{float}
\usepackage{subfig}
\usepackage{placeins}
\usepackage{cancel}
\usepackage{framed}
\usepackage{wrapfig}
\usepackage[subfigure]{tocloft}
\usepackage[font=footnotesize,labelfont=sl]{caption} % Figure caption
\usepackage{bm}
\usepackage[dvipsnames, table]{xcolor}
\definecolor{shadecolor}{rgb}{0.105469, 0.613281, 1}
\colorlet{shadecolor}{Emerald!15} 
\usepackage{icomma}
\makeatother
\usepackage[many]{tcolorbox}
\usepackage{multicol}
\usepackage{stackengine}

\usepackage{esvect} %For vectors with capital letters

% For tabular
\usepackage{array}
\usepackage{multirow}
\usepackage{longtable} %breakable table

% Ligningsreferanser
\usepackage{mathtools}
\mathtoolsset{showonlyrefs}

% index
\usepackage{imakeidx}
\makeindex[title=Indeks]

%Footnote:
\usepackage[bottom, hang, flushmargin]{footmisc}
\usepackage{perpage} 
\MakePerPage{footnote}
\addtolength{\footnotesep}{2mm}
\renewcommand{\thefootnote}{\arabic{footnote}}
\renewcommand\footnoterule{\rule{\linewidth}{0.4pt}}
\renewcommand{\thempfootnote}{\arabic{mpfootnote}}

%colors
\definecolor{c1}{cmyk}{0,0.5,1,0}
\definecolor{c2}{cmyk}{1,0.25,1,0}
\definecolor{n3}{cmyk}{1,0.,1,0}
\definecolor{neg}{cmyk}{1,0.,0.,0}

% Lister med bokstavar
\usepackage[inline]{enumitem}

\newcounter{rg}
\numberwithin{rg}{chapter}
\newcommand{\reg}[2][]{\begin{tcolorbox}[boxrule=0.3 mm,arc=0mm,colback=blue!3] {\refstepcounter{rg}\phantomsection \large \textbf{\therg \;#1} \vspace{5 pt}}\newline #2  \end{tcolorbox}\vspace{-5pt}}

\newcommand\alg[1]{\begin{align} #1 \end{align}}

\newcommand\eks[2][]{\begin{tcolorbox}[boxrule=0.3 mm,arc=0mm,enhanced jigsaw,breakable,colback=green!3] {\large \textbf{Eksempel #1} \vspace{5 pt}\\} #2 \end{tcolorbox}\vspace{-5pt} }

\newcommand{\st}[1]{\begin{tcolorbox}[boxrule=0.0 mm,arc=0mm,enhanced jigsaw,breakable,colback=yellow!12]{ #1} \end{tcolorbox}}

\newcommand{\spr}[1]{\begin{tcolorbox}[boxrule=0.3 mm,arc=0mm,enhanced jigsaw,breakable,colback=yellow!7] {\large \textbf{Språkboksen} \vspace{5 pt}\\} #1 \end{tcolorbox}\vspace{-5pt} }

\newcommand{\sym}[1]{\colorbox{blue!15}{#1}}

\newcommand{\info}[2]{\begin{tcolorbox}[boxrule=0.3 mm,arc=0mm,enhanced jigsaw,breakable,colback=cyan!6] {\large \textbf{#1} \vspace{5 pt}\\} #2 \end{tcolorbox}\vspace{-5pt} }

\newcommand\algv[1]{\vspace{-11 pt}\begin{align*} #1 \end{align*}}

\newcommand{\regv}{\vspace{5pt}}
\newcommand{\mer}{\textsl{Merk}: }
\newcommand{\mers}[1]{{\footnotesize \mer #1}}
\newcommand\vsk{\vspace{11pt}}
\newcommand\vs{\vspace{-11pt}}
\newcommand\vsb{\vspace{-16pt}}
\newcommand\sv{\vsk \textbf{Svar} \vspace{4 pt}\\}
\newcommand\br{\\[5 pt]}
\newcommand{\figp}[1]{../fig/#1}
\newcommand\algvv[1]{\vs\vs\begin{align*} #1 \end{align*}}
\newcommand{\y}[1]{$ {#1} $}
\newcommand{\os}{\\[5 pt]}
\newcommand{\prbxl}[2]{
\parbox[l][][l]{#1\linewidth}{#2
	}}
\newcommand{\prbxr}[2]{\parbox[r][][l]{#1\linewidth}{
		\setlength{\abovedisplayskip}{5pt}
		\setlength{\belowdisplayskip}{5pt}	
		\setlength{\abovedisplayshortskip}{0pt}
		\setlength{\belowdisplayshortskip}{0pt} 
		\begin{shaded}
			\footnotesize	#2 \end{shaded}}}

\renewcommand{\cfttoctitlefont}{\Large\bfseries}
\setlength{\cftaftertoctitleskip}{0 pt}
\setlength{\cftbeforetoctitleskip}{0 pt}

\newcommand{\bs}{\\[3pt]}
\newcommand{\vn}{\\[6pt]}
\newcommand{\fig}[1]{\begin{figure}
		\centering
		\includegraphics[]{\figp{#1}}
\end{figure}}

\newcommand{\figc}[2]{\begin{figure}
		\centering
		\includegraphics[]{\figp{#1}}
		\caption{#2}
\end{figure}}

\newcommand{\sectionbreak}{\clearpage} % New page on each section

\newcommand{\nn}[1]{
\begin{equation}
	#1
\end{equation}
}

% Equation comments
\newcommand{\cm}[1]{\llap{\color{blue} #1}}

\newcommand\fork[2]{\begin{tcolorbox}[boxrule=0.3 mm,arc=0mm,enhanced jigsaw,breakable,colback=yellow!7] {\large \textbf{#1 (forklaring)} \vspace{5 pt}\\} #2 \end{tcolorbox}\vspace{-5pt} }
 
%colors
\newcommand{\colr}[1]{{\color{red} #1}}
\newcommand{\colb}[1]{{\color{blue} #1}}
\newcommand{\colo}[1]{{\color{orange} #1}}
\newcommand{\colc}[1]{{\color{cyan} #1}}
\definecolor{projectgreen}{cmyk}{100,0,100,0}
\newcommand{\colg}[1]{{\color{projectgreen} #1}}

% Methods
\newcommand{\metode}[2]{
	\textsl{#1} \\[-8pt]
	\rule{#2}{0.75pt}
}

%Opg
\newcommand{\abc}[1]{
	\begin{enumerate}[label=\alph*),leftmargin=18pt]
		#1
	\end{enumerate}
}
\newcommand{\abcs}[2]{
	\begin{enumerate}[label=\alph*),start=#1,leftmargin=18pt]
		#2
	\end{enumerate}
}
\newcommand{\abcn}[1]{
	\begin{enumerate}[label=\arabic*),leftmargin=18pt]
		#1
	\end{enumerate}
}
\newcommand{\abch}[1]{
	\hspace{-2pt}	\begin{enumerate*}[label=\alph*), itemjoin=\hspace{1cm}]
		#1
	\end{enumerate*}
}
\newcommand{\abchs}[2]{
	\hspace{-2pt}	\begin{enumerate*}[label=\alph*), itemjoin=\hspace{1cm}, start=#1]
		#2
	\end{enumerate*}
}

% Oppgaver
\newcommand{\opgt}{\phantomsection \addcontentsline{toc}{section}{Oppgaver} \section*{Oppgaver for kapittel \thechapter}\vs \setcounter{section}{1}}
\newcounter{opg}
\numberwithin{opg}{section}
\newcommand{\op}[1]{\vspace{15pt} \refstepcounter{opg}\large \textbf{\color{blue}\theopg} \vspace{2 pt} \label{#1} \\}
\newcommand{\ekspop}[1]{\vsk\textbf{Gruble \thechapter.#1}\vspace{2 pt} \\}
\newcommand{\nes}{\stepcounter{section}
	\setcounter{opg}{0}}
\newcommand{\opr}[1]{\vspace{3pt}\textbf{\ref{#1}}}
\newcommand{\oeks}[1]{\begin{tcolorbox}[boxrule=0.3 mm,arc=0mm,colback=white]
		\textit{Eksempel: } #1	  
\end{tcolorbox}}
\newcommand\opgeks[2][]{\begin{tcolorbox}[boxrule=0.1 mm,arc=0mm,enhanced jigsaw,breakable,colback=white] {\footnotesize \textbf{Eksempel #1} \\} \footnotesize #2 \end{tcolorbox}\vspace{-5pt} }
\newcommand{\rknut}{
Rekn ut.
}

%License
\newcommand{\lic}{\textit{Matematikken sine byggesteinar by Sindre Sogge Heggen is licensed under CC BY-NC-SA 4.0. To view a copy of this license, visit\\ 
		\net{http://creativecommons.org/licenses/by-nc-sa/4.0/}{http://creativecommons.org/licenses/by-nc-sa/4.0/}}}

%referances
\newcommand{\net}[2]{{\color{blue}\href{#1}{#2}}}
\newcommand{\hrs}[2]{\hyperref[#1]{\color{blue}\textsl{#2 \ref*{#1}}}}
\newcommand{\rref}[1]{\hrs{#1}{regel}}
\newcommand{\refkap}[1]{\hrs{#1}{kapittel}}
\newcommand{\refsec}[1]{\hrs{#1}{seksjon}}

\newcommand{\mb}{\net{https://sindrsh.github.io/FirstPrinciplesOfMath/}{MB}}


%line to seperate examples
\newcommand{\linje}{\rule{\linewidth}{1pt} }

\usepackage{datetime2}
%%\usepackage{sansmathfonts} for dyslexia-friendly math
\usepackage[]{hyperref}


\newcommand{\note}{Merk}

% Geometry
\newcommand{\hlikb}{Midtnormalen i en likebeint trekant}
\newcommand{\arealsetn}{Arealsetningen}
\newcommand{\trkmedian}{Medianer i trekanter}
\newcommand{\midtrk}{Midtnormaler i trekanter}
\newcommand{\innskrsirk}{Halveringslinjer og innskrevet sirkel i trekanter}
\newcommand{\cossetn}{Cosinussetningen}
\newcommand{\perfvink}{Sentral- og periferivinkel}
\newcommand{\tang}{Tangent}


% Vectors
\newcommand{\detar}{Arealformler med determinanter}

\begin{document}
\section{Mengder}
En samling av tall kalles en \textit{mengde}\footnote{En mengde kan også være en samling av andre matematiske objekter, som for eksempel funksjoner, men i denne boka holder det å se på mengder av tall.
}, og et tall som er en del av en mengde kalles et \textit{element} i denne mengden. Mengder kan inneholde et endelig antall elementer og de kan inneholde uendelig mange elementer. \regv
\reg[Mengder]{
For to reelle tall $ a $ og $ b $, hvor $ a<b $, har vi at
\begin{center}
	\begin{tabular}{c l}
		$ [a, b] $ & er mengden av alle reelle tall større eller lik $ a $ \\
		&og mindre eller lik $ b $. \\
		$ (a, b] $ & er mengden av alle reelle tall større enn $ a $ \\
		&og mindre eller lik $ b $.\\ 
		$ [a, b) $ & er mengden av alle reelle tall større eller lik $ a $ \\
		&og mindre enn $ b $.
	\end{tabular}
\end{center}
$ [a, b] $ kalles et lukket intervall, mens både $ (a, b] $ og $ [a, b) $ kalles halvåpne intervall.\vsk

Mengden av tre tall $ a $, $ b $ og $ c $ skrives som $ \{a, b, c\} $.\vsk

At $ x $ er et element i en mengde $ M $ skrives som $ x\in M $.\vsk

At $ x $ ikke er et element i en mengde $ M $ skrives som $ x\not \in M $. 
}
\spr{
$ x \in M$ uttales ''$ x $ inneholdt i $ M $'' eller ''$ x $ er et element i $ M $''.
}
\eks[1]{
Mengden av alle heltall større enn 0 og mindre enn 10 skriver vi som 
\[ \{1, 2, 3, 4, 5, 6, 7, 8, 9\} \]
Denne mengden inneholder 9 elementer. 3 er et element i denne mengden, og da kan vi skrive $ 3\in\{1, 2, 3, 4, 5, 6, 7, 8, 9\}  $\vsk 

10 er ikke et element i denne mengden, og da kan vi skrive \\$ 10 \not\in  \{1, 2, 3, 4, 5, 6, 7, 8, 9\} $.
} 
\eks[2]{
Skriv opp ulikhetene som gjelder for alle $ {x\in M} $, og om 1 er inneholdt i $ M $.
\abc{
\item $ M = [0, 1] $
\item $ M = (0, 1] $
\item $ M = [0, 1) $
}
\sv \vs
\abc{
\item $ 0\leq x \leq 1 $. Videre er $ 1\in M $.
\item $ 0< x \leq 1 $. Videre er $ 1\in M $.
\item $ 0\leq x < 1 $. Videre er $ 1\not \in M $.
}
}\vsk

\reg[Navn på mengder]{\vs
\begin{center}
	\begin{tabular}{c l}
		$ \mathbb{N} $ & Mengden av alle positive heltall\footnote{Inneholder \textit{ikke} 0.}\\
		$ \mathbb{Z} $ & Mengden av alle heltall\footnote{Inneholder 0.}\\
		$ \mathbb{Q} $ & Mengden av alle rasjonale tall\\
		$ \mathbb{R} $ & Mengden av alle reelle tall\\
		$ \mathbb{C} $ & Mengden av alle komplekse tall\\
	\end{tabular}
\end{center}
}
\section{Verdi- og definisjonsmengder}
Alle funksjoner har en definisjonsmengde og en verdimengde. For en funksjon $ f(x) $, er definisjonsmengden den mengden som utelukkende inneholder alle verdier $ x $ kan ha. Denne mengden skrives da som $ D_f $. Hvilke verdier $ x $ kan ha er bestemt av to ting:
\begin{itemize}
	\item Hvilken sammenheng $ x $ skal brukes i.
	\item Om $ f $ ikke er definert for visse $ x $-verdier.
\end{itemize}

La oss først bruke $ f(x)=2x+1 $ som et eksempel. Denne funksjonene er definert for alle $ x\in \mathbb{R} $. Vi kunne derfor latt $ \mathbb{R} $ være definisjonsmengden til $ f $, men for enhelhets skyld velger vi her $ D_f=[0, 1] $. Mengden som utelukkende inneholder alle verdier $ f $ kan ha når $ x\in D_f $, er verdimengden til $ f $. Denne mengden skrives som $ V_f $. I dette tilfellet er (forklar for deg selv hvorfor) $ f\in [1, 3] $, altså er $ V_f=[1, 3] $.\vsk

La oss videre se på funksjonen $ g(x)=\frac{1}{x} $. Denne funksjonen er ikke definert for $ x=0 $, noe som betyr at vi allerede har fått en restriksjon på definisjonsmengden til $ g $. Også her gjør vi det enkelt, og unngår\footnote{I \refsec{??} skal vi se nærmere på funksjoner som $ g $ når $ x $ nærmer seg 0.} 0 med god klaring ved å sette $ D_g=[1, 2] $. Da er (forklar for deg selv hvorfor) $ V_g=\left[\frac{1}{2}, 1\right] $.\regv
 
\reg[Verdi- og definisjonsmengder]{
Gitt en funksjon $ f(x) $. Mengden som utelukkende inneholder alle verdier $ x $ kan ha, er da definisjonsmengden til $ f $. Denne mengden skrives som $ D_f $.\vsk

Mengden som utelukkende inneholder alle verdier $ f $ kan ha når $ x\in D_f $, er verdimengden til $ f $. 
}
\section{Betingelser}
Symbolet \sym{$ \Rightarrow $} bruker vi for å vise til at hvis én ting er sann, så er en annen (eller flere) ting sann også. For eksempel, alle tall som er delelige med 2 er partall. For et tall $ n $ kan vi skrive dette slik:
\[ \frac{n}{2}=\text{heltall} \Rightarrow  n\text{ er et partall} \]
Videre kan man spørre seg om det omvendte gjelder; hvis $ n $ er et partall, er det da delelig med 2?
\end{document}