\documentclass[english, 11 pt, class=article, crop=false]{standalone}
\usepackage[T1]{fontenc}
%\renewcommand*\familydefault{\sfdefault} % For dyslexia-friendly text
\usepackage{lmodern} % load a font with all the characters
\usepackage{geometry}
\geometry{verbose,paperwidth=16.1 cm, paperheight=24 cm, inner=2.3cm, outer=1.8 cm, bmargin=2cm, tmargin=1.8cm}
\setlength{\parindent}{0bp}
\usepackage{import}
\usepackage[subpreambles=false]{standalone}
\usepackage{amsmath}
\usepackage{amssymb}
\usepackage{esint}
\usepackage{babel}
\usepackage{tabu}
\makeatother
\makeatletter

\usepackage{titlesec}
\usepackage{ragged2e}
\RaggedRight
\raggedbottom
\frenchspacing

% Norwegian names of figures, chapters, parts and content
\addto\captionsenglish{\renewcommand{\figurename}{Figur}}
\makeatletter
\addto\captionsenglish{\renewcommand{\chaptername}{Kapittel}}
\addto\captionsenglish{\renewcommand{\partname}{Del}}


\usepackage{graphicx}
\usepackage{float}
\usepackage{subfig}
\usepackage{placeins}
\usepackage{cancel}
\usepackage{framed}
\usepackage{wrapfig}
\usepackage[subfigure]{tocloft}
\usepackage[font=footnotesize,labelfont=sl]{caption} % Figure caption
\usepackage{bm}
\usepackage[dvipsnames, table]{xcolor}
\definecolor{shadecolor}{rgb}{0.105469, 0.613281, 1}
\colorlet{shadecolor}{Emerald!15} 
\usepackage{icomma}
\makeatother
\usepackage[many]{tcolorbox}
\usepackage{multicol}
\usepackage{stackengine}

\usepackage{esvect} %For vectors with capital letters

% For tabular
\usepackage{array}
\usepackage{multirow}
\usepackage{longtable} %breakable table

% Ligningsreferanser
\usepackage{mathtools}
\mathtoolsset{showonlyrefs}

% index
\usepackage{imakeidx}
\makeindex[title=Indeks]

%Footnote:
\usepackage[bottom, hang, flushmargin]{footmisc}
\usepackage{perpage} 
\MakePerPage{footnote}
\addtolength{\footnotesep}{2mm}
\renewcommand{\thefootnote}{\arabic{footnote}}
\renewcommand\footnoterule{\rule{\linewidth}{0.4pt}}
\renewcommand{\thempfootnote}{\arabic{mpfootnote}}

%colors
\definecolor{c1}{cmyk}{0,0.5,1,0}
\definecolor{c2}{cmyk}{1,0.25,1,0}
\definecolor{n3}{cmyk}{1,0.,1,0}
\definecolor{neg}{cmyk}{1,0.,0.,0}

% Lister med bokstavar
\usepackage[inline]{enumitem}

\newcounter{rg}
\numberwithin{rg}{chapter}
\newcommand{\reg}[2][]{\begin{tcolorbox}[boxrule=0.3 mm,arc=0mm,colback=blue!3] {\refstepcounter{rg}\phantomsection \large \textbf{\therg \;#1} \vspace{5 pt}}\newline #2  \end{tcolorbox}\vspace{-5pt}}

\newcommand\alg[1]{\begin{align} #1 \end{align}}

\newcommand\eks[2][]{\begin{tcolorbox}[boxrule=0.3 mm,arc=0mm,enhanced jigsaw,breakable,colback=green!3] {\large \textbf{Eksempel #1} \vspace{5 pt}\\} #2 \end{tcolorbox}\vspace{-5pt} }

\newcommand{\st}[1]{\begin{tcolorbox}[boxrule=0.0 mm,arc=0mm,enhanced jigsaw,breakable,colback=yellow!12]{ #1} \end{tcolorbox}}

\newcommand{\spr}[1]{\begin{tcolorbox}[boxrule=0.3 mm,arc=0mm,enhanced jigsaw,breakable,colback=yellow!7] {\large \textbf{Språkboksen} \vspace{5 pt}\\} #1 \end{tcolorbox}\vspace{-5pt} }

\newcommand{\sym}[1]{\colorbox{blue!15}{#1}}

\newcommand{\info}[2]{\begin{tcolorbox}[boxrule=0.3 mm,arc=0mm,enhanced jigsaw,breakable,colback=cyan!6] {\large \textbf{#1} \vspace{5 pt}\\} #2 \end{tcolorbox}\vspace{-5pt} }

\newcommand\algv[1]{\vspace{-11 pt}\begin{align*} #1 \end{align*}}

\newcommand{\regv}{\vspace{5pt}}
\newcommand{\mer}{\textsl{Merk}: }
\newcommand{\mers}[1]{{\footnotesize \mer #1}}
\newcommand\vsk{\vspace{11pt}}
\newcommand\vs{\vspace{-11pt}}
\newcommand\vsb{\vspace{-16pt}}
\newcommand\sv{\vsk \textbf{Svar} \vspace{4 pt}\\}
\newcommand\br{\\[5 pt]}
\newcommand{\figp}[1]{../fig/#1}
\newcommand\algvv[1]{\vs\vs\begin{align*} #1 \end{align*}}
\newcommand{\y}[1]{$ {#1} $}
\newcommand{\os}{\\[5 pt]}
\newcommand{\prbxl}[2]{
\parbox[l][][l]{#1\linewidth}{#2
	}}
\newcommand{\prbxr}[2]{\parbox[r][][l]{#1\linewidth}{
		\setlength{\abovedisplayskip}{5pt}
		\setlength{\belowdisplayskip}{5pt}	
		\setlength{\abovedisplayshortskip}{0pt}
		\setlength{\belowdisplayshortskip}{0pt} 
		\begin{shaded}
			\footnotesize	#2 \end{shaded}}}

\renewcommand{\cfttoctitlefont}{\Large\bfseries}
\setlength{\cftaftertoctitleskip}{0 pt}
\setlength{\cftbeforetoctitleskip}{0 pt}

\newcommand{\bs}{\\[3pt]}
\newcommand{\vn}{\\[6pt]}
\newcommand{\fig}[1]{\begin{figure}
		\centering
		\includegraphics[]{\figp{#1}}
\end{figure}}

\newcommand{\figc}[2]{\begin{figure}
		\centering
		\includegraphics[]{\figp{#1}}
		\caption{#2}
\end{figure}}

\newcommand{\sectionbreak}{\clearpage} % New page on each section

\newcommand{\nn}[1]{
\begin{equation}
	#1
\end{equation}
}

% Equation comments
\newcommand{\cm}[1]{\llap{\color{blue} #1}}

\newcommand\fork[2]{\begin{tcolorbox}[boxrule=0.3 mm,arc=0mm,enhanced jigsaw,breakable,colback=yellow!7] {\large \textbf{#1 (forklaring)} \vspace{5 pt}\\} #2 \end{tcolorbox}\vspace{-5pt} }
 
%colors
\newcommand{\colr}[1]{{\color{red} #1}}
\newcommand{\colb}[1]{{\color{blue} #1}}
\newcommand{\colo}[1]{{\color{orange} #1}}
\newcommand{\colc}[1]{{\color{cyan} #1}}
\definecolor{projectgreen}{cmyk}{100,0,100,0}
\newcommand{\colg}[1]{{\color{projectgreen} #1}}

% Methods
\newcommand{\metode}[2]{
	\textsl{#1} \\[-8pt]
	\rule{#2}{0.75pt}
}

%Opg
\newcommand{\abc}[1]{
	\begin{enumerate}[label=\alph*),leftmargin=18pt]
		#1
	\end{enumerate}
}
\newcommand{\abcs}[2]{
	\begin{enumerate}[label=\alph*),start=#1,leftmargin=18pt]
		#2
	\end{enumerate}
}
\newcommand{\abcn}[1]{
	\begin{enumerate}[label=\arabic*),leftmargin=18pt]
		#1
	\end{enumerate}
}
\newcommand{\abch}[1]{
	\hspace{-2pt}	\begin{enumerate*}[label=\alph*), itemjoin=\hspace{1cm}]
		#1
	\end{enumerate*}
}
\newcommand{\abchs}[2]{
	\hspace{-2pt}	\begin{enumerate*}[label=\alph*), itemjoin=\hspace{1cm}, start=#1]
		#2
	\end{enumerate*}
}

% Oppgaver
\newcommand{\opgt}{\phantomsection \addcontentsline{toc}{section}{Oppgaver} \section*{Oppgaver for kapittel \thechapter}\vs \setcounter{section}{1}}
\newcounter{opg}
\numberwithin{opg}{section}
\newcommand{\op}[1]{\vspace{15pt} \refstepcounter{opg}\large \textbf{\color{blue}\theopg} \vspace{2 pt} \label{#1} \\}
\newcommand{\ekspop}[1]{\vsk\textbf{Gruble \thechapter.#1}\vspace{2 pt} \\}
\newcommand{\nes}{\stepcounter{section}
	\setcounter{opg}{0}}
\newcommand{\opr}[1]{\vspace{3pt}\textbf{\ref{#1}}}
\newcommand{\oeks}[1]{\begin{tcolorbox}[boxrule=0.3 mm,arc=0mm,colback=white]
		\textit{Eksempel: } #1	  
\end{tcolorbox}}
\newcommand\opgeks[2][]{\begin{tcolorbox}[boxrule=0.1 mm,arc=0mm,enhanced jigsaw,breakable,colback=white] {\footnotesize \textbf{Eksempel #1} \\} \footnotesize #2 \end{tcolorbox}\vspace{-5pt} }
\newcommand{\rknut}{
Rekn ut.
}

%License
\newcommand{\lic}{\textit{Matematikken sine byggesteinar by Sindre Sogge Heggen is licensed under CC BY-NC-SA 4.0. To view a copy of this license, visit\\ 
		\net{http://creativecommons.org/licenses/by-nc-sa/4.0/}{http://creativecommons.org/licenses/by-nc-sa/4.0/}}}

%referances
\newcommand{\net}[2]{{\color{blue}\href{#1}{#2}}}
\newcommand{\hrs}[2]{\hyperref[#1]{\color{blue}\textsl{#2 \ref*{#1}}}}
\newcommand{\rref}[1]{\hrs{#1}{regel}}
\newcommand{\refkap}[1]{\hrs{#1}{kapittel}}
\newcommand{\refsec}[1]{\hrs{#1}{seksjon}}

\newcommand{\mb}{\net{https://sindrsh.github.io/FirstPrinciplesOfMath/}{MB}}


%line to seperate examples
\newcommand{\linje}{\rule{\linewidth}{1pt} }

\usepackage{datetime2}
%%\usepackage{sansmathfonts} for dyslexia-friendly math
\usepackage[]{hyperref}


% note
\newcommand{\note}{Note}
\newcommand{\notesm}[1]{{\footnotesize \textsl{\note:} #1}}
\newcommand{\selos}{See the solutions manual.}

\newcommand{\texandasy}{The text is written in \LaTeX\ and the figures are made with the aid of Asymptote.}

\newcommand{\ekstitle}{Example }
\newcommand{\sprtitle}{The language box}
\newcommand{\expl}{explanation}

%%% SECTION HEADLINES %%%

% Our numbers
\newcommand{\likteikn}{The equal sign}
\newcommand{\talsifverd}{Numbers, digits and values}
\newcommand{\koordsys}{Coordinate systems}

% Calculations
\newcommand{\adi}{Addition}
\newcommand{\sub}{Subtraction}
\newcommand{\gong}{Multiplication}
\newcommand{\del}{Division}

%Factorization and order of operations
\newcommand{\fak}{Factorization}
\newcommand{\rrek}{Order of operations}

%Fractions
\newcommand{\brgrpr}{Introduction}
\newcommand{\brvu}{Values, expanding and simplifying}
\newcommand{\bradsub}{Addition and subtraction}
\newcommand{\brgngheil}{Fractions multiplied by integers}
\newcommand{\brdelheil}{Fractions divided by integers}
\newcommand{\brgngbr}{Fractions multiplied by fractions}
\newcommand{\brkans}{Cancelation of fractions}
\newcommand{\brdelmbr}{Division by fractions}
\newcommand{\Rasjtal}{Rational numbers}

%Negative numbers
\newcommand{\negintro}{Introduction}
\newcommand{\negrekn}{The elementary operations}
\newcommand{\negmeng}{Negative numbers as amounts}

%Calculation methods
\newcommand{\delmedtihundre}{Deling med 10, 100, 1\,000 osv.}

% Geometry 1
\newcommand{\omgr}{Terms}
\newcommand{\eignsk}{Attributes of triangles and quadrilaterals}
\newcommand{\omkr}{Perimeter}
\newcommand{\area}{Area}

%Algebra 
\newcommand{\algintro}{Introduction}
\newcommand{\pot}{Powers}
\newcommand{\irrasj}{Irrational numbers}

%Equations
\newcommand{\ligintro}{Introduction}
\newcommand{\liglos}{Solving with the elementary operations}
\newcommand{\ligloso}{Solving with elementary operations summarized}

%Functions
\newcommand{\fintro}{Introduction}
\newcommand{\lingraf}{Linear functions and graphs}

%Geometry 2
\newcommand{\geoform}{Formulas of area and perimeter}
\newcommand{\kongogsim}{Congruent and similar triangles}
\newcommand{\geofork}{Explanations}

% Names of rules
\newcommand{\adkom}{Addition is commutative}
\newcommand{\gangkom}{Multiplication is commutative}
\newcommand{\brdef}{Fractions as rewriting of division}
\newcommand{\brtbr}{Fractions multiplied by fractions}
\newcommand{\delmbr}{Fractions divided by fractions}
\newcommand{\gangpar}{Distributive law}
\newcommand{\gangparsam}{Paranthesis multiplied together}
\newcommand{\gangmnegto}{Multiplication by negative numbers I}
\newcommand{\gangmnegtre}{Multiplication by negative numbers II}
\newcommand{\konsttre}{Unique construction of triangles}
\newcommand{\kongtre}{Congruent triangles}
\newcommand{\topv}{Vertical angles}
\newcommand{\trisum}{The sum of angles in a triangle}
\newcommand{\firsum}{The sum of angles in a quadrilateral}
\newcommand{\potgang}{Multiplication by powers}
\newcommand{\potdivpot}{Division by powers}
\newcommand{\potanull}{The special case of \boldmath $a^0$}
\newcommand{\potneg}{Powers with negative exponents}
\newcommand{\potbr}{Fractions as base}
\newcommand{\faktgr}{Factors as base}
\newcommand{\potsomgrunn}{Powers as base}
\newcommand{\arsirk}{The area of a circle}
\newcommand{\artrap}{The area of a trapezoid}
\newcommand{\arpar}{The area of a parallelogram}
\newcommand{\pyt}{Pythagoras's theorem}
\newcommand{\forform}{Ratios in similar triangles}
\newcommand{\vilkform}{Terms of similar triangles}
\newcommand{\omkrsirk}{The perimeter of a circle (and the value of $ \bm \pi $)}
\newcommand{\artri}{The area of a triangle}
\newcommand{\arrekt}{The area of a rectangle}
\newcommand{\liknflyt}{Moving terms across the equal sign}
\newcommand{\funklin}{Linear functions}



\begin{document}
\section{Sets}
\subsection{Definition}
A collection of numbers is called a \outl{set}\footnote{A set can also be a collection of other mathematical objects, such as functions, but in this book, it is sufficient to consider sets of numbers.}, and a number that is part of a set is called an \outl{element}. Sets can contain a finite number of elements and they can contain infinitely many elements. \regv
\regdef[Sets]{
	For two numbers $ a $ and $ b $, where $ a\leq b $, we have that
	\begin{center}
		\begin{tabular}{c l}
			$ [a, b] $ & is the set of all numbers greater than or equal to $ a $ \\
			&and less than or equal to $ b $. \\
			$ (a, b] $ & is the set of all numbers greater than $ a $ \\
			&and less than or equal to $ b $.\\ 
			$ [a, b) $ & is the set of all real numbers greater than or equal to $ a $ \\
			&and less than $ b $.
		\end{tabular}
	\end{center}
	$ [a, b] $ is called a \outl{closed interval}, $ (a, b) $ is called an \outl{open interval}, and both $ (a, b] $ and $ [a, b) $ are called \outl{half-open intervals}.\vsk
	
	The set that contains only $ a $ and $ b $ is written as $ \{a, b\} $. \vsk
	
	That $ x $ is an element in a set $ M $ is written as $ x\in M $.\vsk
	
	That $ x $ \textsl{is not} an element in a set $ M $ is written as $ x\not \in M $. \vsk
	
	That the set $ M $ consists of the sets $ M_1 $ and $ M_2 $ is written as \\$ M = M_1 \cup M_2 $. \vsk
	
	That $ x $ is omitted from a set $ M $ is written as $ M \setminus x $
}
\spr{
	$ x \in M$ is pronounced ''$ x $ contained in $ M $''.\vsk
	
	Many texts use \sym{$ \langle $} instead of \sym{$ ( $} to indicate open (or half-open) intervals.
}
\info{\note}{
	When we define an interval described by $ a $ and $ b $ hereafter in the book, we take it for granted that $ a $ and $ b $ are two numbers, and that $ a\leq b $.
}

\eks[1]{
	The set of all integers greater than 0 and less than 10 can be written as
	\[ \{1, 2, 3, 4, 5, 6, 7, 8, 9\} \]
	This set contains 9 elements. 3 is an element in this set, and thus we can write $ 3\in\{1, 2, 3, 4, 5, 6, 7, 8, 9\}  $\vsk 
	
	10 is not an element in this set, and thus we can write \\$ 10 \not\in  \{1, 2, 3, 4, 5, 6, 7, 8, 9\} $.
} 
\eks[2]{
	In the expression $ 0\ ?\  x\  ?\  1 $, replace \sym{?} with an inequality symbol so that the expression applies to all $ {x\in M} $, and determine if 1 is contained in $ M $.
	\abc{
		\item $ M = [0, 1] $
		\item $ M = (0, 1] $
		\item $ M = [0, 1) $
	}
	\sv \vs
	\abc{
		\item $ 0\leq x \leq 1 $. Furthermore, $ 1\in M $.
		\item $ 0< x \leq 1 $. Furthermore, $ 1\in M $.
		\item $ 0\leq x < 1 $. Furthermore, $ 1\not \in M $.
	}
}\vsk

\regdef[Names of sets \label{sets}]{\vs
	\begin{center}
		\begin{tabular}{c l}
			$ \mathbb{N} $ & The set of all positive integers\footnote{Does \textsl{not} include 0.}\os
			$ \mathbb{Z} $ & The set of all integers\footnote{Includes 0.}\os
			$ \mathbb{Q} $ & The set of all rational numbers\os
			$ \mathbb{R} $ & The set of all real numbers\os
			$ \mathbb{C} $ & The set of all complex numbers\\
		\end{tabular}
	\end{center}
}

\subsection{The Symbol for Infinity}
The sets in \dref{sets} contain infinitely many elements. Sometimes we wish to limit parts of an infinite set, and then there arises a need for a symbol that helps symbolize this. \sym{$ \infty $} is the symbol for an infinitely large, positive value.\regv 

\eks[]{
	A condition that $ x\geq2 $ can be written as $ x\in[2 ,\, \infty) $.\vsk
	
	A condition that $ x<-7 $ can be written as $ x\in(-\infty ,-7) $.\vsk
}
\spr{
	The two intervals in the example above can also be written as $ [2, \rightarrow) $ and $ (\leftarrow, -7) $.
}
\info{\note}{
	\sym{$ \infty $} is not any specific number. Therefore, using the four basic operations alone with this symbol makes no sense.
}
\subsection{Domain and Range}
\regdef[Domain and Range]{
	Given a function $ f(x) $. 
	\begin{itemize}
		\item The set that exclusively contains all the values $ x $ can have, is the \outl{domain} of $ f $. This set is written as $ D_f $.
		\item The set that exclusively contains all the values $ f $ can have when $ {x\in D_f} $, is the \outl{range} of $ f $. This set is written as $ V_f $. 
	\end{itemize}
}
\eks[1]
{ \label{setex1}
	Below figure shows $ {f(x)=2x+1} $, where $ {D_f=[1, 3]} $. Then ${ V_f=[3, 7]} $. \vs
	\fig{defmengeks1}
}
\eks[2]{
	Below figure shows $ f(x)={\frac{1}{x}} $, where $ {D_f}=[-3, -1] \cup [2, 5] $. Then $ {V_f=\left[-1, -\frac{1}{3}\right]\cup \left[\frac{1}{5},\frac{1}{2}\right]} $. \vs
	\fig{defmengeks2}
}
\newpage
\info{\note}{
	The domain of a function is determined by two things; the context in which the function is used, and any values that result in an undefined function expression. In \textsl{Example 1} on page \pageref{setex1}, the domain is arbitrarily chosen, since the function is defined for all $ x $. In \textsl{Example 2}, however, the function is not defined for $ {x=0} $, so a domain including this value for $ x $ would not make sense.
}
\section{Conditions}
\subsection{Symbols for Conditions}
The symbol \sym{$ \Rightarrow $} is used to indicate that if one condition is satisfied, then another (or several) condition(s) are also satisfied. For example; in \mb\; we saw that if a triangle is right-angled, then Pythagoras' theorem is valid. We can write this as:
\[ \text{the triangle is right-angled} \Rightarrow \text{Pythagoras' theorem is valid}\]
But we also saw that the converse is true; if Pythagoras' theorem is valid, then the triangle must be right-angled. Thus, we can write
\[ \text{the triangle is right-angled} \iff \text{Pythagoras' theorem is valid}\]
It is very important to be aware of the difference between \sym{$ \Rightarrow $} and \sym{$ \iff $}; that condition A satisfied implies condition B satisfied does not necessarily mean that condition B satisfied implies condition A satisfied! \regv

\eks[1]{ \vs \vs
	\[ \text{the square is a square}\Rightarrow\text{the square has four equal sides} \]
}
\eks[2]{ \vs 
	\[ \text{the number is a prime number greater than 2}\Rightarrow\text{the number is an odd number} \]
}
\eks[3]{\vs
	\[ \text{the number is an even number}\iff\text{the number is divisible by 2} \]
}
\newpage
\subsection{Functions with Conditions}
Functions can have multiple expressions that apply under different conditions. Let us define a function $ f(x) $ as follows:
\alg{
	&\text{For }x<1 \text{ the function expression is } {-2x+1}\\
	&\text{For }x\geq 1\text{ the function expression is } x^2-2x
}
\figc{funkvilk}{The graph of $ f $ on the interval $ [-1, 3] $.}
This can be written as
\begin{equation*}f(x)= \left\lbrace{
		\begin{array}{rcr}
			-2x+1 &,&x<1 \br
			x^2-2x   &,& x\geq 1
		\end{array}
	}\right. \label{fsplit}
\end{equation*}


\end{document}