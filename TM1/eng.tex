% note
\newcommand{\note}{Note}
\newcommand{\notesm}[1]{{\footnotesize \textsl{\note:} #1}}
\newcommand{\selos}{See the solutions manual.}

\newcommand{\texandasy}{The text is written in \LaTeX\ and the figures are made with the aid of Asymptote.}

\newcommand{\rknut}{Calculate.}
\newcommand\sv{\vsk \textbf{Answer} \vspace{4 pt}\\}
\newcommand{\ekstitle}{Example }
\newcommand{\sprtitle}{The language box}
\newcommand{\expl}{explanation}

% answers
\newcommand{\mulansw}{\notesm{Multiple possible answers.}}	
\newcommand{\faskap}{Chapter}

% exercises
\newcommand{\opgt}{\newpage \phantomsection \addcontentsline{toc}{section}{Exercises} \section*{Exercises for Chapter \thechapter}\vs \setcounter{section}{1}}

% references
\newcommand{\reftab}[1]{\hrs{#1}{Table}}
\newcommand{\rref}[1]{\hrs{#1}{Rule}}
\newcommand{\dref}[1]{\hrs{#1}{Definition}}
\newcommand{\refkap}[1]{\hrs{#1}{Chapter}}
\newcommand{\refsec}[1]{\hrs{#1}{Section}}
\newcommand{\refdsec}[1]{\hrs{#1}{Subsection}}
\newcommand{\refved}[1]{\hrs{#1}{Appendix}}
\newcommand{\eksref}[1]{\textsl{#1}}
\newcommand\fref[2][]{\hyperref[#2]{\textsl{Figure \ref*{#2}#1}}}
\newcommand{\refop}[1]{{\color{blue}Exercise \ref{#1}}}
\newcommand{\refops}[1]{{\color{blue}Exercise \ref{#1}}}

%Algebra
\newcommand{\kvadset}{The Quadratic Identities}
\newcommand{\aenato}{The Sum-Product Method}

% Geometry
\newcommand{\hlikb}{The Perpendicular Bisector of An Equilateral Triangle}
\newcommand{\arealsetn}{The Law of Sines}
\newcommand{\trkmedian}{The Median}
\newcommand{\midtrk}{Perpendicular Bisector (in a triangle)}
\newcommand{\innskrsirk}{The Inscribed Circle}
\newcommand{\cossetn}{The Cosine Rule}
\newcommand{\perfvink}{Central Angles and Inscribed Angles}
\newcommand{\tang}{The Tangent}

% Derivative
\newcommand{\derel}{The Derivative of Elementary functions}
\newcommand{\divder}{The Division Rule}
\newcommand{\kjernereg}{The Chain Rule}
\newcommand{\prodregder}{The Product Rule}
\newcommand{\lhop}{L'Hopitals rule}

% Funksjonsdrofting
\newcommand{\monder}{Monotony Properties and The Derivative}
\newcommand{\fderekstr}{$ \bm{f'=0} $ for local Extremums}
\newcommand{\andredertest}{The Second Derivative Test}

% Vectors
\newcommand{\detar}{Area Rules for Determinants}
\newcommand{\avstpunktlin}{The Distance Between a Point and a Line}

%Appendix
\newcommand{\rolle}{Rolles theorem}
\newcommand{\meanval}{The Mean Value Sentence}
