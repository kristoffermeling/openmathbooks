\documentclass[english, 11 pt, class=article, crop=false]{standalone}

\newcommand{\note}{Merk}

% Geometry
\newcommand{\hlikb}{Midtnormalen i en likebeint trekant}
\newcommand{\arealsetn}{Arealsetningen}
\newcommand{\trkmedian}{Medianer i trekanter}
\newcommand{\midtrk}{Midtnormaler i trekanter}
\newcommand{\innskrsirk}{Halveringslinjer og innskrevet sirkel i trekanter}
\newcommand{\cossetn}{Cosinussetningen}
\newcommand{\perfvink}{Sentral- og periferivinkel}
\newcommand{\tang}{Tangent}


% Vectors
\newcommand{\detar}{Arealformler med determinanter}
\usepackage[T1]{fontenc}
%\renewcommand*\familydefault{\sfdefault} % For dyslexia-friendly text
\usepackage{lmodern} % load a font with all the characters
\usepackage{geometry}
\geometry{verbose,paperwidth=16.1 cm, paperheight=24 cm, inner=2.3cm, outer=1.8 cm, bmargin=2cm, tmargin=1.8cm}
\setlength{\parindent}{0bp}
\usepackage{import}
\usepackage[subpreambles=false]{standalone}
\usepackage{amsmath}
\usepackage{amssymb}
\usepackage{esint}
\usepackage{babel}
\usepackage{tabu}
\makeatother
\makeatletter

\usepackage{titlesec}
\usepackage{ragged2e}
\RaggedRight
\raggedbottom
\frenchspacing

% Norwegian names of figures, chapters, parts and content
\addto\captionsenglish{\renewcommand{\figurename}{Figur}}
\makeatletter
\addto\captionsenglish{\renewcommand{\chaptername}{Kapittel}}
\addto\captionsenglish{\renewcommand{\partname}{Del}}


\usepackage{graphicx}
\usepackage{float}
\usepackage{subfig}
\usepackage{placeins}
\usepackage{cancel}
\usepackage{framed}
\usepackage{wrapfig}
\usepackage[subfigure]{tocloft}
\usepackage[font=footnotesize,labelfont=sl]{caption} % Figure caption
\usepackage{bm}
\usepackage[dvipsnames, table]{xcolor}
\definecolor{shadecolor}{rgb}{0.105469, 0.613281, 1}
\colorlet{shadecolor}{Emerald!15} 
\usepackage{icomma}
\makeatother
\usepackage[many]{tcolorbox}
\usepackage{multicol}
\usepackage{stackengine}

\usepackage{esvect} %For vectors with capital letters

% For tabular
\usepackage{array}
\usepackage{multirow}
\usepackage{longtable} %breakable table

% Ligningsreferanser
\usepackage{mathtools}
\mathtoolsset{showonlyrefs}

% index
\usepackage{imakeidx}
\makeindex[title=Indeks]

%Footnote:
\usepackage[bottom, hang, flushmargin]{footmisc}
\usepackage{perpage} 
\MakePerPage{footnote}
\addtolength{\footnotesep}{2mm}
\renewcommand{\thefootnote}{\arabic{footnote}}
\renewcommand\footnoterule{\rule{\linewidth}{0.4pt}}
\renewcommand{\thempfootnote}{\arabic{mpfootnote}}

%colors
\definecolor{c1}{cmyk}{0,0.5,1,0}
\definecolor{c2}{cmyk}{1,0.25,1,0}
\definecolor{n3}{cmyk}{1,0.,1,0}
\definecolor{neg}{cmyk}{1,0.,0.,0}

% Lister med bokstavar
\usepackage[inline]{enumitem}

\newcounter{rg}
\numberwithin{rg}{chapter}
\newcommand{\reg}[2][]{\begin{tcolorbox}[boxrule=0.3 mm,arc=0mm,colback=blue!3] {\refstepcounter{rg}\phantomsection \large \textbf{\therg \;#1} \vspace{5 pt}}\newline #2  \end{tcolorbox}\vspace{-5pt}}

\newcommand\alg[1]{\begin{align} #1 \end{align}}

\newcommand\eks[2][]{\begin{tcolorbox}[boxrule=0.3 mm,arc=0mm,enhanced jigsaw,breakable,colback=green!3] {\large \textbf{Eksempel #1} \vspace{5 pt}\\} #2 \end{tcolorbox}\vspace{-5pt} }

\newcommand{\st}[1]{\begin{tcolorbox}[boxrule=0.0 mm,arc=0mm,enhanced jigsaw,breakable,colback=yellow!12]{ #1} \end{tcolorbox}}

\newcommand{\spr}[1]{\begin{tcolorbox}[boxrule=0.3 mm,arc=0mm,enhanced jigsaw,breakable,colback=yellow!7] {\large \textbf{Språkboksen} \vspace{5 pt}\\} #1 \end{tcolorbox}\vspace{-5pt} }

\newcommand{\sym}[1]{\colorbox{blue!15}{#1}}

\newcommand{\info}[2]{\begin{tcolorbox}[boxrule=0.3 mm,arc=0mm,enhanced jigsaw,breakable,colback=cyan!6] {\large \textbf{#1} \vspace{5 pt}\\} #2 \end{tcolorbox}\vspace{-5pt} }

\newcommand\algv[1]{\vspace{-11 pt}\begin{align*} #1 \end{align*}}

\newcommand{\regv}{\vspace{5pt}}
\newcommand{\mer}{\textsl{Merk}: }
\newcommand{\mers}[1]{{\footnotesize \mer #1}}
\newcommand\vsk{\vspace{11pt}}
\newcommand\vs{\vspace{-11pt}}
\newcommand\vsb{\vspace{-16pt}}
\newcommand\sv{\vsk \textbf{Svar} \vspace{4 pt}\\}
\newcommand\br{\\[5 pt]}
\newcommand{\figp}[1]{../fig/#1}
\newcommand\algvv[1]{\vs\vs\begin{align*} #1 \end{align*}}
\newcommand{\y}[1]{$ {#1} $}
\newcommand{\os}{\\[5 pt]}
\newcommand{\prbxl}[2]{
\parbox[l][][l]{#1\linewidth}{#2
	}}
\newcommand{\prbxr}[2]{\parbox[r][][l]{#1\linewidth}{
		\setlength{\abovedisplayskip}{5pt}
		\setlength{\belowdisplayskip}{5pt}	
		\setlength{\abovedisplayshortskip}{0pt}
		\setlength{\belowdisplayshortskip}{0pt} 
		\begin{shaded}
			\footnotesize	#2 \end{shaded}}}

\renewcommand{\cfttoctitlefont}{\Large\bfseries}
\setlength{\cftaftertoctitleskip}{0 pt}
\setlength{\cftbeforetoctitleskip}{0 pt}

\newcommand{\bs}{\\[3pt]}
\newcommand{\vn}{\\[6pt]}
\newcommand{\fig}[1]{\begin{figure}
		\centering
		\includegraphics[]{\figp{#1}}
\end{figure}}

\newcommand{\figc}[2]{\begin{figure}
		\centering
		\includegraphics[]{\figp{#1}}
		\caption{#2}
\end{figure}}

\newcommand{\sectionbreak}{\clearpage} % New page on each section

\newcommand{\nn}[1]{
\begin{equation}
	#1
\end{equation}
}

% Equation comments
\newcommand{\cm}[1]{\llap{\color{blue} #1}}

\newcommand\fork[2]{\begin{tcolorbox}[boxrule=0.3 mm,arc=0mm,enhanced jigsaw,breakable,colback=yellow!7] {\large \textbf{#1 (forklaring)} \vspace{5 pt}\\} #2 \end{tcolorbox}\vspace{-5pt} }
 
%colors
\newcommand{\colr}[1]{{\color{red} #1}}
\newcommand{\colb}[1]{{\color{blue} #1}}
\newcommand{\colo}[1]{{\color{orange} #1}}
\newcommand{\colc}[1]{{\color{cyan} #1}}
\definecolor{projectgreen}{cmyk}{100,0,100,0}
\newcommand{\colg}[1]{{\color{projectgreen} #1}}

% Methods
\newcommand{\metode}[2]{
	\textsl{#1} \\[-8pt]
	\rule{#2}{0.75pt}
}

%Opg
\newcommand{\abc}[1]{
	\begin{enumerate}[label=\alph*),leftmargin=18pt]
		#1
	\end{enumerate}
}
\newcommand{\abcs}[2]{
	\begin{enumerate}[label=\alph*),start=#1,leftmargin=18pt]
		#2
	\end{enumerate}
}
\newcommand{\abcn}[1]{
	\begin{enumerate}[label=\arabic*),leftmargin=18pt]
		#1
	\end{enumerate}
}
\newcommand{\abch}[1]{
	\hspace{-2pt}	\begin{enumerate*}[label=\alph*), itemjoin=\hspace{1cm}]
		#1
	\end{enumerate*}
}
\newcommand{\abchs}[2]{
	\hspace{-2pt}	\begin{enumerate*}[label=\alph*), itemjoin=\hspace{1cm}, start=#1]
		#2
	\end{enumerate*}
}

% Oppgaver
\newcommand{\opgt}{\phantomsection \addcontentsline{toc}{section}{Oppgaver} \section*{Oppgaver for kapittel \thechapter}\vs \setcounter{section}{1}}
\newcounter{opg}
\numberwithin{opg}{section}
\newcommand{\op}[1]{\vspace{15pt} \refstepcounter{opg}\large \textbf{\color{blue}\theopg} \vspace{2 pt} \label{#1} \\}
\newcommand{\ekspop}[1]{\vsk\textbf{Gruble \thechapter.#1}\vspace{2 pt} \\}
\newcommand{\nes}{\stepcounter{section}
	\setcounter{opg}{0}}
\newcommand{\opr}[1]{\vspace{3pt}\textbf{\ref{#1}}}
\newcommand{\oeks}[1]{\begin{tcolorbox}[boxrule=0.3 mm,arc=0mm,colback=white]
		\textit{Eksempel: } #1	  
\end{tcolorbox}}
\newcommand\opgeks[2][]{\begin{tcolorbox}[boxrule=0.1 mm,arc=0mm,enhanced jigsaw,breakable,colback=white] {\footnotesize \textbf{Eksempel #1} \\} \footnotesize #2 \end{tcolorbox}\vspace{-5pt} }
\newcommand{\rknut}{
Rekn ut.
}

%License
\newcommand{\lic}{\textit{Matematikken sine byggesteinar by Sindre Sogge Heggen is licensed under CC BY-NC-SA 4.0. To view a copy of this license, visit\\ 
		\net{http://creativecommons.org/licenses/by-nc-sa/4.0/}{http://creativecommons.org/licenses/by-nc-sa/4.0/}}}

%referances
\newcommand{\net}[2]{{\color{blue}\href{#1}{#2}}}
\newcommand{\hrs}[2]{\hyperref[#1]{\color{blue}\textsl{#2 \ref*{#1}}}}
\newcommand{\rref}[1]{\hrs{#1}{regel}}
\newcommand{\refkap}[1]{\hrs{#1}{kapittel}}
\newcommand{\refsec}[1]{\hrs{#1}{seksjon}}

\newcommand{\mb}{\net{https://sindrsh.github.io/FirstPrinciplesOfMath/}{MB}}


%line to seperate examples
\newcommand{\linje}{\rule{\linewidth}{1pt} }

\usepackage{datetime2}
%%\usepackage{sansmathfonts} for dyslexia-friendly math
\usepackage[]{hyperref}



\begin{document}
\section{Monotoniegenskaper}
De fleste funksjonsverdier varierer. Beskrivelser av hvordan funksjonene varierer kaller vi beskrivelser av funksjonenes \textit{monotoniegenskaper}.\regv
\reg[Voksende og avtagende funskjoner \label{voksogav}]{
Gitt en funksjon $ f(x) $.
\begin{itemize}
\item $ f $ er \textit{voksende} på intervallet $ [a, b] $ hvis vi for alle $ x_1, x_2 \in[a, b]$ har at
\begin{equation}\label{voksende}
	x_1<x_2 \Rightarrow f(x_1)\leq f(x_2)
\end{equation}
Hvis $ {f(x_1)\leq f(x_2) }$ kan erstattes med $ {f(x_1)< f(x_2)} $, er $ f $ \textit{strengt voksende}.
\fig{voksende}
\item $ f $ er \textit{avtagende} på intervallet $ [a, b] $ hvis vi for alle $ {x_1, x_2 \in[a, b]}$ har at
\begin{equation}\label{avtagende}
	x_1<x_2 \Rightarrow f(x_1)\geq f(x_2)
\end{equation}
Hvis $ {f(x_1)\geq f(x_2) }$ kan erstattes med $ {f(x_1)> f(x_2)} $, er $ f $ \textit{strengt avtagende}.
\fig{avtagende}
\end{itemize}
} \newpage

\reg[\monder \label{monder}]{
Gitt $ f(x) $ deriverbar på intervallet $ [a, b] $.
\begin{itemize}
\item Hvis $ {f'\geq0} $ for $ {x\in[a, b]} $, er $ f $ voksende for $ x\in(a, b) $
\item Hvis $ {f'\leq0} $ for $ {x\in[a, b]} $, er $ f $ avtagende for $ x\in(a, b) $
\end{itemize}
Hvis henholdsvis $ \geq $ og $ \leq $ kan erstattes med $ > $ og $ < $, er $ f $ strengt voksende/avtagende. 
} \regv

\eks[]{
Avgjør på hvilke intervaller $ f $ er voksende/avtagende når
\[ f(x)=\frac{1}{3} x^3 - 4x^2 + 12x\qquad,\qquad x\in[0, 8] \]
\sv \vs
Vi har at
\[  f'(x)=x^2-8x+12 \]
For å tydeliggjøre når $ f' $ er positiv, negativ eller lik 0 gjør vi to ting; vi faktoriserer uttrykket til $ f' $, og tegner et \textit{fortegnsskjema}:
\[ f'(x)=(x-2)(x-6) \]
\fig{fortgnskj1}
Fortegnsskjemaet illustrerer følgende:
\begin{itemize}
	\item Uttrykket $ x-2 $ er negativt når  $ x\in[0, 2) $, lik 0 når $ x=2 $, og positivt når $ x\in(2, 8]  $.
	\item Uttrykket $ x-6 $ er negativt når $ x\in[0, 8) $, lik 0 når $ x=6 $, og positivt når $ x\in(6, 8]  $.
	\item Siden $ f'=(x-2)(x-6) $, er \vspace{-5pt}
	\begin{center}
		\begin{tabular}{l}
			$ f'\geq0 $ når $ {{x\in[0, 2]} \cup (6, 8]} $ \br
			$ f'=0 $ når $ {x\in\{2, 6\}} $ \br
			$ f'\leq0 $ når $ x\in[2, 6]$
		\end{tabular}
	\end{center}
\end{itemize}
Dette betyr at
\begin{center}
	\begin{tabular}{l}
		$ f $ er voksende når $ x\in(0, 2) \cup (6, 8) $ \br
		$ f $ er avtagende når $ x\in(2, 6)$
	\end{tabular}
\end{center}
}
\fork{\ref{monder}\,\monder}{
Gitt $ f(x) $, hvor $ {f'\geq0} $ for $ {x\in[a, b]} $. La $ {x_1, x_2 \in(a, b)}$ og $ {x_2>x_1} $. Av middelverdisetningen\footnote{Se vedlegg??} finnes det et tall $ {c\in(x_1, x_2)} $ slik at
\alg{
f'(c) &=\frac{f(x_2)-f(x_1)}{x_2-x_1}
}
Da $ c\in[a, b] $, er $ {f'(x)\geq 0} $, og da er
\alg{
0\geq \frac{f(x_2)-f(x_1)}{x_2-x_1}
}
Følgelig er $ {f(x_2)\geq f(x_1)} $, og av \dref{voksogav} er da $ f $ voksende på intervallet $ (a, b) $. \vs
}


\section{Ekstremalpunkt}
\reg[Maksimum og minimum]{\label{max}\index{maksimum}\index{minimum}	
	\notesm{Et tall $ c $ kan omtales som et punkt i funskjonsdrøftinger, underforstått at det er snakk om punktet $ (c, 0) $.} \br
	
	Gitt en funksjon $ f(x) $ og et tall $ c $.\bs
	\textbf{Absolutt maksimum og minimum}
	\begin{itemize}
		\item $ f $ har absolutt maksimum $ f(c) $ hvis $ {f(c)\geq f(x)} $ for alle $ x\in D_f $.
		
		\item $ f $ har absolutt minimum $ f(c) $ hvis ${ f(c)\leq f(x) }$ for alle $ x\in D_f $.
	\end{itemize}
	\textbf{Lokalt maksimum og minimum}
	\begin{itemize}
		\item $ f $ har et lokalt maksimum $ f(c) $ hvis det finnes et åpent intervall $ I $ om $ c $ slik at $ f(c)\geq f(x) $ for $ x\in I  $.
		
		\item $ f $ har et lokalt minimum $ f(c) $ hvis det finnes et åpent intervall $ I $ om $ c $ slik at $ f(c)\leq f(x) $ for $ x\in I  $.
	\end{itemize}\vs
}
\spr{
Et \textit{maksimum/minimum} blir også kalt en\\ \textit{maksimumsverdi/minimumsverdi}.
}
\reg[Ekstremalverdi og ekstremalpunkt]{\index{ekstremalpunkt}\index{ekstremalverdi}
	Gitt en funksjon $ f(x) $ med maksimum/minimum $ f(c) $. Da er
	\begin{itemize}
		\item $ f(c) $ en ekstremalverdi for $ f $.
		\item $ c $ et ekstremalpunkt for $ f $. Nærmere bestemt et maksimalpunkt/minimumspunkt for $ f $.
		\item $ (c, f(c)) $ et toppunkt/bunnpunkt for $ f $.
	\end{itemize}
}
\reg[Kritiske punkt]{
Et tall $ c $ er et kritisk punkt for en funksjon $ f(x) $ hvis én av følgende gjelder:
\begin{itemize}
	\item $ f $ er ikke deriverbar i $ c $
	\item $ f'(c)=0 $
\end{itemize}
}
\reg[\fderekstr \label{fderekstr}]{
Gitt en deriverbar funksjon $ f(x) $ og $ c\in[a, b] $.
\begin{enumerate}[label=(\roman*)]
	\item Hvis $ c $ er et lokalt ekstremalpunkt for $ f $, er $ {f'(c)= 0}$
	\item Hvis $ {f'>0} $ for $ {x\in(a, c)} $ og $ {f'<0} $ for $ {x\in(c, b)} $, er $ c $ et lokalt maksimumspunkt for $ f $
	\item Hvis $ {f'<0} $ for $ {x\in(a, c)} $ og $ {f'>0} $ for $ {x\in(c, b)} $, er $ c $ et lokalt minimumspunkt for $ f $
\end{enumerate}
}
\spr{
Det som blir beskrevet i punkt ii) og iii) omtales ofte som at \outl{$ \bm f $ skifter fortegn i $ \bm c $} 
}
\fork{\ref{fderekstr} \fderekstr}{
\textbf{Punkt (i)} \os

La $ c $ være et lokalt maksimumspunkt for $ f $. For et tall $ h $ må vi da ha at $ c\geq x $ for $ x\in(c-|h|, c+|h|) $. Da er
\[ f(c+h)-f(c)\leq 0 \] 
Dette betyr at
\alg{
\lim\limits_{h\to 0^+}\frac{f(c+h)-f(c)}{h}\leq 0 
}
og at
\alg{
	\lim\limits_{h\to 0^-}\frac{f(c+h)-f(c)}{h}\geq 0 
}
Følgelig er
\[ \lim\limits_{h\to 0^-}\frac{f(c+h)-f(c)}{h}=\lim\limits_{h\to 0^+}\frac{f(c+h)-f(c)}{h} \]
Altså er $ {f'(c)=0} $, og $ f' $ skifter fortegn fra positiv til negativ i $ c $. Med samme framgangsmåte kan det vises at dette også gjelder dersom $ c $ er et minimumspunkt, bare at da skifter $ f' $ fra negativ til positiv.\vsk

\textbf{Punkt (ii)} \os
Hvis $ {f'>0} $ på intervallet $ (a, c) $, har vi av \rref{monder} at $ f $ er sterkt voksende der. Hvis $ {f'<0} $ på $ (c, b) $, er $ f $ sterkt avtagende der. Dette må nødvendigvis bety at $ {f(c)\geq f(x)} $ for $ {x\in(a, b)} $, og da er $ c $ et maksimumspunkt.\vsk

\textbf{Punkt (iii)} \os
Tilsvarende resonnement som for punkt (ii).
}
\newpage

\reg[\andredertest \label{andredertest}]{
	Gitt en deriverba funksjon $ f(x) $ og et tall $ c $.
	\begin{itemize}
		\item Hvis $ {f'(c)=0} $ og $ {f''(c)<0} $, er $ f(c) $ et lokalt maksimum.
		\item Hvis $ {f'(c)=0 }$ og $ {f''(c)>0} $, er $ f(c) $ et lokalt minimum.
		\item Hvis ${ f'(c)=f''(c)=0 }$, kan man ikke ut ifra denne informasjonen alene si om $ f(c) $ er et lokalt maksimum eller minimum.
	\end{itemize}	
}
\fork{\ref{andredertest} \andredertest}{
	Av definisjonen for den deriverte har vi at
	\[ f''(c)=\lim\limits_{h \to 0}\frac{f'(c+h)-f'(c)}{h} \]
	Når $ f'(c)=0$, er
	\[ f''(c)=\lim\limits_{h \to 0}\frac{f'(c+h)}{h} \]
	Når $ f''(c)<0 $, betyr dette at
	\[\lim\limits_{h \to 0} \frac{f'(c+h)}{h}<0 \]
	Altså må $ {f'(c+h)} $ være positiv når $ h $ går mot 0 fra venstre og negativ når $ h $ går mot 0 fra høgre. Dermed skifter $ f' $ fortegn i $ c $, som da må være et maksimalpunkt for $ f $. Tilsvarende må $ c $ være et minimumspunkt for $ f $ hvis $ {f(c)=0} $ og $ {f''(c)<0} $.
}
\newpage
\reg[Infleksjonspunkt og vendepunkt]{\index{infleksjonspunkt}\index{vendepunkt}
	For en kontinuerlig funksjon $ f(x) $ har vi at
	\begin{itemize}
		\item Hvis $ {f''(c)=0} $ og $ f'' $ skifter fortegn i $ c $, er $ c $ et \textit{infleksjonspunkt} for $ f $.
		\item Hvis $ c $ er et infleksjonspunkt for $ f $, er $ (c, f(c)) $ et \textit{vendepunkt}.
		\item Hvis $ f'' $ går fra positiv til negativ, går $ f $ fra konveks til konkav (og omvendt).
	\end{itemize}\vs
}
\newpage
\eks[]{
\[ f(x)=x^3-3x^2-144x-140 \]
\begin{center}
	\begin{tabular}{c | c}
		\textbf{punkt/verdi} & \textbf{type} \\ \hline
		$ A=(-14, -1456) $ & absolutt bunnpunkt \\
		$ -14 $ & ekstremalpunkt; absolutt minimum \\
		$ -1456 $& absolutt minimum \\ \hline
		$ B=(-6, 400) $ & lokalt toppunkt \\		
		$ -6 $ & ekstremalpunkt; lokalt maksimalpunkt \\
		$ 400 $ & lokalt maksimum \\ \hline
		$ C=(-1, -286) $ & vendepunkt  \\
		$ -1 $ & infleksjonspunkt \\ \hline 
		$ D = (8, -972) $ & lokalt bunnpunkt \\
		$ 8 $ & ekstremalpunkt; lokalt minimumspunkt \\
		$ -972 $ & lokal minimum \\ \hline
		$ E= (16, 884) $ & absolutt maksimum \\
		$ 16 $ & ekstremalpunkt; absolutt maksimumspunkt \\
		$ 884 $ & absolutt maksimum \\ \hline
		$ -10 $, $ -1 $ og $ 14 $ & nullpunkt \\ \hline
	\end{tabular}
\end{center}
\fig{funkdroft}
}
\newpage
\eks{
	Gitt funksjonen
	\[ f(x)=\sin x\quad,\quad x\in[-2, 4] \]	
	\textbf{a)} Finn infleksjonspunktene til $ f $.
	
	\textbf{b)} Finn vendepunktene til $ f $. \\
	
	\sv
	\textbf{a)} Infleksjonspunktene finner vi der hvor $ f''(x)=0 $:
	\alg{f''(x) &= 0\\
		(\sin x)'' &= 0 \\
		-\sin x &= 0 	}
	Av $ {x\in D_f} $ er det $ {x=0} $ og $ {x=\pi }$ som oppfyller kravet fra ligningen over. For å finne ut om $ f'' $ skifter fortegn i disse punktene, setter vi opp et fortegnsskjema:
	\begin{figure}[H]
		\centering
		\begin{tikzpicture}[scale=2]	
			\draw[color=black] (0,-0.25) -- (0,1.25);
			\node[anchor=south] at (0,1.25) { $-2$};  
			\draw[color=black] (2,-0.25) -- (2,1.25);
			\node[anchor=south] at (2,1.25) { $\pi$};	
			\draw[color=black] (3,-0.25) -- (3,1.25);
			\node[anchor=south] at (3,1.25) { $4$};		
			\draw[dashed,color=black] (0,1) -- (3,1);
			\draw[dashed,color=black] (0,0.5) -- (1,0.5);
			\draw[dashed,color=black] (2,0.5) -- (3,0.5);	
			\draw[color=black] (1,0.5) -- (2,0.5);	
			\node[anchor=east] at (0,0.5) { $\sin x$};  		
			\node[anchor=east] at (0,1) { $-1$};  
			\draw[color=black] (1,-0.25) -- (1,1.25);
			\node[anchor=south] at (1,1.25) { $0$};    
			\draw[color=black] (0,0) -- (1,0);
			\draw[color=black] (2,0) -- (3,0);	    
			\draw[dashed,color=black] (1,0) -- (2,0);    
			\node[anchor=east] at (0,0) {$f''$};     
			\filldraw (1,0) circle[radius=1pt] ;   	
			\filldraw (2,0) circle[radius=1pt] ; 	
		\end{tikzpicture}
	\end{figure}
	
	$ f'' $ går altså fra positiv til negativ i $ {x=0} $ og fra negativ til positiv i $ {x=\pi} $. Dette betyr at $ f $ går fra konveks til konkav i $ {x=0} $ og fra konkav til konveks i $ {x=\pi }$.
}
\section{Asymptoter}
\reg[Vertikale asymptoter]{
Gitt en funksjon $ f(x) $ og en konstant $ c $.
\begin{itemize}
	\item Hvis $ {\lim\limits_{x\to c^+} f(x)=\pm \infty}  $, er $ c $ en \outl{vertikal asymptote ovenfra} for $ f $.
	\item Hvis $ {\lim\limits_{x\to c^-} f(x)=\pm \infty}  $, er $ c $ en \outl{vertikal asymptote \\nedenfra} for $ f $.
	\item Hvis $ {\lim\limits_{x\to c} f(x)=\pm \infty}  $, er $ c $ en \outl{vertikal asymptote} for $ f $.
\end{itemize}
}
\eks[]{
Finn den vertikale asymptoten til
\[ f(x)=\frac{1}{x-3}+2 \]
\sv
Vi observerer at
\[ \lim\limits_{x\to 3}\left[\frac{1}{x-3}+2\right]=\pm \infty  \]
Altså er $ x=3 $ en vertikal asymptote for $ f $
} \vsk

\reg[Horisontale asymptoter]{
Gitt en funksjon $ f(x) $. Da er $ {y=c} $ en \outl{horisontal asymptote} for $ f $ hvis 
\[  {\lim\limits_{x\to |\infty|} f(x)=c} \]
}
\eks[]{
	Finn den horisontale asymptoten til
	\[ f(x)=\frac{1}{x-3}+2 \]
	\sv
	Vi observerer at
	\[ \lim\limits_{x\to |\infty|}\left[\frac{1}{x-3}+2\right]=2 \]
	Altså er $ y=2 $ en horisontal asymptote for $ f $.
}
\section{Konvekse og konkave funksjoner}
\reg[Konvekse og konkave funksjoner]{
	Gitt en kontinuerlig funksjon $ f(x) $.	\vsk
	
	Hvis hele linja mellom $ (a, f(a)) $ og $ (b, f(b)) $ ligger over grafen til $ f $ på intervallet $ [a, b] $, er $ f $ konveks for $ x\in[a, b] $.	
	\begin{figure}
		\centering
		\includegraphics[]{\figp{konv}}
	\end{figure}	
	Hvis hele linja mellom $ (a, f(a)) $ og $ (b, f(b)) $ ligger under grafen til $ f $ på intervallet $ [a, b] $, er $ f $ konkav for $ x\in[a, b] $.
	\begin{figure}
		\centering
		\includegraphics[]{\figp{konk}}
	\end{figure}
	\vs
}

\section{Injektive funksjoner}
\reg[Injektive funksoner]{
Gitt en funksjon $ f(x) $. Hvis alle verdier til $ f $ er unike på intervallet $ x\in[a, b] $, er $ f $ \textit{injektiv} på dette intervallet.
}
\spr{
Et annet ord for injektiv er \textit{én-entydig}.
}

\section{Omvendte funksjoner}
Gitt funksjonen $ f(x)=2x+1 $, som åpenbart er injektiv for alle $ x\in \mathbb{R} $. Dette betyr at likningen $ f=2x+1 $ bare har én løsning, uavhengig om vi løser med hensyn på $ x $ eller $ f $. Løser vi med hensyn på $ x $, får vi at
\[ x=\frac{f-1}{2} \]
Nå har vi gått fra å ha et uttrykk for $ f $ til, det ''omvendte'', et uttrykk for $ x $. Siden $ x $ og $ f $ begge er variabler, er $ x $ en funksjon av $ f $, og for å tydeliggjøre dette kunne vi ha skrevet
\[ x(f)=\frac{f-1}{2} \]
Denne funksjonen kalles den \textit{omvendte} til $ f $. Setter vi uttrykket til $ f $ inn i uttrykket til $ x(f) $, får vi nødvendigvis $ x $:
\alg{
x\left(2x+1\right)&=\frac{2x+1-1}{2} \\
&= x
} 
Likningen over synliggjør et problem; det er veldig rotete å behandle $ x $ som en funksjon og som en variabel samtidig. Det er derfor vanlig å omdøpe både $ f $ og $ x $, slik at den omvendte funksjonen og variabelen den avhenger av får nye symboler. For eksempel kan vi sette $ y=f$ og $ g=x $. Den omvendte funksjonen $ g $ til $ f $ er da at
\[ g(y)=\frac{y-1}{2} \]
\reg[Omvendte funksjoner]{
Gitt to injektive funksjoner $ f(x) $ og $ g(y) $. Hvis
\[ g(f)=x \]
er $ f $ og $ g $ \textit{omvendte} funksjoner.
} \newpage
\eks[1]{Gitt funksjonen $ f(x)=5x-3 $.
\abc{
\item Finn den omvendte funksjonen $ g $ til $ f $.
\item Vis at $ g(f)=x $. 
} 

\sv
\abc{
\item Vi setter $ y=f $, og løser likningen med hensyn på $ x $:
\alg{
	y&= 5x-3 \\
	x &= \frac{y+3}{5}
}
Da er $ g(y)=\frac{y+3}{5} $.
\item Når $ y = f $, har vi at
\alg{
g(y)&=g(5x-3)\br 
&= \frac{5x-3+3}{5} \\
&= x
}
}
}
\newpage
\info{$ \bm {f^{-1}} $}{
Hvis $ f $ og $ g $ er omvendte funksjoner, skrives $ g $ ofte som $ f^{-1} $. Da er det veldig viktig å merke seg at $ f^{-1} $ ikke er det samme som $ (f)^{-1} $. For eksempel, gitt $ f(x)= x+1 $. Da er
\[ f^{-1}=x-1 \qquad,\qquad (f)^{-1}=\frac{1}{x+1}\] 
I alle andre tilfeller enn ved $ {n=-1} $, vil det i denne boka være slik at
\[ f^n=(f)^n \] 
}

\end{document}