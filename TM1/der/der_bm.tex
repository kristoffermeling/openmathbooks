\documentclass[english, 11 pt, class=article, crop=false]{standalone}
\usepackage[T1]{fontenc}
%\renewcommand*\familydefault{\sfdefault} % For dyslexia-friendly text
\usepackage{lmodern} % load a font with all the characters
\usepackage{geometry}
\geometry{verbose,paperwidth=16.1 cm, paperheight=24 cm, inner=2.3cm, outer=1.8 cm, bmargin=2cm, tmargin=1.8cm}
\setlength{\parindent}{0bp}
\usepackage{import}
\usepackage[subpreambles=false]{standalone}
\usepackage{amsmath}
\usepackage{amssymb}
\usepackage{esint}
\usepackage{babel}
\usepackage{tabu}
\makeatother
\makeatletter

\usepackage{titlesec}
\usepackage{ragged2e}
\RaggedRight
\raggedbottom
\frenchspacing

% Norwegian names of figures, chapters, parts and content
\addto\captionsenglish{\renewcommand{\figurename}{Figur}}
\makeatletter
\addto\captionsenglish{\renewcommand{\chaptername}{Kapittel}}
\addto\captionsenglish{\renewcommand{\partname}{Del}}


\usepackage{graphicx}
\usepackage{float}
\usepackage{subfig}
\usepackage{placeins}
\usepackage{cancel}
\usepackage{framed}
\usepackage{wrapfig}
\usepackage[subfigure]{tocloft}
\usepackage[font=footnotesize,labelfont=sl]{caption} % Figure caption
\usepackage{bm}
\usepackage[dvipsnames, table]{xcolor}
\definecolor{shadecolor}{rgb}{0.105469, 0.613281, 1}
\colorlet{shadecolor}{Emerald!15} 
\usepackage{icomma}
\makeatother
\usepackage[many]{tcolorbox}
\usepackage{multicol}
\usepackage{stackengine}

\usepackage{esvect} %For vectors with capital letters

% For tabular
\usepackage{array}
\usepackage{multirow}
\usepackage{longtable} %breakable table

% Ligningsreferanser
\usepackage{mathtools}
\mathtoolsset{showonlyrefs}

% index
\usepackage{imakeidx}
\makeindex[title=Indeks]

%Footnote:
\usepackage[bottom, hang, flushmargin]{footmisc}
\usepackage{perpage} 
\MakePerPage{footnote}
\addtolength{\footnotesep}{2mm}
\renewcommand{\thefootnote}{\arabic{footnote}}
\renewcommand\footnoterule{\rule{\linewidth}{0.4pt}}
\renewcommand{\thempfootnote}{\arabic{mpfootnote}}

%colors
\definecolor{c1}{cmyk}{0,0.5,1,0}
\definecolor{c2}{cmyk}{1,0.25,1,0}
\definecolor{n3}{cmyk}{1,0.,1,0}
\definecolor{neg}{cmyk}{1,0.,0.,0}

% Lister med bokstavar
\usepackage[inline]{enumitem}

\newcounter{rg}
\numberwithin{rg}{chapter}
\newcommand{\reg}[2][]{\begin{tcolorbox}[boxrule=0.3 mm,arc=0mm,colback=blue!3] {\refstepcounter{rg}\phantomsection \large \textbf{\therg \;#1} \vspace{5 pt}}\newline #2  \end{tcolorbox}\vspace{-5pt}}

\newcommand\alg[1]{\begin{align} #1 \end{align}}

\newcommand\eks[2][]{\begin{tcolorbox}[boxrule=0.3 mm,arc=0mm,enhanced jigsaw,breakable,colback=green!3] {\large \textbf{Eksempel #1} \vspace{5 pt}\\} #2 \end{tcolorbox}\vspace{-5pt} }

\newcommand{\st}[1]{\begin{tcolorbox}[boxrule=0.0 mm,arc=0mm,enhanced jigsaw,breakable,colback=yellow!12]{ #1} \end{tcolorbox}}

\newcommand{\spr}[1]{\begin{tcolorbox}[boxrule=0.3 mm,arc=0mm,enhanced jigsaw,breakable,colback=yellow!7] {\large \textbf{Språkboksen} \vspace{5 pt}\\} #1 \end{tcolorbox}\vspace{-5pt} }

\newcommand{\sym}[1]{\colorbox{blue!15}{#1}}

\newcommand{\info}[2]{\begin{tcolorbox}[boxrule=0.3 mm,arc=0mm,enhanced jigsaw,breakable,colback=cyan!6] {\large \textbf{#1} \vspace{5 pt}\\} #2 \end{tcolorbox}\vspace{-5pt} }

\newcommand\algv[1]{\vspace{-11 pt}\begin{align*} #1 \end{align*}}

\newcommand{\regv}{\vspace{5pt}}
\newcommand{\mer}{\textsl{Merk}: }
\newcommand{\mers}[1]{{\footnotesize \mer #1}}
\newcommand\vsk{\vspace{11pt}}
\newcommand\vs{\vspace{-11pt}}
\newcommand\vsb{\vspace{-16pt}}
\newcommand\sv{\vsk \textbf{Svar} \vspace{4 pt}\\}
\newcommand\br{\\[5 pt]}
\newcommand{\figp}[1]{../fig/#1}
\newcommand\algvv[1]{\vs\vs\begin{align*} #1 \end{align*}}
\newcommand{\y}[1]{$ {#1} $}
\newcommand{\os}{\\[5 pt]}
\newcommand{\prbxl}[2]{
\parbox[l][][l]{#1\linewidth}{#2
	}}
\newcommand{\prbxr}[2]{\parbox[r][][l]{#1\linewidth}{
		\setlength{\abovedisplayskip}{5pt}
		\setlength{\belowdisplayskip}{5pt}	
		\setlength{\abovedisplayshortskip}{0pt}
		\setlength{\belowdisplayshortskip}{0pt} 
		\begin{shaded}
			\footnotesize	#2 \end{shaded}}}

\renewcommand{\cfttoctitlefont}{\Large\bfseries}
\setlength{\cftaftertoctitleskip}{0 pt}
\setlength{\cftbeforetoctitleskip}{0 pt}

\newcommand{\bs}{\\[3pt]}
\newcommand{\vn}{\\[6pt]}
\newcommand{\fig}[1]{\begin{figure}
		\centering
		\includegraphics[]{\figp{#1}}
\end{figure}}

\newcommand{\figc}[2]{\begin{figure}
		\centering
		\includegraphics[]{\figp{#1}}
		\caption{#2}
\end{figure}}

\newcommand{\sectionbreak}{\clearpage} % New page on each section

\newcommand{\nn}[1]{
\begin{equation}
	#1
\end{equation}
}

% Equation comments
\newcommand{\cm}[1]{\llap{\color{blue} #1}}

\newcommand\fork[2]{\begin{tcolorbox}[boxrule=0.3 mm,arc=0mm,enhanced jigsaw,breakable,colback=yellow!7] {\large \textbf{#1 (forklaring)} \vspace{5 pt}\\} #2 \end{tcolorbox}\vspace{-5pt} }
 
%colors
\newcommand{\colr}[1]{{\color{red} #1}}
\newcommand{\colb}[1]{{\color{blue} #1}}
\newcommand{\colo}[1]{{\color{orange} #1}}
\newcommand{\colc}[1]{{\color{cyan} #1}}
\definecolor{projectgreen}{cmyk}{100,0,100,0}
\newcommand{\colg}[1]{{\color{projectgreen} #1}}

% Methods
\newcommand{\metode}[2]{
	\textsl{#1} \\[-8pt]
	\rule{#2}{0.75pt}
}

%Opg
\newcommand{\abc}[1]{
	\begin{enumerate}[label=\alph*),leftmargin=18pt]
		#1
	\end{enumerate}
}
\newcommand{\abcs}[2]{
	\begin{enumerate}[label=\alph*),start=#1,leftmargin=18pt]
		#2
	\end{enumerate}
}
\newcommand{\abcn}[1]{
	\begin{enumerate}[label=\arabic*),leftmargin=18pt]
		#1
	\end{enumerate}
}
\newcommand{\abch}[1]{
	\hspace{-2pt}	\begin{enumerate*}[label=\alph*), itemjoin=\hspace{1cm}]
		#1
	\end{enumerate*}
}
\newcommand{\abchs}[2]{
	\hspace{-2pt}	\begin{enumerate*}[label=\alph*), itemjoin=\hspace{1cm}, start=#1]
		#2
	\end{enumerate*}
}

% Oppgaver
\newcommand{\opgt}{\phantomsection \addcontentsline{toc}{section}{Oppgaver} \section*{Oppgaver for kapittel \thechapter}\vs \setcounter{section}{1}}
\newcounter{opg}
\numberwithin{opg}{section}
\newcommand{\op}[1]{\vspace{15pt} \refstepcounter{opg}\large \textbf{\color{blue}\theopg} \vspace{2 pt} \label{#1} \\}
\newcommand{\ekspop}[1]{\vsk\textbf{Gruble \thechapter.#1}\vspace{2 pt} \\}
\newcommand{\nes}{\stepcounter{section}
	\setcounter{opg}{0}}
\newcommand{\opr}[1]{\vspace{3pt}\textbf{\ref{#1}}}
\newcommand{\oeks}[1]{\begin{tcolorbox}[boxrule=0.3 mm,arc=0mm,colback=white]
		\textit{Eksempel: } #1	  
\end{tcolorbox}}
\newcommand\opgeks[2][]{\begin{tcolorbox}[boxrule=0.1 mm,arc=0mm,enhanced jigsaw,breakable,colback=white] {\footnotesize \textbf{Eksempel #1} \\} \footnotesize #2 \end{tcolorbox}\vspace{-5pt} }
\newcommand{\rknut}{
Rekn ut.
}

%License
\newcommand{\lic}{\textit{Matematikken sine byggesteinar by Sindre Sogge Heggen is licensed under CC BY-NC-SA 4.0. To view a copy of this license, visit\\ 
		\net{http://creativecommons.org/licenses/by-nc-sa/4.0/}{http://creativecommons.org/licenses/by-nc-sa/4.0/}}}

%referances
\newcommand{\net}[2]{{\color{blue}\href{#1}{#2}}}
\newcommand{\hrs}[2]{\hyperref[#1]{\color{blue}\textsl{#2 \ref*{#1}}}}
\newcommand{\rref}[1]{\hrs{#1}{regel}}
\newcommand{\refkap}[1]{\hrs{#1}{kapittel}}
\newcommand{\refsec}[1]{\hrs{#1}{seksjon}}

\newcommand{\mb}{\net{https://sindrsh.github.io/FirstPrinciplesOfMath/}{MB}}


%line to seperate examples
\newcommand{\linje}{\rule{\linewidth}{1pt} }

\usepackage{datetime2}
%%\usepackage{sansmathfonts} for dyslexia-friendly math
\usepackage[]{hyperref}


\newcommand{\note}{Merk}
\newcommand{\notesm}[1]{{\footnotesize \textsl{\note:} #1}}
\newcommand{\ekstitle}{Eksempel }
\newcommand{\sprtitle}{Språkboksen}
\newcommand{\expl}{forklaring}

\newcommand{\vedlegg}[1]{\refstepcounter{vedl}\section*{Vedlegg \thevedl: #1}  \setcounter{vedleq}{0}}

\newcommand\sv{\vsk \textbf{Svar} \vspace{4 pt}\\}

%references
\newcommand{\reftab}[1]{\hrs{#1}{tabell}}
\newcommand{\rref}[1]{\hrs{#1}{regel}}
\newcommand{\dref}[1]{\hrs{#1}{definisjon}}
\newcommand{\refkap}[1]{\hrs{#1}{kapittel}}
\newcommand{\refsec}[1]{\hrs{#1}{seksjon}}
\newcommand{\refdsec}[1]{\hrs{#1}{delseksjon}}
\newcommand{\refved}[1]{\hrs{#1}{vedlegg}}
\newcommand{\eksref}[1]{\textsl{#1}}
\newcommand\fref[2][]{\hyperref[#2]{\textsl{figur \ref*{#2}#1}}}
\newcommand{\refop}[1]{{\color{blue}Oppgave \ref{#1}}}
\newcommand{\refops}[1]{{\color{blue}oppgave \ref{#1}}}
\newcommand{\refgrubs}[1]{{\color{blue}gruble \ref{#1}}}

\newcommand{\openmathser}{\openmath\,-\,serien}

% Exercises
\newcommand{\opgt}{\newpage \phantomsection \addcontentsline{toc}{section}{Oppgaver} \section*{Oppgaver for kapittel \thechapter}\vs \setcounter{section}{1}}


% Sequences and series
\newcommand{\sumarrek}{Summen av en aritmetisk rekke}
\newcommand{\sumgerek}{Summen av en geometrisk rekke}
\newcommand{\regnregsum}{Regneregler for summetegnet}

% Trigonometry
\newcommand{\sincoskomb}{Sinus og cosinus kombinert}
\newcommand{\cosfunk}{Cosinusfunksjonen}
\newcommand{\trid}{Trigonometriske identiteter}
\newcommand{\deravtri}{Den deriverte av de trigonometriske funksjonene}
% Solutions manual
\newcommand{\selos}{Se løsningsforslag.}
\newcommand{\se}[1]{Se eksempel på side \pageref{#1}}

%Vectors
\newcommand{\parvek}{Parallelle vektorer}
\newcommand{\vekpro}{Vektorproduktet}
\newcommand{\vekproarvol}{Vektorproduktet som areal og volum}


% 3D geometries
\newcommand{\linrom}{Linje i rommet}
\newcommand{\avstplnpkt}{Avstand mellom punkt og plan}


% Integral
\newcommand{\bestminten}{Bestemt integral I}
\newcommand{\anfundteo}{Analysens fundamentalteorem}
\newcommand{\intuf}{Integralet av utvalge funksjoner}
\newcommand{\bytvar}{Bytte av variabel}
\newcommand{\intvol}{Integral som volum}
\newcommand{\andordlindif}{Andre ordens lineære differensialligninger}




\begin{document}

\section{Definisjoner \label{grense}}
Gitt en funksjon $ f(x) $ og to $ x $-verdier $ a $ og $ b $. Endringen til $ f $ relativ til endringen til $ x $ for disse verdiene er gitt som
\begin{equation}\label{releq}
	\frac{f(b)-f(a)}{b-a}
\end{equation}
I \mb\ har vi sett at uttrykket over gir stigningstallet til linja som går gjennom punktene $ (a, f(a)) $ og $ (b, f(b)) $. I en matematisk sammenheng er det ekstra interessant å undersøke \eqref{releq} når $ b $ nærmer seg $ a $.
\begin{figure}
	\centering
	\subfloat[]{
		\includegraphics[scale=0.8]{\figp{der2a}}
	}\qquad
	\subfloat[]{
		\includegraphics[scale=0.8]{\figp{der2b}}
	}\\
	\subfloat[]{
		\includegraphics[scale=0.8]{\figp{der2c}}
	} \qquad
	\subfloat[]{
	\includegraphics[scale=0.8]{\figp{der2d}}
}
\end{figure}
Ved å innføre tallet $ h $, og å sette $ b=a+h $, kan vi skrive \eqref{releq} som
\[ \frac{f(a+h)-f(a)}{h} \]
Å \outl{derivere} innebærer å undersøke grenseverdien til denne brøken når $ h $ går mot 0.
\info{\note}{
I teksten og figurene over har vi tatt utgangspunkt i at $ {b>a }$, men dette er ikke en forutsetning for at uttrykkene er gyldige.
}
\regdef[Den deriverte\label{derdefa}]{
Gitt en funksjon $ f(x) $. \outl{Den deriverte av $ f $ i $ {x=a} $} er da gitt som
\begin{equation}\label{derdefaeq}
	f'(a) =\lim\limits_{h \to 0} \frac{f(a+h)-f(a)}{h}
\end{equation}
Linja som har stigningstall $ f'(a) $, og som går gjennom punktet $ (a, f(a)) $, kalles \outl{tangeringslinja} til $ f $ for $ x=a $.
\fig{der1}
}
\eks[1]{
Gitt $ f(x)=x^2 $. Finn $ f'(2) $.

\sv
Vi har at
\algv{
	f'(2) &=\lim\limits_{h \to 0} \frac{f(2+h)-f(2)}{h} \\
	&=\lim\limits_{h \to 0} \frac{(2+h)^2-2^2}{h} \\
	&= \lim\limits_{h \to 0} \frac{2^2+4h + (h)^2-2^2}{h} \\
	&= 4
}
}
\newpage
\eks[2]{ \label{dereks2}
Gitt $ f(x)=x^3 $. Finn $ f'(a) $ .\vsk

\sv
Vi har at
\alg{
f'(a) &=\lim\limits_{h \to 0} \frac{(a+h)^3-a^3}{h} \br
&= \lim\limits_{h \to 0}\frac{a^3+3a^2h +3 ah^2+ h^3-a^3}{h} \br
&= \lim\limits_{h \to 0}\left(3a^2 +3ah+h^2\right) \\
&= 3a^2
}
Altså er $ f'(a)=3a^2 $.
}
\info{Alternativ definisjon}{
En ekvivalent utgave av \eqref{derdefaeq} er
\begin{equation}\label{derdefalt}
f'(a)=	\lim\limits_{b\to a} \frac{f(b)-f(a)}{b-a}
\end{equation}
}
\subsubsection{Linearisering av en funksjon}
Gitt en funksjon $ f(x) $ og en variabel $ k $. Siden $ f'(a) $ angir stigningstallet til $ f(a) $ for $ x=a $, vil en tilnærming til $ f(a+k) $ være
\[ f(a+k)\approx f(a)+f'(a)k \]
Det er ofte nyttig å vite differansen $ \varepsilon $ mellom en tilnærming og den faktiske verdien:
\begin{equation}\label{vareps}
	\varepsilon = f(a+k)-\left[f(a)+f'(a)k\right]
\end{equation}
Vi legger merket til at\footnote{Dette overlates til leseren å vise.} $ \lim\limits_{h\to0}\frac{\varepsilon_f}{k} =0 $, og skriver om \eqref{vareps} til en formel for $ f(x+k) $: \regv
\reg[Linearisering av en funskjon \label{linfunk}]{
Gitt en funskjon $ f(x) $ og en variabel $ k $. Da finnes en funksjon $ \varepsilon $ slik at
\begin{equation} \label{linfunkeq}
f(a+k)=f(a)+f'(a)k+\varepsilon
\end{equation}
hvor $ \lim\limits_{h\to0}\frac{\varepsilon_f}{k}=0 $. \vsk

Tilnærmingen
\begin{equation}\label{linaprx}
	f(a+k)\approx f(a)+f'(x)k
\end{equation}
kalles \outl{lineæarapproksimasjonen} av $ f(a+k) $.
}

\section{Derivasjonsregler}
\subsection{Den deriverte}
\textsl{Eksempel 2} på side \pageref{dereks2} belyser noe viktig; hvis grenseverdien i \eqref{derdefaeq} eksisterer, vil $ f'(a) $ være uttrykt ved $ a $. Og selv om $ a $ betraktes som en konstant langs veien som fører til dette uttrykket, er det ingenting som hindrer oss i å etterpå behandle $ a $ som en variabel. Hvis $ f'(a) $ er et resultat av derivasjon av funksjonen $ f(x) $ er det også hendig å omdøpe $ a $ til $ x $:\regv
\reg[Den deriverte funksjonen]{
	Gitt en funksjon $ f(x) $. \outl{Den deriverte av $ f $} er funksjonen som fremkommer ved å erstatte $ a $ i \eqref{derdefaeq} med $ x $. Denne funksjonen skriver vi som $ f'(x) $.
}
\eks[]{
Gitt $ {f(x)=x^3} $. Siden\footnote{Se \textsl{Eksempel 2}, side \pageref{dereks2}.} $ {f'(a)=3a^2} $, er $ f'(x)=3x^2 $. 
}
\info{Alternative skrivemåter}{
Alternative skrivemåter for \sym{$f'$} er 
\sym{$ (f)'$} og \sym{$\frac{\text{d}}{\text{d}x} f $}.
}
\info{Derivert med hensyn på}{
Derivasjon som vi har sett på så langt har vært en brøk med en differanse av $ x $-verdier i nevner og den tilknyttede differansen av $ f $-verdier i teller. Da sier vi at $ f $ er derivert med \outl{hensyn på $ x $}. I denne bokserien skal vi i all hovedsak se på funksjoner som bare er avhengige av én variabel. Gitt en funksjon $ f(x) $, er det da underforstått at \sym{$ f' $} symboliserer $ f $ derivert med hensyn på $ x $.\vsk

Samtidig er det greit å være klar over at en funksjon gjerne kan være avhengig av flere variabler. For eksempel er funksjonen
\[ f(x, y)= x^2+y^3 \]
en \outl{flervariabel funksjon}, avhengig av både $ x $ og $ y $. I dette tilfellet kan vi bruke skrive \sym{$ \frac{\text{d}}{\text{d}x} f$} for å indikere derivasjon med hensyn på $ x $, og \sym{$ \frac{\text{d}}{\text{d}x}f $} for å indikere derivasjon med hensyn på $ y $. Leseren må gjerne forklare for seg selv hvorfor følgende stemmer:
\[ \frac{\text{d}}{\text{d}x}f = 2x \qquad,\qquad  \frac{\text{d}}{\text{d}y}f = 3y^2 ,\]
}
\subsection{Den deriverte av elementære funksjoner}
\reg[\derel \label{derel}]{
For en variabel $ x $ og $ r\in\mathbb{R} $ er
\begin{align}
& (e^x)' =e^{x} \label{derex} \vn	
&  (x^r)' = r x^{r-1} \label{derxr} \vn
& (\ln x)'=  \frac{1}{x} \label{derlnx}
\end{align}
}	
\reg[Den deriverte av sammensatte funksjoner]{
Gitt $ c\in \mathbb{R} $ og funksjonene $ f(x) $ og $ g(x) $. Da er
\alg{
& \left(a\cdot f\right)'=a\cdot f' \vn
& \left(f+g\right)' =f'+g' \vn
& \left(f-g\right)'=f'-g'
}
}
\regdef[Den andrederiverte]{
Gitt en deriverbar funksjon $ f(x) $. Da er den \outl{andrederiverte av $ f $} gitt som
\begin{equation}\label{andredereq}
	\left(f'\right)'=f''
\end{equation}
}
\regdef[Den deriverte av en vektorfunksjon \label{vektfunkder}]{
Gitt funksjonene $ f(t) $, $ g(t) $ og $ v(t)=[f(t), g(t)] $. Da er
\begin{equation}\label{vektfunkdereq}
v'(t)=[f'(t), g'(t)]	
\end{equation}
}
\subsection{Kjerneregelen}
La oss se på tre funksjoner $ f $, $ g $ og $ u $, hvor
\[ f(x)= g\left[u(x)\right] \]
$ f $ beskrives direkte av $ x $, mens $ g $ beskrives indirekte av $ x $, via $ u(x) $. \vsk

La oss bruke $f(x) =e^{x^2} $ som eksempel. Kjenner vi verdien til \textit{x}, kan vi fort regne ut hva verdien til $ f(x) $ er. For eksempel er
\[ f(2)=e^4 \]
Men vi kan også skrive $ {g\left[u(x)\right]=e^{u(x)}} $, hvor $ u(x)=x^2 $. Denne skrivemåten impliserer at når vi kjenner verdien til \textit{x}, regner vi først ut verdien til \textit{u}, før vi så finner verdien av $g$: 
\[ u(2) =4\qquad,\qquad 
g\left[u(2)\right]=e^{u(2)}=e^4 \]
Av \eqref{derdefaeq} har vi at
\alg{
f(x)&=\lim\limits_{h \to 0}\frac{f(x+h)-f(x)}{h} \br
&= \lim\limits_{h \to 0}\frac{g\left[u(x+h)\right]-g\left[u(x)\right]}{h}
}
Vi setter $ k = u(x+h)-u(x) $. Da er
\[ \lim\limits_{h \to 0}\frac{g\left[u(x+h)\right]-g\left[u(x)\right]}{h}= \lim\limits_{h \to 0}\frac{g(u+k)-g(u)}{h} \]
Av \eqref{linfunkeq} har vi at
\[ g(u)-g(u+k)= g'(u)k+\varepsilon_g  \]
Altså er
\alg{
\lim\limits_{h \to 0}\frac{g(u+k)-g(u)}{h}& = \lim\limits_{h\to 0}\frac{g'(u)k+\varepsilon_g}{h}\br
&=\lim\limits_{h\to 0}\left(g'(u)+\frac{\varepsilon_g}{k}\right)\frac{k}{h}
}
Da $ \lim\limits_{h\to0}k=0 $, er $ \lim\limits_{h\to0}\frac{\varepsilon_g}{k}=0 $. Videre har vi at $ \lim\limits_{h\to0}\frac{k}{h}=u'(x) $. Altså har vi at
\nn{
\lim\limits_{h\to 0}\left(g'(u)+\frac{\varepsilon_g}{k}\right)\frac{k}{h}=g'(u)u'(x)
}
\reg[Kjerneregelen \label{kjerne}]{
For en funksjon $ f(x)=g\left[u(x)\right] $ har vi at
\begin{equation}\label{kjerneeq}
	f'(x) = g'(u)u'(x)
\end{equation}
}
\regv
\eks{
Finn $ f'(x) $ når  $ f(x)=e^{x^2+x+1} $.\\
\sv

Vi setter $ u=x^2+x+1 $, og får at
\begin{align*}
& g(u)=e^u \vn
& g'(u)=e^u \vn
& u'(x) = 2x + 1
\end{align*}
Altså er
\algv{
 f'(x) &= g'(u)u'(x) \\
 &= e^u(2x+1)\\
 &= e^{x^2+x+1}(2x+1) 
}
}
\subsection{Produkt- og divisjonsregelen}
Gitt funksjonene $ f $, $ u $ og $ v $, hvor
\[ f(x)=u(x)v(x) \] 
Av \eqref{derdefa} er da
\[ f'=\lim\limits_{h\to 0}\frac{u(x+h)v(x+h)-u(x)v(x)}{h} \]
La oss nå skrive $ u(x) $ og $ v(x) $ som henholdsvis $ u $ og $ v ,$ og $ u(x+h) $ og $ v(x+h) $ som henholdsvis $ \tilde{u} $ og $ \tilde{v} $:
\alg{
f' &= \lim\limits_{h\to 0} \frac{\tilde{u}\tilde{v}- u v}{h}
}
Vi kan alltids legge til 0 i form av $ \dfrac{u\tilde{v}}{h}- \dfrac{u\tilde{v}}{h} $:
\alg{f'&= \lim\limits_{h\to 0} \left[\frac{\tilde{u}\tilde{v}- u v}{h} + \frac{u\tilde{v}}{h}- \frac{u\tilde{v}}{h}\right] \\
&= \lim\limits_{h\to 0}\left[\frac{(\tilde{u}- u)\tilde{v}}{h} +\frac{u(\tilde{v}- v)}{h} \right]
	}
Siden vi for enhver kontinuerlig funksjon $ g $ har at $ {\lim\limits_{h\to 0} \tilde{g}=g} $ og \\$ {\lim\limits_{h\to 0}  \frac{g(x+k)-g(x)}{h}=g'}$, er
\[ f'=u'v+uv' \]
\reg[Produktregelen ved derivasjon]{
Gitt funksjonene $ f(x) $, $ u(x) $ og $ v(x) $, hvor $ {f=uv} $ da er
\[ f'=u'v+uv' \]	
}
\eks[1]{
Finn den deriverte av funksjonen $ f(x)=x^2e^x $.

\sv
Vi setter $ {u(x)=x^2} $ og $ v(x)=e^x $, da er
\alg{
f &= uv
&u'&=2x
&v'&=e^x
}
Altså er
\algv{
f' &= 2xe^x +x^2 e^x \\
&=xe^x(2+x)
}
}
\vsk

\reg[\divder \label{divder}]{
Gitt funksjonene $ f(x) $, $ u(x) $ og $ v(x) $, hvor $ f=\dfrac{u}{v} $. Da er
\begin{equation}\label{divdereq}
	f'=\frac{u'v-uv'}{v^2} 
\end{equation}
}
\newpage
\eks{
Finn den deriverte av funksjonen $ f(x)=\frac{\cos x}{x^4} $.

\sv
Vi setter $ u(x)=\cos x $ og $ v(x)=x $, da er
\alg{
f&=uv & u'&=-\sin x & v' &= 4x^3
}
Altså er
\alg{
f' &= \frac{-\sin x\cdot x^4-\cos x\cdot 4x^3}{x^8} \br
&= -\frac{x\sin x+4\cos x}{x^5}
}
\notesm{Vi kan også finne $ f $' ved å sette $ {u(x)=\cos x} $ og $ {v(x)=x^{-4}} $, for så å bruke produktregelen.}
}
\newpage
\reg[L'Hopitals regel \label{Lhop}]{
Gitt to deriverbare funksjoner $ f(x) $ og $ g(x) $, hvor
\[ f(a)=g(a)=0 \]
Da er
\nn{
\lim\limits_{x\to a} \frac{f}{g}=\lim\limits_{x\to a}\frac{f'}{g'}\label{Lhopeq}
}
}
\eks[]{
Finn grenseverdien til $ \lim\limits_{x\to 0} \dfrac{e^x-1}{x}$.

\sv
Vi setter $ {f(x)=e^x-1} $ og $ {g(x)=x} $, og merker oss at $ {f(0)=g(0)=0} $. Dermed har vi at
\alg{
\lim\limits_{x\to 0} \frac{e^x-1}{x}&=\lim\limits_{x\to0}\frac{f}{g}
\br
&=\lim\limits_{x\to0} \frac{f'}{g'}\br
&=\lim\limits_{x\to0}\frac{e^x}{1} \\
&= 1
} 
}

\section*{Forklaringer}
\fork{L'hoptial}{
Siden $ f(a)=g(a)=0 $, er
\alg{
\lim\limits_{x\to a}\frac{f(x)}{g(x)}=\lim\limits_{x\to a} \frac{f(x)-f(a)}{g(x)-g(a)}
}
Av \eqref{derdefalt} har vi da at
\[ \lim\limits_{x\to a}\frac{f(x)}{g(x)}=\lim\limits_{x\to a}\frac{f'(x)}{g'(x)} \]}
\vsk
\fork{\ref{divder} \divder}{
Vi har at
\[ f'=\left(\dfrac{u}{v} \right)'=\left(u v^{-1}\right)' \]
Av produktregelen og kjerneregelen er da
\alg{
f'	&= u' v^{-1} - u v^{-2}v' \\
	&= \frac{u'v - u v'}{v^2}
}
}
\newpage
\fork{\ref{derel} \derel}{
\textbf{Likning (\ref{derxr})}\\
Vi starter med å merke oss at
\alg{(\ln x^r)'&=(r \ln x)' \\
	&= \frac{r}{x}
}  
Vi setter $ u=x^r $. Av kjerneregelen har vi da at
\alg{
\frac{r}{x}&=	(\ln u)'\\
	&=\frac{1}{u} u' \\
	&=\frac{1}{x^r}(x^r)' 
}
Altså er
\[ 	(x^r)'= \frac{r}{x}x^r = r x^{r-1} \]

\textbf{Likning (\ref{derlnx})}\\	
Vi har at $ {x=e^{\ln x}} $. Vi setter $ {u=\ln x} $ og $ g(u)=e^u $. Da har vi at $ x=g(u) $, og at
\alg{
g'(u) &= e^u=e^{\ln x}=x \br
u'(x) &= \left(\ln x\right)'
}
Av kjerneregelen har vi at
\alg{
\left(x\right)' &= g'(u)u'(x)\\
&= x \left(\ln x\right)'
}
Da\footnote{Se oppgave ??.}  ${\left(x\right)'=1} $, har vi at
\[ 1 = x\left(\ln x \right)' \]
Altså er
\[ \left(\ln x\right)'= \frac{1}{x} \]
}



\end{document}