\documentclass[english, 11 pt, class=article, crop=false]{standalone}
\usepackage[T1]{fontenc}
%\renewcommand*\familydefault{\sfdefault} % For dyslexia-friendly text
\usepackage{lmodern} % load a font with all the characters
\usepackage{geometry}
\geometry{verbose,paperwidth=16.1 cm, paperheight=24 cm, inner=2.3cm, outer=1.8 cm, bmargin=2cm, tmargin=1.8cm}
\setlength{\parindent}{0bp}
\usepackage{import}
\usepackage[subpreambles=false]{standalone}
\usepackage{amsmath}
\usepackage{amssymb}
\usepackage{esint}
\usepackage{babel}
\usepackage{tabu}
\makeatother
\makeatletter

\usepackage{titlesec}
\usepackage{ragged2e}
\RaggedRight
\raggedbottom
\frenchspacing

% Norwegian names of figures, chapters, parts and content
\addto\captionsenglish{\renewcommand{\figurename}{Figur}}
\makeatletter
\addto\captionsenglish{\renewcommand{\chaptername}{Kapittel}}
\addto\captionsenglish{\renewcommand{\partname}{Del}}


\usepackage{graphicx}
\usepackage{float}
\usepackage{subfig}
\usepackage{placeins}
\usepackage{cancel}
\usepackage{framed}
\usepackage{wrapfig}
\usepackage[subfigure]{tocloft}
\usepackage[font=footnotesize,labelfont=sl]{caption} % Figure caption
\usepackage{bm}
\usepackage[dvipsnames, table]{xcolor}
\definecolor{shadecolor}{rgb}{0.105469, 0.613281, 1}
\colorlet{shadecolor}{Emerald!15} 
\usepackage{icomma}
\makeatother
\usepackage[many]{tcolorbox}
\usepackage{multicol}
\usepackage{stackengine}

\usepackage{esvect} %For vectors with capital letters

% For tabular
\usepackage{array}
\usepackage{multirow}
\usepackage{longtable} %breakable table

% Ligningsreferanser
\usepackage{mathtools}
\mathtoolsset{showonlyrefs}

% index
\usepackage{imakeidx}
\makeindex[title=Indeks]

%Footnote:
\usepackage[bottom, hang, flushmargin]{footmisc}
\usepackage{perpage} 
\MakePerPage{footnote}
\addtolength{\footnotesep}{2mm}
\renewcommand{\thefootnote}{\arabic{footnote}}
\renewcommand\footnoterule{\rule{\linewidth}{0.4pt}}
\renewcommand{\thempfootnote}{\arabic{mpfootnote}}

%colors
\definecolor{c1}{cmyk}{0,0.5,1,0}
\definecolor{c2}{cmyk}{1,0.25,1,0}
\definecolor{n3}{cmyk}{1,0.,1,0}
\definecolor{neg}{cmyk}{1,0.,0.,0}

% Lister med bokstavar
\usepackage[inline]{enumitem}

\newcounter{rg}
\numberwithin{rg}{chapter}
\newcommand{\reg}[2][]{\begin{tcolorbox}[boxrule=0.3 mm,arc=0mm,colback=blue!3] {\refstepcounter{rg}\phantomsection \large \textbf{\therg \;#1} \vspace{5 pt}}\newline #2  \end{tcolorbox}\vspace{-5pt}}

\newcommand\alg[1]{\begin{align} #1 \end{align}}

\newcommand\eks[2][]{\begin{tcolorbox}[boxrule=0.3 mm,arc=0mm,enhanced jigsaw,breakable,colback=green!3] {\large \textbf{Eksempel #1} \vspace{5 pt}\\} #2 \end{tcolorbox}\vspace{-5pt} }

\newcommand{\st}[1]{\begin{tcolorbox}[boxrule=0.0 mm,arc=0mm,enhanced jigsaw,breakable,colback=yellow!12]{ #1} \end{tcolorbox}}

\newcommand{\spr}[1]{\begin{tcolorbox}[boxrule=0.3 mm,arc=0mm,enhanced jigsaw,breakable,colback=yellow!7] {\large \textbf{Språkboksen} \vspace{5 pt}\\} #1 \end{tcolorbox}\vspace{-5pt} }

\newcommand{\sym}[1]{\colorbox{blue!15}{#1}}

\newcommand{\info}[2]{\begin{tcolorbox}[boxrule=0.3 mm,arc=0mm,enhanced jigsaw,breakable,colback=cyan!6] {\large \textbf{#1} \vspace{5 pt}\\} #2 \end{tcolorbox}\vspace{-5pt} }

\newcommand\algv[1]{\vspace{-11 pt}\begin{align*} #1 \end{align*}}

\newcommand{\regv}{\vspace{5pt}}
\newcommand{\mer}{\textsl{Merk}: }
\newcommand{\mers}[1]{{\footnotesize \mer #1}}
\newcommand\vsk{\vspace{11pt}}
\newcommand\vs{\vspace{-11pt}}
\newcommand\vsb{\vspace{-16pt}}
\newcommand\sv{\vsk \textbf{Svar} \vspace{4 pt}\\}
\newcommand\br{\\[5 pt]}
\newcommand{\figp}[1]{../fig/#1}
\newcommand\algvv[1]{\vs\vs\begin{align*} #1 \end{align*}}
\newcommand{\y}[1]{$ {#1} $}
\newcommand{\os}{\\[5 pt]}
\newcommand{\prbxl}[2]{
\parbox[l][][l]{#1\linewidth}{#2
	}}
\newcommand{\prbxr}[2]{\parbox[r][][l]{#1\linewidth}{
		\setlength{\abovedisplayskip}{5pt}
		\setlength{\belowdisplayskip}{5pt}	
		\setlength{\abovedisplayshortskip}{0pt}
		\setlength{\belowdisplayshortskip}{0pt} 
		\begin{shaded}
			\footnotesize	#2 \end{shaded}}}

\renewcommand{\cfttoctitlefont}{\Large\bfseries}
\setlength{\cftaftertoctitleskip}{0 pt}
\setlength{\cftbeforetoctitleskip}{0 pt}

\newcommand{\bs}{\\[3pt]}
\newcommand{\vn}{\\[6pt]}
\newcommand{\fig}[1]{\begin{figure}
		\centering
		\includegraphics[]{\figp{#1}}
\end{figure}}

\newcommand{\figc}[2]{\begin{figure}
		\centering
		\includegraphics[]{\figp{#1}}
		\caption{#2}
\end{figure}}

\newcommand{\sectionbreak}{\clearpage} % New page on each section

\newcommand{\nn}[1]{
\begin{equation}
	#1
\end{equation}
}

% Equation comments
\newcommand{\cm}[1]{\llap{\color{blue} #1}}

\newcommand\fork[2]{\begin{tcolorbox}[boxrule=0.3 mm,arc=0mm,enhanced jigsaw,breakable,colback=yellow!7] {\large \textbf{#1 (forklaring)} \vspace{5 pt}\\} #2 \end{tcolorbox}\vspace{-5pt} }
 
%colors
\newcommand{\colr}[1]{{\color{red} #1}}
\newcommand{\colb}[1]{{\color{blue} #1}}
\newcommand{\colo}[1]{{\color{orange} #1}}
\newcommand{\colc}[1]{{\color{cyan} #1}}
\definecolor{projectgreen}{cmyk}{100,0,100,0}
\newcommand{\colg}[1]{{\color{projectgreen} #1}}

% Methods
\newcommand{\metode}[2]{
	\textsl{#1} \\[-8pt]
	\rule{#2}{0.75pt}
}

%Opg
\newcommand{\abc}[1]{
	\begin{enumerate}[label=\alph*),leftmargin=18pt]
		#1
	\end{enumerate}
}
\newcommand{\abcs}[2]{
	\begin{enumerate}[label=\alph*),start=#1,leftmargin=18pt]
		#2
	\end{enumerate}
}
\newcommand{\abcn}[1]{
	\begin{enumerate}[label=\arabic*),leftmargin=18pt]
		#1
	\end{enumerate}
}
\newcommand{\abch}[1]{
	\hspace{-2pt}	\begin{enumerate*}[label=\alph*), itemjoin=\hspace{1cm}]
		#1
	\end{enumerate*}
}
\newcommand{\abchs}[2]{
	\hspace{-2pt}	\begin{enumerate*}[label=\alph*), itemjoin=\hspace{1cm}, start=#1]
		#2
	\end{enumerate*}
}

% Oppgaver
\newcommand{\opgt}{\phantomsection \addcontentsline{toc}{section}{Oppgaver} \section*{Oppgaver for kapittel \thechapter}\vs \setcounter{section}{1}}
\newcounter{opg}
\numberwithin{opg}{section}
\newcommand{\op}[1]{\vspace{15pt} \refstepcounter{opg}\large \textbf{\color{blue}\theopg} \vspace{2 pt} \label{#1} \\}
\newcommand{\ekspop}[1]{\vsk\textbf{Gruble \thechapter.#1}\vspace{2 pt} \\}
\newcommand{\nes}{\stepcounter{section}
	\setcounter{opg}{0}}
\newcommand{\opr}[1]{\vspace{3pt}\textbf{\ref{#1}}}
\newcommand{\oeks}[1]{\begin{tcolorbox}[boxrule=0.3 mm,arc=0mm,colback=white]
		\textit{Eksempel: } #1	  
\end{tcolorbox}}
\newcommand\opgeks[2][]{\begin{tcolorbox}[boxrule=0.1 mm,arc=0mm,enhanced jigsaw,breakable,colback=white] {\footnotesize \textbf{Eksempel #1} \\} \footnotesize #2 \end{tcolorbox}\vspace{-5pt} }
\newcommand{\rknut}{
Rekn ut.
}

%License
\newcommand{\lic}{\textit{Matematikken sine byggesteinar by Sindre Sogge Heggen is licensed under CC BY-NC-SA 4.0. To view a copy of this license, visit\\ 
		\net{http://creativecommons.org/licenses/by-nc-sa/4.0/}{http://creativecommons.org/licenses/by-nc-sa/4.0/}}}

%referances
\newcommand{\net}[2]{{\color{blue}\href{#1}{#2}}}
\newcommand{\hrs}[2]{\hyperref[#1]{\color{blue}\textsl{#2 \ref*{#1}}}}
\newcommand{\rref}[1]{\hrs{#1}{regel}}
\newcommand{\refkap}[1]{\hrs{#1}{kapittel}}
\newcommand{\refsec}[1]{\hrs{#1}{seksjon}}

\newcommand{\mb}{\net{https://sindrsh.github.io/FirstPrinciplesOfMath/}{MB}}


%line to seperate examples
\newcommand{\linje}{\rule{\linewidth}{1pt} }

\usepackage{datetime2}
%%\usepackage{sansmathfonts} for dyslexia-friendly math
\usepackage[]{hyperref}


\newcommand{\note}{Merk}
\newcommand{\notesm}[1]{{\footnotesize \textsl{\note:} #1}}
\newcommand{\ekstitle}{Eksempel }
\newcommand{\sprtitle}{Språkboksen}
\newcommand{\expl}{forklaring}

\newcommand{\vedlegg}[1]{\refstepcounter{vedl}\section*{Vedlegg \thevedl: #1}  \setcounter{vedleq}{0}}

\newcommand\sv{\vsk \textbf{Svar} \vspace{4 pt}\\}

%references
\newcommand{\reftab}[1]{\hrs{#1}{tabell}}
\newcommand{\rref}[1]{\hrs{#1}{regel}}
\newcommand{\dref}[1]{\hrs{#1}{definisjon}}
\newcommand{\refkap}[1]{\hrs{#1}{kapittel}}
\newcommand{\refsec}[1]{\hrs{#1}{seksjon}}
\newcommand{\refdsec}[1]{\hrs{#1}{delseksjon}}
\newcommand{\refved}[1]{\hrs{#1}{vedlegg}}
\newcommand{\eksref}[1]{\textsl{#1}}
\newcommand\fref[2][]{\hyperref[#2]{\textsl{figur \ref*{#2}#1}}}
\newcommand{\refop}[1]{{\color{blue}Oppgave \ref{#1}}}
\newcommand{\refops}[1]{{\color{blue}oppgave \ref{#1}}}
\newcommand{\refgrubs}[1]{{\color{blue}gruble \ref{#1}}}

\newcommand{\openmathser}{\openmath\,-\,serien}

% Exercises
\newcommand{\opgt}{\newpage \phantomsection \addcontentsline{toc}{section}{Oppgaver} \section*{Oppgaver for kapittel \thechapter}\vs \setcounter{section}{1}}


% Sequences and series
\newcommand{\sumarrek}{Summen av en aritmetisk rekke}
\newcommand{\sumgerek}{Summen av en geometrisk rekke}
\newcommand{\regnregsum}{Regneregler for summetegnet}

% Trigonometry
\newcommand{\sincoskomb}{Sinus og cosinus kombinert}
\newcommand{\cosfunk}{Cosinusfunksjonen}
\newcommand{\trid}{Trigonometriske identiteter}
\newcommand{\deravtri}{Den deriverte av de trigonometriske funksjonene}
% Solutions manual
\newcommand{\selos}{Se løsningsforslag.}
\newcommand{\se}[1]{Se eksempel på side \pageref{#1}}

%Vectors
\newcommand{\parvek}{Parallelle vektorer}
\newcommand{\vekpro}{Vektorproduktet}
\newcommand{\vekproarvol}{Vektorproduktet som areal og volum}


% 3D geometries
\newcommand{\linrom}{Linje i rommet}
\newcommand{\avstplnpkt}{Avstand mellom punkt og plan}


% Integral
\newcommand{\bestminten}{Bestemt integral I}
\newcommand{\anfundteo}{Analysens fundamentalteorem}
\newcommand{\intuf}{Integralet av utvalge funksjoner}
\newcommand{\bytvar}{Bytte av variabel}
\newcommand{\intvol}{Integral som volum}
\newcommand{\andordlindif}{Andre ordens lineære differensialligninger}



\begin{document}
\section{Definisjoner}
\regdef[Halveringslinje]{
Gitt $ \angle BAC $. For et punkt $ P $ som ligger på \textit{halveringslinja} til vinkelen, er
\begin{equation}
	\angle BAP = PAC=\frac{1}{2}\angle BAC
\end{equation}
\fig{halfline}
}	
\regdef[Midtpunkt]{
Midtpunktet $ C $ til $ AB $ er punktet på linjestykket slik at $ {AC=CB} $.
\fig{midtp} 
}
\regdef[Midtnormal]{
Midtnormalen til $ AB $ står normalt på, og går gjennom midtpunktet til, $ AB $.
\fig{midnorm0}
}
\newpage
\regdef[Sinus, cosinus og tangens]{
Gitt en rettvinklet trekant med katetene $ a $ og $ b $, hypotenus $ c $, og vinkel $ v $, som vist i figuren under.
\fig{sincostan}
Da er \vs
\begin{align}
\sin v &= \frac{a}{c} \label{sindef1}\vn
\cos v &= \frac{b}{c} \label{cosdef1}\vn
\tan v &= \frac{a}{b} 	
\end{align}
}
\spr{
I figuren over blir $ a $ kalt den \textit{motstående} kateten til vinkel $ v $, og $ b $ den \textit{hosliggende}.
}
\info{Eksaktverdier}{
	De aller fleste sinus-, cosinus- og tangensverdier er irrasjonale tall, i praktiske anvendelser av verdiene er det derfor vanlig å benytte digitale hjelpemidler. Verdiene som er viktigst for teoretiske formål er gitt i vedlegg ??.
}
\eks[]{ \vs
\fig{sincostaneks}
\[ \sin v =\frac{3}{5}\quad,\quad \cos v = \frac{4}{5}\quad,\quad \tan v = \frac{3}{4} \]
}

\reg[Sinus, cosinus og tangens I]{
	Gitt $ \triangle ABC$, hvor $ {v=\angle BAC>90^\circ} $, som vist i figuren under.
	\fig{sincostan2}
	Da er \vs
\begin{align}
	\sin v &= \frac{CD}{AC} \label{sindef2} \vn
	\cos v &= -\frac{AD}{AC} \label{cosdef2}  \vn
	\tan v &= -\frac{CD}{AD} \label{tandef2}
\end{align}
}
\eks[]{ \vs
\fig{sincostaneks2}
I figuren over er $ CD=\sqrt{3} $, $ AD=1 $ og $ AC=2 $. Da er
\[ \sin 120^\circ =\frac{\sqrt{3}}{2}\quad,\quad \cos 120^\circ = -\frac{1}{2}\quad,\quad \tan 120^\circ =-\sqrt{3} \]
}

\section{Egenskaper til trekanter}

\reg[\arealsetn \label{arealsetn}]{
\begin{figure}
	\centering
	\subfloat[]{\includegraphics{\figp{arealsetn}}}\qquad
	\subfloat[]{\includegraphics{\figp{arealsetnb}}}
\end{figure}
Arealet $ T $ til $ \triangle ABC $ er
\begin{equation}
T=\frac{1}{2} AB\cdot AC\cdot\sin \angle A
\end{equation}
}
\eks[]{\vs
\fig{arealsetneks1}
Da $ \sin 60^\circ =\frac{\sqrt{3}}{2} $
Arealet $ T $ til $ \triangle ABC $ er
\[ T=\frac{1}{2}\cdot5\cdot 2\cdot \frac{\sqrt{3}}{2} =\frac{5\sqrt{3}}{2} \]
}
\reg[Sinussetningen]{
\begin{figure}
	\centering
	\subfloat[]{\includegraphics{\figp{arealsetn}}}\qquad
	\subfloat[]{\includegraphics{\figp{arealsetnb}}}
\end{figure}
	\begin{equation}
		\frac{\sin \angle A}{BC}= \frac{\sin \angle B}{AC}=\frac{\sin \angle C}{AB}
	\end{equation}
}
\reg[\cossetn \label{cossetn}]{
Gitt en trekant med sidelengder $ a $, $ b $ og $ c $, og vinkel $ v $, som vist i figurene under.
\begin{figure}
	\centering
	\subfloat[]{\includegraphics{\figp{cossetna}}} \qquad \quad
	\subfloat[]{\includegraphics{\figp{cossetnb}}}
\end{figure}
Da er
\begin{equation}
a^2 = b^2+c^2-ab\cos v	
\end{equation}
}
\reg[\hlikb \label{hlikb}]{
Gitt en likebeint trekant $ \triangle ABC $, hvor $ AC=BC $, som vist i figuren under. 
\fig{hlikb}
Høgda $ DC $ ligger da på midtnormalen til $ AB $.
}
\reg[\trkmedian \label{trkmedian}]{
En \outl{median} er et linjestykke som går fra et hjørne i en trekant til midtpunktet på den motstående siden i trekanten. \vsk

De tre medianene i en trekant skjærer hverandre i ett og samme punkt.
\fig{median1a}
Gitt $ \triangle ABC $ med medianer $ CD $, $ BF $ og $ AE $, som skjærer hverandre i $ G $. Da er
\alg{	
\frac{CG}{GD}=\frac{BG}{GF}=\frac{AG}{GE} = 2
}
}
\reg[\midtrk \label{midtrk}]{
Midtnormalene i en trekant møtes i ett og samme punkt. Dette punktet er sentrum i den \outl{omskrevne sirkelen} til trekantne, som har hjørnene til trekanten på sin bue.	
\fig{midnorm1a}
}

\reg[\innskrsirk \label{innskrsirk}]{
Halveringslinjene til vinklene i en trekant møtes i ett og samme punkt. Dette punktet er sentrum i den \outl{innskrevne sirkelen} til trekanten, som tangerer hver av sidene til trekanten.
\fig{innskr1a}
}

\section{Egenskaper til sirkler}
\reg[\tang \label{tang}]{
En linje som skjærer en sirkel i bare ett punkt, kalles en \textit{tangent} til sirkelen.\vsk

La $ S $ være sentrum i en sirkel, og la $ A $ være skjæringspunktet til denne sirkelen og ei linje. Da har vi at
\nn{
\text{linja er en tangent til sirkelen} \Longleftrightarrow \vv{AS}\text{ står vinkelrett på linja} 
}
	
\fig{tang0a}
}\regv

\reg[\perfvink \label{perfvink}]{
	Både periferi- og sentralvinkler har vinkelbein som ligger (delvis) inni en sirkel. \vsk
	
	En sentralvinkel har toppunkt i sentrum av en sirkel.\vsk
	
	En periferivinkel har toppunkt på sirkelbuen.\vsk
	
	Gitt en periferivinkel $ u $ og en sentralvinkel $ v $, som er innskrevet i samme sirkel og som spenner over samme sirkelbue. Da er\vs
	\begin{equation}\label{v2u}
		v=2u
	\end{equation}
	\centering
	\includegraphics[]{\figp{pervink0a}}\qquad
	\includegraphics[]{\figp{pervink0b}}\qquad
	\includegraphics[]{\figp{pervink0c}}	
}


\section{Forklaringer}
\fork{\ref{hlikb} \hlikb}{
	Da både $ \triangle ADC $ og $ \triangle DBC $ er rettvinklede, har $ CD $ som korteste katet, og $ {AC=BC} $, følger det av Pytagoras' setning at $ {AD=BD} $.
} \vsk

\fork{\ref{trkmedian} \trkmedian}{
	\begin{figure}
		\centering
		\subfloat[]{
			\includegraphics{\figp{median1b}}
		}\;\;
		\subfloat[]{
			\includegraphics{\figp{median1c}}
		}
	\end{figure}
	Vi lar $ G $ være skjæringspunktet til $ BF $ og $ AE $, og tar det for gitt at dette ligger inne i $ \triangle ABC $. Da $ {AF=\frac{1}{2}AC} $ og $ {BE=\frac{1}{2}BC }$, er $ {ABF= BAE=\frac{1}{2}ABC} $. Dermed har $ F $ og $ E $ lik avstand til $ AB $, som betyr at $ {FE\parallel AB} $. Videre har vi også at
	\alg{
		ABG + AFG &= ABG + BGE \\
		AFG &= BGE 
	}
	$ G $ har lik avstand til $ AF $ og $ FC $, og $ {AF=FC} $. Dermed er $ AFG=GFC $. Tilsvarende er $ BGE=GEC $. Altså har disse fire trekantene likt areal. Videre er
	\alg{
		AFG+GFC+GEC&= AEC \\
		GEC &= \frac{1}{6}ABC
	}
	La $ H $ være skjæringspunktet til $ AE $ og $ CD $. Med samme framgangsmåte som over kan det vises at
	\[HEC=\frac{1}{6}ABC \]
	Da både $ \triangle GEC $ og $ \triangle HEC $ har $ CE $ som side, likt areal, og både $ G $ og $ H $ ligger på $ AE $, må $ G=H $. Altså skjærer medianene hverandre i ett og samme punkt. \vsk
	
	$ \triangle ABC\sim\triangle FEC$ fordi de har parvis parallelle sider. Dermed er
	\[ \frac{AB}{FE} = \frac{BC}{CE}=2  \]
	$ {\triangle ABG\sim\triangle EFG }$ fordi $ \angle EGF $ og $ \angle AGB $ er toppvinkler og $ AB\parallel FE $. Dermed er
	\[ \frac{GB}{FG}=\frac{AB}{FE}=2 \]
	Tilsvarende kan det vises at
	\[ \frac{CG}{GD}=\frac{AG}{GE}=2 \]
}
\newpage
\fork{\ref{midtrk} \midtrk}{
	\fig{midnorm1b}
	Gitt $ \triangle ABC $ med midtpunktene $ D $, $ E $ og $ F $. Vi lar $ S $ være skjæringspunktet til de respektive midtnormalene til $ AC $ og $ AB $.
	$ {\triangle AFS\sim\triangle CFS} $ fordi begge er rettvinklede, begge har $ FS $ som korteste katet, og $ AF=FC $. Tilsvarende er $ {\triangle ADS\sim\triangle BDS} $. Følgelig er $ {CS=AS=BS} $. Dette betyr at 
	\begin{itemize}
		\item $ \triangle BSC $ er likebeint, og da går midtnormalen til $ BC $ gjennom $ S $.
		\item $ A $, $ B $ og $ C $ må nødvendigvis ligge på sirkelen med sentrum $ S $ og radius $ AS=BS=CS $ 
	\end{itemize}
}
\newpage
\fork{\ref{innskrsirk} \innskrsirk}{
	\fig{innskr1b}
	Gitt $ \triangle ABC $. Vi lar $ S $ være skjæringspunktet til de respective halveringslinjene til $ \angle BAC $ og $ \angle CBA $. Videre plasserer vi $ D $, $ E $ og $ F $ slik at $ {DS\perp AB} $, $ {ES\perp BC} $ og $ {FS\perp AC} $. $ {\triangle ASD\cong\triangle ASF} $ fordi begge er rettvinklede og har hypotenus $ AS $, og $ {\angle DAS=\angle SAF} $. Tilsvarende er $ {\triangle BSD \cong \triangle BSE} $. Dermed er $ {SE=SD=SF} $. Følgelig er $ F $, $ C $ og $ E $ de respektive tangeringspunktene til $ AB $, $ BC $ og $ AC $ og sirkelen med sentrum $ S $ og radius $ SE $. \vsk
	
	Videre har vi at $ {\triangle CSE \cong  \triangle CSF} $, fordi begge er rettvinklede og har hypotenus $ CS $, og $ {SF=SE} $. Altså er $ {\angle FCS=\angle ECS}$, som betyr at $ CS $ ligger på halveringslinja til $ \angle ACB $.
}
\newpage
\fork{\ref{tang} \tang}{
\textbf{Linja er en tangent til sirkelen ${\boldmath \Longrightarrow \vv{AS}} $ står vinkelrett på linja}
\fig{tangforkla}
Vi antar at vinkelen mellom linja og $ \vv{AS} $ er ulik $ 90^\circ $. Da må det finnes et punkt $ B $ på linja slik at $ {\angle BAS=\angle SBA} $, som betyr at $ \triangle ASB $ er likebeint. Følgelig er $ {AS=BS} $, og da $ AS $ er lik radien i sirkelen, må dette bety at $ B $ også ligger på sirkelen. Dette motsier det faktum at $ A $ er det eneste skjæringspunktet til sirkelen og linja, og dermed må vinkelen mellom linja og $ \vv{AS} $ være $ 90^\circ $. \vsk

\textbf{Linja er en tangent til sirkelen ${\boldmath \Longleftarrow \vv{AS}} $ står vinkelrett på linja}
\fig{tangforklb}
Gitt et vilkårlig punkt $ B $, som ikke samsvarer med $ A $, på linja. Da er $ BS $ hypotenusen i $ \triangle ABC $. Dette innebærer at $ BS $ er større enn radien til sirkelen ($ {BS>AS} $), og da kan $ B $ umulig ligge på sirkelen. Altså er $ A $ det eneste punktet som ligger på både linja og sirkelen, og dermed er linja en tangent til sirkelen.
}

\fork{\ref{arealsetn} \arealsetn }{
	Gitt to tilfeller av $ \triangle ABC$, som vist i figuren under. Det éne hvor $ \angle BAC\in (0^\circ, 90^\circ] $, det andre hvor $ \angle BAC\in(90^\circ, 0^\circ) $ og la $ h $ være høyden med grunnlinje $ AB $.
	\begin{figure}
		\centering
		\subfloat[]{
			\includegraphics{\figp{arealsetnforkla}}
		}
		\subfloat[]{
			\includegraphics{\figp{arealsetnforklb}}
		}
	\end{figure}
	Arealet $ T $ til $ \triangle ABC$ er i begge tilfeller
	\begin{equation}
		T = \frac{1}{2}AB\cdot h
	\end{equation}
	Av henholdsvis \eqref{sindef1} og \eqref{sindef2} har vi at $ h=AC\cdot \sin \angle BAC $, og da er
	\[ T = \frac{1}{2}AB\cdot h=\frac{1}{2}AB\cdot AC\sin \angle BAC \]
}
\newpage
\fork{\ref{cossetn} \cossetn}{
	{\boldmath $ v\in(90^\circ, 180^\circ) $ }\\	
	\fig{cossetnforkla}	
	Av Pytagoras' setning har vi at
	\begin{equation}
		x^2=b^2-h^2 \label{cossetnforkl1}
	\end{equation}
	og at
	\begin{align}
		a^2&=(x+c)^2+h^2 \label{cosetnforkla2}\\
		a^2&=x^2+2xc+c^2+h^2 \label{cossetnforkl2}
	\end{align}
	Ved å sette uttrykket for $ x^2 $ fra \eqref{cossetnforkl1} inn i \eqref{cossetnforkl2}, får vi at
	\begin{align}
		a^2&=b^2-h^2+2xc+c^2+h^2 \\
		a^2&=b^2+c^2+2xc \label{cossetnforkl3}
	\end{align}
	Av \eqref{cosdef2} har vi at $ x=-b\cos v $, og da er
	\nn{
		a^2=b^2+c^2-2bc\cos v
	}
	{\boldmath $ v\in[(0^\circ, 90^\circ] $ }\\	
	\fig{cossetnforklb}	
	Dette tilfellet skiller seg ut fra tilfellet hvor $ {v\in(90^\circ, 180^\circ]} $ på to måter:
	\begin{enumerate}[label=(\roman*)]
		\item I \eqref{cosetnforkla2} får vi $ (c-x)^2 $ i steden for $ (x+c)^2 $. I \eqref{cossetnforkl3} får vi da $ -2xc $ i steden for $ +2xc $.
		\item Av \eqref{cosdef1} er $ x=b\cos v $. Av punkt (ii) følger det da at
		\[ a^2=b^2+c^2-2bc\cos v \]
	\end{enumerate}
}
\fork{\ref{perfvink} \perfvink}{
	Tilhørende periferi- og sentralvinkler kan deles inn i tre tilfeller.\vsk
	
	\textbf{i) En diameter i sirkelen er høyre eller venstre vinkelbein i begge vinklene}\\[5pt]
	I figuren under er $ S $ sentrum i sirkelen, ${\angle BAC= u}$ en periferi-vinkel og $ {\angle BSC=v} $ den tilhørende sentralvinkelen. Vi setter $ {\angle {SCB}=a} $. $ {\angle {ACS}=\angle {SAC}=u} $ og ${\angle {CBS}=\angle {SCB}=a} $ fordi både $ \triangle {ASC} $ og $ \triangle {SBC} $ er likebeinte.\vspace{-1pt}
	\fig{pervinkforkla}
	Vi har at \vs
	\begin{align}
		2a &= 180^\circ-v \label{aogv} \vn
		2u+2a &= 180^\circ \label{uoga}
	\end{align}
	Vi setter uttrykket for $ 2a $ fra \eqref{aogv} inn i \eqref{uoga}:
	\alg{
		2u+180^\circ -v &= 180^\circ\\
		2u &= v
	}
	
	\textbf{ii) Vinklene ligger innenfor samme halvdel av sirkelen}\\[5pt]
	I figuren under er $ u $ en periferivinkel og $ v $ den tilhørende sentralvinkelen. I tillegg har vi tegnet inn en diameter, som er med på å danne vinklene $ a $ og $ b $. Både $ u $ og $ v $ ligger i sin helhet på samme side av denne diameteren.
	\fig{pervinkforkld}
	Ettersom $  {u+a }$  er en periferivinkel, og $ {v+b }$ den tilhørende sentralvinkelen, vet vi av tilfelle 1 at
	\[ 2(u+a) = v+b  \]
	Men ettersom $ a $ og $ b $ også er samhørende periferi- og sentralvinkler, er $ {2a=b} $. Det betyr at
	\algv{
		2u+b &= v+b \\
		2u &= v
	}
	\newpage
	\textbf{iii) Vinklene ligger ikke innenfor samme halvdel av sirkelen}\\[5pt]
	I figuren under er $ u $ en periferivinkel og $ v $ den tilhørende sentralvinkelen. I figuren til høyre har vi tegnet inn en diameter. Den deler $ u $ inn i vinklene $ a $ og $ c $, og $ v $ inn i $ b $ og $ d $.
	\begin{center}
		\includegraphics[]{\figp{pervinkforklc}}\qquad
		\includegraphics[]{\figp{pervinkforklb}}
	\end{center}
	$ a $ og $ c $ er begge periferivinkler, med henholdsvis $ b $ og $ d $ som tilhørende sentralvinkler. Av tilfelle i) har vi da at
	\algv{
		2a &= b \vn
		2c &= d
	}
	Dermed er
	\algv{
		2a+2c &= b+d \\
		2(a+c) &= v \\
		2u &= v
	}
}
\end{document}


