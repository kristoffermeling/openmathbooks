\documentclass[english, 11 pt, class=article, crop=false]{standalone}
\usepackage[T1]{fontenc}
%\renewcommand*\familydefault{\sfdefault} % For dyslexia-friendly text
\usepackage{lmodern} % load a font with all the characters
\usepackage{geometry}
\geometry{verbose,paperwidth=16.1 cm, paperheight=24 cm, inner=2.3cm, outer=1.8 cm, bmargin=2cm, tmargin=1.8cm}
\setlength{\parindent}{0bp}
\usepackage{import}
\usepackage[subpreambles=false]{standalone}
\usepackage{amsmath}
\usepackage{amssymb}
\usepackage{esint}
\usepackage{babel}
\usepackage{tabu}
\makeatother
\makeatletter

\usepackage{titlesec}
\usepackage{ragged2e}
\RaggedRight
\raggedbottom
\frenchspacing

% Norwegian names of figures, chapters, parts and content
\addto\captionsenglish{\renewcommand{\figurename}{Figur}}
\makeatletter
\addto\captionsenglish{\renewcommand{\chaptername}{Kapittel}}
\addto\captionsenglish{\renewcommand{\partname}{Del}}


\usepackage{graphicx}
\usepackage{float}
\usepackage{subfig}
\usepackage{placeins}
\usepackage{cancel}
\usepackage{framed}
\usepackage{wrapfig}
\usepackage[subfigure]{tocloft}
\usepackage[font=footnotesize,labelfont=sl]{caption} % Figure caption
\usepackage{bm}
\usepackage[dvipsnames, table]{xcolor}
\definecolor{shadecolor}{rgb}{0.105469, 0.613281, 1}
\colorlet{shadecolor}{Emerald!15} 
\usepackage{icomma}
\makeatother
\usepackage[many]{tcolorbox}
\usepackage{multicol}
\usepackage{stackengine}

\usepackage{esvect} %For vectors with capital letters

% For tabular
\usepackage{array}
\usepackage{multirow}
\usepackage{longtable} %breakable table

% Ligningsreferanser
\usepackage{mathtools}
\mathtoolsset{showonlyrefs}

% index
\usepackage{imakeidx}
\makeindex[title=Indeks]

%Footnote:
\usepackage[bottom, hang, flushmargin]{footmisc}
\usepackage{perpage} 
\MakePerPage{footnote}
\addtolength{\footnotesep}{2mm}
\renewcommand{\thefootnote}{\arabic{footnote}}
\renewcommand\footnoterule{\rule{\linewidth}{0.4pt}}
\renewcommand{\thempfootnote}{\arabic{mpfootnote}}

%colors
\definecolor{c1}{cmyk}{0,0.5,1,0}
\definecolor{c2}{cmyk}{1,0.25,1,0}
\definecolor{n3}{cmyk}{1,0.,1,0}
\definecolor{neg}{cmyk}{1,0.,0.,0}

% Lister med bokstavar
\usepackage[inline]{enumitem}

\newcounter{rg}
\numberwithin{rg}{chapter}
\newcommand{\reg}[2][]{\begin{tcolorbox}[boxrule=0.3 mm,arc=0mm,colback=blue!3] {\refstepcounter{rg}\phantomsection \large \textbf{\therg \;#1} \vspace{5 pt}}\newline #2  \end{tcolorbox}\vspace{-5pt}}

\newcommand\alg[1]{\begin{align} #1 \end{align}}

\newcommand\eks[2][]{\begin{tcolorbox}[boxrule=0.3 mm,arc=0mm,enhanced jigsaw,breakable,colback=green!3] {\large \textbf{Eksempel #1} \vspace{5 pt}\\} #2 \end{tcolorbox}\vspace{-5pt} }

\newcommand{\st}[1]{\begin{tcolorbox}[boxrule=0.0 mm,arc=0mm,enhanced jigsaw,breakable,colback=yellow!12]{ #1} \end{tcolorbox}}

\newcommand{\spr}[1]{\begin{tcolorbox}[boxrule=0.3 mm,arc=0mm,enhanced jigsaw,breakable,colback=yellow!7] {\large \textbf{Språkboksen} \vspace{5 pt}\\} #1 \end{tcolorbox}\vspace{-5pt} }

\newcommand{\sym}[1]{\colorbox{blue!15}{#1}}

\newcommand{\info}[2]{\begin{tcolorbox}[boxrule=0.3 mm,arc=0mm,enhanced jigsaw,breakable,colback=cyan!6] {\large \textbf{#1} \vspace{5 pt}\\} #2 \end{tcolorbox}\vspace{-5pt} }

\newcommand\algv[1]{\vspace{-11 pt}\begin{align*} #1 \end{align*}}

\newcommand{\regv}{\vspace{5pt}}
\newcommand{\mer}{\textsl{Merk}: }
\newcommand{\mers}[1]{{\footnotesize \mer #1}}
\newcommand\vsk{\vspace{11pt}}
\newcommand\vs{\vspace{-11pt}}
\newcommand\vsb{\vspace{-16pt}}
\newcommand\sv{\vsk \textbf{Svar} \vspace{4 pt}\\}
\newcommand\br{\\[5 pt]}
\newcommand{\figp}[1]{../fig/#1}
\newcommand\algvv[1]{\vs\vs\begin{align*} #1 \end{align*}}
\newcommand{\y}[1]{$ {#1} $}
\newcommand{\os}{\\[5 pt]}
\newcommand{\prbxl}[2]{
\parbox[l][][l]{#1\linewidth}{#2
	}}
\newcommand{\prbxr}[2]{\parbox[r][][l]{#1\linewidth}{
		\setlength{\abovedisplayskip}{5pt}
		\setlength{\belowdisplayskip}{5pt}	
		\setlength{\abovedisplayshortskip}{0pt}
		\setlength{\belowdisplayshortskip}{0pt} 
		\begin{shaded}
			\footnotesize	#2 \end{shaded}}}

\renewcommand{\cfttoctitlefont}{\Large\bfseries}
\setlength{\cftaftertoctitleskip}{0 pt}
\setlength{\cftbeforetoctitleskip}{0 pt}

\newcommand{\bs}{\\[3pt]}
\newcommand{\vn}{\\[6pt]}
\newcommand{\fig}[1]{\begin{figure}
		\centering
		\includegraphics[]{\figp{#1}}
\end{figure}}

\newcommand{\figc}[2]{\begin{figure}
		\centering
		\includegraphics[]{\figp{#1}}
		\caption{#2}
\end{figure}}

\newcommand{\sectionbreak}{\clearpage} % New page on each section

\newcommand{\nn}[1]{
\begin{equation}
	#1
\end{equation}
}

% Equation comments
\newcommand{\cm}[1]{\llap{\color{blue} #1}}

\newcommand\fork[2]{\begin{tcolorbox}[boxrule=0.3 mm,arc=0mm,enhanced jigsaw,breakable,colback=yellow!7] {\large \textbf{#1 (forklaring)} \vspace{5 pt}\\} #2 \end{tcolorbox}\vspace{-5pt} }
 
%colors
\newcommand{\colr}[1]{{\color{red} #1}}
\newcommand{\colb}[1]{{\color{blue} #1}}
\newcommand{\colo}[1]{{\color{orange} #1}}
\newcommand{\colc}[1]{{\color{cyan} #1}}
\definecolor{projectgreen}{cmyk}{100,0,100,0}
\newcommand{\colg}[1]{{\color{projectgreen} #1}}

% Methods
\newcommand{\metode}[2]{
	\textsl{#1} \\[-8pt]
	\rule{#2}{0.75pt}
}

%Opg
\newcommand{\abc}[1]{
	\begin{enumerate}[label=\alph*),leftmargin=18pt]
		#1
	\end{enumerate}
}
\newcommand{\abcs}[2]{
	\begin{enumerate}[label=\alph*),start=#1,leftmargin=18pt]
		#2
	\end{enumerate}
}
\newcommand{\abcn}[1]{
	\begin{enumerate}[label=\arabic*),leftmargin=18pt]
		#1
	\end{enumerate}
}
\newcommand{\abch}[1]{
	\hspace{-2pt}	\begin{enumerate*}[label=\alph*), itemjoin=\hspace{1cm}]
		#1
	\end{enumerate*}
}
\newcommand{\abchs}[2]{
	\hspace{-2pt}	\begin{enumerate*}[label=\alph*), itemjoin=\hspace{1cm}, start=#1]
		#2
	\end{enumerate*}
}

% Oppgaver
\newcommand{\opgt}{\phantomsection \addcontentsline{toc}{section}{Oppgaver} \section*{Oppgaver for kapittel \thechapter}\vs \setcounter{section}{1}}
\newcounter{opg}
\numberwithin{opg}{section}
\newcommand{\op}[1]{\vspace{15pt} \refstepcounter{opg}\large \textbf{\color{blue}\theopg} \vspace{2 pt} \label{#1} \\}
\newcommand{\ekspop}[1]{\vsk\textbf{Gruble \thechapter.#1}\vspace{2 pt} \\}
\newcommand{\nes}{\stepcounter{section}
	\setcounter{opg}{0}}
\newcommand{\opr}[1]{\vspace{3pt}\textbf{\ref{#1}}}
\newcommand{\oeks}[1]{\begin{tcolorbox}[boxrule=0.3 mm,arc=0mm,colback=white]
		\textit{Eksempel: } #1	  
\end{tcolorbox}}
\newcommand\opgeks[2][]{\begin{tcolorbox}[boxrule=0.1 mm,arc=0mm,enhanced jigsaw,breakable,colback=white] {\footnotesize \textbf{Eksempel #1} \\} \footnotesize #2 \end{tcolorbox}\vspace{-5pt} }
\newcommand{\rknut}{
Rekn ut.
}

%License
\newcommand{\lic}{\textit{Matematikken sine byggesteinar by Sindre Sogge Heggen is licensed under CC BY-NC-SA 4.0. To view a copy of this license, visit\\ 
		\net{http://creativecommons.org/licenses/by-nc-sa/4.0/}{http://creativecommons.org/licenses/by-nc-sa/4.0/}}}

%referances
\newcommand{\net}[2]{{\color{blue}\href{#1}{#2}}}
\newcommand{\hrs}[2]{\hyperref[#1]{\color{blue}\textsl{#2 \ref*{#1}}}}
\newcommand{\rref}[1]{\hrs{#1}{regel}}
\newcommand{\refkap}[1]{\hrs{#1}{kapittel}}
\newcommand{\refsec}[1]{\hrs{#1}{seksjon}}

\newcommand{\mb}{\net{https://sindrsh.github.io/FirstPrinciplesOfMath/}{MB}}


%line to seperate examples
\newcommand{\linje}{\rule{\linewidth}{1pt} }

\usepackage{datetime2}
%%\usepackage{sansmathfonts} for dyslexia-friendly math
\usepackage[]{hyperref}


\newcommand{\note}{Merk}
\newcommand{\notesm}[1]{{\footnotesize \textsl{\note:} #1}}
\newcommand{\ekstitle}{Eksempel }
\newcommand{\sprtitle}{Språkboksen}
\newcommand{\expl}{forklaring}

\newcommand{\vedlegg}[1]{\refstepcounter{vedl}\section*{Vedlegg \thevedl: #1}  \setcounter{vedleq}{0}}

\newcommand\sv{\vsk \textbf{Svar} \vspace{4 pt}\\}

%references
\newcommand{\reftab}[1]{\hrs{#1}{tabell}}
\newcommand{\rref}[1]{\hrs{#1}{regel}}
\newcommand{\dref}[1]{\hrs{#1}{definisjon}}
\newcommand{\refkap}[1]{\hrs{#1}{kapittel}}
\newcommand{\refsec}[1]{\hrs{#1}{seksjon}}
\newcommand{\refdsec}[1]{\hrs{#1}{delseksjon}}
\newcommand{\refved}[1]{\hrs{#1}{vedlegg}}
\newcommand{\eksref}[1]{\textsl{#1}}
\newcommand\fref[2][]{\hyperref[#2]{\textsl{figur \ref*{#2}#1}}}
\newcommand{\refop}[1]{{\color{blue}Oppgave \ref{#1}}}
\newcommand{\refops}[1]{{\color{blue}oppgave \ref{#1}}}
\newcommand{\refgrubs}[1]{{\color{blue}gruble \ref{#1}}}

\newcommand{\openmathser}{\openmath\,-\,serien}

% Exercises
\newcommand{\opgt}{\newpage \phantomsection \addcontentsline{toc}{section}{Oppgaver} \section*{Oppgaver for kapittel \thechapter}\vs \setcounter{section}{1}}


% Sequences and series
\newcommand{\sumarrek}{Summen av en aritmetisk rekke}
\newcommand{\sumgerek}{Summen av en geometrisk rekke}
\newcommand{\regnregsum}{Regneregler for summetegnet}

% Trigonometry
\newcommand{\sincoskomb}{Sinus og cosinus kombinert}
\newcommand{\cosfunk}{Cosinusfunksjonen}
\newcommand{\trid}{Trigonometriske identiteter}
\newcommand{\deravtri}{Den deriverte av de trigonometriske funksjonene}
% Solutions manual
\newcommand{\selos}{Se løsningsforslag.}
\newcommand{\se}[1]{Se eksempel på side \pageref{#1}}

%Vectors
\newcommand{\parvek}{Parallelle vektorer}
\newcommand{\vekpro}{Vektorproduktet}
\newcommand{\vekproarvol}{Vektorproduktet som areal og volum}


% 3D geometries
\newcommand{\linrom}{Linje i rommet}
\newcommand{\avstplnpkt}{Avstand mellom punkt og plan}


% Integral
\newcommand{\bestminten}{Bestemt integral I}
\newcommand{\anfundteo}{Analysens fundamentalteorem}
\newcommand{\intuf}{Integralet av utvalge funksjoner}
\newcommand{\bytvar}{Bytte av variabel}
\newcommand{\intvol}{Integral som volum}
\newcommand{\andordlindif}{Andre ordens lineære differensialligninger}



\begin{document}	
\vedlegg{Eulers tall} \label{eulerstallfork}
\subsubsection{Den deriverte som motivajon}
Gitt funksjonen $ f(x)=a^x $. Da har vi at
\alg{
\left(a^x\right)'&=\lim\limits_{h\to 0}\frac{a^{x+h}-a^x}{h}\br
&= \lim\limits_{h\to 0}\frac{a^{x}a^h-a^x}{h}
}
Da $ x $ er uavhengig av $ h $, får vi at
\alg{
\left(a^x\right)'=a^x\lim\limits_{h\to 0}\frac{a^h-1}{h}
}
Likningen over peker mot noe fantastisk; hvis det finnes et tall $ a $ som er slik at $ {\lim\limits_{h\to 0}\frac{a^h-1}{h}=1} $, så vil funksjonen $ a^x $ være sin egen deriverte funksjon! Altså er da $ \left(a^x\right)'=a^x $. Vi legger nå merke til at hvis
\[ a=\lim\limits_{h\to 0}\left(1+h\right)^\frac{1}{h} \]
så er 
\algv{
\lim\limits_{h\to 0}\frac{a^h-1}{h}&=\lim\limits_{h\to 0}\frac{\left(\left(1+h\right)^\frac{1}{h}\right)^h-1}{h}\br
&=\frac{1+h-1}{h}\\
&=1
}
Hvis vi kan vise at grenseverdien $ \lim\limits_{h\to 0}\left(1+h\right)^\frac{1}{h} $ eksisterer, har vi altså funnet akkurat det uttrykket for $ a $ som vi ønsker oss.

\subsubsection{Undersøking av grenseverdien}
Vi innfører følgende to funksjoner (motivasjonen for å innføre $ g $ vil komme fram senere):
\[ f(h)=1+h \qquad,\qquad g(h)=2-\left(\frac{1}{4}\right)^{h}\]
Videre undersøker vi for hvilke verdier $ f $ er mindre enn  $ g $. Når $ f=g $, har vi at
\begin{equation}\label{eforkleqh}
	1+h=2-\left(\frac{1}{4}\right)^h 
\end{equation}
Vi gjør nå følgende observasjon: Gitt to tall $ c $ og $ k $, og funksjonen $ p(h)=b^h $, hvor $ k>0 $ og $ 0<b<1 $. Da har vi at
\alg{
p(c+k)-p(c)=b^{c+k}-b^{c}=b^c\left(b^{k}-1\right)
}
Tilsvarende er
\alg{
p(c+2k)-p(c+k)&=b^{c+k}\left(b^{k}-1\right)
}
Videre er $ b^{c+k}<b^c $ og $ b^k-1<1 $, som betyr at
\[ \frac{p(c+k)-p(c)}{k}<\frac{p(c+2k)-p(c+k)}{k} \]
Dermed må linja mellom $ (c, p(c)) $ og $ (c+k, p(c+k)) $ være brattere enn linja mellom $ (c+k, p(c+k)) $ og $ (c+2k, p(c+2k)) $, og da må $ (c+k, p(c+k)) $ ligge under linja mellom $ (c, p(c)) $ og $ (c+2k, p(c+2k)) $.
\fig{ekspfunk3}
Da $ p(h) $ ikke er en lineær funksjon, må én av disse tre påstandene være gyldig:
\begin{itemize}
	\item $ p $ er konveks for alle $ h$
	\item $ p $ er konkav for alle $ h $
	\item $ p $ er skiftvis konkav/konveks
\end{itemize}
Men hvis $ p $ er konkav, må det finnes et intervall hvor $ (c+k, p(c+k)) $ ligger over linja mellom $ (c, p(c)) $ og $ (c+2k, p(c+2k)) $, og dette er selvmotsigende. Altså må $ p $ nødvendigvis være konveks for alle $ h $.\vsk

\newpage
Av det vi akkurat har funnet, kan vi konkludere med at funksjonen $2-\left(\frac{1}{4}\right)^h $ er konkav for alle $ h $, og da $ 1+h$ er et lineært uttrykk, har \eqref{eforkleqh} maksimalt to løsninger. Det er enkelt å vise at $ h=0 $ og $ h=\frac{1}{2} $ er løsningene til \eqref{eforkleqh}, og dette betyr at
\begin{equation}\label{1plushleq}
1+h\leq 2-\left(\frac{1}{4}\right)^h\qquad,\qquad x\in\left[0, \frac{1}{2}\right] 	
\end{equation}
\fig{ekspfunk2}
Vi setter $z=\frac{1}{h} $ for $ h\neq0 $. Da er
\[ \lim\limits_{h\to 0} (1+h)^h=\lim\limits_{z\to\infty}=\left(1+\frac{1}{z}\right)^\frac{1}{z} \]
Videre kan \eqref{1plushleq} omskrives til
\begin{equation}\label{eforkleqz}
	1+\frac{1}{z}\leq 2-\left(\frac{1}{4}\right)^\frac{1}{z}\qquad,\qquad z\in[2, \infty]
\end{equation}

For $ z\to\infty $ kan vi derfor være sikre på at
\[ 1+\frac{1}{z}<1+1-\left(\frac{1}{4}\right)^\frac{1}{z}+\left(1-\left(\frac{1}{4}\right)^\frac{1}{z}\right)^2+\left(1-\left(\frac{1}{4}\right)^\frac{1}{z}\right)^3+... \]
Høgresiden i ulikheten over kjenner vi igjen\footnote{Se om geometriske rekker i \tmto.} som en uendelig geometrisk rekke hvor summen er gitt som
\[ \frac{1}{1-\left(1-\left(\frac{1}{4}\right)^\frac{1}{z}\right)} =\frac{1}{\left(\frac{1}{4}\right)^\frac{1}{z}}=4^\frac{1}{z} \]
Altså er
\begin{equation}\label{eforkllim4}
	\lim\limits_{z\to\infty}\left(1+\frac{1}{z}\right)^z\leq \lim\limits_{z\to\infty}\left(4^\frac{1}{z}\right)^z=4
\end{equation}
\newpage
Videre er det enkelt å vise at ligningen
\[ 1+h=2-\left(\frac{1}{2}\right)^h \]
har løsningene $ h=-1 $ og $ h=0 $, som betyr at
\[ 1+h\geq2-\left(\frac{1}{2}\right)^h \qquad,\qquad h\in[0, \infty]\]
På tilsvarende måte som vi kom fram til en øvre grense, kan vi bruke dette til å slå fast at
\[ \lim\limits_{z\to \infty}\left(1 + \frac{1}{z}\right)^z\geq 2 \] 
Dermed vet vi at $ \lim\limits_{z\to\infty }\left(1+\frac{1}{z}\right)^n $ ligger et sted mellom 2 og 4. Da\\ uttrykket inneholder utelukkende positive ledd for $ {z\to\infty} $, kan vi også være sikre på at grenseverdien er endelig\footnote{I motsetning til å være ubestemt. For eksempel vil $ \lim\limits_{x\to \infty} \cos x $  være ubestemt, fordi $ \cos x $ svinger mellom $ -1 $ og $ 1 $.}. Det gir derfor mening å behandle grenseverdien som et tall, som vi kaller for $ e $:
\[ e=\lim\limits_{z\to\infty }\left(1+\frac{1}{z}\right)^z=\lim\limits_{h\to0}\left(1+h\right)^{\frac{1}{h}}  \]
\info{Merk}{
Den mest klassiske metoden for å finne en øvre og en nedre grense for $ \lim\limits_{z\to\infty }\left(1+\frac{1}{z}\right)^n $ er ved å bruke \net{https://en.wikipedia.org/wiki/Binomial\_theorem\#General\_case}{Binomialteoremet}. 
}
\subsubsection{Et tilbakeblikk på den deriverte}
Derivasjon av potensfunksjoner var det som motiverte oss til å undersøke tallet $ e $. Av det vi har drøftet i de foregående avsnittene, følger det at
\[ \left(e^x\right)'=e^x \]
Likningen over er rett og slett én av de viktigste likningene i\\
matematikk.
\end{document}