\documentclass[english,a4paper,hidelinks,pdftex, 11 pt, class=report,crop=false]{standalone}
\usepackage[T1]{fontenc}
\usepackage[utf8]{luainputenc}
\usepackage{geometry}
\setlength{\parindent}{0bp}
\usepackage{import}
\usepackage[subpreambles=false]{standalone}
\usepackage{amsmath}
\usepackage{amssymb}
\usepackage{esint}
\usepackage{babel}
\usepackage{tabu}
\usepackage{lmodern}
\usepackage[dvipsnames]{xcolor}
\geometry{verbose, inner=2.3cm, outer=1.8 cm, bmargin=2cm, tmargin=1.8cm}
% Lister med bokstavar
\usepackage{enumitem}

\newcommand{\os}{\\[5pt]}
\newcommand{\vsk}{\\[12pt]}
\newcommand{\net}[2]{{\href{#1}{\color{blue}#2}}}

\usepackage{bm}

\usepackage{hyperref}


\usepackage[many]{tcolorbox}
\newcommand{\reg}[2][]{\begin{tcolorbox}[boxrule=0.3 mm,arc=0mm,colback=blue!3] {\Large \textbf{#1} \vspace{5 pt}}\newline #2  \end{tcolorbox}\vspace{-5pt}}

\newcommand\eks[2][]{\begin{tcolorbox}[boxrule=0.3 mm,arc=0mm,enhanced jigsaw,breakable,colback=green!3] {\Large \textbf{Eksempel #1} \vspace{5 pt}\\} #2 \end{tcolorbox}\vspace{-5pt} }

\newcommand{\asym}[1]{/home/sindre/G/fig/#1}
\newcommand{\fig}[1]{\begin{figure}
		\centering
		\includegraphics[]{\asym{#1}}
\end{figure}}

\newcommand{\ca}[1]{{\color{blue} #1}}
\newcommand{\cb}[1]{{\color{orange} #1}}
\newcommand{\cc}[1]{{\color{ForestGreen} #1}}
\newcommand{\cd}[1]{{\color{cyan} #1}}

\begin{document}
\huge \textbf{Lekseark matematikk måndag veke 12}\\
\footnotesize OBS! Hugs å skrive namn på arket \\[25pt]
\large

{\Large \textbf{Oppgåve 1}}\\[10pt]
Rekn ut. \\[10pt]
\fbox{\textit{Eksempel:} $ \displaystyle
\frac{2}{9}+\frac{3}{9}=\frac{5}{9} $ \\[10pt]}
\vspace{12pt}
\begin{enumerate}[label=\alph*)]
	\item $\displaystyle \dfrac{2}{3}+\dfrac{5}{3} $\\[10pt]
	\item $\displaystyle \dfrac{8}{7}+\frac{2}{7} $\\[10pt]
	\item $\displaystyle \dfrac{20}{12}-\frac{4}{12} $\\[10pt]
	\item $\displaystyle \dfrac{15}{17}-\frac{3}{17}$\\[10pt]	
\end{enumerate} \vspace{20pt}

{\Large \textbf{Oppgåve 2}}\\[10pt]
Rekn ut. \\[10pt]
\fbox{\textit{Eksempel:} $ \displaystyle
	{\color{red}4}\cdot\frac{\color{blue}3}{7}=\frac{{\color{red}4}\cdot{\color{blue}3}}{7}=\frac{12}{7} $ \\[10pt]}
\vspace{12pt}
\begin{enumerate}[label=\alph*)]
	\item $\displaystyle 7\cdot\dfrac{2}{3} $\\[10pt]
	\item $\displaystyle 5\cdot\dfrac{8}{7}$\\[10pt]
	\item $\displaystyle \dfrac{20}{12}\cdot9 $\\[10pt]
	\item $\displaystyle \dfrac{15}{17}\cdot4$\\[10pt]	
\end{enumerate}
\newpage

\reg[Prosent]{Brøkar med 100 i nemnar kan vi skrive med symbolet $ \% $. Symbolet uttalar vi som ''prosent''.}
\eks[1]{ \vspace{-11pt}
\[ \frac{15}{100}=15\% \]
}
\eks[2]{ \vspace{-11pt}
	\[ \frac{83}{100}=83\% \]
}
\vspace{20pt}
{\Large \textbf{Oppgåve 3}}\\[10pt]
Skriv som prosent.
\begin{enumerate}[label=\alph*)]
	\item $\displaystyle \frac{12}{100} $\\[10pt]
		\item $\displaystyle \frac{49}{100} $\\[10pt]
		\item $\displaystyle \frac{97}{100} $\\[10pt]
	\item $\displaystyle \frac{205}{100} $\\[10pt]
\end{enumerate}
\vspace{20pt}
{\Large \textbf{Ekstraoppgåve}}\\[10pt]
Skriv som prosent.
\begin{enumerate}[label=\alph*)]
	\item $\displaystyle \frac{7}{10} $\\[10pt]
	\item $\displaystyle \frac{8}{20} $\\[10pt]
	\item $\displaystyle \frac{9}{25} $\\[10pt]
	\item $\displaystyle \frac{6}{5} $\\[10pt]
	\item $\displaystyle \frac{70}{50} $\\[10pt]
\end{enumerate}

\end{document}

