\documentclass[english, 11 pt, class=article, crop=false]{standalone}
\usepackage[T1]{fontenc}
%\renewcommand*\familydefault{\sfdefault} % For dyslexia-friendly text
\usepackage{lmodern} % load a font with all the characters
\usepackage{geometry}
\geometry{verbose,paperwidth=16.1 cm, paperheight=24 cm, inner=2.3cm, outer=1.8 cm, bmargin=2cm, tmargin=1.8cm}
\setlength{\parindent}{0bp}
\usepackage{import}
\usepackage[subpreambles=false]{standalone}
\usepackage{amsmath}
\usepackage{amssymb}
\usepackage{esint}
\usepackage{babel}
\usepackage{tabu}
\makeatother
\makeatletter

\usepackage{titlesec}
\usepackage{ragged2e}
\RaggedRight
\raggedbottom
\frenchspacing

% Norwegian names of figures, chapters, parts and content
\addto\captionsenglish{\renewcommand{\figurename}{Figur}}
\makeatletter
\addto\captionsenglish{\renewcommand{\chaptername}{Kapittel}}
\addto\captionsenglish{\renewcommand{\partname}{Del}}


\usepackage{graphicx}
\usepackage{float}
\usepackage{subfig}
\usepackage{placeins}
\usepackage{cancel}
\usepackage{framed}
\usepackage{wrapfig}
\usepackage[subfigure]{tocloft}
\usepackage[font=footnotesize,labelfont=sl]{caption} % Figure caption
\usepackage{bm}
\usepackage[dvipsnames, table]{xcolor}
\definecolor{shadecolor}{rgb}{0.105469, 0.613281, 1}
\colorlet{shadecolor}{Emerald!15} 
\usepackage{icomma}
\makeatother
\usepackage[many]{tcolorbox}
\usepackage{multicol}
\usepackage{stackengine}

\usepackage{esvect} %For vectors with capital letters

% For tabular
\usepackage{array}
\usepackage{multirow}
\usepackage{longtable} %breakable table

% Ligningsreferanser
\usepackage{mathtools}
\mathtoolsset{showonlyrefs}

% index
\usepackage{imakeidx}
\makeindex[title=Indeks]

%Footnote:
\usepackage[bottom, hang, flushmargin]{footmisc}
\usepackage{perpage} 
\MakePerPage{footnote}
\addtolength{\footnotesep}{2mm}
\renewcommand{\thefootnote}{\arabic{footnote}}
\renewcommand\footnoterule{\rule{\linewidth}{0.4pt}}
\renewcommand{\thempfootnote}{\arabic{mpfootnote}}

%colors
\definecolor{c1}{cmyk}{0,0.5,1,0}
\definecolor{c2}{cmyk}{1,0.25,1,0}
\definecolor{n3}{cmyk}{1,0.,1,0}
\definecolor{neg}{cmyk}{1,0.,0.,0}

% Lister med bokstavar
\usepackage[inline]{enumitem}

\newcounter{rg}
\numberwithin{rg}{chapter}
\newcommand{\reg}[2][]{\begin{tcolorbox}[boxrule=0.3 mm,arc=0mm,colback=blue!3] {\refstepcounter{rg}\phantomsection \large \textbf{\therg \;#1} \vspace{5 pt}}\newline #2  \end{tcolorbox}\vspace{-5pt}}

\newcommand\alg[1]{\begin{align} #1 \end{align}}

\newcommand\eks[2][]{\begin{tcolorbox}[boxrule=0.3 mm,arc=0mm,enhanced jigsaw,breakable,colback=green!3] {\large \textbf{Eksempel #1} \vspace{5 pt}\\} #2 \end{tcolorbox}\vspace{-5pt} }

\newcommand{\st}[1]{\begin{tcolorbox}[boxrule=0.0 mm,arc=0mm,enhanced jigsaw,breakable,colback=yellow!12]{ #1} \end{tcolorbox}}

\newcommand{\spr}[1]{\begin{tcolorbox}[boxrule=0.3 mm,arc=0mm,enhanced jigsaw,breakable,colback=yellow!7] {\large \textbf{Språkboksen} \vspace{5 pt}\\} #1 \end{tcolorbox}\vspace{-5pt} }

\newcommand{\sym}[1]{\colorbox{blue!15}{#1}}

\newcommand{\info}[2]{\begin{tcolorbox}[boxrule=0.3 mm,arc=0mm,enhanced jigsaw,breakable,colback=cyan!6] {\large \textbf{#1} \vspace{5 pt}\\} #2 \end{tcolorbox}\vspace{-5pt} }

\newcommand\algv[1]{\vspace{-11 pt}\begin{align*} #1 \end{align*}}

\newcommand{\regv}{\vspace{5pt}}
\newcommand{\mer}{\textsl{Merk}: }
\newcommand{\mers}[1]{{\footnotesize \mer #1}}
\newcommand\vsk{\vspace{11pt}}
\newcommand\vs{\vspace{-11pt}}
\newcommand\vsb{\vspace{-16pt}}
\newcommand\sv{\vsk \textbf{Svar} \vspace{4 pt}\\}
\newcommand\br{\\[5 pt]}
\newcommand{\figp}[1]{../fig/#1}
\newcommand\algvv[1]{\vs\vs\begin{align*} #1 \end{align*}}
\newcommand{\y}[1]{$ {#1} $}
\newcommand{\os}{\\[5 pt]}
\newcommand{\prbxl}[2]{
\parbox[l][][l]{#1\linewidth}{#2
	}}
\newcommand{\prbxr}[2]{\parbox[r][][l]{#1\linewidth}{
		\setlength{\abovedisplayskip}{5pt}
		\setlength{\belowdisplayskip}{5pt}	
		\setlength{\abovedisplayshortskip}{0pt}
		\setlength{\belowdisplayshortskip}{0pt} 
		\begin{shaded}
			\footnotesize	#2 \end{shaded}}}

\renewcommand{\cfttoctitlefont}{\Large\bfseries}
\setlength{\cftaftertoctitleskip}{0 pt}
\setlength{\cftbeforetoctitleskip}{0 pt}

\newcommand{\bs}{\\[3pt]}
\newcommand{\vn}{\\[6pt]}
\newcommand{\fig}[1]{\begin{figure}
		\centering
		\includegraphics[]{\figp{#1}}
\end{figure}}

\newcommand{\figc}[2]{\begin{figure}
		\centering
		\includegraphics[]{\figp{#1}}
		\caption{#2}
\end{figure}}

\newcommand{\sectionbreak}{\clearpage} % New page on each section

\newcommand{\nn}[1]{
\begin{equation}
	#1
\end{equation}
}

% Equation comments
\newcommand{\cm}[1]{\llap{\color{blue} #1}}

\newcommand\fork[2]{\begin{tcolorbox}[boxrule=0.3 mm,arc=0mm,enhanced jigsaw,breakable,colback=yellow!7] {\large \textbf{#1 (forklaring)} \vspace{5 pt}\\} #2 \end{tcolorbox}\vspace{-5pt} }
 
%colors
\newcommand{\colr}[1]{{\color{red} #1}}
\newcommand{\colb}[1]{{\color{blue} #1}}
\newcommand{\colo}[1]{{\color{orange} #1}}
\newcommand{\colc}[1]{{\color{cyan} #1}}
\definecolor{projectgreen}{cmyk}{100,0,100,0}
\newcommand{\colg}[1]{{\color{projectgreen} #1}}

% Methods
\newcommand{\metode}[2]{
	\textsl{#1} \\[-8pt]
	\rule{#2}{0.75pt}
}

%Opg
\newcommand{\abc}[1]{
	\begin{enumerate}[label=\alph*),leftmargin=18pt]
		#1
	\end{enumerate}
}
\newcommand{\abcs}[2]{
	\begin{enumerate}[label=\alph*),start=#1,leftmargin=18pt]
		#2
	\end{enumerate}
}
\newcommand{\abcn}[1]{
	\begin{enumerate}[label=\arabic*),leftmargin=18pt]
		#1
	\end{enumerate}
}
\newcommand{\abch}[1]{
	\hspace{-2pt}	\begin{enumerate*}[label=\alph*), itemjoin=\hspace{1cm}]
		#1
	\end{enumerate*}
}
\newcommand{\abchs}[2]{
	\hspace{-2pt}	\begin{enumerate*}[label=\alph*), itemjoin=\hspace{1cm}, start=#1]
		#2
	\end{enumerate*}
}

% Oppgaver
\newcommand{\opgt}{\phantomsection \addcontentsline{toc}{section}{Oppgaver} \section*{Oppgaver for kapittel \thechapter}\vs \setcounter{section}{1}}
\newcounter{opg}
\numberwithin{opg}{section}
\newcommand{\op}[1]{\vspace{15pt} \refstepcounter{opg}\large \textbf{\color{blue}\theopg} \vspace{2 pt} \label{#1} \\}
\newcommand{\ekspop}[1]{\vsk\textbf{Gruble \thechapter.#1}\vspace{2 pt} \\}
\newcommand{\nes}{\stepcounter{section}
	\setcounter{opg}{0}}
\newcommand{\opr}[1]{\vspace{3pt}\textbf{\ref{#1}}}
\newcommand{\oeks}[1]{\begin{tcolorbox}[boxrule=0.3 mm,arc=0mm,colback=white]
		\textit{Eksempel: } #1	  
\end{tcolorbox}}
\newcommand\opgeks[2][]{\begin{tcolorbox}[boxrule=0.1 mm,arc=0mm,enhanced jigsaw,breakable,colback=white] {\footnotesize \textbf{Eksempel #1} \\} \footnotesize #2 \end{tcolorbox}\vspace{-5pt} }
\newcommand{\rknut}{
Rekn ut.
}

%License
\newcommand{\lic}{\textit{Matematikken sine byggesteinar by Sindre Sogge Heggen is licensed under CC BY-NC-SA 4.0. To view a copy of this license, visit\\ 
		\net{http://creativecommons.org/licenses/by-nc-sa/4.0/}{http://creativecommons.org/licenses/by-nc-sa/4.0/}}}

%referances
\newcommand{\net}[2]{{\color{blue}\href{#1}{#2}}}
\newcommand{\hrs}[2]{\hyperref[#1]{\color{blue}\textsl{#2 \ref*{#1}}}}
\newcommand{\rref}[1]{\hrs{#1}{regel}}
\newcommand{\refkap}[1]{\hrs{#1}{kapittel}}
\newcommand{\refsec}[1]{\hrs{#1}{seksjon}}

\newcommand{\mb}{\net{https://sindrsh.github.io/FirstPrinciplesOfMath/}{MB}}


%line to seperate examples
\newcommand{\linje}{\rule{\linewidth}{1pt} }

\usepackage{datetime2}
%%\usepackage{sansmathfonts} for dyslexia-friendly math
\usepackage[]{hyperref}


\newcommand{\note}{Merk}
\newcommand{\notesm}[1]{{\footnotesize \textsl{\note:} #1}}
\newcommand{\ekstitle}{Eksempel }
\newcommand{\sprtitle}{Språkboksen}
\newcommand{\expl}{forklaring}

\newcommand{\vedlegg}[1]{\refstepcounter{vedl}\section*{Vedlegg \thevedl: #1}  \setcounter{vedleq}{0}}

\newcommand\sv{\vsk \textbf{Svar} \vspace{4 pt}\\}

%references
\newcommand{\reftab}[1]{\hrs{#1}{tabell}}
\newcommand{\rref}[1]{\hrs{#1}{regel}}
\newcommand{\dref}[1]{\hrs{#1}{definisjon}}
\newcommand{\refkap}[1]{\hrs{#1}{kapittel}}
\newcommand{\refsec}[1]{\hrs{#1}{seksjon}}
\newcommand{\refdsec}[1]{\hrs{#1}{delseksjon}}
\newcommand{\refved}[1]{\hrs{#1}{vedlegg}}
\newcommand{\eksref}[1]{\textsl{#1}}
\newcommand\fref[2][]{\hyperref[#2]{\textsl{figur \ref*{#2}#1}}}
\newcommand{\refop}[1]{{\color{blue}Oppgave \ref{#1}}}
\newcommand{\refops}[1]{{\color{blue}oppgave \ref{#1}}}
\newcommand{\refgrubs}[1]{{\color{blue}gruble \ref{#1}}}

\newcommand{\openmathser}{\openmath\,-\,serien}

% Exercises
\newcommand{\opgt}{\newpage \phantomsection \addcontentsline{toc}{section}{Oppgaver} \section*{Oppgaver for kapittel \thechapter}\vs \setcounter{section}{1}}


% Sequences and series
\newcommand{\sumarrek}{Summen av en aritmetisk rekke}
\newcommand{\sumgerek}{Summen av en geometrisk rekke}
\newcommand{\regnregsum}{Regneregler for summetegnet}

% Trigonometry
\newcommand{\sincoskomb}{Sinus og cosinus kombinert}
\newcommand{\cosfunk}{Cosinusfunksjonen}
\newcommand{\trid}{Trigonometriske identiteter}
\newcommand{\deravtri}{Den deriverte av de trigonometriske funksjonene}
% Solutions manual
\newcommand{\selos}{Se løsningsforslag.}
\newcommand{\se}[1]{Se eksempel på side \pageref{#1}}

%Vectors
\newcommand{\parvek}{Parallelle vektorer}
\newcommand{\vekpro}{Vektorproduktet}
\newcommand{\vekproarvol}{Vektorproduktet som areal og volum}


% 3D geometries
\newcommand{\linrom}{Linje i rommet}
\newcommand{\avstplnpkt}{Avstand mellom punkt og plan}


% Integral
\newcommand{\bestminten}{Bestemt integral I}
\newcommand{\anfundteo}{Analysens fundamentalteorem}
\newcommand{\intuf}{Integralet av utvalge funksjoner}
\newcommand{\bytvar}{Bytte av variabel}
\newcommand{\intvol}{Integral som volum}
\newcommand{\andordlindif}{Andre ordens lineære differensialligninger}



%\usepackage{xr}
%\externaldocument{../MB_bm}
\begin{document}

\footnotesize

\subsection*{\faskap \ref{Talavare}}
\opr{opgtelverdi}
\abch{
\item 22 
\item 13
\item 36
}

\opr{opgtelverditl}
\abch{
\item 17
\item 29
\item 11
}

\opr{opgtelverdidestl}
\abch{
\item 1,7
\item 2,3
}

\opr{opgtelverdides}
\abch{
\item 13,9
\item 32,8
\item 0,7
\item 2,4
}

\opr{opgeininndelta} a)

\opr{opgeininndeltc} c)

\opr{opgeininndeltb} b)

\opr{opg100inndelta} a)

\opr{opgeininndeltb} c)

\subsection*{\faskap \ref{Rekneartane}}
\opr{opgsumtotall}
\mulansw
\abch{
\item $ 4=1+3 $
\item $ 5=2+3 $
\item $ 6=2+4 $
\item $ 7=1+6 $
\item $ 8=3+5 $
\item $ 9=2+7 $
}

\opr{opgsumtretall}
\abch{
	\item $ 5 =2+2+1 $ 
	\item $ 6 = 1+3+2 $
	\item $ 7 = 3+2+2 $
	\item $ 8 = 1+1+6 $
	\item $ 9 = 3+3+3 $
	\item $ 2+5+3 $
}

\opr{opgtiervenn} \vspace{-5pt}

\abcn{
\item
	\abch{
		\item 8
		\item 7 
		\item 6
		\item 5
	}
\item  Fordi de er svarene i oppgave 1a), 1b) og 1c). 
}

\opr{opgsumparodd}
\abch{
\item 1)
\item 1)
\item 2)
}

\opr{opgdifto} \mulansw
\abch{
\item $ 2=7-5 $
\item $ 3=10-7 $
\item $ 4=5-1 $
\item $ 5=10-5 $
\item $ 6 = 9-3 $
\item $ 7 = 9-2 $
\item $ 8=10-2 $
}

\opr{opgdifparodd}
\abch{
	\item 1)
	\item 1)
	\item 2)
}

\opr{opgadsomgang}
\abc{
	\item $ 2+2+2=2\cdot 4 = 4+4 $
	\item $ 3+3+3+3+3+3= 3\cdot 6=6+6+6 $
	\item $ 4+4=\cdot 4\cdot2 = 2+2+2+2 $ \\
	\item $ 5+5+5+5+5+5+5+5+5+5 = 5\cdot10=10+10+10+10+10 $
	\item $ 6+6+6+6=6\cdot5 = 5+5+5+5+5+5 $
	\item $ 7+7+7+7=7\cdot4=4+4+4+4+4+4+4 $
}

\opr{opggangrut}
\abch{
	\item $ 20 $ 
	\item $ 24 $
	\item $ 18 $
	\item $ 30 $
	\item $ 56$
}

\opr{opggangparod} 
\abch{
	\item partall
	\item partall, 0
	\item oddetall, 5
}

\subsection*{\faskap \ref{Faktogrek}}

\opr{opgfaktto} \mulansw
\abch{
	\item $ 5\cdot20 $
	\item $ 3\cdot10 $
	\item $ 2\cdot 20 $
	\item $ 2\cdot 35 $
} \os
\abchs{5}{
	\item $ 7\cdot 6 $
	\ \ \item $ 8\cdot 4 $
	\item $ 42\cdot2 $
	\item $ 3\cdot30 $
}

\opr{opgprimfakt}
\abch{
	\item $ 2\cdot 2\cdot2\cdot2\cdot 3 $ 
	\item $2\cdot 2\cdot3 \cdot5 $
	\item $ 2\cdot 5 $
} 

\opr{opgperf} 28

\grubr{opgoddgangodd}

\grubr{opgfindprime} \selos

\opr{opggangutv}
\abch{
\item 34
\item 177
\item 100
\item 664
\item 2943
}

\subsection*{\faskap \ref{Brok}}

\opr{opgbrverdidelelig}
\abch{
\item 6
\item 5
\item 2
\item 7
\item 9
\item 4
}

\opr{opgbrverdiikkedelelig}
\abch{
\item 0,5
\item 0,25
\item 0,2
\item 0,75
\item 0,4
\item 0,6
\item 0,8
\item 1,5
\item 0,33...
\item 2,5
\item 0,833...
\item 1,4
\item 2,75
\item 0,7
}

\opr{opgbrtallin}
\abch{
\item $ \frac{3}{4} $
\item $ \frac{2}{7} $
\item $ \frac{2}{5} $
}

\opr{opgbrtallin2}
\abch{
\item $ \frac{14}{3} $
\item $ \frac{13}{2} $
\item $ \frac{11}{5} $
}

\opr{opgbrutvid}
\abch{
\item $ \frac{20}{6} $
\item $ \frac{9}{12} $
\item $ \frac{12}{28} $
\item $ \frac{45}{40} $
\item $ \frac{54}{30} $
\item $ \frac{77}{28} $
}

\opr{opgbrutvidtil}
\abch{
\item $ \frac{9}{4} $
\item $ \frac{9}{5} $
\item $ \frac{2}{7} $
}

\opr{opgbrad}
\abch{
\item $ \frac{10}{3} $
\item $ \frac{14}{4} $
\item $ \frac{11}{6} $
\item $ \frac{10}{7} $
\item 1
} \\
\mers{Éin av brøkane kan forkortast.}

\opr{opgbrad2}
\abch{
\item $ \frac{22}{3} $
\item $ \frac{8}{5} $
\item $ \frac{17}{7} $
}

\opr{opgbrsub}
\abch{
	\item $ \frac{1}{3} $
	\item $ \frac{2}{4} $
	\item $ \frac{9}{6} $
	\item 1
	\item 0
} \\
\mers{To av brøkane kan forkortast.}

\opr{opgbradandsub}
\abch{
\item $ \frac{6}{5} $	
\item $ \frac{5}{7} $
\item 4
}

\opr{opgbrad3}
\abch{
\item $ \frac{9}{10} $
\item $ \frac{73}{63} $
\item $ \frac{101}{24} $
\item $ \frac{73}{20} $
\item $ \frac{5}{6} $
}

\opr{opgbrsub2}
\abch{
\item $ \frac{1}{10} $
\item $ \frac{29}{36} $
\item $ \frac{71}{72} $
\item $ \frac{11}{20} $
\item $ \frac{5}{6} $
}

\opr{opgbradandsubmix}
\abch{
\item $ \frac{5}{12} $
\item $ \frac{157}{30} $
\item $ \frac{229}{56} $
}

\opr{opgbrgongheil}
\abch{
\item $ \frac{20}{3} $
\item $ \frac{40}{7} $
\item $ \frac{54}{10} $
\item $ \frac{80}{7} $
\item $ \frac{21}{2} $
\item $ \frac{28}{3} $
\item $ \frac{35}{3} $
\item $ \frac{30}{7} $
\item $ \frac{5}{11} $
\item $ \frac{72}{17} $ 
}

\opr{opgbrdelheil}
\abch{
\item $ \frac{4}{15} $
\item $ \frac{5}{56} $
\item $ \frac{9}{60} $
\item $ \frac{8}{70} $
\item $ \frac{3}{14} $
\item $ \frac{9}{110} $
\item $ \frac{1}{60} $
\item $ \frac{9}{290} $
\item $ \frac{8}{459} $
\item $ \frac{4}{158} $
}

\opr{opgbrgongbr}
\abch{
\item $ \frac{20}{27} $
\item $ \frac{7}{32} $
\item $ \frac{18}{21} $
\item $ \frac{60}{5} $
\item $ \frac{21}{10} $
\item $ \frac{10}{21} $
\item $ \frac{16}{21} $
\item $ \frac{80}{9} $
\item $ \frac{36}{35} $
\item $ \frac{35}{12} $
}

\opr{opgbrgongbr2}
\abch{
\item $ \frac{15}{40} $
\item $ \frac{153}{32} $
\item $ \frac{46}{32} $
\item $ \frac{21}{648} $
\item $ \frac{203}{328} $
}

\opr{kansfakt}
\abch{
\item $ \frac{4}{11} $
\item $ \frac{35}{8} $
\item $ \frac{1}{9} $
\item 4
}

\opr{opgbrforkortmfaktor}
\abch{
\item $ \frac{7}{4} $
\item $ \frac{3}{7} $
\item $ \frac{2}{3} $
\item $ \frac{8}{7} $
\item $ \frac{1}{2} $
\item $ \frac{7}{2} $
} \vsk


\opr{opgbrgongbrfakt}
\abch{
\item 49
\item 54
\item 70
\item 16
\item 30
\item 12
\item 25
\item 14
\item 7
\item 63
}

\opr{opgbrdelbr}
\abch{
\item $ \frac{14}{15} $
\item $ \frac{24}{45} $
\item $ \frac{30}{21} $
\item $ \frac{7}{20} $
\item $ \frac{66}{15} $
}

\opr{opgbrdelbr2}
\abch{
	\item $ \frac{4}{9} $
	\item $ \frac{15}{4} $
	\item $ \frac{14}{5} $
} \os

\grubr{opgfracval} 
\abch{
\item 2
\item 4
\item 5
\item 2
\item 4
\item 5
\item 3, 4
\item 2, 5
\item 3, 5
\item 4, 5	
} \os

\grubr{opgfracval2}
\abch{
	\item 2
	\item 4
	\item 5
	\item 4, 3
	\item 5, 2
	\item 5, 3
	\item 5, 4
} \os

\grubr{opgfracprim}
\abch{
\item $ \frac{403}{6732} $
\item $ \frac{269}{3150} $
}

\subsection*{\faskap \ref{Negtal}}

\opr{opgnegfyllinn}
\abc{
	\item 9 har retning mot høgre og lengde 9.
	\item 4 har retning mot høgre og lengde 4
	\item -3 har retning mot venstre og lengde 3
	\item 12 har retning mot høgre og lengde 12
	\item -11 har retning mot venstre og lengde 11
	\item -25 har retning mot venstre og lengde 25
}	
\opr{opgnegplas}\\
\includegraphics{\figp{opgneg1fas}}


\opr{opgnegogpos}
\abch{
	\item 9, 4 og 12.
	\item $ -3 $, $ -11 $ og $ -25 $.
}


\opr{opgnegad}
\abch{
	\item $ 1 $
	\item $ 8 $
	\item $ 6 $
	\item $ 0 $
} \vsk

\abchs{5}{
	\item $ 3 $
	\ \ \ \item $ 1 $
	\ \ \ \item $ 10 $
	\item $ 5 $
}

\opr{opgnegad2}
\abch{
	\item $ -16 $
	\item $ -8 $
	\item $ -23 $
	\item $ 0 $
} \\[12pt]

\abchs{5}{
	\item $ -23 $
	\item $ -17 $
	\item $ -3 $
	\item $ 0 $
}

\opr{opgnegsub}
\abch{
	\item $ 15 $
	\item $ 17 $
	\item $ 12 $
	\item $ 14 $
} \vsk

\abchs{5}{
	\item $ -12 $
	\ \ \ \item $ -19 $
	\ \ \ \item $ -12 $
	\item $ -13 $
}

\opr{opgnegsub2}
\abch{
	\item $ 22 $
	\item $ 22 $
	\item $ -17 $
	\item $ 14 $
	\item $ 15 $
	\item $ 13 $
	\item $ -13 $
	\item $ -12 $
}

\opr{opgnegmult}
\abch{
	\item $ -12 $
	\item $ -50 $
	\item $ -63 $
	\item $ -24 $
	\item $ (-7)\cdot8 $
	\item $ (-3)\cdot9 $
	\item $ (-1)\cdot12 $
	\item $ (-10)\cdot4 $
	\item $ -21 $
	\item $ -25 $
	\item $ -12 $
	\item $ -72 $
}

\opr{opgnegdiv}
\abch{
	\item $ -4 $
	\item $ -6 $
	\item $ -5 $
	\item $ -4 $
	\item $ -8 $
	\item $ -9 $
	\item $ -5 $
	\item $ -5$
	\item $ 8 $
	\item $ 9$
	\item $ 5 $
}

\subsection*{\faskap \ref{Geometri}}
\opr{opgomkr1}
\abch{
\item 14
\item 20
\item 24
}

\opr{opgarfirk}
\abch{
\item 2
\item 16
\item 27
}

\opr{opgarfinn}
\abch{
\item 2 og 8
\item 3 og 4
\item 3 og 6
}

\opr{opgarstorst}
\abch{
\item 81
\item 1) Et åpenbart eksempel er kvadratet fra oppgave a), som har bredde og høgde 9, og areal 81. 2) Bredde 15 og høgde 3, areal 45. 3) Bredde 12 og høgde 6, areal 72. \mulansw
}

\opr{opgartrek}
\abch{
\item 3
\item 10
\item 6 
\item 6
\item 10
\item 6
\item 4
\item 3
\item 28
}

\opr{opggeovolprism}
\abch{
\item 90. \mers{Grunnflaten kan også være 72 eller 80, avhengig av hvilken side man velger ut som grunnflate.}
}

\opr{opgtr1} \selos 

\opr{opgsp1} \selos

\grubr{opgomkrogarparodd} \selos

\grubr{opgtrkkvadv} \selos

\grubr{opggeolikar} \selos

\subsection*{\faskap \ref{Algebra}}
\opr{opgalggjentad}
\abch{
	\item $ 3a $ \item $ 4a $ \item $ 7a $
	\item $ -2b $
	\item $ -5b $
	\item $ -3k $
}

\opr{opgalgadogsub}
\abch{
	\item $ a+b $
	\item $ a+2b $
	\item $ 9b-3a $
} \vsk

\opr{opgalgadogsub2}
\abch{
	\item $ 2b-5a+c $
	\item $ 3b-9a $
	\item $ 11b-3a $
}

\opr{opgalggong}
\abch{
	\item $ 7a+14 $
	\item $ 9b+27 $
	\item $ 8b-24c $
	\item $ -6a-10b $
	\item $ (9a+2) $
	\item $ (3b+8)a $
	\item $ 3ac-ab $
	\item $ 2a+6b+8c$
	\item $ 27b-9c+63a $
	\item $ 2c-6b-14a $
}

\opr{opgalgfaktoriser}
Bruk \rref{gangpar} til å faktorisere uttrykket.\os
\abch{
	\item $ 2(a+b) $
	\item $ b(4a+5) $
	\item $ c(9b-1) $
	\item $ 2a(2c-1) $
}

\opr{opgalgkvad}

\opr{GV21D1opg10}
\abch{
	\item $ \frac{1}{4} $
	\item $ \frac{1}{2} $ når $ x=4 $ og $ 2 $ når $ x=-2 $.
}

\opr{eksu22opg2}
Uttrykkene gitt av a) og c) stemmer.

\opr{opgpot}
\abch{
	\item $ 3^4 $
	\item $ 5^2$
	\item $ 7^6 $
	\item $ a^3 $
	\item $ b^2 $
	\item $ (-c)^4 $ \mers{$ (-c)^4=c^4 $}
}

\opr{opgpotverdi}
\abch{
	\item $ 64 $
	\item $ 32 $
	\item $ 64 $
	\item $ -8$
	\item $ -243 $
	\item $ 256 $
} 

\opr{opgpotskrivtilpot}
\abch{
	\item $ 2^{16} $
	\item $ 3^{11} $
	\item $ 9^6 $
	\item $ 6^5 $
	\item $ 5^{-4} $
	\item $ 10^{11} $
	\item $ a^{16} $
	\item $ k^7 $
	\item $ x^3 $
	\item $ x$
	\item $ 1 $
	\item $ a^5\cdot b^{-3} $
}

\opr{opgrotter}
\abch{
	\item $ 5 $
	\item $ 10 $
	\item $ 12 $
	\item $ 3 $
	\item $ 9 $
	\item $ 10 $
}

\grubr{1TH21D1opg6} $ 2^{-\frac{5}{4}} $ 
 
\subsection*{\faskap \ref{Likningar}}

\opr{opglig3ledd}
\abch{
\item $ x= 10$
\item $ x= 5$
\item $ x= 9$
\item $ x= 2$
\item $ x=3 $
\item $ x=4 $
\item $ x= 10$
\item $ x= 8$
\item $ x= 10$
}

\opr{opglig4ledd}
\abch{
\item $ x= 37$
\item $ x= 29$
\item $ x= 23$
\item $ x= 15$
\item $ x= 8$
\item $ x= 11$
\item $ x= 23$
\item $ x= 19$
\item $ x= 10$
\item $ x= 28$
}

\opr{opgligax1}

\abch{
\item $ x= 4$
\item $ x= 5$
\item $ x= 9$
\item $ x= 15$
}

\opr{opgligxovera1}
\abch{
\item $ x= 8$
\item $ x= 72$
\item $ x= 49$
\item $ x= 150$
}

\opr{opgligax2}
\abch{
\item $ x= 7$
\item $ x= 10 $
\item $ x= 6$
\item $ x= 9$
\item $ x= 9$
\item $ x= 8$
\item $ x= 5$
\item $ x=2 $
}

\opr{opgligvink} \selos

\opr{eksu22opg3} $ x=18 $

\opr{1TV21D1opg1} $ x=1, y=-2 $

\grubr{opgrepdes} \selos

\subsection*{\faskap \ref{Funksjoner}}
\opr{opgfunfinnlin}
\abch{
	\item $ f(x)=2x+4 $.
	\item 204
	\item $ x=24 $.
}

\opr{opgfunkfem}
\abch{
	\item $ a(x)=5x^2 $.
	\item $ x = 2000 $?
	\item $ x =9 $?
}

\opr{opgfunkbluegreen}
\abch{
	\item $ b(x)=x^2+2x $
	\item 440?
	\item $ x= 8 $?
}

\opr{EG22D1opg1}
22 trekanter og 10 firkanter.

\opr{opgfunkparodd}
La $ x $ være et positivt heltall.
\abch{
	\item $ p(x)=2n $
	\item $ o(x)=2n-1 $
}

\opr{opgfunkstigogkonst}

\abch{
	\item Stigningstall $ 5 $ og konstantledd $ 10 $.
	\item Stigningstall $ 3 $ og konstantledd $ -12 $.
	\item  Stigningstall $ -\frac{1}{7} $ og konstantledd $ -9 $.
	\item Stigningstall $ \frac{3}{2} $ og konstantledd $ -\frac{1}{4} $.
}

\opr{funkopgtegn} \selos


\opr{funkopgligset}
\abch{
	\item \eqref{eks3a} og \eqref{eks3b} gir hver for seg en ligning som beksriver en rett linje. Disse linjene kan også representeres ved $ f $ og $ g $ slik som de er definert.
	\item $ x=7 $ og $ y=2 $.
}

\opr{funkopgfinngraf1} $ f(x)=-8x+16 $

\opr{funkopgfinngraf2} $ f(x)=4x+3 $

\opr{funkopgfinngraf3} $ f(x)=x-3 $ og $ g(x)=\frac{1}{4}x+1 $

\grubr{opgaddifparodd}
\selos

\subsection*{\faskap \ref{Geometri2}}
\opr{geoopgformlvink} $ 80^\circ $

\opr{geoopgsamsv}
$ AC $ og $ DE $, $ BC $ og $ EF $, $ AB $ og $ DF $.

\opr{geofinnlen} $ EF=3 $ og $ DF=2 $.

\opr{geofinnlen2} $ AC=15 $ og $ DF=6 $.

\opr{opggeoarforhold}
\abch{
	\item 9
	\item $ a^2 $
}

\opr{opggeovolkjegl}
\abch{
	\item $ 100\pi $
	\item $ 400\pi $
}

\opr{opggepvolforhold}
\abch{
	\item 27
	\item $ a^3 $
}

\grubr{opglikbmidtn} Se side ?? i \tmen.

\grubr{opgeqlheight} \selos

\grubr{opg306090} \selos

\grubr{opgdoublear} \selos

\grubr{opgmedian} \selos

\grubr{opgtresirkar} \selos

\grubr{GV21D1opg12} \selos

\end{document}