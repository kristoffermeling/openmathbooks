\documentclass[english, 11 pt, class=article, crop=false]{standalone}
\usepackage[T1]{fontenc}
\usepackage[utf8]{luainputenc}
\usepackage{lmodern} % load a font with all the characters
\usepackage{geometry}
\geometry{verbose,paperwidth=16.1 cm, paperheight=24 cm, inner=2.3cm, outer=1.8 cm, bmargin=2cm, tmargin=1.8cm}
\setlength{\parindent}{0bp}
\usepackage{import}
\usepackage[subpreambles=false]{standalone}
\usepackage{amsmath}
\usepackage{amssymb}
\usepackage{esint}
\usepackage{babel}
\usepackage{tabu}
\makeatother
\makeatletter

\usepackage{titlesec}
\usepackage{ragged2e}
\RaggedRight
\raggedbottom
\frenchspacing

% Norwegian names of figures, chapters, parts and content
\addto\captionsenglish{\renewcommand{\figurename}{Figure}}
\makeatletter
\addto\captionsenglish{\renewcommand{\chaptername}{Chapter}}
%\addto\captionsenglish{\renewcommand{\partname}{Part}}

%\addto\captionsenglish{\renewcommand{\contentsname}{Content}}

\usepackage{graphicx}
\usepackage{float}
\usepackage{subfig}
\usepackage{placeins}
\usepackage{cancel}
\usepackage{framed}
\usepackage{wrapfig}
\usepackage[subfigure]{tocloft}
\usepackage[font=footnotesize,labelfont=sl]{caption} % Figure caption
\usepackage{bm}
\usepackage[dvipsnames, table]{xcolor}
\definecolor{shadecolor}{rgb}{0.105469, 0.613281, 1}
\colorlet{shadecolor}{Emerald!15} 
\usepackage{icomma}
\makeatother
\usepackage[many]{tcolorbox}
\usepackage{multicol}
\usepackage{stackengine}

% For tabular
\usepackage{array}
\usepackage{multirow}
\usepackage{longtable} %breakable table

% Ligningsreferanser
\usepackage{mathtools}
\mathtoolsset{showonlyrefs}

% index
\usepackage{imakeidx}
\makeindex[title=Index]

%Footnote:
\usepackage[bottom, hang, flushmargin]{footmisc}
\usepackage{perpage} 
\MakePerPage{footnote}
\addtolength{\footnotesep}{2mm}
\renewcommand{\thefootnote}{\arabic{footnote}}
\renewcommand\footnoterule{\rule{\linewidth}{0.4pt}}
\renewcommand{\thempfootnote}{\arabic{mpfootnote}}

%colors
\definecolor{c1}{cmyk}{0,0.5,1,0}
\definecolor{c2}{cmyk}{1,0.25,1,0}
\definecolor{n3}{cmyk}{1,0.,1,0}
\definecolor{neg}{cmyk}{1,0.,0.,0}

% Lister med bokstavar
\usepackage{enumitem}

\newcounter{rg}
\numberwithin{rg}{chapter}
\newcommand{\reg}[2][]{\begin{tcolorbox}[boxrule=0.3 mm,arc=0mm,colback=blue!3] {\refstepcounter{rg}\phantomsection \large \textbf{\therg \;#1} \vspace{5 pt}}\newline #2  \end{tcolorbox}\vspace{-5pt}}

\newcommand\alg[1]{\begin{align} #1 \end{align}}

\newcommand\eks[2][]{\begin{tcolorbox}[boxrule=0.3 mm,arc=0mm,enhanced jigsaw,breakable,colback=green!3] {\large \textbf{Example #1} \vspace{5 pt}\\} #2 \end{tcolorbox}\vspace{-5pt} }

\newcommand{\st}[1]{\begin{tcolorbox}[boxrule=0.0 mm,arc=0mm,enhanced jigsaw,breakable,colback=yellow!12]{ #1} \end{tcolorbox}}

\newcommand{\spr}[1]{\begin{tcolorbox}[boxrule=0.3 mm,arc=0mm,enhanced jigsaw,breakable,colback=yellow!7] {\large \textbf{The language box} \vspace{5 pt}\\} #1 \end{tcolorbox}\vspace{-5pt} }

\newcommand{\sym}[1]{\colorbox{blue!15}{#1}}

\newcommand{\info}[2]{\begin{tcolorbox}[boxrule=0.3 mm,arc=0mm,enhanced jigsaw,breakable,colback=cyan!6] {\large \textbf{#1} \vspace{5 pt}\\} #2 \end{tcolorbox}\vspace{-5pt} }

\newcommand\algv[1]{\vspace{-11 pt}\begin{align*} #1 \end{align*}}

\newcommand{\regv}{\vspace{5pt}}
\newcommand{\mer}{\textsl{Note}: }
\newcommand{\merk}{Note}
\newcommand\vsk{\vspace{11pt}}
\newcommand\vs{\vspace{-11pt}}
\newcommand\vsb{\vspace{-16pt}}
\newcommand\sv{\vsk \textbf{Answer} \vspace{4 pt}\\}
\newcommand\br{\\[5 pt]}
\newcommand{\asym}[1]{../fig/#1}
\newcommand\algvv[1]{\vs\vs\begin{align*} #1 \end{align*}}
\newcommand{\y}[1]{$ {#1} $}
\newcommand{\os}{\\[5 pt]}
\newcommand{\prbxl}[2]{
\parbox[l][][l]{#1\linewidth}{#2
	}}
\newcommand{\prbxr}[2]{\parbox[r][][l]{#1\linewidth}{
		\setlength{\abovedisplayskip}{5pt}
		\setlength{\belowdisplayskip}{5pt}	
		\setlength{\abovedisplayshortskip}{0pt}
		\setlength{\belowdisplayshortskip}{0pt} 
		\begin{shaded}
			\footnotesize	#2 \end{shaded}}}

\renewcommand{\cfttoctitlefont}{\Large\bfseries}
\setlength{\cftaftertoctitleskip}{0 pt}
\setlength{\cftbeforetoctitleskip}{0 pt}

\newcommand{\bs}{\\[3pt]}
\newcommand{\vn}{\\[6pt]}
\newcommand{\fig}[1]{\begin{figure}
		\centering
		\includegraphics[]{\asym{#1}}
\end{figure}}

\newcommand{\sectionbreak}{\clearpage} % New page on each section

% Equation comments
\newcommand{\cm}[1]{\llap{\color{blue} #1}}

\newcommand\fork[2]{\begin{tcolorbox}[boxrule=0.3 mm,arc=0mm,enhanced jigsaw,breakable,colback=yellow!7] {\large \textbf{#1 (explanation)} \vspace{5 pt}\\} #2 \end{tcolorbox}\vspace{-5pt} }

% Colors
\newcommand{\colr}[1]{{\color{red} #1}}
\newcommand{\colb}[1]{{\color{blue} #1}}
\newcommand{\colo}[1]{{\color{orange} #1}}
\newcommand{\colc}[1]{{\color{cyan} #1}}
\definecolor{projectgreen}{cmyk}{100,0,100,0}
\newcommand{\colg}[1]{{\color{projectgreen} #1}}

%%% SECTION HEADLINES %%%

% Our numbers
\newcommand{\likteikn}{The equal sign}
\newcommand{\talsifverd}{Numbers, digits and values}
\newcommand{\koordsys}{Coordinate systems}

% Calculations
\newcommand{\adi}{Addition}
\newcommand{\sub}{Subtraction}
\newcommand{\gong}{Multiplication}
\newcommand{\del}{Division}

%Factorization and order of operations
\newcommand{\fak}{Factorization}
\newcommand{\rrek}{Order of operations}

%Fractions
\newcommand{\brgrpr}{Introduction}
\newcommand{\brvu}{Values, expanding and simplifying}
\newcommand{\bradsub}{Addition and subtraction}
\newcommand{\brgngheil}{Fractions multiplied by integers}
\newcommand{\brdelheil}{Fractions divided by integers}
\newcommand{\brgngbr}{Fractions multiplied by fractions}
\newcommand{\brkans}{Cancelation of fractions}
\newcommand{\brdelmbr}{Division by fractions}
\newcommand{\Rasjtal}{Rational numbers}

%Negative numbers
\newcommand{\negintro}{Introduction}
\newcommand{\negrekn}{The elementary operations}
\newcommand{\negmeng}{Negative numbers as amounts}

% Geometry 1
\newcommand{\omgr}{Terms}
\newcommand{\eignsk}{Attributes of triangles and quadrilaterals}
\newcommand{\omkr}{Perimeter}
\newcommand{\area}{Area}

%Algebra 
\newcommand{\algintro}{Introduction}
\newcommand{\pot}{Powers}
\newcommand{\irrasj}{Irrational numbers}

%Equations
\newcommand{\ligintro}{Introduction}
\newcommand{\liglos}{Solving with the elementary operations}
\newcommand{\ligloso}{Solving with elementary operations summarized}

%Functions
\newcommand{\fintro}{Introduction}
\newcommand{\lingraf}{Linear functions and graphs}

%Geometry 2
\newcommand{\geoform}{Formulas of area and perimeter}
\newcommand{\kongogsim}{Congruent and similar triangles}
\newcommand{\geofork}{Explanations}

% Names of rules
\newcommand{\adkom}{Addition is commutative}
\newcommand{\gangkom}{Multiplication is commutative}
\newcommand{\brdef}{Fractions as rewriting of division}
\newcommand{\brtbr}{Fractions multiplied by fractions}
\newcommand{\delmbr}{Fractions divided by fractions}
\newcommand{\gangpar}{Distributive law}
\newcommand{\gangparsam}{Paranthesis multiplied together}
\newcommand{\gangmnegto}{Multiplication by negative numbers I}
\newcommand{\gangmnegtre}{Multiplication by negative numbers II}
\newcommand{\konsttre}{Unique construction of triangles}
\newcommand{\kongtre}{Congruent triangles}
\newcommand{\topv}{Vertical angles}
\newcommand{\trisum}{The sum of angles in a triangle}
\newcommand{\firsum}{The sum of angles in a quadrilateral}
\newcommand{\potgang}{Multiplication by powers}
\newcommand{\potdivpot}{Division by powers}
\newcommand{\potanull}{The special case of \boldmath $a^0$}
\newcommand{\potneg}{Powers with negative exponents}
\newcommand{\potbr}{Fractions as base}
\newcommand{\faktgr}{Factors as base}
\newcommand{\potsomgrunn}{Powers as base}
\newcommand{\arsirk}{The area of a circle}
\newcommand{\artrap}{The area of a trapezoid}
\newcommand{\arpar}{The area of a parallelogram}
\newcommand{\pyt}{Pythagoras's theorem}
\newcommand{\forform}{Ratios in similar triangles}
\newcommand{\vilkform}{Terms of similar triangles}
\newcommand{\omkrsirk}{The perimeter of a circle (and the value of $ \bm \pi $)}
\newcommand{\artri}{The area of a triangle}
\newcommand{\arrekt}{The area of a rectangle}
\newcommand{\liknflyt}{Moving terms across the equal sign}
\newcommand{\funklin}{Linear functions}

%License
\newcommand{\lic}{\textit{First Principles of Math by Sindre Sogge Heggen is licensed under CC BY-NC-SA 4.0. To view a copy of this license, visit\\ 
		\net{http://creativecommons.org/licenses/by-nc-sa/4.0/}{http://creativecommons.org/licenses/by-nc-sa/4.0/}}}

%referances
\newcommand{\net}[2]{{\color{blue}\href{#1}{#2}}}
\newcommand{\hrs}[2]{\hyperref[#1]{\color{blue}\textsl{#2 \ref*{#1}}}}
\newcommand{\rref}[1]{\hrs{#1}{Rule}}
\newcommand{\refkap}[1]{\hrs{#1}{Chapter}}
\newcommand{\refsec}[1]{\hrs{#1}{Section}}

\usepackage{datetime2}
\usepackage[]{hyperref}


\begin{document}
\section{\algintro}
Simply said, \textit{algebra}\index{algebra} is mathematics where letters represent numbers. This makes it easier working with \textsl{general} cases. For example, $ 3\cdot 2=2\cdot3 $ and $ {6\cdot7=7\cdot6} $ but these are only two of the infinitely many examples of the commutative property of multiplication! One of the aims of algebra is giving \textsl{one} example that explains \textsl{all} cases, and since our digits (0-9) are inevitably connected to specific numbers, we apply letters to reach this target. \vsk

The value of the numbers represented by letters will often vary, in that case we call the letter-numbers \textit{variables}\index{variable}. If letter-numbers on the other hand have a specific value, they are called \textit{constants}\index{constant}.

\vsk

In \hrs{Del1}{Part}, we studied calculations through examples with specific numbers, however, most of these rules are \textsl{general}; they are valid for all numbers. On page \pageref{regstart}\,-\,\pageref{regslutt}, many of these rules are reproduced in a general form. A good way of getting acquainted with algebra is comparing the rules here presented by the way they are expressed in\footnote{The number of the rules as found in \hrs{Del1}{Part} are written inside parentheses.} \hrs{Del1}{Part}. \vsk

\regv
\label{regstart}
\reg[\adkom\;(\ref{adkom}) \label{adkomalg}]{\vs
\[ a+ b =b+a \]
}
\eks{ \vsb
\[ 7+ 5=5+7 \]
} \vsk \vsk

\reg[\gangkom\;(\ref{gangkom})]{\vs
	\[ a\cdot b =b\cdot a \]
}
\eks[1]{ \vsb
	\[ 9\cdot 8=8\cdot9 \]
}
\eks[2]{ \vsb
\[  8\cdot a= a\cdot 8  \]
}
\newpage
\info{Multiplication involving letters}{When multiplication involves letters, it is common to omit the symbol of multiplication. If a specific number and a letter are multiplied together, the specific number is written first. For example,
	\[ a\cdot b= ab \]
	and
	\[ a\cdot 8 =8a \]
We also write
\[ 1\cdot a=a \]
In addition, it is common to omit the symbol of multiplication when an expression with parentheses is involved:
\[ 
3\cdot(a+b)=3(a+b) \]
}
\vsk 

\reg[\brdef\;(\ref{brdef})]{
\[ a:b=\frac{a}{b} \]
}
\eks[]{ \vs
\[a:2= \frac{a}{2} \]
}
 \vsk 

\reg[\brtbr\; (\ref{brtbr})]{
\[ \frac{a}{b}\cdot\frac{c}{d}=\frac{a c}{b d} \]
}
\eks[1]{ \vs
\algv{
\frac{2}{11}\cdot \frac{13}{21}&=\frac{2\cdot 13}{11\cdot21} =\frac{26}{231}
}
}
\eks[2]{ \vs
	\[ \frac{3}{b}\cdot \frac{a}{7}=\frac{3 a}{7b} \]
}
\newpage
\reg[\brdelmbr\;(\ref{delmbr})]{ 
\[ \frac{a}{b}:\frac{c}{d}=\frac{a}{b}\cdot \frac{d}{c} \]
}
\eks[1]{ \vs
\[ \frac{1}{2}:\frac{5}{7}=\frac{1}{2}\cdot \frac{7}{5} \]
}
\eks[2]{ \vsb \vs
\alg{
\frac{a}{13}:\frac{b}{3}&=\frac{a}{13}\cdot \frac{3}{b} \br
&=\frac{3a}{13b}
}
} \vsk \vsk

\reg[\gangpar\;(\ref{gangpar}) \label{gangpara}]{ \vs
\[ (a+b)c = a c + b c \]
} 
\eks[1]{ \vs
\[ (2+a)b =2b+ab \]
}
\eks[2]{\vs
\[ a(5b-3)=5ab-3a \]
} 
\vsk \vsk
\reg[\gangmnegto\;(\ref{gangmnegto})]{
\[ a\cdot(-b)=-(a\cdot b) \]
}
\eks[1]{ \vsb \vs
\alg{
3\cdot(-4)&=-(3\cdot 4) \\
&= 	-12
}
}
\eks[2]{ \vsb
	\algv{
	(-a)\cdot7&=-(a\cdot 7)\\
	&=-7a 
}
} \vsk \vsk

\reg[\gangmnegtre\;(\ref{gangmnegtre}) \label{gangmnegtrea}]{
\[ (-a)\cdot(-b)=a\cdot b \]
}
\eks[1]{ \vs \vs
\alg{
(-2)\cdot(-8)&=2\cdot 8 \\
&= 	16
}	
}
\eks[2]{ \vs 
\[ (-a)\cdot(-15)=15a \]
}
\label{regslutt}
\vsk \vsk

\newpage
\info{Extensions of the rules}{
One of the strengths of algebra is that we can express compact rules which are easily extended to apply for other cases. Let's, as an example, find another expression of
	\[ (a+b+c)d \]
\rref{gangpara} does not directly imply how to calculate between the expression inside the parentheses and $ d $, but there is no wrongdoing in defining $ a+b $ as $ k $:
\[ a+b=k \]
Then
\[ (a+b+c)d=(k+c)d \]
Now, by \rref{gangpara}, we have
\[ (k+c)d = kd+cd \]
Inserting the expression for $ k $, we have 
\[ kd+cd=(a+b)d+cd \]
By applying \rref{gangpara} once more we can write
\[ (a+b)d+cd=ad+bc+cd \]
Then
\[ (a+b+c)d=ad+bc+cd \]
\it Notice! This example is \textsl{not} meant to show how to handle expressions not directly covered by \textsl{Rule} \ref{adkomalg}\,-\,\ref{gangmnegtrea}, but to emphasize why it's always sufficient to write rules with the least amount of terms, factors etc. Usually you apply extension of the rules without even thinking about it, and surely not in such meticulous manner as here provided.
}


\end{document}


