\documentclass[english, 11 pt, class=article, crop=false]{standalone}
\usepackage[T1]{fontenc}
%\renewcommand*\familydefault{\sfdefault} % For dyslexia-friendly text
\usepackage{lmodern} % load a font with all the characters
\usepackage{geometry}
\geometry{verbose,paperwidth=16.1 cm, paperheight=24 cm, inner=2.3cm, outer=1.8 cm, bmargin=2cm, tmargin=1.8cm}
\setlength{\parindent}{0bp}
\usepackage{import}
\usepackage[subpreambles=false]{standalone}
\usepackage{amsmath}
\usepackage{amssymb}
\usepackage{esint}
\usepackage{babel}
\usepackage{tabu}
\makeatother
\makeatletter

\usepackage{titlesec}
\usepackage{ragged2e}
\RaggedRight
\raggedbottom
\frenchspacing

% Norwegian names of figures, chapters, parts and content
\addto\captionsenglish{\renewcommand{\figurename}{Figur}}
\makeatletter
\addto\captionsenglish{\renewcommand{\chaptername}{Kapittel}}
\addto\captionsenglish{\renewcommand{\partname}{Del}}


\usepackage{graphicx}
\usepackage{float}
\usepackage{subfig}
\usepackage{placeins}
\usepackage{cancel}
\usepackage{framed}
\usepackage{wrapfig}
\usepackage[subfigure]{tocloft}
\usepackage[font=footnotesize,labelfont=sl]{caption} % Figure caption
\usepackage{bm}
\usepackage[dvipsnames, table]{xcolor}
\definecolor{shadecolor}{rgb}{0.105469, 0.613281, 1}
\colorlet{shadecolor}{Emerald!15} 
\usepackage{icomma}
\makeatother
\usepackage[many]{tcolorbox}
\usepackage{multicol}
\usepackage{stackengine}

\usepackage{esvect} %For vectors with capital letters

% For tabular
\usepackage{array}
\usepackage{multirow}
\usepackage{longtable} %breakable table

% Ligningsreferanser
\usepackage{mathtools}
\mathtoolsset{showonlyrefs}

% index
\usepackage{imakeidx}
\makeindex[title=Indeks]

%Footnote:
\usepackage[bottom, hang, flushmargin]{footmisc}
\usepackage{perpage} 
\MakePerPage{footnote}
\addtolength{\footnotesep}{2mm}
\renewcommand{\thefootnote}{\arabic{footnote}}
\renewcommand\footnoterule{\rule{\linewidth}{0.4pt}}
\renewcommand{\thempfootnote}{\arabic{mpfootnote}}

%colors
\definecolor{c1}{cmyk}{0,0.5,1,0}
\definecolor{c2}{cmyk}{1,0.25,1,0}
\definecolor{n3}{cmyk}{1,0.,1,0}
\definecolor{neg}{cmyk}{1,0.,0.,0}

% Lister med bokstavar
\usepackage[inline]{enumitem}

\newcounter{rg}
\numberwithin{rg}{chapter}
\newcommand{\reg}[2][]{\begin{tcolorbox}[boxrule=0.3 mm,arc=0mm,colback=blue!3] {\refstepcounter{rg}\phantomsection \large \textbf{\therg \;#1} \vspace{5 pt}}\newline #2  \end{tcolorbox}\vspace{-5pt}}

\newcommand\alg[1]{\begin{align} #1 \end{align}}

\newcommand\eks[2][]{\begin{tcolorbox}[boxrule=0.3 mm,arc=0mm,enhanced jigsaw,breakable,colback=green!3] {\large \textbf{Eksempel #1} \vspace{5 pt}\\} #2 \end{tcolorbox}\vspace{-5pt} }

\newcommand{\st}[1]{\begin{tcolorbox}[boxrule=0.0 mm,arc=0mm,enhanced jigsaw,breakable,colback=yellow!12]{ #1} \end{tcolorbox}}

\newcommand{\spr}[1]{\begin{tcolorbox}[boxrule=0.3 mm,arc=0mm,enhanced jigsaw,breakable,colback=yellow!7] {\large \textbf{Språkboksen} \vspace{5 pt}\\} #1 \end{tcolorbox}\vspace{-5pt} }

\newcommand{\sym}[1]{\colorbox{blue!15}{#1}}

\newcommand{\info}[2]{\begin{tcolorbox}[boxrule=0.3 mm,arc=0mm,enhanced jigsaw,breakable,colback=cyan!6] {\large \textbf{#1} \vspace{5 pt}\\} #2 \end{tcolorbox}\vspace{-5pt} }

\newcommand\algv[1]{\vspace{-11 pt}\begin{align*} #1 \end{align*}}

\newcommand{\regv}{\vspace{5pt}}
\newcommand{\mer}{\textsl{Merk}: }
\newcommand{\mers}[1]{{\footnotesize \mer #1}}
\newcommand\vsk{\vspace{11pt}}
\newcommand\vs{\vspace{-11pt}}
\newcommand\vsb{\vspace{-16pt}}
\newcommand\sv{\vsk \textbf{Svar} \vspace{4 pt}\\}
\newcommand\br{\\[5 pt]}
\newcommand{\figp}[1]{../fig/#1}
\newcommand\algvv[1]{\vs\vs\begin{align*} #1 \end{align*}}
\newcommand{\y}[1]{$ {#1} $}
\newcommand{\os}{\\[5 pt]}
\newcommand{\prbxl}[2]{
\parbox[l][][l]{#1\linewidth}{#2
	}}
\newcommand{\prbxr}[2]{\parbox[r][][l]{#1\linewidth}{
		\setlength{\abovedisplayskip}{5pt}
		\setlength{\belowdisplayskip}{5pt}	
		\setlength{\abovedisplayshortskip}{0pt}
		\setlength{\belowdisplayshortskip}{0pt} 
		\begin{shaded}
			\footnotesize	#2 \end{shaded}}}

\renewcommand{\cfttoctitlefont}{\Large\bfseries}
\setlength{\cftaftertoctitleskip}{0 pt}
\setlength{\cftbeforetoctitleskip}{0 pt}

\newcommand{\bs}{\\[3pt]}
\newcommand{\vn}{\\[6pt]}
\newcommand{\fig}[1]{\begin{figure}
		\centering
		\includegraphics[]{\figp{#1}}
\end{figure}}

\newcommand{\figc}[2]{\begin{figure}
		\centering
		\includegraphics[]{\figp{#1}}
		\caption{#2}
\end{figure}}

\newcommand{\sectionbreak}{\clearpage} % New page on each section

\newcommand{\nn}[1]{
\begin{equation}
	#1
\end{equation}
}

% Equation comments
\newcommand{\cm}[1]{\llap{\color{blue} #1}}

\newcommand\fork[2]{\begin{tcolorbox}[boxrule=0.3 mm,arc=0mm,enhanced jigsaw,breakable,colback=yellow!7] {\large \textbf{#1 (forklaring)} \vspace{5 pt}\\} #2 \end{tcolorbox}\vspace{-5pt} }
 
%colors
\newcommand{\colr}[1]{{\color{red} #1}}
\newcommand{\colb}[1]{{\color{blue} #1}}
\newcommand{\colo}[1]{{\color{orange} #1}}
\newcommand{\colc}[1]{{\color{cyan} #1}}
\definecolor{projectgreen}{cmyk}{100,0,100,0}
\newcommand{\colg}[1]{{\color{projectgreen} #1}}

% Methods
\newcommand{\metode}[2]{
	\textsl{#1} \\[-8pt]
	\rule{#2}{0.75pt}
}

%Opg
\newcommand{\abc}[1]{
	\begin{enumerate}[label=\alph*),leftmargin=18pt]
		#1
	\end{enumerate}
}
\newcommand{\abcs}[2]{
	\begin{enumerate}[label=\alph*),start=#1,leftmargin=18pt]
		#2
	\end{enumerate}
}
\newcommand{\abcn}[1]{
	\begin{enumerate}[label=\arabic*),leftmargin=18pt]
		#1
	\end{enumerate}
}
\newcommand{\abch}[1]{
	\hspace{-2pt}	\begin{enumerate*}[label=\alph*), itemjoin=\hspace{1cm}]
		#1
	\end{enumerate*}
}
\newcommand{\abchs}[2]{
	\hspace{-2pt}	\begin{enumerate*}[label=\alph*), itemjoin=\hspace{1cm}, start=#1]
		#2
	\end{enumerate*}
}

% Oppgaver
\newcommand{\opgt}{\phantomsection \addcontentsline{toc}{section}{Oppgaver} \section*{Oppgaver for kapittel \thechapter}\vs \setcounter{section}{1}}
\newcounter{opg}
\numberwithin{opg}{section}
\newcommand{\op}[1]{\vspace{15pt} \refstepcounter{opg}\large \textbf{\color{blue}\theopg} \vspace{2 pt} \label{#1} \\}
\newcommand{\ekspop}[1]{\vsk\textbf{Gruble \thechapter.#1}\vspace{2 pt} \\}
\newcommand{\nes}{\stepcounter{section}
	\setcounter{opg}{0}}
\newcommand{\opr}[1]{\vspace{3pt}\textbf{\ref{#1}}}
\newcommand{\oeks}[1]{\begin{tcolorbox}[boxrule=0.3 mm,arc=0mm,colback=white]
		\textit{Eksempel: } #1	  
\end{tcolorbox}}
\newcommand\opgeks[2][]{\begin{tcolorbox}[boxrule=0.1 mm,arc=0mm,enhanced jigsaw,breakable,colback=white] {\footnotesize \textbf{Eksempel #1} \\} \footnotesize #2 \end{tcolorbox}\vspace{-5pt} }
\newcommand{\rknut}{
Rekn ut.
}

%License
\newcommand{\lic}{\textit{Matematikken sine byggesteinar by Sindre Sogge Heggen is licensed under CC BY-NC-SA 4.0. To view a copy of this license, visit\\ 
		\net{http://creativecommons.org/licenses/by-nc-sa/4.0/}{http://creativecommons.org/licenses/by-nc-sa/4.0/}}}

%referances
\newcommand{\net}[2]{{\color{blue}\href{#1}{#2}}}
\newcommand{\hrs}[2]{\hyperref[#1]{\color{blue}\textsl{#2 \ref*{#1}}}}
\newcommand{\rref}[1]{\hrs{#1}{regel}}
\newcommand{\refkap}[1]{\hrs{#1}{kapittel}}
\newcommand{\refsec}[1]{\hrs{#1}{seksjon}}

\newcommand{\mb}{\net{https://sindrsh.github.io/FirstPrinciplesOfMath/}{MB}}


%line to seperate examples
\newcommand{\linje}{\rule{\linewidth}{1pt} }

\usepackage{datetime2}
%%\usepackage{sansmathfonts} for dyslexia-friendly math
\usepackage[]{hyperref}


\newcommand{\note}{Merk}
\newcommand{\notesm}[1]{{\footnotesize \textsl{\note:} #1}}
\newcommand{\ekstitle}{Eksempel }
\newcommand{\sprtitle}{Språkboksen}
\newcommand{\expl}{forklaring}

\newcommand{\vedlegg}[1]{\refstepcounter{vedl}\section*{Vedlegg \thevedl: #1}  \setcounter{vedleq}{0}}

\newcommand\sv{\vsk \textbf{Svar} \vspace{4 pt}\\}

%references
\newcommand{\reftab}[1]{\hrs{#1}{tabell}}
\newcommand{\rref}[1]{\hrs{#1}{regel}}
\newcommand{\dref}[1]{\hrs{#1}{definisjon}}
\newcommand{\refkap}[1]{\hrs{#1}{kapittel}}
\newcommand{\refsec}[1]{\hrs{#1}{seksjon}}
\newcommand{\refdsec}[1]{\hrs{#1}{delseksjon}}
\newcommand{\refved}[1]{\hrs{#1}{vedlegg}}
\newcommand{\eksref}[1]{\textsl{#1}}
\newcommand\fref[2][]{\hyperref[#2]{\textsl{figur \ref*{#2}#1}}}
\newcommand{\refop}[1]{{\color{blue}Oppgave \ref{#1}}}
\newcommand{\refops}[1]{{\color{blue}oppgave \ref{#1}}}
\newcommand{\refgrubs}[1]{{\color{blue}gruble \ref{#1}}}

\newcommand{\openmathser}{\openmath\,-\,serien}

% Exercises
\newcommand{\opgt}{\newpage \phantomsection \addcontentsline{toc}{section}{Oppgaver} \section*{Oppgaver for kapittel \thechapter}\vs \setcounter{section}{1}}


% Sequences and series
\newcommand{\sumarrek}{Summen av en aritmetisk rekke}
\newcommand{\sumgerek}{Summen av en geometrisk rekke}
\newcommand{\regnregsum}{Regneregler for summetegnet}

% Trigonometry
\newcommand{\sincoskomb}{Sinus og cosinus kombinert}
\newcommand{\cosfunk}{Cosinusfunksjonen}
\newcommand{\trid}{Trigonometriske identiteter}
\newcommand{\deravtri}{Den deriverte av de trigonometriske funksjonene}
% Solutions manual
\newcommand{\selos}{Se løsningsforslag.}
\newcommand{\se}[1]{Se eksempel på side \pageref{#1}}

%Vectors
\newcommand{\parvek}{Parallelle vektorer}
\newcommand{\vekpro}{Vektorproduktet}
\newcommand{\vekproarvol}{Vektorproduktet som areal og volum}


% 3D geometries
\newcommand{\linrom}{Linje i rommet}
\newcommand{\avstplnpkt}{Avstand mellom punkt og plan}


% Integral
\newcommand{\bestminten}{Bestemt integral I}
\newcommand{\anfundteo}{Analysens fundamentalteorem}
\newcommand{\intuf}{Integralet av utvalge funksjoner}
\newcommand{\bytvar}{Bytte av variabel}
\newcommand{\intvol}{Integral som volum}
\newcommand{\andordlindif}{Andre ordens lineære differensialligninger}



\begin{document}
\section{\algintro}
\outl{Algebra}\index{algebra} er matematikk der bokstaver representerer tall. Dette gjør at vi lettere kan jobbe med \textsl{generelle} tilfeller. For eksempel er $ {3\cdot 2=2\cdot3} $ og $ 6\cdot7=7\cdot6 $, men disse er bare to av de uendelig mange eksemplene på at multiplikasjon er kommutativ! En av hensiktene med algebra er at vi ønsker å gi \textsl{ett} eksempel som forklarer \textsl{alle} tilfeller, og siden sifrene våre (0-9) er uløselig knyttet til bestemte tall, bruker vi bokstaver for å nå dette målet. \vsk

Verdien til tallene som er representert ved bokstaver vil ofte variere ut ifra en sammenheng, og da kaller vi disse bokstavtallene for \outl{variabler}\index{variabel}. Hvis bokstavtallene derimot har en bestemt verdi, kaller vi dem for \outl{konstanter}\index{konstant}.

\vsk

I \textsl{Del I} av boka har vi sett på regning med konkrete tal, likevel er de fleste reglene vi har utledet \textsl{generelle}; de gjelder for alle tall. På side \pageref{regstart}\,-\,\pageref{regslutt} har vi gjengitt mange av disse reglene på en mer generell form. En fin introduksjon til algebra er å sammenligne reglene du finner her med slik du finner dem\footnote{Reglene sine nummer i \textsl{Del I} står i parentes.} i \textsl{Del I}. \vsk

\regv
\label{regstart}
\reg[\adkom\;(\ref{adkom}) \label{adkomalg}]{\vs
\[ a+ b =b+a \]
}
\eks{ \vsb
\[ 7+ 5=5+7 \]
} \vsk \vsk

\reg[\gangkom\;(\ref{gangkom})]{\vs
	\[ a\cdot b =b\cdot a \]
}
\eks[1]{ \vsb
	\[ 9\cdot 8=8\cdot9 \]
}
\eks[2]{ \vsb
\[  8\cdot a= a\cdot 8  \]
}
\newpage
\info{Ganging med bokstavuttrykk}{Når man ganger sammen bokstaver, er det vanlig å utelate gangetegnet. Og om man ganger sammen en bokstav og et konkret tal, skriver man det konkrete tallet først. Dette betyr for eksempel at
	\[ a\cdot b= ab \]
	og at
	\[ a\cdot 8 =8a \]
I tillegg skriver vi også
\[ 1\cdot a=a \]
Det er også vanlig å utelate gangetegn der parentesuttrykk er en faktor:\[ 
3\cdot(a+b)=3(a+b) \]
}
\vsk 

\reg[\brdef\;(\ref{brdef})]{
\[ a:b=\frac{a}{b} \]
}
\eks[]{ \vs
\[a:2= \frac{a}{2} \]
}
 \vsk 

\reg[\brtbr\; (\ref{brtbr})]{
\[ \frac{a}{b}\cdot\frac{c}{d}=\frac{a c}{b d} \]
}
\eks[1]{ \vs
\algv{
\frac{2}{11}\cdot \frac{13}{21}&=\frac{2\cdot 13}{11\cdot21} =\frac{26}{231}
}
}
\eks[2]{ \vs
	\[ \frac{3}{b}\cdot \frac{a}{7}=\frac{3 a}{7b} \]
}
\newpage
\reg[\brdelmbr\;(\ref{delmbr})]{ 
\[ \frac{a}{b}:\frac{c}{d}=\frac{a}{b}\cdot \frac{d}{c} \]
}
\eks[1]{ \vs
\[ \frac{1}{2}:\frac{5}{7}=\frac{1}{2}\cdot \frac{7}{5} \]
}
\eks[2]{ \vsb \vs
\alg{
\frac{a}{13}:\frac{b}{3}&=\frac{a}{13}\cdot \frac{3}{b} \br
&=\frac{3a}{13b}
}
} \vsk \vsk

\reg[\gangpar\;(\ref{gangpar}) \label{gangpara}]{ \vs
\[ (a+b)c = a c + b c \]
} 
\eks[1]{ \vs
\[ (2+a)b =2b+ab \]
}
\eks[2]{\vs
\[ a(5b-3)=5ab-3a \]
} 
\vsk \vsk
\reg[\gangmnegto\;(\ref{gangmnegto})]{\vs
\[ a\cdot(-b)=-(a\cdot b) \]
}
\eks[1]{ \vsb \vs
\alg{
3\cdot(-4)&=-(3\cdot 4) \\
&= 	-12
}
}
\newpage
\eks[2]{ \vsb
	\algv{
	(-a)\cdot7&=-(a\cdot 7)\\
	&=-7a 
}
} \vsk \vsk

\reg[\gangmnegtre\;(\ref{gangmnegtre}) \label{gangmnegtrea}]{\vs
\[ (-a)\cdot(-b)=a\cdot b \]
}
\eks[1]{ \vs \vs
\alg{
(-2)\cdot(-8)&=2\cdot 8 \\
&= 	16
}	
}
\eks[2]{ \vs 
\[ (-a)\cdot(-15)=15a \]
}
\label{regslutt}
\vsk \vsk

\info{Utvidelser av reglene}{
Noe av styrken til algebra er at vi kan lage oss kompakte regler som det er lett å utvide også til andre tilfeller. La oss som et eksempel finne et annet uttrykk for
	\[ (a+b+c)d \]
\hrs{gangpara}{Regel} forteller oss ikke direkte hvordan vi kan regne mellom parentesuttrykket og $ d $, men det er ingenting som hindrer oss i å omdøpe $ a+b $ til $ k $:
\[ a+b=k \]
Da er
\[ (a+b+c)d=(k+c)d \]
Av \rref{gangpara} har vi nå at
\[ (k+c)d = kd+cd \]
Om vi setter inn igjen uttrykket for $ k $, får vi 
\[ kd+cd=(a+b)d+cd \]
Ved å utnytte \rref{gangpara} enda en gang kan vi skrive
\[ (a+b)d+cd=ad+bc+cd \]
Altså er
\[ (a+b+c)d=ad+bc+cd \]
{\footnotesize
\it Obs! Dette eksempelet er \textsl{ikke} ment for å vise hvordan man skal gå fram når man har uttrykk som ikke direkte er omfattet av regel \ref{adkomalg}\,-\,\ref{gangmnegtrea}, men for å vise hvorfor det alltid er nok å skrive regler med færrest mulige ledd, faktorer og lignende. Oftest vil man bruke utvidelser av reglene uten engang å tenke over det, og i alle fall langt ifra så pertentlig som det vi gjorde her.}
}

\section{\pot \label{Potensar}}
\fig{pot}
En potens består av et \outl{grunntall}\index{grunntall} og en \outl{eksponent}\index{eksponent}. For eksempel er $2^{3}$ en potens med grunntall 2 og
eksponent 3. En positiv, heltalls eksponent sier hvor mange eksemplar
av grunntallet som skal ganges sammen, altså er
\[ 2^3 =2\cdot2\cdot2 \]

\reg[Potenstall]{
	$ {a^n} $ er et potenstall med grunntall $ a $ og eksponent $ n $. 
	\vsk
	
	Hvis $ n $ er et naturlig tall, vil $ a^n $ svare til $ n $ eksemplar av $ a $\\ multiplisert med hverandre.
}
\eks[1]{\vs \vs
	\algv{
		5^3 &= 5\cdot5\cdot5 \\
		&= 125
	}
}
\eks[2]{\vs \vs
	\[ c^4 = c\cdot c \cdot c \cdot c \]
}
\eks[3]{ \vs \vs
	\algv{
		(-7)^2 &= (-7)\cdot(-7) \\
		&= 49
	}
} 
\eks[4]{\vs \vs
	\[ a^1=a \]
}
\spr{
	Vanlige måter å si $ 2^3 $ på er
	\begin{itemize}
		\item ''2 i tredje''
		\item ''2 opphøyd i 3''
	\end{itemize}
	I programmeringsspråk brukes gjerne symbolet \sym{\^{}} eller symbolene \sym{**} mellom grunntall og eksponent.\vsk
	
	Å opphøye et tall i 2 kalles ''å kvadrere'' tallet.
}
\newpage
\info{Merk}{
	De kommende sidene vil inneholde regler for potenser med tilhørende forklaringer. Selv om det er ønskelig at de har en så generell form som mulig, har vi i forklaringene valgt å bruke eksempel der eksponentene ikke er variabler. Å bruke variabler som eksponenter ville gitt mye mindre leservennlige uttrykk, og vi vil påstå at de generelle tilfellene kommer godt til synes også ved å studere konkrete tilfeller. 
} \vsk \vsk

\reg[\potgang \label{potgang}]{
	\begin{equation}
		a^{m}\cdot a^{n}=a^{m+n}	
	\end{equation}
}
\eks[1]{\vs \vs
	\algv{3^{5}\cdot3^{2}&=3^{5+2}\\&=3^{7}}
}
\eks[2]{\vs \vs
	\algv{
		b^4\cdot b^{11}&= b^{3+11}\\
		&=b^{14}
	}
}
\eks[3]{ \vs \vs
	\algv{
		a^5\cdot a^{-7} &= a^{5+(-7)} \\
		&=a^{5-7} \\
		&= a^{-2} 
	}
	(Se \rref{potneg} for hvordan en potens med negativ eksponent kan tolkes.)	
} 
\newpage
\fork{\ref{potgang} \potgang}{
	La oss se på tilfellet 
	\[ a^{2}\cdot a^{3} \]
	Vi har at
	\algv{
		a^{2} & =2\cdot2\vn
		a^{3} & =2\cdot2\cdot2
	}
	Med andre ord kan vi skrive 
	\begin{align*}
		a^{2}\cdot a^{3} & =\overbrace{a \cdot a}^{a^{2}}\cdot\overbrace{a\cdot a\cdot a}^{a^{3}}\\
		& =a^{5}
	\end{align*}
}
\reg[\potdivpot \label{potdivpot}]{\vs
	\[ \frac{a^{m}}{a^{n}}=a^{m-n} \] }

\eks[1]{\vspace{-20 pt}
	\[
	\frac{3^{5}}{3^{2}}=3^{5-2}=3^{3}
	\]
} 
\eks[2]{ \vs \vsb
	\alg{
		\frac{2^{4}\cdot a^{7}}{a^{6}\cdot2^{2}}&=2^{4-2}\cdot a^{7-6}\\
		&=2^{2}a \\
		&=4a
	}
}
\newpage
\fork{\ref{potdivpot} \potdivpot}{
	La
	oss undersøke brøken
	\[ \frac{a^{5}}{a^{2}} \]
	Vi skriver
	ut potensene i teller og nevner: 
	\begin{align*}
		\frac{a^{5}}{a^{2}} & =\frac{a\cdot a\cdot a\cdot a\cdot a}{a\cdot a}\br
		& =\frac{\bcancel{a}\cdot\bcancel{a}\cdot a\cdot a\cdot a}{\bcancel{a}\cdot\bcancel{a}}\\
		& =a\cdot a\cdot a\\
		& =a^{3}
	\end{align*}
	Dette kunne vi ha skrevet som
	\begin{align*}
		\frac{a^{5}}{a^{2}} & =a^{5-2}\\
		& =a^{3}
	\end{align*}
} \vsk \vsk

\reg[\potanull \label{pota0}]{\vs \vs
	\[
	a^{0}=1
	\]
}
\eks[1]{\vs \vs\[
	1000^{0}=1
	\]}
\eks[2]{\vs \vs\[
	(-b)^{0}=1
	\]}
\fork{\ref{pota0} \potanull}{
	Et tall delt på seg selv er alltid lik 1, derfor er 
	\[
	\frac{a^{n}}{a^{n}}=1
	\]
	Av dette, og \rref{potdivpot}, har vi at
	\algv{
		1&=\frac{a^{n}}{a^{n}}
		\\& =a^{n-n}\\
		& =a^{0}
	}
} \vsk \vsk

\reg[\potneg \label{potneg}]{
	\[ a^{-n}=\frac{1}{a^n} \]
}
\eks[1]{ \vs \vs
	\alg{
		a^{-8}&=\frac{1}{a^8}  
	}	
}
\eks[2]{ \vs \vs
	\alg{
		(-4)^{-3}&=\frac{1}{(-4)^3} 
		=-\frac{1}{64}
	}
}
\fork{\ref{potneg} \potneg}{
	Av \rref{pota0} har vi at $ a^0=1 $. Altså er
	\alg{
		\frac{1}{a^n}=\frac{a^0}{a^n}
	}
	Av \rref{potdivpot}  er
	\algv{
		\frac{a^0}{a^n}&=a^{0-n} \\
		&=a^{-n}
	}
} \vsk \vsk


\reg[\potbr \label{potbr}]{\vs
	\begin{equation}\label{pbrg}
		\left(\frac{a}{b}\right)^{m}=\frac{a^{m}}{b^{m}}
\end{equation}} 
\eks[1]{ \vs \vs
	\alg{
		\left(\frac{3}{4}\right)^2=\frac{3^2}{4^2} 
		=\frac{9}{16}
	}
}
\eks[2]{ \vs \vs
	\alg{
		\left(\frac{a}{7}\right)^3=\frac{a^3}{7^3} 
		=\frac{a^3}{343}
	}
}
\fork{\ref{potbr} \potbr}{
	La oss studere
	\[ \left(\frac{a}{b}\right)^3 \]
	Vi har at
	\begin{align*}
		\left(\frac{a}{b}\right)^3 	&=\frac{a}{b}\cdot \frac{a}{b}\cdot \frac{a}{b}\br
		& =\frac{a\cdot a\cdot a}{b\cdot b\cdot b}\br
		& =\frac{a^{3}}{b^{3}}
	\end{align*}
}\vsk \vsk

\reg[\faktgr \label{faktgr}]{
	\begin{equation}\label{key}
		\left(ab\right)^{m}=a^{m}b^{m}
	\end{equation}
}
\eks[1]{ \vs \vs \vs
	\alg{
		(3a)^5&=3^5a^5 \\
		&=243a^5 
	}	
}
\eks[2]{\vs\vs
	\[
	(ab)^{4}=a^{4}b^{4}
	\]
}
\fork{\ref{faktgr} \faktgr}{
	La oss
	bruke ${(a\cdot b)^{3}}$ som eksempel. Vi har at
	\alg{
		(a\cdot b)^{3}&=(a\cdot b)\cdot(a\cdot b)\cdot(a\cdot b) \\
		&=a\cdot a\cdot a \cdot b \cdot b \cdot b \\
		&=a^3b^3
	}
}\vsk \vsk

\newpage
\reg[\potsomgrunn \label{potsomgrunn}]{\vs
	\begin{equation}
		\left(a^{m}\right)^{n}=a^{m\cdot n}
\end{equation}}
\eks[1]{ \vs \vs
	\alg{
		\left(c^4\right)^5&=c^{4\cdot5}\\
		&=c^{20}	
	}	
}
\eks[2]{ \vs \vs 
	\alg{
		\left(3^\frac{5}{4}\right)^8&=3^{\frac{5}{4}\cdot8} \\
		&=3^{10}
	}	
}
\fork{\ref{potsomgrunn} \potsomgrunn}{
	La oss bruke $\left(a^{3}\right)^{4}$ som eksempel. Vi har at
	\begin{align*}
		\left(a^{3}\right)^{4} & =a^{3}\cdot a^{3}\cdot a^{3}\cdot a^{3}
	\end{align*}
	
	
	Av \rref{potgang} er
	\algv{
		a^{3}\cdot a^{3}\cdot a^{3}\cdot a^{3} & =a^{3+3+3+3}\\
		& =a^{3\cdot4}\\
		&=a^{12}
	}	
}

\newpage
\reg[\textit{n}-rot]{ \vs
	\[ a^\frac{1}{n}=\sqrt[n]{a} \]
	Symbolet \sym{$ \sqrt{\phantom{a}} $} kalles et \outl{rottegn}\index{rottegn}. For eksponenten $ \frac{1}{2} $ er det vanlig å utelate 2 i rottegnet:
	\[ a^\frac{1}{2}=\sqrt{a} \]
}
\eks{
	Av \rref{potsomgrunn} har vi at
	\alg{
		\left(a^b\right)^\frac{1}{b}&=a^{b\cdot \frac{1}{b}} \\
		&=a	
	}
	For eksempel er	
	\algv{
		9^\frac{1}{2}=\sqrt{9}=3 &\text{, siden } 3^2 =9 \vn
		125^\frac{1}{3}=\sqrt[3]{125}=5 &\text{, siden } 5^3 =125 \vn	
		16^\frac{1}{4}=\sqrt[4]{16}=2 &\text{, siden } 2^4 =16
	}	
}
\spr{
	$\sqrt{9} $ kalles ''kvadratrota til 9'' \vsk
	
	$ \sqrt[5]{9} $ kalles ''femterota til 9''.
}
\newpage
\section{\irrasj}
\reg[Irrasjonale tall]{
	Et tall som \textsl{ikke} er et rasjonalt tall, er et irrasjonalt tall\index{tall!irrasjonalt}\footnotemark.\vsk
	
	Verdien til et irrasjonalt tall har uendelig mange desimaler med et ikke-repeterende mønster.
}
\footnotetext{Strengt tatt er irrasjonale tall alle \textit{reelle} tall som ikke er rasjonale tall. Men for å forklare hva \textit{reelle} tall er, må vi forklare hva \textit{imaginære} tall er, og det har vi valgt å ikke gjøre i denne boka. }
\eks[1]{
	$ \sqrt{2} $ er et irrasjonalt tall.
	\[ \sqrt{2}=1.414213562373... \]
}

\end{document}


