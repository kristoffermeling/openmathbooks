\documentclass[english, 11 pt, class=article, crop=false]{standalone}
\usepackage[T1]{fontenc}
%\renewcommand*\familydefault{\sfdefault} % For dyslexia-friendly text
\usepackage{lmodern} % load a font with all the characters
\usepackage{geometry}
\geometry{verbose,paperwidth=16.1 cm, paperheight=24 cm, inner=2.3cm, outer=1.8 cm, bmargin=2cm, tmargin=1.8cm}
\setlength{\parindent}{0bp}
\usepackage{import}
\usepackage[subpreambles=false]{standalone}
\usepackage{amsmath}
\usepackage{amssymb}
\usepackage{esint}
\usepackage{babel}
\usepackage{tabu}
\makeatother
\makeatletter

\usepackage{titlesec}
\usepackage{ragged2e}
\RaggedRight
\raggedbottom
\frenchspacing

% Norwegian names of figures, chapters, parts and content
\addto\captionsenglish{\renewcommand{\figurename}{Figur}}
\makeatletter
\addto\captionsenglish{\renewcommand{\chaptername}{Kapittel}}
\addto\captionsenglish{\renewcommand{\partname}{Del}}


\usepackage{graphicx}
\usepackage{float}
\usepackage{subfig}
\usepackage{placeins}
\usepackage{cancel}
\usepackage{framed}
\usepackage{wrapfig}
\usepackage[subfigure]{tocloft}
\usepackage[font=footnotesize,labelfont=sl]{caption} % Figure caption
\usepackage{bm}
\usepackage[dvipsnames, table]{xcolor}
\definecolor{shadecolor}{rgb}{0.105469, 0.613281, 1}
\colorlet{shadecolor}{Emerald!15} 
\usepackage{icomma}
\makeatother
\usepackage[many]{tcolorbox}
\usepackage{multicol}
\usepackage{stackengine}

\usepackage{esvect} %For vectors with capital letters

% For tabular
\usepackage{array}
\usepackage{multirow}
\usepackage{longtable} %breakable table

% Ligningsreferanser
\usepackage{mathtools}
\mathtoolsset{showonlyrefs}

% index
\usepackage{imakeidx}
\makeindex[title=Indeks]

%Footnote:
\usepackage[bottom, hang, flushmargin]{footmisc}
\usepackage{perpage} 
\MakePerPage{footnote}
\addtolength{\footnotesep}{2mm}
\renewcommand{\thefootnote}{\arabic{footnote}}
\renewcommand\footnoterule{\rule{\linewidth}{0.4pt}}
\renewcommand{\thempfootnote}{\arabic{mpfootnote}}

%colors
\definecolor{c1}{cmyk}{0,0.5,1,0}
\definecolor{c2}{cmyk}{1,0.25,1,0}
\definecolor{n3}{cmyk}{1,0.,1,0}
\definecolor{neg}{cmyk}{1,0.,0.,0}

% Lister med bokstavar
\usepackage[inline]{enumitem}

\newcounter{rg}
\numberwithin{rg}{chapter}
\newcommand{\reg}[2][]{\begin{tcolorbox}[boxrule=0.3 mm,arc=0mm,colback=blue!3] {\refstepcounter{rg}\phantomsection \large \textbf{\therg \;#1} \vspace{5 pt}}\newline #2  \end{tcolorbox}\vspace{-5pt}}

\newcommand\alg[1]{\begin{align} #1 \end{align}}

\newcommand\eks[2][]{\begin{tcolorbox}[boxrule=0.3 mm,arc=0mm,enhanced jigsaw,breakable,colback=green!3] {\large \textbf{Eksempel #1} \vspace{5 pt}\\} #2 \end{tcolorbox}\vspace{-5pt} }

\newcommand{\st}[1]{\begin{tcolorbox}[boxrule=0.0 mm,arc=0mm,enhanced jigsaw,breakable,colback=yellow!12]{ #1} \end{tcolorbox}}

\newcommand{\spr}[1]{\begin{tcolorbox}[boxrule=0.3 mm,arc=0mm,enhanced jigsaw,breakable,colback=yellow!7] {\large \textbf{Språkboksen} \vspace{5 pt}\\} #1 \end{tcolorbox}\vspace{-5pt} }

\newcommand{\sym}[1]{\colorbox{blue!15}{#1}}

\newcommand{\info}[2]{\begin{tcolorbox}[boxrule=0.3 mm,arc=0mm,enhanced jigsaw,breakable,colback=cyan!6] {\large \textbf{#1} \vspace{5 pt}\\} #2 \end{tcolorbox}\vspace{-5pt} }

\newcommand\algv[1]{\vspace{-11 pt}\begin{align*} #1 \end{align*}}

\newcommand{\regv}{\vspace{5pt}}
\newcommand{\mer}{\textsl{Merk}: }
\newcommand{\mers}[1]{{\footnotesize \mer #1}}
\newcommand\vsk{\vspace{11pt}}
\newcommand\vs{\vspace{-11pt}}
\newcommand\vsb{\vspace{-16pt}}
\newcommand\sv{\vsk \textbf{Svar} \vspace{4 pt}\\}
\newcommand\br{\\[5 pt]}
\newcommand{\figp}[1]{../fig/#1}
\newcommand\algvv[1]{\vs\vs\begin{align*} #1 \end{align*}}
\newcommand{\y}[1]{$ {#1} $}
\newcommand{\os}{\\[5 pt]}
\newcommand{\prbxl}[2]{
\parbox[l][][l]{#1\linewidth}{#2
	}}
\newcommand{\prbxr}[2]{\parbox[r][][l]{#1\linewidth}{
		\setlength{\abovedisplayskip}{5pt}
		\setlength{\belowdisplayskip}{5pt}	
		\setlength{\abovedisplayshortskip}{0pt}
		\setlength{\belowdisplayshortskip}{0pt} 
		\begin{shaded}
			\footnotesize	#2 \end{shaded}}}

\renewcommand{\cfttoctitlefont}{\Large\bfseries}
\setlength{\cftaftertoctitleskip}{0 pt}
\setlength{\cftbeforetoctitleskip}{0 pt}

\newcommand{\bs}{\\[3pt]}
\newcommand{\vn}{\\[6pt]}
\newcommand{\fig}[1]{\begin{figure}
		\centering
		\includegraphics[]{\figp{#1}}
\end{figure}}

\newcommand{\figc}[2]{\begin{figure}
		\centering
		\includegraphics[]{\figp{#1}}
		\caption{#2}
\end{figure}}

\newcommand{\sectionbreak}{\clearpage} % New page on each section

\newcommand{\nn}[1]{
\begin{equation}
	#1
\end{equation}
}

% Equation comments
\newcommand{\cm}[1]{\llap{\color{blue} #1}}

\newcommand\fork[2]{\begin{tcolorbox}[boxrule=0.3 mm,arc=0mm,enhanced jigsaw,breakable,colback=yellow!7] {\large \textbf{#1 (forklaring)} \vspace{5 pt}\\} #2 \end{tcolorbox}\vspace{-5pt} }
 
%colors
\newcommand{\colr}[1]{{\color{red} #1}}
\newcommand{\colb}[1]{{\color{blue} #1}}
\newcommand{\colo}[1]{{\color{orange} #1}}
\newcommand{\colc}[1]{{\color{cyan} #1}}
\definecolor{projectgreen}{cmyk}{100,0,100,0}
\newcommand{\colg}[1]{{\color{projectgreen} #1}}

% Methods
\newcommand{\metode}[2]{
	\textsl{#1} \\[-8pt]
	\rule{#2}{0.75pt}
}

%Opg
\newcommand{\abc}[1]{
	\begin{enumerate}[label=\alph*),leftmargin=18pt]
		#1
	\end{enumerate}
}
\newcommand{\abcs}[2]{
	\begin{enumerate}[label=\alph*),start=#1,leftmargin=18pt]
		#2
	\end{enumerate}
}
\newcommand{\abcn}[1]{
	\begin{enumerate}[label=\arabic*),leftmargin=18pt]
		#1
	\end{enumerate}
}
\newcommand{\abch}[1]{
	\hspace{-2pt}	\begin{enumerate*}[label=\alph*), itemjoin=\hspace{1cm}]
		#1
	\end{enumerate*}
}
\newcommand{\abchs}[2]{
	\hspace{-2pt}	\begin{enumerate*}[label=\alph*), itemjoin=\hspace{1cm}, start=#1]
		#2
	\end{enumerate*}
}

% Oppgaver
\newcommand{\opgt}{\phantomsection \addcontentsline{toc}{section}{Oppgaver} \section*{Oppgaver for kapittel \thechapter}\vs \setcounter{section}{1}}
\newcounter{opg}
\numberwithin{opg}{section}
\newcommand{\op}[1]{\vspace{15pt} \refstepcounter{opg}\large \textbf{\color{blue}\theopg} \vspace{2 pt} \label{#1} \\}
\newcommand{\ekspop}[1]{\vsk\textbf{Gruble \thechapter.#1}\vspace{2 pt} \\}
\newcommand{\nes}{\stepcounter{section}
	\setcounter{opg}{0}}
\newcommand{\opr}[1]{\vspace{3pt}\textbf{\ref{#1}}}
\newcommand{\oeks}[1]{\begin{tcolorbox}[boxrule=0.3 mm,arc=0mm,colback=white]
		\textit{Eksempel: } #1	  
\end{tcolorbox}}
\newcommand\opgeks[2][]{\begin{tcolorbox}[boxrule=0.1 mm,arc=0mm,enhanced jigsaw,breakable,colback=white] {\footnotesize \textbf{Eksempel #1} \\} \footnotesize #2 \end{tcolorbox}\vspace{-5pt} }
\newcommand{\rknut}{
Rekn ut.
}

%License
\newcommand{\lic}{\textit{Matematikken sine byggesteinar by Sindre Sogge Heggen is licensed under CC BY-NC-SA 4.0. To view a copy of this license, visit\\ 
		\net{http://creativecommons.org/licenses/by-nc-sa/4.0/}{http://creativecommons.org/licenses/by-nc-sa/4.0/}}}

%referances
\newcommand{\net}[2]{{\color{blue}\href{#1}{#2}}}
\newcommand{\hrs}[2]{\hyperref[#1]{\color{blue}\textsl{#2 \ref*{#1}}}}
\newcommand{\rref}[1]{\hrs{#1}{regel}}
\newcommand{\refkap}[1]{\hrs{#1}{kapittel}}
\newcommand{\refsec}[1]{\hrs{#1}{seksjon}}

\newcommand{\mb}{\net{https://sindrsh.github.io/FirstPrinciplesOfMath/}{MB}}


%line to seperate examples
\newcommand{\linje}{\rule{\linewidth}{1pt} }

\usepackage{datetime2}
%%\usepackage{sansmathfonts} for dyslexia-friendly math
\usepackage[]{hyperref}


\newcommand{\note}{Merk}
\newcommand{\notesm}[1]{{\footnotesize \textsl{\note:} #1}}
\newcommand{\ekstitle}{Eksempel }
\newcommand{\sprtitle}{Språkboksen}
\newcommand{\expl}{forklaring}

\newcommand{\vedlegg}[1]{\refstepcounter{vedl}\section*{Vedlegg \thevedl: #1}  \setcounter{vedleq}{0}}

\newcommand\sv{\vsk \textbf{Svar} \vspace{4 pt}\\}

%references
\newcommand{\reftab}[1]{\hrs{#1}{tabell}}
\newcommand{\rref}[1]{\hrs{#1}{regel}}
\newcommand{\dref}[1]{\hrs{#1}{definisjon}}
\newcommand{\refkap}[1]{\hrs{#1}{kapittel}}
\newcommand{\refsec}[1]{\hrs{#1}{seksjon}}
\newcommand{\refdsec}[1]{\hrs{#1}{delseksjon}}
\newcommand{\refved}[1]{\hrs{#1}{vedlegg}}
\newcommand{\eksref}[1]{\textsl{#1}}
\newcommand\fref[2][]{\hyperref[#2]{\textsl{figur \ref*{#2}#1}}}
\newcommand{\refop}[1]{{\color{blue}Oppgave \ref{#1}}}
\newcommand{\refops}[1]{{\color{blue}oppgave \ref{#1}}}
\newcommand{\refgrubs}[1]{{\color{blue}gruble \ref{#1}}}

\newcommand{\openmathser}{\openmath\,-\,serien}

% Exercises
\newcommand{\opgt}{\newpage \phantomsection \addcontentsline{toc}{section}{Oppgaver} \section*{Oppgaver for kapittel \thechapter}\vs \setcounter{section}{1}}


% Sequences and series
\newcommand{\sumarrek}{Summen av en aritmetisk rekke}
\newcommand{\sumgerek}{Summen av en geometrisk rekke}
\newcommand{\regnregsum}{Regneregler for summetegnet}

% Trigonometry
\newcommand{\sincoskomb}{Sinus og cosinus kombinert}
\newcommand{\cosfunk}{Cosinusfunksjonen}
\newcommand{\trid}{Trigonometriske identiteter}
\newcommand{\deravtri}{Den deriverte av de trigonometriske funksjonene}
% Solutions manual
\newcommand{\selos}{Se løsningsforslag.}
\newcommand{\se}[1]{Se eksempel på side \pageref{#1}}

%Vectors
\newcommand{\parvek}{Parallelle vektorer}
\newcommand{\vekpro}{Vektorproduktet}
\newcommand{\vekproarvol}{Vektorproduktet som areal og volum}


% 3D geometries
\newcommand{\linrom}{Linje i rommet}
\newcommand{\avstplnpkt}{Avstand mellom punkt og plan}


% Integral
\newcommand{\bestminten}{Bestemt integral I}
\newcommand{\anfundteo}{Analysens fundamentalteorem}
\newcommand{\intuf}{Integralet av utvalge funksjoner}
\newcommand{\bytvar}{Bytte av variabel}
\newcommand{\intvol}{Integral som volum}
\newcommand{\andordlindif}{Andre ordens lineære differensialligninger}



\begin{document}

\section{\ligintro}
Ethvert matematisk uttrykk som inneholder \sym{$=$} er en \outl{likning}\index{likning}, likevel er ordet \textit{likning} tradisjonelt knyttet til at vi har et \textsl{ukjent} tall.\vsk

Si at vi ønsker å finne et tall som er slik at hvis vi legger til $4$, så får vi $7$. Dette tallet kan vi kalle for hva som helst, men det vanligste er å kalle det for $ x $, som altså er det ukjente tallet vårt. Likningen vår kan nå skrives slik:
\[ x+4=7 \]
$ x $-verdien\footnote{I andre tilfeller kan det være flere verdier.} som gjør at det blir samme verdi på begge sider av likhets-tegnet kalles \outl{løsningen} av likningen.
Det er alltids lov til å se eller prøve seg fram for å finne verdien til $ x $. Kanskje har du allerede merket at $ {x=3} $ er løsningen av likningen, siden
\[ 3+4=7 \]
Men de fleste likninger er det vanskelig å se eller gjette seg fram til svaret på, og da må vi ty til mer generelle løsningsmetoder. Egentlig er det bare ett prinsipp vi følger: \regv
\st{\label{principle}Vi kan alltid utføre en matematisk operasjon på den ene siden av likhetstegnet, så lenge vi utfører den også på den andre siden.}
De matematiske operasjonene vi har presentert i denne boka er de fire rekneartene. Med disse lyder prinsippet slik:\regv
\st{Vi kan alltid legge til, trekke ifra, gange eller dele med et tall på den ene siden av likhetstegnet, så lenge vi gjør det også på den andre siden.}\regv
Prinsippet følger av betydningen til \sym{$=$}. Når to uttrykk har samme verdi, må de nødvendigvis fortsette å ha lik verdi, så lenge vi utfører de samme matematiske operasjonane på dem. I kommende seksjon skal vi likevel konkretisere dette prinsippet for hver enkelt rekneoperasjon, men hvis du føler dette allerede gir god mening kan du hoppe til \hrs{ligsaml}{seksjon}.
\section{\liglos \label{likloysfire}}
\textit{I figurene til denne seksjonen skal vi forstå likninger ut ifra et vektprinsipp. \sym{$=$} vil da indikere\footnote{\sym{$\neq$} er symbolet for ''er \textsl{ikke} lik''.} at det er like mye vekt (lik verdi) på venstre side som på høyre side.} \vspace{-5pt}
\begin{figure}
	\centering
	\includegraphics[]{\figp{lig}}\qquad
	\includegraphics[]{\figp{lig1}}
\end{figure}
\subsection*{Addisjon og subtraksjon; tall som skifter side}
\subsubsection*{Første eksempel}
Vi har allerede funnet løsningen på denne likningen, men la oss løse den på en annen måte\footnote{\textsl{Merk:} I tidligere figurer har det vært samsvar mellom størrelsen på rutene og tallverdien til tallet de symboliserer. Dette gjelder ikke rutene som representerer $ x $.}: \vspace{-3pt}
\[ x+4=7 \]
\fig{lig2}
Det blir tydelig hva verdien til $ x $ er hvis $ x $ står alene på en av sidene, og $ x $ blir isolert på venstresiden hvis vi tar bort 4. Men skal vi ta bort 4 fra venstresiden, må vi ta bort 4 fra høyresiden også, skal begge sidene ha samme verdi. \vs
\[ x+4-{\color{red}4}=7-{\color{red}4}  \]
\fig{lig2b}
Siden $ 4-{\color{red}4}=0 $ og $ 7-{\color{red}4}=3 $, får vi at \vspace{-3pt}
\[ x=3 \]
\fig{lig2c}
Dette kunne vi ha skrevet noe mer kortfattet slik:

\prbxl{0.5}{
	\alg{
		x+4 &= 7 \\
		x&= 7-4 \\
		x &= 3
}}
\prbxr{0.5}{Mellom første og andre linje er det vanlig å si at \textsl{4 har skiftet side, og derfor også fortegn (fra $ + $ til $ - $).}}
\textbf{Andre eksempel}\os
La oss gå videre til å se på en litt vanskeligere likning\footnote{Legg merke til at figuren illustrerer $ {4x+(-2)} $ (se \hrs{negmeng}{seksjon}) på venstre side. Men  $ {4x+(-2)} $ er det samme som $ {4x-2} $ (se \hrs{rekmneg}{seksjon}).}:
\[ 4x-2=3x+5 \]
\fig{lig4}
For å skaffe et uttrykk med $ x $ bare på én side, tar vi vekk $ 3x $ på begge sider:
\[ 4x-2-{\color{red}3x}=3x+5-{\color{red}3x} \]
\fig{lig5}
Da får vi at
\[ x-2=5 \]
\fig{lig5b}
For å isolere $ x $, legger vi til 2 på venstre side. Da må vi også legge til $ 2 $ på høyre side:
\[ x-2+{\color{blue}2}=5+{\color{blue}2} \]
\fig{lig6}
\newpage
Da får vi at
\[ x=7 \]
\fig{lig7}
Stegene vi har tatt kan oppsummeres slik:
\begin{flalign*}
&& 4x-2&=3x+5 && \llap{1. figur} \\
&& 4x-{\color{red}3x}-2&=3x-{\color{red}3x}+5 &&  \llap{2. figur} \\
&& x -2 &= 5 &&\llap{3. figur}\\
&& x-2+\color{blue}2&=  5+\color{blue}2 &&\llap{4. figur}\\
&& x &= 7 &&\llap{5. figur}
\end{flalign*}
Dette kan vi på en forenklet måte skrive slik:
\alg{
4x-2&=3x+5 \\
4x-{\color{red}3x}&=5+\color{blue}2\\
x &= 7
}

\reg[Flytting av tall over likhetstegnet \label{bytt}]{I en likning ønsker vi å samle alle $x$-ledd og alle kjente ledd på hver sin side av likhetstegnet. Skifter et ledd side, skifter det fortegn.}

\eks[1]{Løs likningen
\[ 3x+5 =2x+9 \]
\sv	
	 \vs \vs \vs \vs
	\begin{align*}
	3x-2x &=9-5 \\
	x &=4
	\end{align*}  \vspace{-20pt}}
\eks[2]{Løs likningen
\[ -4x-3 =-5x+12 \\ \]	
\sv
	 \vs \vs \vs \vs
	\begin{align*}
	-4x+5x &=12+3 \\
	x &=15
	\end{align*}}
\subsection*{Ganging og deling}	
\subsubsection{Deling}
Hittil har vi sett på likninger der vi endte opp med én $x $ på den ene siden av likhetstegnet. Ofte har vi flere $ x $-er, som for eksempel i likningen
\[ 3x=6 \]
\fig{lig8}
Deler vi venstresiden vår i tre like grupper, får vi én $ x $ i hver gruppe. Deler vi også høyresiden inn i tre like grupper, må alle gruppene ha den samme verdien
\[ \frac{3x}{3}=\frac{6}{3} \]
\fig{lig9}
Altså er
\[ x=2 \]
\fig{lig10}
La oss oppsummere utregningen vår:\\ 
\prbxl{0.6}{\begin{flalign*}
	&& 3x&=6 && \llap{1. figur} \br
	&& \frac{3x}{3}&=\frac{6}{3} && \llap{2. figur} \br
	&& x&=2 && \llap{3. figur}
	\end{flalign*}}\qquad\qquad
\prbxr{0.25}{Du husker kanskje at vi gjerne skriver
\[ \frac{\cancel{3}x}{\cancel{3}} \] 
}
\newpage
\reg[Deling på begge sider av en likning \label{ligdel}]{Vi kan dele begge sider av en likning med det samme tallet.}
\eks[1]{ Løs likningen\vspace{-3pt}
	\[ 	4x = 20  \]
	\sv \vs \vs \vsb
	\begin{align*}
	\frac{\cancel{4}x}{\cancel{4}}&=\frac{20}{4} \\
	x &=5
	\end{align*}
	\vspace{-20 pt}
	}
	
\eks[2]{Løs likningen \vspace{-3pt}
	\[2x+6 =3x-2 \]
\sv \vs \vs \vs \vs
	\begin{flalign*}
	&& 2x-3x &= -2-6 &&\\
	&&-x &= -8&& \\
	&& \frac{\cancel{-1}x}{\cancel{-1}} &= \frac{-8}{-1} &&\cm{($-x=-1x$)}\\
	&& x &= 8&&
	\end{flalign*}
}

\subsection*{Ganging}
Det siste tilfellet vi skal se på er når likninger inneholder brøkdeler av den ukjente, som for eksempel i likningen
\[ \frac{x}{3}=4 \]
\fig{lig11}
Vi kan få én $ x $ på venstresiden hvis vi legger til to eksemplar av $ \frac{x}{3} $. Likningen forteller oss at $ \frac{x}{3} $ har samme verdi som 4. Dette betyr at 
for hver $ \frac{x}{3} $ vi legger til på venstresiden, må vi legge til 4 på høyresiden, skal sidene ha samme verdi.
\[ \frac{x}{3}+\frac{x}{3}+\frac{x}{3}=4+4+4 \]
\fig{lig12}
Vi legger nå merke til at \y{\frac{x}{3}+\frac{x}{3}+\frac{x}{3}=\frac{x}{3}\cdot3} og at \y{4+4+4=4\cdot3}:
\[ \frac{x}{3}\cdot 3 = 4\cdot 3 \]
\fig{lig12b}
Og da $ \frac{x}{3}\cdot3=x $ og $ 4\cdot3=12 $, har vi at
\[ x=12 \]
\fig{lig13}
En oppsummering av stegene våre kan vi skrive slik:
\begin{flalign*}
&& \frac{x}{3}&=4 && \llap{1. figur} \br 
&& \frac{x}{3}+\frac{x}{3}+\frac{x}{3} &= 4+4+4  &&\llap{2. figur} \br
&& \frac{x}{3}\cdot 3&=4\cdot3 && \llap{3. figur} \\
&& x&=12 && \llap{4. figur}
\end{flalign*}
Dette kan vi kortere skrive som
\alg{
\frac{x}{3}&= 4 \br 
\frac{x}{\cancel{3}}\cdot \cancel{3} &= 4\cdot 3 \br
x &= 12
}
\newpage
\reg[Ganging på begge sider av en likning]{
Vi kan gange begge sider av en likning med det samme tallet.
}
\eks[1]{
Løs likningen
\[ \frac{x}{5}=2 \]
\sv \vsb \vs

\algv{
\frac{x}{\cancel{5}}\cdot\cancel{5}&=2\cdot5 \\
x &= 10
}
}
\eks[2]{
Løs likningen 
\[ \frac{7x}{10}-5=13+\frac{x}{10} \]
\sv \vsb \vs

\algv{
\frac{7x}{10}-\frac{x}{10}&=13+5\br
\frac{6x}{10}&=18 \br
\frac{6x}{\cancel{10}}\cdot \cancel{10}&=18\cdot10 \\
6x&=180 \br
\frac{\cancel{6}x}{\cancel{6}}&=\frac{180}{6}\br
x&=30
}
}
\newpage
\section{\ligloso \label{ligsaml}}
\reg[Løsningsmetoder for likninger \label{lsmlig}]{
For å løse en likning, er det ønskelig å isolere den ukjente på én side av likhetstegnet. For å få til dette kan vi alltid
\begin{itemize}
\item addere eller subtrahere begge sider av en likning med det samme tallet. 
Dette er ekvivalent til å flytte et ledd fra den ene siden av likningen til den andre, så lenge vi også skifter fortegn på leddet.
\item gange eller dele begge sider av en likning med det samme tallet.
\end{itemize} 
}
\eks[1]{
Løs likningen 
\[ 3x-4=6+2x \]

\sv \vs \vs

\algv{
3x-2x&=6+4 \\
x&=10
}
}
\eks[2]{
Løs likningen
\[ 9-7x=-8x+3 \]
\sv \vs \vs
\algv{
  8x-7x&=3-9 \\
  x&=-6
}
}
\newpage
\eks[3]{
	Løs likningen
	\[ 10x-20=7x-5 \]
	\sv \vs \vs
	\algv{
		10x-7x&=20-5 \\
		3x&=15 \\
		\frac{\cancel{3}x}{\cancel{3}}&=\frac{15}{3} \\
		x&=5
	}
}
\eks[4]{
Løs likningen
\[ 15-4x=x+5 \]

\sv \vs \vsb
\alg{
15-5&=x+4x \\
10&=5x\\
\frac{10}{5}&=\frac{\cancel{5}x}{\cancel{5}}\\
2&=x
}
{\footnotesize \mer I de andre eksemplene har vi valgt å samle $ x $-ene på venstre side av likningen, men vi kan likså gjerne samle dem på høgre side. Ved å gjøre det her har vi unngått utregninger med negative tall.}
}	

\eks[5]{
Løs likningen
\[ \frac{4x}{9}-20=8-\frac{3x}{9} \]

\sv \vs \vsb
\alg{
\frac{4x}{9}+\frac{3x}{9}&=20+8 \\
\frac{\cancel{7}x}{9\cdot \cancel{7}}&=\frac{28}{7} \br
\frac{x}{\cancel{9}}\cdot \cancel{9}&=4\cdot 9\\
x&=36
}
}

\newpage
\eks[6]{ Løs likningen
	\[ \frac{1}{3}x+\frac{1}{6}=\frac{5}{12}x+2  \vs\]	
	
	\sv
	For å unngå brøker, ganger vi begge sider med fellesnevneren 12:
	\begin{align}
	\left(\frac{1}{3}x+\frac{1}{6}\right)12&=\left(\frac{5}{12}x+2\right)12 \br	
	\frac{1}{3}x\cdot12+\frac{1}{6}\cdot12&=\frac{5}{12}x\cdot12+2\cdot12 \tag{$ \ast $}\br	
	4x+2 &= 5x+24 \\
	4x-5x &= 24-2 \\
	-x &= 22 \\
	\frac{\cancel{-1}\,x}{\cancel{-1}} &= \frac{22}{-1} \\
	x &= -22
	\end{align} 
}
\info{Tips}{
	Mange liker å lage seg en regel om at ''{vi kan gange eller dele alle ledd med det samme tallet}''. I eksempelet over kunne vi da hoppet direkte til andre linje i utrekningen.
}
\eks[7]{
Løs likningen
\[ 3-\frac{6}{x}= 2+\frac{5}{2x} \vs\]

\sv
Vi ganger begge sider med fellesnevneren $ 2x $:
\alg{
	 2x\left(3-\frac{6}{x}\right)&= 2x\left(2+\frac{5}{2x}\right) \br
6x -12 &= 4x+5\\
6x-4x &= 5+12\\
2x &= 17\\
x&=\frac{17}{2}
}
}
\section{Potenslikninger}
La oss løse likningen
\[ x^2=9 \]
Dette kalles en \textit{potenslikning}\index{potenslikning}. Potenslikninger er vanligvis vanskelige å løse bare ved hjelp av de fire regneartane, så her må vi også nytte oss av potensregler. Vi opphøyer begge sidene av likningen med den omvendte brøken\footnote{Husk at $ 2=\frac{2}{1} $.} til 2:
\[ \left(x^2\right)^\frac{1}{2}=9^\frac{1}{2} \]
Av \rref{potsomgrunn} er
\alg{
	x^{2\cdot\frac{1}{2}}&=9^\frac{1}{2} \\
	x&=9^\frac{1}{2}
}
Siden $ 3^2=9 $, er $ 9^\frac{1}{2}=3 $. Altså er $ {x=3} $ en løsning. For ordens skyld kan vi bekrefte dette med utregningen
\[ 3^2=3\cdot3=9 \]
Men vi har også at
\[ (-3)^2=(-3)(-3)=9 \]
Altså er $ -3 $ også en løsning av likningen vi startet med! \vsk

Nå legger vi merke til dette: \textsl{Prinsippet erklært på side \pageref{principle} sier at vi kan, som vi nå gjorde, utføre en matematisk operasjon på begge sider av likningen. Men, å følge dette prinsippet garanterer ikke at alle løsninger er funnet.}\regv

\reg[Potenslikninger]{
	En likning som kan bli skrevet som
	\[ x^a=b \]
	der $ a $ og $ b $ er konstanter,
	er en \outl{potenslikning}. \vsk
	
	Likningen har $ a $ forskjellige løsninger.
} 
\newpage
\eks[1]{
	Løs likningen
	\[ x^2+5= 21\]
	\sv \vs \vs \vs
	\algv{
		x^2+5&= 21\\
		x^2 &= 21-5\\
		x^2 &= 16
	}
	Siden $ {4\cdot4 =16} $ og $ {(-4)\cdot(-4)=16} $, har vi at
	\[ x=4\qquad\vee\qquad x=-4 \]
}
\eks[2]{
	Løs likningen
	\[ 3x^2+1=7 \]
	\sv \vs \vs \vs
	\alg{
		3x^2&=7-1 \\
		3x^2&=6 \\
		\frac{\cancel{3}x^2}{\cancel{3}}&=\frac{6}{3}\\
		x^2&=2}
	Altså er
	\[ x=\sqrt{2}\qquad\vee\qquad x=-\sqrt{2} \]
}
\info{Merk}{
	Selv om likningen
	\[ x^a=b \]
	har $ a $ løsninger, er ikke alle nødvendigvis \textsl{reelle}\footnote{Som nevnt, \textit{reelle} og \textit{imaginære} tall er noe vi ikke går nærmere inn på i denne boka}. I denne boka nøyer vi oss med å finne alle rasjonale eller irrasjonale tall som løser likningen. For eksempel har likningen
	\[ x^3=8 \]
	3 løsninger, men vi nøyer oss med å finne at $ {x=2} $ er en løsning.
}

\section{Ulikheter}
\subsection{Introduksjon}
Mens en likning viser to uttrykk som er like, vil en \outl{ulikhet}\index{ulikhet} vise to uttrykk som er ulike. For å skrive ulikheter har vi disse symbolene:\regv
\st{
	\renewcommand{\arraystretch}{1.2}
	\begin{tabular}{@{}cp{0.4cm}l}
		$ < $ && ''er mindre enn'' \\
		$ > $ && ''er større enn'' \\
		$ \leq $ && ''er mindre enn eller lik'' \\
		$ \geq $ && ''er større enn eller lik'' \\		
	\end{tabular}
	\renewcommand{\arraystretch}{1}
} \vsk

En ulikhet mellom to tall er avgjort av verdien og fortegnet til tallene:\regv

\reg[Ulikheter]{
Et positivt tall er større enn et negativt tal.\vsk

For tall med same fortegn, er det tallet med størst\\ absoluttverdi som er størst.\vsk

0 er større enn ethvert negativt tal og mindre enn ethvert positivt tal.	
}
\eks[1]{ \vsb
	\[ 9>8 \]
}
\eks[2]{ \vsb
	\[ -9<-8 \]
}
\eks[3]{ \vsb
	\[ -7<1 \]
}
\newpage
\eks[4]{ \vs
	\[ a-3\geq 1  \]
	Undersøk om ulikheten er sann hvis
	\abc{
		\item $ a=5 $
		\item $ a=4 $
		\item $ a=3 $
	}
	\sv
	\abc{
		\item Når $ a=5 $, har vi at
		\[ a-3=5-3 =2 \]
		Siden $ {2>1} $, er ulikheten sann.	
		\item Når $ a=4 $, har vi at
		\[ a-3=4-3=1 \]
		Siden $ {1=1} $, er ulikheten sann.	
		\item Når $ a=3 $, har vi at
		\[ a-3=3-3=0 \]
		Siden $ 0<1 $, er ulikheten usann.
	}
}
\newpage
\subsection{Løsing av ulikheter}
Vi kan løse ulikheter ved å bruke metodene fra \rref{lsmlig}, men med ett unntak; \textsl{om vi ganger eller deler med negative tall, skifter ulikheten symbol}. For å forklare kva som skjer, la oss bruke den enkle ulikheten 
\[ 9>8 \]
Om vi ganger begge sider av ulikheten med $ -1 $, får vi $ -9 $ på venstre side og $ -8 $ på høgre side. Men $ {-9 < -8}$. Symbolet fra vår opprinnelege ulikhet har altså endret seg fra \sym{>} til \sym{<}. \regv

\reg[Løsing av ulikheter]{
	Ulikheter kan løses på same måte som likninger, men med ett unntak: Hvis man ganger eller deler begge sider av en ulikhet med et negativt tal, vil $ \sym{>} $ endre seg til $ \sym{<} $, og omvendt.
}
\eks[1]{
	Løs ulikheten
	\[ 5x-3\geq2x+6 \]
	\sv \vsb \vsb 
	\alg{
		5x-3&\geq2x+6 \\
		3x&\geq 9 \\
		x &\geq 3
	}
}
\eks[2]{
	Løs ulikheten 
	\[ 4x-8\leq6x+12 \]
	\sv \vsb \vsb
	\alg{
		4x-8&\leq6x+12 \\
		-2x&\leq20 \\
		\frac{\cancel{-2}x}{\cancel{-2}}&\geq \frac{20}{-2} \\
		x&\geq -10
	}
	\mers{
	Her kan vi selvsagt unngå å dele med $ -2 $ ved å isolere $ x $-ene på høgre side av ulikheten i steden for venstre.	
	} 
}
\section{Likninger med flere ukjente; likningssett}
\spr{
Hvis vi har en likning med flere ukjente, bruker vi uttrykket ''å løse med hensyn på'' for å vise til hvilke av de ukjente vi isolerer på én side av likhetstegnet.
}
\eks[]{
Gitt likningen
\[ 4x+5y=3x+7y+10 \]
\abc{
\item Løs likningen med hensyn på $ x $.
\item Løs likningen med hensyn på $ y $.
}
\sv
\abc{
\item Vi bruker metodene beskrevet i \rref{lsmlig} for å løse likningen med hensyn på $ x $:
\alg{
4x-3x &=7y-5y+10 \\
x &= 2y+10
}
\item Vi bruker metodene beskrevet i \rref{lsmlig} for å løse likningen med hensyn på $ y $:
\alg{
4x-3x-10&=7y-5y \\
x-10&=2y \\
\frac{x-10}{2}&=\frac{2y}{2} \br
\frac{x}{2}-5&=y
}
}
}
\newpage
\subsubsection{Likningssett}
Hvis vi har to eller flere tall ukjente tall er det som regel slik at \regv

\st{\begin{itemize}
		\item er det to ukjente, trengs minst to likninger for å finne løsninger med bestemt verdi.
		\item er det tre ukjente, trengs minst tre likninger for å finne løsninger med bestemt verdi.
\end{itemize}}
Og slik fortsetter det. Likningene som gir oss den nødvendige informasjonen om de ukjente, kalles et \outl{likningssett}. I denne boka skal vi konsentrere oss om \textsl{lineære likninger med to ukjente}, som betyr at likningssettet består av uttrykk for lineære funksjoner\footnote{Se \refkap{Funksjoner}.}. For å løse likningssettene skal vi anvende følgende to metoder:\regv

\regdef[Innsettingsmetoden]{
	Et lineært likningssett bestående av to likninger med to ukjente, $ x $ og $ y $, kan løses ved å 
	\begin{enumerate}
		\item bruke den éne likningen til å finne et uttrykk for $ x $.
		\item sette uttrykket fra punkt 1 inn i den andre likningen, og løse denne med hensyn på $ y $.
		\item sette løsningen for $ y $ inn i uttrykket for $ x $.
	\end{enumerate}
	{\footnotesize \mer I punktene over kan selvsagt $ x $ og $ y $ bytte roller.}
} \regv
\regdef[Eliminasjonsmetoden]{
	Et lineært likningssett bestående av to likninger med to ukjente, $ x $ og $ y $, kan løses ved å (eventuelt) omskrive den éne likningen slik at den kan brukes til å eliminere $ x $ eller $ y $ i den andre likningen.
}
\newpage
\eks[1]{
	Løs likningssettet. \vs
	\alg{
		x-y&=5 \tag{I} \label{eks4a}\vn
		x+y&=9 \tag{II} \label{eks4b}
	}
	\sv
\metode{Innsettingsmetoden}{150pt}\os	
	Av \eqref{eks4a} har vi at 
	\algv{
		x-y &= 5 \\
		x&=5+y
	}
	Vi setter dette uttrykket for $ x $ inn i \eqref{eks4b}:
	\alg{
		5+y+y &=9 \\
		2y&=9-5 \\
		2y&=4 \\
		y&=2
	}
	Vi setter løsningen for $ y $ inn i uttrykket for $ x $:
	\alg{
		x&=5+y \\
		&=5+2 \\
		&=7
	}
	Altså er $ x=7 $ og $ y=2 $. \vsk
	
\metode{Eliminasjonssmetoden}{150pt}\\	
Vi legger sammen \eqref{eks4a} og \eqref{eks4b}, og får at
\algv{
x-y+(x+y)&=5+9 \\
2x&=14 \\
x&=7
}
Vi setter løsningen for $ x $ inn i én av likningene, i dette tilfellet \eqref{eks4b}:
\algv{
7+y&=9 \\
y&=2
}
Altså er $ {x=7} $ og $ {y=2} $
}
\newpage
\eks[2]{
	Løs likningssettet \vs
	\alg{
		7x-5y&=-8 \tag{I} \label{eks1a}\vn
		5x-2y&=4x-5 \tag{II} \label{eks1b}
	}
	\sv
	
	\metode{Innsettingsmetoden}{150pt}\os
	Ved innsettingsmetoden kan man ofte spare seg for en del utregning ved å velge likningen og den ukjente som gir det fineste uttrykket innledningsvis. Vi observerer at \eqref{eks1b} gir et fint uttrykk for $ x $:
	\alg{
		7x-5y&=-6 \\
		x&=2y-5
	}
	Vi setter uttrykket for $ x $ inn i \eqref{eks1a}:
	\alg{
		7x-5y&=-8\\
		7(2y-5)-5y&=-8 \\
		14y-35-5y&=-8 \\
		9y &=27 \\
		y&=3
	}
	Vi setter løsningen for $ y $ inn i uttrykket for $ x $:
	\[ x=2y-5=2\cdot 3-=1 \]
	Altså er $ x=1 $ og $ y=3 $.\vsk
	
\metode{Eliminasjonssmetoden}{150pt}\os
Vi starter med å omskrive \eqref{eks1b}:
\alg{
5x-4x-2y&=-5 \\
x-2y&=-5\\
7x-14y&=-35 \tag{II*} \label{eks1bstar}
}
Vi trekker \eqref{eks1bstar} fra \eqref{eks1a}:
\alg{
7x-5y-(7x-14y) &= -8-(-35) \\
9y &= 27
}	
Herfra er utregningene identiske med de vi så på for innsettingsmetoden.
}
\newpage
\begin{comment}
\eks[3]{
	Løs likningssettet \vs
	\alg{
		3x-4y&=-2 \tag{I} \label{eks2a}\vn
		9y-5x&=6x+y \tag{II} \label{eks2b}
	} \vs \vs
	
	\sv
	
	Vi velger her å bruke \eqref{eks1a} til å finne et uttrykk for $ y $:
	\alg{
		3x-4y&=-2 \\
		3x+2&=4y \\
		\frac{3x+2}{4}&=y
	}
	Vi setter dette uttrykket for $ y $ inn i \eqref{eks2b}:
	\alg{
		9y-5x&=6x+y \\	
		9\cdot\frac{3x+2}{4}-5x&=6x+\frac{3x+2}{4} \\
		9(3x+2)-20x&=24x+3x+2 \\
		27x+18-20x &=24x+3x+2 \\
		-20x&=-16 \\
		x&=\frac{4}{5}
	}
	Vi setter løsningen for $ x $ inn i uttrykket for $ y $:
	\alg{
		y&=\frac{3x+2}{4} \br
		&=\frac{3\cdot\frac{4}{5}+2}{4} \br
		&=\frac{\frac{22}{5}}{4} \\
		&=\frac{11}{10}
	}
	Altså er $ x=\frac{4}{5} $ og $ y=\frac{11}{10} $.
}
\end{comment}
\end{document}


