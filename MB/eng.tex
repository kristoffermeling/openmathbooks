
% note
\newcommand{\note}{Note}
\newcommand{\notesm}[1]{{\footnotesize \textsl{\note:} #1}}
\newcommand{\selos}{See the solutions manual.}

\newcommand{\texandasy}{The text is written in \LaTeX\ and the figures are made with the aid of Asymptote.}

\newcommand{\rknut}{Calculate.}
\newcommand\sv{\vsk \textbf{Answer} \vspace{4 pt}\\}
\newcommand{\ekstitle}{Example }
\newcommand{\sprtitle}{The language box}
\newcommand{\expl}{explanation}

% answers
\newcommand{\mulansw}{\notesm{Multiple possible answers.}}	

% references
\newcommand{\reftab}[1]{\hrs{#1}{Table}}
\newcommand{\rref}[1]{\hrs{#1}{Rule}}
\newcommand{\dref}[1]{\hrs{#1}{Definition}}
\newcommand{\refkap}[1]{\hrs{#1}{Chapter}}
\newcommand{\refsec}[1]{\hrs{#1}{Section}}
\newcommand{\refdsec}[1]{\hrs{#1}{Subsection}}
\newcommand{\refved}[1]{\hrs{#1}{Appendix}}
\newcommand{\eksref}[1]{\textsl{#1}}
\newcommand\fref[2][]{\hyperref[#2]{\textsl{Figure \ref*{#2}#1}}}
\newcommand{\refop}[1]{{\color{blue}Exercise \ref{#1}}}
\newcommand{\refops}[1]{{\color{blue}Exercise \ref{#1}}}

%%% SECTION HEADLINES %%%

% Our numbers
\newcommand{\likteikn}{The equal sign}
\newcommand{\talsifverd}{Numbers, digits and values}
\newcommand{\koordsys}{Coordinate systems}

% Calculations
\newcommand{\adi}{Addition}
\newcommand{\sub}{Subtraction}
\newcommand{\gong}{Multiplication}
\newcommand{\del}{Division}

%Factorization and order of operations
\newcommand{\fak}{Factorization}
\newcommand{\rrek}{Order of operations}

%Fractions
\newcommand{\brgrpr}{Introduction}
\newcommand{\brvu}{Values, expanding and simplifying}
\newcommand{\bradsub}{Addition and subtraction}
\newcommand{\brgngheil}{Fractions multiplied by integers}
\newcommand{\brdelheil}{Fractions divided by integers}
\newcommand{\brgngbr}{Fractions multiplied by fractions}
\newcommand{\brkans}{Cancelation of fractions}
\newcommand{\brdelmbr}{Division by fractions}
\newcommand{\Rasjtal}{Rational numbers}

%Negative numbers
\newcommand{\negintro}{Introduction}
\newcommand{\negrekn}{The elementary operations}
\newcommand{\negmeng}{Negative numbers as amounts}

%Calculation methods
\newcommand{\delmedtihundre}{Deling med 10, 100, 1\,000 osv.}

% Geometry 1
\newcommand{\omgr}{Terms}
\newcommand{\eignsk}{Attributes of triangles and quadrilaterals}
\newcommand{\omkr}{Perimeter}
\newcommand{\area}{Area}

%Algebra 
\newcommand{\algintro}{Introduction}
\newcommand{\pot}{Powers}
\newcommand{\irrasj}{Irrational numbers}

%Equations
\newcommand{\ligintro}{Introduction}
\newcommand{\liglos}{Solving with the elementary operations}
\newcommand{\ligloso}{Solving with elementary operations summarized}

%Functions
\newcommand{\fintro}{Introduction}
\newcommand{\lingraf}{Linear functions and graphs}

%Geometry 2
\newcommand{\geoform}{Formulas of area and perimeter}
\newcommand{\kongogsim}{Congruent and similar triangles}
\newcommand{\geofork}{Explanations}

% Names of rules
\newcommand{\adkom}{Addition is commutative}
\newcommand{\gangkom}{Multiplication is commutative}
\newcommand{\brdef}{Fractions as rewriting of division}
\newcommand{\brtbr}{Fractions multiplied by fractions}
\newcommand{\delmbr}{Fractions divided by fractions}
\newcommand{\gangpar}{Distributive law}
\newcommand{\gangparsam}{Paranthesis multiplied together}
\newcommand{\gangmnegto}{Multiplication by negative numbers I}
\newcommand{\gangmnegtre}{Multiplication by negative numbers II}
\newcommand{\konsttre}{Unique construction of triangles}
\newcommand{\kongtre}{Congruent triangles}
\newcommand{\topv}{Vertical angles}
\newcommand{\trisum}{The sum of angles in a triangle}
\newcommand{\firsum}{The sum of angles in a quadrilateral}
\newcommand{\potgang}{Multiplication by powers}
\newcommand{\potdivpot}{Division by powers}
\newcommand{\potanull}{The special case of \boldmath $a^0$}
\newcommand{\potneg}{Powers with negative exponents}
\newcommand{\potbr}{Fractions as base}
\newcommand{\faktgr}{Factors as base}
\newcommand{\potsomgrunn}{Powers as base}
\newcommand{\arsirk}{The area of a circle}
\newcommand{\artrap}{The area of a trapezoid}
\newcommand{\arpar}{The area of a parallelogram}
\newcommand{\pyt}{Pythagoras's theorem}
\newcommand{\forform}{Ratios in similar triangles}
\newcommand{\vilkform}{Terms of similar triangles}
\newcommand{\omkrsirk}{The perimeter of a circle (and the value of $ \bm \pi $)}
\newcommand{\artri}{The area of a triangle}
\newcommand{\arrekt}{The area of a rectangle}
\newcommand{\liknflyt}{Moving terms across the equal sign}
\newcommand{\funklin}{Linear functions}

