\documentclass[english, 11 pt, class=article, crop=false]{standalone}
\usepackage[T1]{fontenc}
%\renewcommand*\familydefault{\sfdefault} % For dyslexia-friendly text
\usepackage{lmodern} % load a font with all the characters
\usepackage{geometry}
\geometry{verbose,paperwidth=16.1 cm, paperheight=24 cm, inner=2.3cm, outer=1.8 cm, bmargin=2cm, tmargin=1.8cm}
\setlength{\parindent}{0bp}
\usepackage{import}
\usepackage[subpreambles=false]{standalone}
\usepackage{amsmath}
\usepackage{amssymb}
\usepackage{esint}
\usepackage{babel}
\usepackage{tabu}
\makeatother
\makeatletter

\usepackage{titlesec}
\usepackage{ragged2e}
\RaggedRight
\raggedbottom
\frenchspacing

% Norwegian names of figures, chapters, parts and content
\addto\captionsenglish{\renewcommand{\figurename}{Figur}}
\makeatletter
\addto\captionsenglish{\renewcommand{\chaptername}{Kapittel}}
\addto\captionsenglish{\renewcommand{\partname}{Del}}


\usepackage{graphicx}
\usepackage{float}
\usepackage{subfig}
\usepackage{placeins}
\usepackage{cancel}
\usepackage{framed}
\usepackage{wrapfig}
\usepackage[subfigure]{tocloft}
\usepackage[font=footnotesize,labelfont=sl]{caption} % Figure caption
\usepackage{bm}
\usepackage[dvipsnames, table]{xcolor}
\definecolor{shadecolor}{rgb}{0.105469, 0.613281, 1}
\colorlet{shadecolor}{Emerald!15} 
\usepackage{icomma}
\makeatother
\usepackage[many]{tcolorbox}
\usepackage{multicol}
\usepackage{stackengine}

\usepackage{esvect} %For vectors with capital letters

% For tabular
\usepackage{array}
\usepackage{multirow}
\usepackage{longtable} %breakable table

% Ligningsreferanser
\usepackage{mathtools}
\mathtoolsset{showonlyrefs}

% index
\usepackage{imakeidx}
\makeindex[title=Indeks]

%Footnote:
\usepackage[bottom, hang, flushmargin]{footmisc}
\usepackage{perpage} 
\MakePerPage{footnote}
\addtolength{\footnotesep}{2mm}
\renewcommand{\thefootnote}{\arabic{footnote}}
\renewcommand\footnoterule{\rule{\linewidth}{0.4pt}}
\renewcommand{\thempfootnote}{\arabic{mpfootnote}}

%colors
\definecolor{c1}{cmyk}{0,0.5,1,0}
\definecolor{c2}{cmyk}{1,0.25,1,0}
\definecolor{n3}{cmyk}{1,0.,1,0}
\definecolor{neg}{cmyk}{1,0.,0.,0}

% Lister med bokstavar
\usepackage[inline]{enumitem}

\newcounter{rg}
\numberwithin{rg}{chapter}
\newcommand{\reg}[2][]{\begin{tcolorbox}[boxrule=0.3 mm,arc=0mm,colback=blue!3] {\refstepcounter{rg}\phantomsection \large \textbf{\therg \;#1} \vspace{5 pt}}\newline #2  \end{tcolorbox}\vspace{-5pt}}

\newcommand\alg[1]{\begin{align} #1 \end{align}}

\newcommand\eks[2][]{\begin{tcolorbox}[boxrule=0.3 mm,arc=0mm,enhanced jigsaw,breakable,colback=green!3] {\large \textbf{Eksempel #1} \vspace{5 pt}\\} #2 \end{tcolorbox}\vspace{-5pt} }

\newcommand{\st}[1]{\begin{tcolorbox}[boxrule=0.0 mm,arc=0mm,enhanced jigsaw,breakable,colback=yellow!12]{ #1} \end{tcolorbox}}

\newcommand{\spr}[1]{\begin{tcolorbox}[boxrule=0.3 mm,arc=0mm,enhanced jigsaw,breakable,colback=yellow!7] {\large \textbf{Språkboksen} \vspace{5 pt}\\} #1 \end{tcolorbox}\vspace{-5pt} }

\newcommand{\sym}[1]{\colorbox{blue!15}{#1}}

\newcommand{\info}[2]{\begin{tcolorbox}[boxrule=0.3 mm,arc=0mm,enhanced jigsaw,breakable,colback=cyan!6] {\large \textbf{#1} \vspace{5 pt}\\} #2 \end{tcolorbox}\vspace{-5pt} }

\newcommand\algv[1]{\vspace{-11 pt}\begin{align*} #1 \end{align*}}

\newcommand{\regv}{\vspace{5pt}}
\newcommand{\mer}{\textsl{Merk}: }
\newcommand{\mers}[1]{{\footnotesize \mer #1}}
\newcommand\vsk{\vspace{11pt}}
\newcommand\vs{\vspace{-11pt}}
\newcommand\vsb{\vspace{-16pt}}
\newcommand\sv{\vsk \textbf{Svar} \vspace{4 pt}\\}
\newcommand\br{\\[5 pt]}
\newcommand{\figp}[1]{../fig/#1}
\newcommand\algvv[1]{\vs\vs\begin{align*} #1 \end{align*}}
\newcommand{\y}[1]{$ {#1} $}
\newcommand{\os}{\\[5 pt]}
\newcommand{\prbxl}[2]{
\parbox[l][][l]{#1\linewidth}{#2
	}}
\newcommand{\prbxr}[2]{\parbox[r][][l]{#1\linewidth}{
		\setlength{\abovedisplayskip}{5pt}
		\setlength{\belowdisplayskip}{5pt}	
		\setlength{\abovedisplayshortskip}{0pt}
		\setlength{\belowdisplayshortskip}{0pt} 
		\begin{shaded}
			\footnotesize	#2 \end{shaded}}}

\renewcommand{\cfttoctitlefont}{\Large\bfseries}
\setlength{\cftaftertoctitleskip}{0 pt}
\setlength{\cftbeforetoctitleskip}{0 pt}

\newcommand{\bs}{\\[3pt]}
\newcommand{\vn}{\\[6pt]}
\newcommand{\fig}[1]{\begin{figure}
		\centering
		\includegraphics[]{\figp{#1}}
\end{figure}}

\newcommand{\figc}[2]{\begin{figure}
		\centering
		\includegraphics[]{\figp{#1}}
		\caption{#2}
\end{figure}}

\newcommand{\sectionbreak}{\clearpage} % New page on each section

\newcommand{\nn}[1]{
\begin{equation}
	#1
\end{equation}
}

% Equation comments
\newcommand{\cm}[1]{\llap{\color{blue} #1}}

\newcommand\fork[2]{\begin{tcolorbox}[boxrule=0.3 mm,arc=0mm,enhanced jigsaw,breakable,colback=yellow!7] {\large \textbf{#1 (forklaring)} \vspace{5 pt}\\} #2 \end{tcolorbox}\vspace{-5pt} }
 
%colors
\newcommand{\colr}[1]{{\color{red} #1}}
\newcommand{\colb}[1]{{\color{blue} #1}}
\newcommand{\colo}[1]{{\color{orange} #1}}
\newcommand{\colc}[1]{{\color{cyan} #1}}
\definecolor{projectgreen}{cmyk}{100,0,100,0}
\newcommand{\colg}[1]{{\color{projectgreen} #1}}

% Methods
\newcommand{\metode}[2]{
	\textsl{#1} \\[-8pt]
	\rule{#2}{0.75pt}
}

%Opg
\newcommand{\abc}[1]{
	\begin{enumerate}[label=\alph*),leftmargin=18pt]
		#1
	\end{enumerate}
}
\newcommand{\abcs}[2]{
	\begin{enumerate}[label=\alph*),start=#1,leftmargin=18pt]
		#2
	\end{enumerate}
}
\newcommand{\abcn}[1]{
	\begin{enumerate}[label=\arabic*),leftmargin=18pt]
		#1
	\end{enumerate}
}
\newcommand{\abch}[1]{
	\hspace{-2pt}	\begin{enumerate*}[label=\alph*), itemjoin=\hspace{1cm}]
		#1
	\end{enumerate*}
}
\newcommand{\abchs}[2]{
	\hspace{-2pt}	\begin{enumerate*}[label=\alph*), itemjoin=\hspace{1cm}, start=#1]
		#2
	\end{enumerate*}
}

% Oppgaver
\newcommand{\opgt}{\phantomsection \addcontentsline{toc}{section}{Oppgaver} \section*{Oppgaver for kapittel \thechapter}\vs \setcounter{section}{1}}
\newcounter{opg}
\numberwithin{opg}{section}
\newcommand{\op}[1]{\vspace{15pt} \refstepcounter{opg}\large \textbf{\color{blue}\theopg} \vspace{2 pt} \label{#1} \\}
\newcommand{\ekspop}[1]{\vsk\textbf{Gruble \thechapter.#1}\vspace{2 pt} \\}
\newcommand{\nes}{\stepcounter{section}
	\setcounter{opg}{0}}
\newcommand{\opr}[1]{\vspace{3pt}\textbf{\ref{#1}}}
\newcommand{\oeks}[1]{\begin{tcolorbox}[boxrule=0.3 mm,arc=0mm,colback=white]
		\textit{Eksempel: } #1	  
\end{tcolorbox}}
\newcommand\opgeks[2][]{\begin{tcolorbox}[boxrule=0.1 mm,arc=0mm,enhanced jigsaw,breakable,colback=white] {\footnotesize \textbf{Eksempel #1} \\} \footnotesize #2 \end{tcolorbox}\vspace{-5pt} }
\newcommand{\rknut}{
Rekn ut.
}

%License
\newcommand{\lic}{\textit{Matematikken sine byggesteinar by Sindre Sogge Heggen is licensed under CC BY-NC-SA 4.0. To view a copy of this license, visit\\ 
		\net{http://creativecommons.org/licenses/by-nc-sa/4.0/}{http://creativecommons.org/licenses/by-nc-sa/4.0/}}}

%referances
\newcommand{\net}[2]{{\color{blue}\href{#1}{#2}}}
\newcommand{\hrs}[2]{\hyperref[#1]{\color{blue}\textsl{#2 \ref*{#1}}}}
\newcommand{\rref}[1]{\hrs{#1}{regel}}
\newcommand{\refkap}[1]{\hrs{#1}{kapittel}}
\newcommand{\refsec}[1]{\hrs{#1}{seksjon}}

\newcommand{\mb}{\net{https://sindrsh.github.io/FirstPrinciplesOfMath/}{MB}}


%line to seperate examples
\newcommand{\linje}{\rule{\linewidth}{1pt} }

\usepackage{datetime2}
%%\usepackage{sansmathfonts} for dyslexia-friendly math
\usepackage[]{hyperref}


\begin{document}
\opgt
\op{opgbrverdidelelig}
Finn verdien til brøken.\os
\abch{
\item $ \dfrac{18}{3} $	
\item $ \dfrac{20}{4} $
\item $ \dfrac{10}{5} $
\item $ \dfrac{42}{6} $
\item $ \dfrac{63}{7} $
\item $ \dfrac{32}{8} $	
}

\op{opgbrverdiikkedelelig}
Finn verdien til brøken. Bruk kalkulator om nødvendig.\os
\abch{
\item $ \dfrac{1}{2} $ 	
\item $ \dfrac{1}{4} $ 	
\item $ \dfrac{1}{5} $
\item $ \dfrac{3}{4} $
\item $ \dfrac{2}{5} $
\item $ \dfrac{3}{5} $
\item $ \dfrac{4}{5} $
} \vsk

\abchs{6}{
\item $ \dfrac{3}{2} $
\ \item $ \dfrac{1}{3} $
\item $ \dfrac{5}{2} $
\item $ \dfrac{5}{6} $
\item $ \dfrac{7}{5} $
\item $ \dfrac{11}{4} $
\item $ \dfrac{7}{10} $
}

\op{opgbrtallin}
Skriv brøken markert med raudt.
\abc{
	\item \includegraphics{\figp{bropg1}}
	\item \includegraphics{\figp{bropg2}}
	\item \includegraphics{\figp{bropg3}}
}

\op{opgbrtallin2}
Skriv brøken markert med raudt.
\abc{
	\item \includegraphics{\figp{bropg4}}
	\item \includegraphics{\figp{bropg5}}
	\item \includegraphics{\figp{bropg6}}
}
\nes
\newpage
\op{opgbrutvid}
\opgeks[]{ \vs
\[ \frac{9}{8}\text{ utvida med \colb{3}}=\frac{9\cdot\colb{3}}{8\cdot\colb{3}}=\frac{27}{24} \] 
} \vsk

Utvid\os
\abch{
\item $ \dfrac{10}{3} $ med 2.	
\item $ \dfrac{3}{4} $ med 3.
\item $ \dfrac{3}{7} $ med 4.
} \vsk

\abchs{3}{
\item $ \dfrac{9}{8} $ med 5.
\ \ \item $ \dfrac{9}{5} $ med 6.
\item $ \dfrac{11}{4} $ med 7.
}

\op{opgbrutvidtil}
Utvid
\abc{
	\item $ \frac{7}{3} $ til ein brøk med 15 som nemnar.
	\item $ \frac{3}{4} $ til ein brøk med 32 som nemnar.
	\item $ \frac{10}{9} $ til ein brøk med 63 som nemnar.
}

\op{opgbrfork}
\opgeks[]{ \vs
	\[ \frac{10}{8}\text{ forkorta med \colb{2}}=\frac{10:\colb{2}}{8:\colb{2}}=\frac{5}{4} \] 
} \vsk

Forkort\os
\abch{
	\item $ \dfrac{14}{26} $ med 2.	
	\item $ \dfrac{15}{12} $ med 3.
	\item $ \dfrac{20}{16} $ med 4.
} \vsk

\abchs{3}{
	\item $ \dfrac{35}{50} $ med 5.
	\item $ \dfrac{54}{18} $ med 6.
	\item $ \dfrac{49}{63} $ med 7.
}

\op{opgbrforktil}
Forkort
\abc{
	\item $ \frac{27}{12} $ til en brøk med 4 som nemnar.
	\item $ \frac{36}{20} $ til en brøk med 5 som nemnar.
	\item $ \frac{18}{63} $ til en brøk med 7 som nemnar.
}
\nes
\newpage

\op{opgbrad}
\rknut \os
\abch{
\item $ \displaystyle \frac{4}{3}+\frac{6}{3}$
\item $ \displaystyle \frac{5}{4}+\frac{9}{4}$
\item $ \displaystyle \frac{1}{6}+\frac{10}{6}$
\item $ \displaystyle \frac{8}{7}+\frac{2}{7}$
\item $ \displaystyle \frac{1}{2}+\frac{1}{2}$
}

\op{opgbrad2}
\rknut \os
\abch{
	\item $ \displaystyle \frac{10}{3}+\frac{4}{3}+\frac{8}{3}$
	\item $ \displaystyle \frac{4}{5}+\frac{3}{5}+\frac{1}{5}$
	\item $ \displaystyle \frac{11}{7}+\frac{2}{7}+\frac{4}{7}$		
}

\op{opgbrsub}
\rknut \os
\abch{
	\item $ \displaystyle \frac{5}{3}-\frac{4}{3}$
	\item $ \displaystyle \frac{9}{4}-\frac{7}{4}$
	\item $ \displaystyle \frac{11}{6}-\frac{1}{6}$
	\item $ \displaystyle \frac{11}{7}-\frac{4}{7}$
	\item $ \displaystyle \frac{1}{2}-\frac{1}{2}$
}

\op{opgbradandsub}
\rknut \os
\abch{
	\item $ \displaystyle \frac{4}{5}+\frac{3}{5}-\frac{1}{5}$
	\item $ \displaystyle \frac{11}{7}-\frac{2}{7}-\frac{4}{7}$
	\item $ \displaystyle \frac{10}{3}-\frac{4}{3}+\frac{8}{3}$		
}

\op{opgbrad3}
\rknut \os
\abch{
\item $\displaystyle \frac{2}{5}+\frac{3}{6} $
\item $\displaystyle \frac{5}{7}+\frac{4}{9} $
\item $\displaystyle \frac{10}{3}+\frac{7}{8} $
\item $\displaystyle \frac{7}{5}+\frac{9}{4} $
\item $\displaystyle \frac{1}{3}+\frac{1}{2} $
}

\op{opgbrsub2}
\rknut \os
\abch{
	\item $\displaystyle \frac{2}{5}-\frac{3}{10} $
	\item $\displaystyle \frac{5}{4}-\frac{4}{9} $
	\item $\displaystyle \frac{10}{9}-\frac{1}{8} $
	\item $\displaystyle \frac{4}{5}-\frac{1}{4} $
	\item $\displaystyle \frac{5}{2}-\frac{5}{3} $
}

\op{opgbradandsubmix}
\rknut \os
\abch{
\item $\displaystyle \frac{2}{3}+\frac{1}{2}-\frac{3}{4} $	
\item $\displaystyle \frac{10}{2}-\frac{1}{6}+\frac{2}{5} $
\item $\displaystyle \frac{9}{2}-\frac{2}{7}-\frac{1}{8} $
}
\nes
\newpage

\op{opgbrgongheil}
\rknut \os
\abch{
	\item $ \displaystyle \frac{4}{3}\cdot5 $
	\item $ \displaystyle \frac{5}{7}\cdot8 $
	\item $ \displaystyle \frac{9}{10}\cdot6 $
	\item $ \displaystyle \frac{8}{7}\cdot10 $
	\item $ \displaystyle \frac{3}{2}\cdot7 $
} \vsk

\abchs{6}{
	\item $ \displaystyle 7\cdot\frac{4}{3} $
	\ \item $ \displaystyle 5\cdot\frac{7}{3} $
	\item $ \displaystyle 3\cdot\frac{10}{7} $
	\item $ \displaystyle 1\cdot\frac{5}{11} $
	\item $ \displaystyle 8\cdot\frac{9}{17} $
}


\nes
\op{opgbrdelheil}
\rknut\os
\abch{
	\item $ \displaystyle \frac{4}{3}:5 $
	\ \ \item $ \displaystyle \frac{5}{7}:8 $
\ 	\item $ \displaystyle \frac{9}{10}:6 $
	\item $ \displaystyle \frac{8}{7}:10 $
	\item $ \displaystyle \frac{3}{2}:7 $
} \\[8pt]

\abchs{6}{
	\item $ \displaystyle \frac{9}{10}:11 $
	\item $ \displaystyle \frac{1}{5}:12 $
	\item $ \displaystyle \frac{9}{10}:29 $
	\item $ \displaystyle \frac{8}{9}:51 $
	\item $ \displaystyle \frac{3}{2}:79 $
}

\nes
\op{opgbrgongbr}
\rknut \os
\abch{
	\item $\displaystyle \frac{4}{3}\cdot\frac{5}{9} $
	\item $\displaystyle \frac{7}{8}\cdot\frac{1}{4} $
	\item $\displaystyle \frac{2}{7}\cdot\frac{9}{3} $
	\item $\displaystyle \frac{10}{3}\cdot\frac{6}{5} $
	\item $\displaystyle \frac{3}{2}\cdot\frac{7}{5} $
} \vsk

\abchs{6}{
	\item $ \dfrac{2}{3}\cdot\dfrac{5}{7} $
	\item $ \dfrac{8}{9}\cdot\dfrac{2}{3} $
	\item $ \dfrac{10}{3}\cdot\dfrac{8}{3} $
	\item $ \dfrac{4}{5}\cdot\dfrac{9}{7} $
	\item $ \dfrac{7}{2}\cdot\dfrac{5}{6} $
}

\nes
\op{opgbrgongbr2}
\rknut \os
\abch{
\item $\displaystyle \frac{3}{10}\cdot\frac{5}{4} $
\item $\displaystyle \frac{17}{8}\cdot\frac{9}{4} $
\item $\displaystyle \frac{23}{8}\cdot\frac{2}{4} $
\item $\displaystyle \frac{7}{81}\cdot\frac{3}{8} $
\item $\displaystyle \frac{7}{8}\cdot\frac{29}{41} $
}

\nes

\op{kansfakt}
Kanseller så mange faktorer som mulig i brøken.\os 
\abch{
	\item $ \dfrac{3\cdot11\cdot8}{4\cdot8\cdot3}$
	\item $ \dfrac{5\cdot12\cdot7\cdot2}{2\cdot8\cdot12}$
	\item $ \dfrac{6\cdot10}{6\cdot9\cdot10}$
	\item $ \dfrac{7\cdot4\cdot3}{7\cdot3}$
}
\newpage
\op{opgbrforkortmfaktor}
Forkort brøken så mykje som mogleg.\os
\abch{
	\item $ \dfrac{28}{16} $
	\item $ \dfrac{18}{42} $
	\item $ \dfrac{24}{36} $
	\item $ \dfrac{56}{49} $
	\item $ \dfrac{25}{50} $
	\item $ \dfrac{21}{14} $
}

\op{opgbrgongbrfakt}
\opgeks[1]{ \vs
\[ \frac{3}{\colb{4}}\cdot 20=\frac{3}{\cancel{\colb{4}}}\cdot \cancel{\colb{4}}\cdot5=3\cdot 5=15 \]
} \regv

Utnytt at nemnaren er ein faktor i talet det blir gonga med, og rekn ut\os
\abch{
\item $ \dfrac{7}{3}\cdot 21 $
\item $ \dfrac{9}{5}\cdot 30 $
\item $ \dfrac{10}{7}\cdot 49 $
\item $ \dfrac{8}{9}\cdot 18 $
\item $ \dfrac{5}{4}\cdot 24 $
} \os
\abchs{6}{
\item $ 8\cdot\dfrac{3}{2} $
\ \ \item $ 35\cdot\dfrac{5}{7} $
\item $ 63\cdot\dfrac{2}{9} $
\ \item $ 48\cdot\dfrac{1}{6} $
\item $ 27\cdot\dfrac{7}{3} $
}
\nes

\op{opgbrdelbr}
\rknut \os
\abch{
	\item $ \dfrac{2}{3}:\dfrac{5}{7} $
	\item $ \dfrac{8}{9}:\dfrac{5}{3} $
	\item $ \dfrac{10}{3}:\dfrac{7}{3} $
	\item $ \dfrac{1}{5}:\dfrac{4}{7} $
	\item $ \dfrac{6}{5}:\dfrac{3}{11} $
}

\op{opgbrdelbr2}
\opgeks{ \vs
\[ \colb{\frac{3}{4}}: \colg{\frac{15}{8}}=\colb{\frac{\cancel{3}}{\cancel{2}\cdot\cancel{2}}}\cdot \colg{\frac{\cancel{2}\cdot\cancel{2}\cdot2}{\cancel{3}\cdot5}}=\frac{2}{5} \]
\mers{Her har vi valgt å primtalsfaktorisere alle tala, men ein treng ikkje gjere det viss ein ser kva faktorar som er felles for tellarane og nemnarane.}
} \regv

Utnytt at tellarane og nemnarane har felles faktorar, og rekn ut. \os
\abch{
\item $ \displaystyle \frac{7}{9}:\frac{21}{12} $
\item $ \displaystyle \frac{35}{24}:\frac{7}{18} $
\item $ \displaystyle \frac{84}{55}:\frac{42}{77} $
}

\nes
\newpage

\ekspop{1} 
Bruk \rref{brdeenk} og \rref{brtbr} til å fylle inn heltallet som mangler der det står ''\_''.
\abc{
\item Å gange med $ \frac{1}{2} $ er det samme som å dele med \_\,.
\item Å gange med $ \frac{1}{4} $ er det samme som å dele med \_\,.
\item Å gange med $ \frac{1}{5} $ er det samme som å dele med \_\,.	
}
Se tilbake til svarene for oppgave \opr{opgbrverdiikkedelelig}\textbf{a)\,-\,g)}. Fyll inn heltallet som mangler der det står ''\_''.
\abcs{4}{
\item Å gange med $ 0,5 $ er det samme som å dele med \_\,.
\item Å gange med $ 0,25 $ er det samme som å dele med \_\,.
\item Å gange med $ 0,2 $ er det samme som å dele med \_\,.
\item Å gange med $ 0,75 $ er det samme som å gange med \_\, og dele med \_\,.
\item Å gange med $ 0,4 $ er det samme som å gange med \_\, og dele med \_\,.
\item Å gange med $ 0,6 $ er det samme som å gange med \_\, og dele med \_\,.
\item Å gange med $ 0,8 $ er det samme som å gange med \_\, og dele med \_\,.
}
\newpage
\ekspop{2}
Se tilbake til \rref{delmbr} og svarene for oppgave \ref{opgbrverdiikkedelelig}a)\,-\,g).
Fyll inn heltallet som mangler der det står ''\_''.
{\renewcommand{\labelenumi}{(\alph{enumi})}
	\begin{enumerate}
		\item Å dele med $ 0,5 $ er det samme som å gange med \_\,.
		\item Å dele med $ 0,25 $ er det samme som å gange med \_\,.
		\item Å dele med $ 0,2 $ er det samme som å gange med \_\,.
		\item Å dele med $ 0,75 $ er det samme som å gange med \_\, og dele med \_\,.
		\item Å dele med $ 0,4 $ er det samme som å gange med \_\, og dele med \_\,..
		\item Å dele med $ 0,6 $ er det samme som å gange med \_\, og dele med \_\,..
		\item Å dele med $ 0,8 $ er det samme som å gange med \_\, og dele med \_\,..	
\end{enumerate} }

\ekspop{3}
Utnytt primtalsfaktorisering til å finne fellesnemar, og \rknut\os
\abch{
\item $ \displaystyle \frac{5}{204}+\frac{7}{198}$
\item $ \displaystyle \frac{11}{350}+\frac{17}{315} $
}
\newpage
\end{document}


