\documentclass[english, 11 pt, class=article, crop=false]{standalone}
\usepackage[T1]{fontenc}
%\renewcommand*\familydefault{\sfdefault} % For dyslexia-friendly text
\usepackage{lmodern} % load a font with all the characters
\usepackage{geometry}
\geometry{verbose,paperwidth=16.1 cm, paperheight=24 cm, inner=2.3cm, outer=1.8 cm, bmargin=2cm, tmargin=1.8cm}
\setlength{\parindent}{0bp}
\usepackage{import}
\usepackage[subpreambles=false]{standalone}
\usepackage{amsmath}
\usepackage{amssymb}
\usepackage{esint}
\usepackage{babel}
\usepackage{tabu}
\makeatother
\makeatletter

\usepackage{titlesec}
\usepackage{ragged2e}
\RaggedRight
\raggedbottom
\frenchspacing

% Norwegian names of figures, chapters, parts and content
\addto\captionsenglish{\renewcommand{\figurename}{Figur}}
\makeatletter
\addto\captionsenglish{\renewcommand{\chaptername}{Kapittel}}
\addto\captionsenglish{\renewcommand{\partname}{Del}}


\usepackage{graphicx}
\usepackage{float}
\usepackage{subfig}
\usepackage{placeins}
\usepackage{cancel}
\usepackage{framed}
\usepackage{wrapfig}
\usepackage[subfigure]{tocloft}
\usepackage[font=footnotesize,labelfont=sl]{caption} % Figure caption
\usepackage{bm}
\usepackage[dvipsnames, table]{xcolor}
\definecolor{shadecolor}{rgb}{0.105469, 0.613281, 1}
\colorlet{shadecolor}{Emerald!15} 
\usepackage{icomma}
\makeatother
\usepackage[many]{tcolorbox}
\usepackage{multicol}
\usepackage{stackengine}

\usepackage{esvect} %For vectors with capital letters

% For tabular
\usepackage{array}
\usepackage{multirow}
\usepackage{longtable} %breakable table

% Ligningsreferanser
\usepackage{mathtools}
\mathtoolsset{showonlyrefs}

% index
\usepackage{imakeidx}
\makeindex[title=Indeks]

%Footnote:
\usepackage[bottom, hang, flushmargin]{footmisc}
\usepackage{perpage} 
\MakePerPage{footnote}
\addtolength{\footnotesep}{2mm}
\renewcommand{\thefootnote}{\arabic{footnote}}
\renewcommand\footnoterule{\rule{\linewidth}{0.4pt}}
\renewcommand{\thempfootnote}{\arabic{mpfootnote}}

%colors
\definecolor{c1}{cmyk}{0,0.5,1,0}
\definecolor{c2}{cmyk}{1,0.25,1,0}
\definecolor{n3}{cmyk}{1,0.,1,0}
\definecolor{neg}{cmyk}{1,0.,0.,0}

% Lister med bokstavar
\usepackage[inline]{enumitem}

\newcounter{rg}
\numberwithin{rg}{chapter}
\newcommand{\reg}[2][]{\begin{tcolorbox}[boxrule=0.3 mm,arc=0mm,colback=blue!3] {\refstepcounter{rg}\phantomsection \large \textbf{\therg \;#1} \vspace{5 pt}}\newline #2  \end{tcolorbox}\vspace{-5pt}}

\newcommand\alg[1]{\begin{align} #1 \end{align}}

\newcommand\eks[2][]{\begin{tcolorbox}[boxrule=0.3 mm,arc=0mm,enhanced jigsaw,breakable,colback=green!3] {\large \textbf{Eksempel #1} \vspace{5 pt}\\} #2 \end{tcolorbox}\vspace{-5pt} }

\newcommand{\st}[1]{\begin{tcolorbox}[boxrule=0.0 mm,arc=0mm,enhanced jigsaw,breakable,colback=yellow!12]{ #1} \end{tcolorbox}}

\newcommand{\spr}[1]{\begin{tcolorbox}[boxrule=0.3 mm,arc=0mm,enhanced jigsaw,breakable,colback=yellow!7] {\large \textbf{Språkboksen} \vspace{5 pt}\\} #1 \end{tcolorbox}\vspace{-5pt} }

\newcommand{\sym}[1]{\colorbox{blue!15}{#1}}

\newcommand{\info}[2]{\begin{tcolorbox}[boxrule=0.3 mm,arc=0mm,enhanced jigsaw,breakable,colback=cyan!6] {\large \textbf{#1} \vspace{5 pt}\\} #2 \end{tcolorbox}\vspace{-5pt} }

\newcommand\algv[1]{\vspace{-11 pt}\begin{align*} #1 \end{align*}}

\newcommand{\regv}{\vspace{5pt}}
\newcommand{\mer}{\textsl{Merk}: }
\newcommand{\mers}[1]{{\footnotesize \mer #1}}
\newcommand\vsk{\vspace{11pt}}
\newcommand\vs{\vspace{-11pt}}
\newcommand\vsb{\vspace{-16pt}}
\newcommand\sv{\vsk \textbf{Svar} \vspace{4 pt}\\}
\newcommand\br{\\[5 pt]}
\newcommand{\figp}[1]{../fig/#1}
\newcommand\algvv[1]{\vs\vs\begin{align*} #1 \end{align*}}
\newcommand{\y}[1]{$ {#1} $}
\newcommand{\os}{\\[5 pt]}
\newcommand{\prbxl}[2]{
\parbox[l][][l]{#1\linewidth}{#2
	}}
\newcommand{\prbxr}[2]{\parbox[r][][l]{#1\linewidth}{
		\setlength{\abovedisplayskip}{5pt}
		\setlength{\belowdisplayskip}{5pt}	
		\setlength{\abovedisplayshortskip}{0pt}
		\setlength{\belowdisplayshortskip}{0pt} 
		\begin{shaded}
			\footnotesize	#2 \end{shaded}}}

\renewcommand{\cfttoctitlefont}{\Large\bfseries}
\setlength{\cftaftertoctitleskip}{0 pt}
\setlength{\cftbeforetoctitleskip}{0 pt}

\newcommand{\bs}{\\[3pt]}
\newcommand{\vn}{\\[6pt]}
\newcommand{\fig}[1]{\begin{figure}
		\centering
		\includegraphics[]{\figp{#1}}
\end{figure}}

\newcommand{\figc}[2]{\begin{figure}
		\centering
		\includegraphics[]{\figp{#1}}
		\caption{#2}
\end{figure}}

\newcommand{\sectionbreak}{\clearpage} % New page on each section

\newcommand{\nn}[1]{
\begin{equation}
	#1
\end{equation}
}

% Equation comments
\newcommand{\cm}[1]{\llap{\color{blue} #1}}

\newcommand\fork[2]{\begin{tcolorbox}[boxrule=0.3 mm,arc=0mm,enhanced jigsaw,breakable,colback=yellow!7] {\large \textbf{#1 (forklaring)} \vspace{5 pt}\\} #2 \end{tcolorbox}\vspace{-5pt} }
 
%colors
\newcommand{\colr}[1]{{\color{red} #1}}
\newcommand{\colb}[1]{{\color{blue} #1}}
\newcommand{\colo}[1]{{\color{orange} #1}}
\newcommand{\colc}[1]{{\color{cyan} #1}}
\definecolor{projectgreen}{cmyk}{100,0,100,0}
\newcommand{\colg}[1]{{\color{projectgreen} #1}}

% Methods
\newcommand{\metode}[2]{
	\textsl{#1} \\[-8pt]
	\rule{#2}{0.75pt}
}

%Opg
\newcommand{\abc}[1]{
	\begin{enumerate}[label=\alph*),leftmargin=18pt]
		#1
	\end{enumerate}
}
\newcommand{\abcs}[2]{
	\begin{enumerate}[label=\alph*),start=#1,leftmargin=18pt]
		#2
	\end{enumerate}
}
\newcommand{\abcn}[1]{
	\begin{enumerate}[label=\arabic*),leftmargin=18pt]
		#1
	\end{enumerate}
}
\newcommand{\abch}[1]{
	\hspace{-2pt}	\begin{enumerate*}[label=\alph*), itemjoin=\hspace{1cm}]
		#1
	\end{enumerate*}
}
\newcommand{\abchs}[2]{
	\hspace{-2pt}	\begin{enumerate*}[label=\alph*), itemjoin=\hspace{1cm}, start=#1]
		#2
	\end{enumerate*}
}

% Oppgaver
\newcommand{\opgt}{\phantomsection \addcontentsline{toc}{section}{Oppgaver} \section*{Oppgaver for kapittel \thechapter}\vs \setcounter{section}{1}}
\newcounter{opg}
\numberwithin{opg}{section}
\newcommand{\op}[1]{\vspace{15pt} \refstepcounter{opg}\large \textbf{\color{blue}\theopg} \vspace{2 pt} \label{#1} \\}
\newcommand{\ekspop}[1]{\vsk\textbf{Gruble \thechapter.#1}\vspace{2 pt} \\}
\newcommand{\nes}{\stepcounter{section}
	\setcounter{opg}{0}}
\newcommand{\opr}[1]{\vspace{3pt}\textbf{\ref{#1}}}
\newcommand{\oeks}[1]{\begin{tcolorbox}[boxrule=0.3 mm,arc=0mm,colback=white]
		\textit{Eksempel: } #1	  
\end{tcolorbox}}
\newcommand\opgeks[2][]{\begin{tcolorbox}[boxrule=0.1 mm,arc=0mm,enhanced jigsaw,breakable,colback=white] {\footnotesize \textbf{Eksempel #1} \\} \footnotesize #2 \end{tcolorbox}\vspace{-5pt} }
\newcommand{\rknut}{
Rekn ut.
}

%License
\newcommand{\lic}{\textit{Matematikken sine byggesteinar by Sindre Sogge Heggen is licensed under CC BY-NC-SA 4.0. To view a copy of this license, visit\\ 
		\net{http://creativecommons.org/licenses/by-nc-sa/4.0/}{http://creativecommons.org/licenses/by-nc-sa/4.0/}}}

%referances
\newcommand{\net}[2]{{\color{blue}\href{#1}{#2}}}
\newcommand{\hrs}[2]{\hyperref[#1]{\color{blue}\textsl{#2 \ref*{#1}}}}
\newcommand{\rref}[1]{\hrs{#1}{regel}}
\newcommand{\refkap}[1]{\hrs{#1}{kapittel}}
\newcommand{\refsec}[1]{\hrs{#1}{seksjon}}

\newcommand{\mb}{\net{https://sindrsh.github.io/FirstPrinciplesOfMath/}{MB}}


%line to seperate examples
\newcommand{\linje}{\rule{\linewidth}{1pt} }

\usepackage{datetime2}
%%\usepackage{sansmathfonts} for dyslexia-friendly math
\usepackage[]{hyperref}


\newcommand{\note}{Merk}
\newcommand{\notesm}[1]{{\footnotesize \textsl{\note:} #1}}
\newcommand{\ekstitle}{Eksempel }
\newcommand{\sprtitle}{Språkboksen}
\newcommand{\expl}{forklaring}

\newcommand{\vedlegg}[1]{\refstepcounter{vedl}\section*{Vedlegg \thevedl: #1}  \setcounter{vedleq}{0}}

\newcommand\sv{\vsk \textbf{Svar} \vspace{4 pt}\\}

%references
\newcommand{\reftab}[1]{\hrs{#1}{tabell}}
\newcommand{\rref}[1]{\hrs{#1}{regel}}
\newcommand{\dref}[1]{\hrs{#1}{definisjon}}
\newcommand{\refkap}[1]{\hrs{#1}{kapittel}}
\newcommand{\refsec}[1]{\hrs{#1}{seksjon}}
\newcommand{\refdsec}[1]{\hrs{#1}{delseksjon}}
\newcommand{\refved}[1]{\hrs{#1}{vedlegg}}
\newcommand{\eksref}[1]{\textsl{#1}}
\newcommand\fref[2][]{\hyperref[#2]{\textsl{figur \ref*{#2}#1}}}
\newcommand{\refop}[1]{{\color{blue}Oppgave \ref{#1}}}
\newcommand{\refops}[1]{{\color{blue}oppgave \ref{#1}}}
\newcommand{\refgrubs}[1]{{\color{blue}gruble \ref{#1}}}

\newcommand{\openmathser}{\openmath\,-\,serien}

% Exercises
\newcommand{\opgt}{\newpage \phantomsection \addcontentsline{toc}{section}{Oppgaver} \section*{Oppgaver for kapittel \thechapter}\vs \setcounter{section}{1}}


% Sequences and series
\newcommand{\sumarrek}{Summen av en aritmetisk rekke}
\newcommand{\sumgerek}{Summen av en geometrisk rekke}
\newcommand{\regnregsum}{Regneregler for summetegnet}

% Trigonometry
\newcommand{\sincoskomb}{Sinus og cosinus kombinert}
\newcommand{\cosfunk}{Cosinusfunksjonen}
\newcommand{\trid}{Trigonometriske identiteter}
\newcommand{\deravtri}{Den deriverte av de trigonometriske funksjonene}
% Solutions manual
\newcommand{\selos}{Se løsningsforslag.}
\newcommand{\se}[1]{Se eksempel på side \pageref{#1}}

%Vectors
\newcommand{\parvek}{Parallelle vektorer}
\newcommand{\vekpro}{Vektorproduktet}
\newcommand{\vekproarvol}{Vektorproduktet som areal og volum}


% 3D geometries
\newcommand{\linrom}{Linje i rommet}
\newcommand{\avstplnpkt}{Avstand mellom punkt og plan}


% Integral
\newcommand{\bestminten}{Bestemt integral I}
\newcommand{\anfundteo}{Analysens fundamentalteorem}
\newcommand{\intuf}{Integralet av utvalge funksjoner}
\newcommand{\bytvar}{Bytte av variabel}
\newcommand{\intvol}{Integral som volum}
\newcommand{\andordlindif}{Andre ordens lineære differensialligninger}



\begin{document}
\newpage
\section{Likhetstegnet, mengder og tallinjer}
\index{tall}
\subsection*{\likteikn}
Som navnet tilsier, viser \outl{likhetstegnet} \index{likhetstegnet} \sym{$ = $} til at noe er likt. I hvilken grad og når man kan si at noe er likt er en filosofisk diskusjon, og innledningsvis er vi bare prisgitt dette: \textsl{Hvilken likhet \sym{$=$} sikter til må bli forstått ut ifra konteksten tegnet blir brukt i}. Med denne forstelsen av \sym{$ = $} kan vi studere noen grunnleggende egenskaper for tallene våre, og så komme tilbake til mer presise betydninger av tegnet. \regv
\spr{
Vanlige måter å si \sym{$=$} på er
\begin{itemize}
	\item ''er lik'' \\
	\item ''er det samme som''
\end{itemize}
}
\subsection*{Mengder og tallinjer}
Tall kan representere så mangt. I denne boka skal vi holde oss til to måter å tolke tallene på; tall som en \textsl{mengde} og tall som en \textsl{plassering på en linje}. Alle representasjoner av tall tar utganspunkt i hva forståelsen er av tallene 0 og 1.

\subsubsection*{Tall som mengde}
	Når vi snakkar om en mengde, vil tallet 0 være\footnote{I \hrs{Rekneartane}{kapittel} skal vi se at det også er andre tolkninger av 0.} knyttet til ''ingenting''. En figur der det ikke er noe til stede vil slik være det samme som 0:
	\[ =0 \]
	1 vil vi tegne som en rute:
	\fig{rut1}

Andre tall vil da være definert ut ifra hvor mange ''enerruter'' (''enere'') vi har:
	\fig{rut2}
\newpage	
\subsubsection*{Tall som plassering på ei linje}
	Når vi plasserer tall på en linje, vil 0 være utgangspunket vårt:
	\fig{lin0a}
	Så plasserer vi 1 en viss lengde til høyre for 0:
	\fig{lin0b}
	Andre tall vil da være definert ut ifra hvor mange enerlengder (enere) vi er unna 0:
	\fig{lin1}
\subsection*{Positive heltall}
Vi skal straks se at tall ikke nødvendigvis trenger å være \textsl{hele} antall enere, men tallene som \textsl{er} det har et eget navn:\regv

\reg[Positive heltall]{
Tall som er et helt antall enere kalles \outl{positive\footnotemark heltall}\index{positive heltall}. De\\ positive heltallene er
\[ 1, 2, 3, 4, 5 \text{ og så videre.} \]
Positive heltall blir også kalt \outl{naturlige tal}\index{tall!naturlige}.
}
\info{Hva med 0?}{
Noen forfattere inkluderer også 0 i begrepet naturlige tal. I noen sammenhenger vil dette lønne seg, i andre ikke.
}
\footnotetext{Hva ordet 'positiv' innebærer skal vi se i \hrs{Negtal}{kapittel}.}

\newpage
\section{\talsifverd}
Tallene våre er bygd opp av \outl{sifrene}\index{siffer} $ 0, 1, 2 , 3, 4, 5, 6, 7, 8 $ og $ 9 $, og \textsl{plasseringen} av dem. Sifrene og deres plassering definerer\footnote{Etter hvert skal vi også se at \textit{fortegn} er med på å definere verdien til tallet (se \hrs{Negtal}{kapittel}).} \outl{verdien} \index{verdi} til tallet.
\subsection*{Heltall større enn 9}
La oss som et eksempel skrive tallet 'fjorten' ved hjelp av sifrene våre.
\fig{tel}
En gruppe med ti enere kaller vi en ''tier''. Av fjorten kan vi lage 1 tier, og i tillegg har vi da 4 enere. Da skriver vi 'fjorten' slik:
\[ \text{fjorten}=14 \]
\fig{tel2_bm}\vsk

\fig{tel2t}
\newpage
\subsection*{Desimaltall}
I mange tilfeller har vi ikke et helt antall enere, og da vil det være behov for å dele 1 inn i mindre biter. La oss starte med å tegne en ener:
\fig{maal}
\fig{des1}
Så deler vi eneren vår inn i 10 mindre biter:
\fig{maal1}
\fig{des1a}
Siden vi har delt 1 inn i 10 biter, kaller vi en slik bit for ''en tidel'':
\fig{maal1a_bm}
\fig{des1b_bm}
\begin{comment}
\eks{\vs
	\fig{maal2}
	\fig{des2}
}\vsk
\end{comment}
Tideler skriver vi ved hjelp av  \outl{desimaltegnet} \sym{,}  :
\fig{maal1b}
\fig{des1c}
\eks[]{\vs 
	\fig{maal2a}
	\fig{des3}
}\regv
\spr{
På engelsk bruker man punktum \sym{.} som desimaltegn:
\alg{
	3,5&\quad(norsk) \\
	3.5&\quad (english)
}
}
\newpage
\subsection*{Titallsystemet}
Vi har nå sett hvordan vi kan uttrykke verdien til tall ved å plassere \\siffer etter antall tiere, enere og tideler, og det stopper selvsagt ikke der: \regv

\reg[Titallsystemet \label{titalsys}]{
Verdien til et tall er gitt av sifrene $ 0, 1, 2, 3, 4, 5, 6, 7, 8  $ og $ 9 $, og plasseringen av dem. Med sifferet som angir enere som utgangspunkt vil
\begin{itemize}
	\item siffer til venstre (i rekkefølge) indikere antall tiere, \\hundrere, tusener og så videre.
	\item siffer til høyre (i rekkefølge) indikere antall tideler, \\hundredeler, tusendeler og så videre.
\end{itemize}
}
\eks[1]{\vs 
	\fig{maal3_bm}
}
\eks[2]{ \vs \vs
\fig{titalsys_bm}
}
\newpage
\reg[Partall og oddetall \label{parogodd}]{
Heltall som har 0, 2, 4, 6 eller 8 på enerplassen kalles \outl{partall}\index{partall}.\vsk

Heltall som har 1, 3, 5, 7 eller 9 på enerplassen kalles \outl{oddetall}\index{oddetall}.
}
\eks{
De ti første (positive) partallene er
\[ \text{0, 2, 4, 6, 8, 10, 12, 14, 16, og 18} \]
De ti første (positive) oddetallene er
\[ \text{1, 3, 5, 7, 9, 11, 13, 15, 17, og 19} \]
}

\newpage
\section{\koordsys \label{Koord}}

\prbxl{0.5}{I mange tilfeller er det nyttig å bruke to tallinjer samtidig. Dette kaller vi et \outl{koordinatsystem}\index{koordinatsystem}. Vi plasserer da én tallinje (en akse) som går \textsl{horisontalt} og én som går \textsl{vertikalt}. En plassering i et koordinatsystem kaller vi et \outl{punkt}\index{punkt}. 
 }\qquad
\prbxr{0.4}{Strengt tatt finnes det mange typer koordinatsystem, men i denne boka bruker vi ordet om bare én sort, nemlig det \outl{kartesiske koordinatsystem}. Det er oppkalt etter den franske filosofen og matematikeren René Descartes.}
\st{Et punkt skriver vi som to tall inni en parentes. De to tallene blir kalt \outl{førstekoordinaten} og \outl{andrekoordinaten} til punktet.
	\begin{itemize}
		\item Førstekoordinaten forteller oss hvor langt vi skal gå langs horisontalaksen.
		\item Andrekoordinaten forteller oss hvor langt vi skal gå langs vertikalaksen.
\end{itemize}
I figuren ser vi punktene $ (2, 3) $, $ (5, 1) $ og $ (0, 0) $. Punktet der aksene møtes, altså $ (0, 0) $, kalles \outl{origo}\index{origo}.
\fig{kord}
}

\end{document}