\documentclass[english, 11 pt, class=article, crop=false]{standalone}
\usepackage[T1]{fontenc}
%\renewcommand*\familydefault{\sfdefault} % For dyslexia-friendly text
\usepackage{lmodern} % load a font with all the characters
\usepackage{geometry}
\geometry{verbose,paperwidth=16.1 cm, paperheight=24 cm, inner=2.3cm, outer=1.8 cm, bmargin=2cm, tmargin=1.8cm}
\setlength{\parindent}{0bp}
\usepackage{import}
\usepackage[subpreambles=false]{standalone}
\usepackage{amsmath}
\usepackage{amssymb}
\usepackage{esint}
\usepackage{babel}
\usepackage{tabu}
\makeatother
\makeatletter

\usepackage{titlesec}
\usepackage{ragged2e}
\RaggedRight
\raggedbottom
\frenchspacing

% Norwegian names of figures, chapters, parts and content
\addto\captionsenglish{\renewcommand{\figurename}{Figur}}
\makeatletter
\addto\captionsenglish{\renewcommand{\chaptername}{Kapittel}}
\addto\captionsenglish{\renewcommand{\partname}{Del}}


\usepackage{graphicx}
\usepackage{float}
\usepackage{subfig}
\usepackage{placeins}
\usepackage{cancel}
\usepackage{framed}
\usepackage{wrapfig}
\usepackage[subfigure]{tocloft}
\usepackage[font=footnotesize,labelfont=sl]{caption} % Figure caption
\usepackage{bm}
\usepackage[dvipsnames, table]{xcolor}
\definecolor{shadecolor}{rgb}{0.105469, 0.613281, 1}
\colorlet{shadecolor}{Emerald!15} 
\usepackage{icomma}
\makeatother
\usepackage[many]{tcolorbox}
\usepackage{multicol}
\usepackage{stackengine}

\usepackage{esvect} %For vectors with capital letters

% For tabular
\usepackage{array}
\usepackage{multirow}
\usepackage{longtable} %breakable table

% Ligningsreferanser
\usepackage{mathtools}
\mathtoolsset{showonlyrefs}

% index
\usepackage{imakeidx}
\makeindex[title=Indeks]

%Footnote:
\usepackage[bottom, hang, flushmargin]{footmisc}
\usepackage{perpage} 
\MakePerPage{footnote}
\addtolength{\footnotesep}{2mm}
\renewcommand{\thefootnote}{\arabic{footnote}}
\renewcommand\footnoterule{\rule{\linewidth}{0.4pt}}
\renewcommand{\thempfootnote}{\arabic{mpfootnote}}

%colors
\definecolor{c1}{cmyk}{0,0.5,1,0}
\definecolor{c2}{cmyk}{1,0.25,1,0}
\definecolor{n3}{cmyk}{1,0.,1,0}
\definecolor{neg}{cmyk}{1,0.,0.,0}

% Lister med bokstavar
\usepackage[inline]{enumitem}

\newcounter{rg}
\numberwithin{rg}{chapter}
\newcommand{\reg}[2][]{\begin{tcolorbox}[boxrule=0.3 mm,arc=0mm,colback=blue!3] {\refstepcounter{rg}\phantomsection \large \textbf{\therg \;#1} \vspace{5 pt}}\newline #2  \end{tcolorbox}\vspace{-5pt}}

\newcommand\alg[1]{\begin{align} #1 \end{align}}

\newcommand\eks[2][]{\begin{tcolorbox}[boxrule=0.3 mm,arc=0mm,enhanced jigsaw,breakable,colback=green!3] {\large \textbf{Eksempel #1} \vspace{5 pt}\\} #2 \end{tcolorbox}\vspace{-5pt} }

\newcommand{\st}[1]{\begin{tcolorbox}[boxrule=0.0 mm,arc=0mm,enhanced jigsaw,breakable,colback=yellow!12]{ #1} \end{tcolorbox}}

\newcommand{\spr}[1]{\begin{tcolorbox}[boxrule=0.3 mm,arc=0mm,enhanced jigsaw,breakable,colback=yellow!7] {\large \textbf{Språkboksen} \vspace{5 pt}\\} #1 \end{tcolorbox}\vspace{-5pt} }

\newcommand{\sym}[1]{\colorbox{blue!15}{#1}}

\newcommand{\info}[2]{\begin{tcolorbox}[boxrule=0.3 mm,arc=0mm,enhanced jigsaw,breakable,colback=cyan!6] {\large \textbf{#1} \vspace{5 pt}\\} #2 \end{tcolorbox}\vspace{-5pt} }

\newcommand\algv[1]{\vspace{-11 pt}\begin{align*} #1 \end{align*}}

\newcommand{\regv}{\vspace{5pt}}
\newcommand{\mer}{\textsl{Merk}: }
\newcommand{\mers}[1]{{\footnotesize \mer #1}}
\newcommand\vsk{\vspace{11pt}}
\newcommand\vs{\vspace{-11pt}}
\newcommand\vsb{\vspace{-16pt}}
\newcommand\sv{\vsk \textbf{Svar} \vspace{4 pt}\\}
\newcommand\br{\\[5 pt]}
\newcommand{\figp}[1]{../fig/#1}
\newcommand\algvv[1]{\vs\vs\begin{align*} #1 \end{align*}}
\newcommand{\y}[1]{$ {#1} $}
\newcommand{\os}{\\[5 pt]}
\newcommand{\prbxl}[2]{
\parbox[l][][l]{#1\linewidth}{#2
	}}
\newcommand{\prbxr}[2]{\parbox[r][][l]{#1\linewidth}{
		\setlength{\abovedisplayskip}{5pt}
		\setlength{\belowdisplayskip}{5pt}	
		\setlength{\abovedisplayshortskip}{0pt}
		\setlength{\belowdisplayshortskip}{0pt} 
		\begin{shaded}
			\footnotesize	#2 \end{shaded}}}

\renewcommand{\cfttoctitlefont}{\Large\bfseries}
\setlength{\cftaftertoctitleskip}{0 pt}
\setlength{\cftbeforetoctitleskip}{0 pt}

\newcommand{\bs}{\\[3pt]}
\newcommand{\vn}{\\[6pt]}
\newcommand{\fig}[1]{\begin{figure}
		\centering
		\includegraphics[]{\figp{#1}}
\end{figure}}

\newcommand{\figc}[2]{\begin{figure}
		\centering
		\includegraphics[]{\figp{#1}}
		\caption{#2}
\end{figure}}

\newcommand{\sectionbreak}{\clearpage} % New page on each section

\newcommand{\nn}[1]{
\begin{equation}
	#1
\end{equation}
}

% Equation comments
\newcommand{\cm}[1]{\llap{\color{blue} #1}}

\newcommand\fork[2]{\begin{tcolorbox}[boxrule=0.3 mm,arc=0mm,enhanced jigsaw,breakable,colback=yellow!7] {\large \textbf{#1 (forklaring)} \vspace{5 pt}\\} #2 \end{tcolorbox}\vspace{-5pt} }
 
%colors
\newcommand{\colr}[1]{{\color{red} #1}}
\newcommand{\colb}[1]{{\color{blue} #1}}
\newcommand{\colo}[1]{{\color{orange} #1}}
\newcommand{\colc}[1]{{\color{cyan} #1}}
\definecolor{projectgreen}{cmyk}{100,0,100,0}
\newcommand{\colg}[1]{{\color{projectgreen} #1}}

% Methods
\newcommand{\metode}[2]{
	\textsl{#1} \\[-8pt]
	\rule{#2}{0.75pt}
}

%Opg
\newcommand{\abc}[1]{
	\begin{enumerate}[label=\alph*),leftmargin=18pt]
		#1
	\end{enumerate}
}
\newcommand{\abcs}[2]{
	\begin{enumerate}[label=\alph*),start=#1,leftmargin=18pt]
		#2
	\end{enumerate}
}
\newcommand{\abcn}[1]{
	\begin{enumerate}[label=\arabic*),leftmargin=18pt]
		#1
	\end{enumerate}
}
\newcommand{\abch}[1]{
	\hspace{-2pt}	\begin{enumerate*}[label=\alph*), itemjoin=\hspace{1cm}]
		#1
	\end{enumerate*}
}
\newcommand{\abchs}[2]{
	\hspace{-2pt}	\begin{enumerate*}[label=\alph*), itemjoin=\hspace{1cm}, start=#1]
		#2
	\end{enumerate*}
}

% Oppgaver
\newcommand{\opgt}{\phantomsection \addcontentsline{toc}{section}{Oppgaver} \section*{Oppgaver for kapittel \thechapter}\vs \setcounter{section}{1}}
\newcounter{opg}
\numberwithin{opg}{section}
\newcommand{\op}[1]{\vspace{15pt} \refstepcounter{opg}\large \textbf{\color{blue}\theopg} \vspace{2 pt} \label{#1} \\}
\newcommand{\ekspop}[1]{\vsk\textbf{Gruble \thechapter.#1}\vspace{2 pt} \\}
\newcommand{\nes}{\stepcounter{section}
	\setcounter{opg}{0}}
\newcommand{\opr}[1]{\vspace{3pt}\textbf{\ref{#1}}}
\newcommand{\oeks}[1]{\begin{tcolorbox}[boxrule=0.3 mm,arc=0mm,colback=white]
		\textit{Eksempel: } #1	  
\end{tcolorbox}}
\newcommand\opgeks[2][]{\begin{tcolorbox}[boxrule=0.1 mm,arc=0mm,enhanced jigsaw,breakable,colback=white] {\footnotesize \textbf{Eksempel #1} \\} \footnotesize #2 \end{tcolorbox}\vspace{-5pt} }
\newcommand{\rknut}{
Rekn ut.
}

%License
\newcommand{\lic}{\textit{Matematikken sine byggesteinar by Sindre Sogge Heggen is licensed under CC BY-NC-SA 4.0. To view a copy of this license, visit\\ 
		\net{http://creativecommons.org/licenses/by-nc-sa/4.0/}{http://creativecommons.org/licenses/by-nc-sa/4.0/}}}

%referances
\newcommand{\net}[2]{{\color{blue}\href{#1}{#2}}}
\newcommand{\hrs}[2]{\hyperref[#1]{\color{blue}\textsl{#2 \ref*{#1}}}}
\newcommand{\rref}[1]{\hrs{#1}{regel}}
\newcommand{\refkap}[1]{\hrs{#1}{kapittel}}
\newcommand{\refsec}[1]{\hrs{#1}{seksjon}}

\newcommand{\mb}{\net{https://sindrsh.github.io/FirstPrinciplesOfMath/}{MB}}


%line to seperate examples
\newcommand{\linje}{\rule{\linewidth}{1pt} }

\usepackage{datetime2}
%%\usepackage{sansmathfonts} for dyslexia-friendly math
\usepackage[]{hyperref}


\newcommand{\note}{Merk}
\newcommand{\notesm}[1]{{\footnotesize \textsl{\note:} #1}}
\newcommand{\ekstitle}{Eksempel }
\newcommand{\sprtitle}{Språkboksen}
\newcommand{\expl}{forklaring}

\newcommand{\vedlegg}[1]{\refstepcounter{vedl}\section*{Vedlegg \thevedl: #1}  \setcounter{vedleq}{0}}

\newcommand\sv{\vsk \textbf{Svar} \vspace{4 pt}\\}

%references
\newcommand{\reftab}[1]{\hrs{#1}{tabell}}
\newcommand{\rref}[1]{\hrs{#1}{regel}}
\newcommand{\dref}[1]{\hrs{#1}{definisjon}}
\newcommand{\refkap}[1]{\hrs{#1}{kapittel}}
\newcommand{\refsec}[1]{\hrs{#1}{seksjon}}
\newcommand{\refdsec}[1]{\hrs{#1}{delseksjon}}
\newcommand{\refved}[1]{\hrs{#1}{vedlegg}}
\newcommand{\eksref}[1]{\textsl{#1}}
\newcommand\fref[2][]{\hyperref[#2]{\textsl{figur \ref*{#2}#1}}}
\newcommand{\refop}[1]{{\color{blue}Oppgave \ref{#1}}}
\newcommand{\refops}[1]{{\color{blue}oppgave \ref{#1}}}
\newcommand{\refgrubs}[1]{{\color{blue}gruble \ref{#1}}}

\newcommand{\openmathser}{\openmath\,-\,serien}

% Exercises
\newcommand{\opgt}{\newpage \phantomsection \addcontentsline{toc}{section}{Oppgaver} \section*{Oppgaver for kapittel \thechapter}\vs \setcounter{section}{1}}


% Sequences and series
\newcommand{\sumarrek}{Summen av en aritmetisk rekke}
\newcommand{\sumgerek}{Summen av en geometrisk rekke}
\newcommand{\regnregsum}{Regneregler for summetegnet}

% Trigonometry
\newcommand{\sincoskomb}{Sinus og cosinus kombinert}
\newcommand{\cosfunk}{Cosinusfunksjonen}
\newcommand{\trid}{Trigonometriske identiteter}
\newcommand{\deravtri}{Den deriverte av de trigonometriske funksjonene}
% Solutions manual
\newcommand{\selos}{Se løsningsforslag.}
\newcommand{\se}[1]{Se eksempel på side \pageref{#1}}

%Vectors
\newcommand{\parvek}{Parallelle vektorer}
\newcommand{\vekpro}{Vektorproduktet}
\newcommand{\vekproarvol}{Vektorproduktet som areal og volum}


% 3D geometries
\newcommand{\linrom}{Linje i rommet}
\newcommand{\avstplnpkt}{Avstand mellom punkt og plan}


% Integral
\newcommand{\bestminten}{Bestemt integral I}
\newcommand{\anfundteo}{Analysens fundamentalteorem}
\newcommand{\intuf}{Integralet av utvalge funksjoner}
\newcommand{\bytvar}{Bytte av variabel}
\newcommand{\intvol}{Integral som volum}
\newcommand{\andordlindif}{Andre ordens lineære differensialligninger}



\begin{document}
\newpage
\section{\adi \label{Addisjon}}	
\subsection*{Addisjon med mengder; å legge til}
Når vi har en mengde og skal legge til mer, bruker vi \outl{plusstegnet} \sym{$ + $}. Har vi 2 og skal legge til 3, skriver vi 
\[ 2+3=5 \]
\fig{plusm1}
Rekkefølgen vi legger sammen tallene på har ikke  noe å si; å starte med 2 og så legge til 3 er det samme som å starte med 3 og så legge til 2:
\[ 3+2=5 \]
\fig{plusm1a}
\spr{
	Et addisjonsstykke består av to eller flere \outl{ledd}\index{ledd} og én \outl{sum}\index{sum}. I regnestykket
	\[ 2+3=5 \]
	er både $ 2 $ og $ 3 $ ledd, mens $ 5 $ er summen.\vsk
	
	Vanlige måter å si $ 2+3 $ på er
	\begin{itemize}
		\item ''2 pluss 3''
		\item ''2 addert med 3''
	 	\item ''2 og 3 lagt sammen''
	\end{itemize}
Det å legge sammen tall kalles også \textit{å summere}.
} \regv
\newpage
\reg[Addisjon er kommutativ \label{adkom}]{
Summen er den samme uansett rekkefølge på leddene.
}
\eks{ \vs \vsb
\alg{
2+5 &= 7 =5+2  \vn
6+3 &=9=3+6
}
}

\subsection*{Addisjon på tallinja; Vandring mot høyre}
På en tallinje vil addisjon med positive tall  innebære vandring \textsl{mot \\høyre}:\regv
\eks[1]{ \vs
\[ 2+7=9 \]
\fig{plus2}
}
\eks[2]{ \vs
	\[ 4+11=15 \]
\fig{plus3}
}
\info{Betydningen av \sym{$\bm=$}}{
\sym{$ + $} gir oss muligheten til å uttrykke tall på mange forskjellige måter, for eksempel er $ {5=2+3} $ og $ {5=1+4} $. I denne sammenhengen vil \sym{$ = $} bety ''har samme verdi som''. Dette gjelder også ved subtraksjon, multiplikasjon og divisjon, som vi skal se på i de neste tre seksjonene.
}

\section{\sub \label{Subtraksjon}}
\subsection*{Subtraksjon med mengder: Å trekke ifra}
Når vi har en mengde og tar bort en del av den, bruker vi \\ \outl{minustegnet} \sym{$ - $}. Har vi 5 og skal ta bort 3, skriver vi
\[ 5-{\color{red} 3}=2 \]
\fig{min1c}

\spr{
	Et subtraksjonsstykke består av to eller flere \outl{ledd} \index{ledd} og én\\ \outl{differanse}\index{differanse}. I subtraksjonsstykket
	\[  5-3=2 \] 
	er både $ 5 $ og $ 3 $ ledd og $ 2 $ er differansen. \vsk
	
	Vanlige måter å si $ 5-3 $ på er
	\begin{itemize}
		\item ''5 minus 3'' \\
		\item ''5 fratrekt 3''
		\item ''3 subtrahert fra 5''
	\end{itemize}
} \vsk \vsk

\info{En ny tolkning av 0}{
	Innledningsvis i denne boka nevnte vi at 0 kan tolkes som \\''ingenting''. Subtraksjon gir oss muligheten til å uttrykke 0 via andre tall. For eksempel er $ {7-7=0} $ og $ {19-19=0} $.
} \vsk

\begin{comment}
\info{Subtraksjon er \textsl{ikkje} kommutativ}{
$ 5-3 $ er ikkje lik\footnote{kva $ 3-5 $ er skal vi sjå på i \hrs{Negtal}{kapittel}} $ 3-5 $. Hadde alle rekneartar vore kommutative ville ikkje reknerekkefølga
}
\end{comment}
\newpage
\subsection*{Subtraksjon på tallinja: Vandring mot venstre}
I \hrs{Addisjon}{seksjon} har vi sett at \sym{$ + $} (med positive tall) innebærer at vi skal gå \textsl{mot høyre} langs tallinja. Med \sym{$ - $} gjør vi omvendt, vi går \textsl{mot \\venstre}\footnote{I figurer med tallinjer vil rødfargede piler indikere at man starter ved pilspissen og vandrer til andre enden.}: \regv
\info{Merk}{
	I \textsl{Eksempel 1} og \textsl{Eksempel 2} under går vi i motsatt retning av den som pila peker i. Dette kan først virke litt rart, men spesielt i \refkap{Negtal} vil det lønne seg å tenke slik.
} 
\eks[1]{ \vs
	\[ 6-4=2 \]
	\fig{mint}
}
\eks[2]{ \vs
	\[ 12-7=5 \]
	\fig{mint2}
}

\section{\gong \label{Gonging} }

\subsection*{Ganging med heltall; innledende definisjon}
Når vi legger sammen like tall, kan vi bruke \outl{gangetegnet} \sym{$ \cdot $}\;for å skrive regnestykkene våre kortere: \regv
\eks[]{\vsb \vs
	\alg{
		4+4+4 &= 4\cdot 3 \vn
		8+8 &=8\cdot 2 \vn
		1+1+1+1+1&= 1\cdot5 
	}
} \regv
\spr{
	Et gangestykke består av to eller flere \outl{faktorer}\index{faktor} og ett \outl{produkt}\index{produkt}. I gangestykket
	\[ {4\cdot 3=12} \]
	er $ 4 $ og $ 3 $ faktorer, mens $ 12 $ er produktet. \vsk
	
	Vanlige måter å si $ 4\cdot3 $ på er
	\begin{itemize}
		\item ''4 ganger 3''  \\
		\item ''4 ganget med 3''\\
		\item ''4 multiplisert med 3''
	\end{itemize}
	
	Mange nettsteder og bøker på engelsk bruker symbolet \sym{$ \times $} i steden for \sym{$ \cdot $}. I de fleste programmeringsspråk er \sym{*} symbolet for multiplikasjon.
}
\subsection*{Ganging av mengder} \label{gangmengd}
La oss nå bruke en figur for å se for oss gangestykket $ 2\cdot3 $:
\fig{2t3}
Og så kan vi legge merke til produktet av $ 3\cdot 2 $:
\fig{3t2}
\reg[\gangkom \label{gangkom}]{
	Produktet er det samme uansett rekkefølge på faktorene.
}
\eks[]{\vsb \vs
	\alg{
		3\cdot 4 &=12= 4\cdot 3 \vn
		6\cdot 7 &=42= 7\cdot6 \vn
		8\cdot 9 &=72= 9\cdot8
	}
}

\subsection*{Ganging på tallinja}
Vi kan også bruke tallinja for å regne ut gangestykker. For eksempel kan vi finne hva $ 2\cdot4 $ er ved å tenke slik:
\[\text{''} 2\cdot 4 \text{ betyr å vandre 2 plasser \textsl{mot høyre}, 4 ganger.}\text{''} \]
\[ 2\cdot4=8 \]
\fig{2t4l}
Også tallinja kan vi bruke for å overbevise oss om at rekkefølgen i et gangestykke ikke har noe å si:
\[\text{''} 4\cdot 2 \text{ betyr å vandre 4 plasser \textsl{mot høyre}, 2 ganger.}\text{''} \]
\[ 4\cdot2=8 \]
\fig{4t2l}

\newpage
\subsection*{Endelig definisjon av ganging med positive heltall}
Det ligger kanskje nærmest å tolke ''2 ganger 3'' som ''3, 2 ganger''. Da er
\[ \text{''2 ganger 3''}=3+3 \] 
På side \pageref{gangmengd} presenterete vi $ {2\cdot3} $, altså ''2 ganger 3'', som $ {2+2+2} $. Med denne tolkningen vil $ {3+3} $ svare til $ {3\cdot2} $, men nettopp det at multiplikasjon er en kommutativ operasjon (\rref{gangkom}) gjør at den ene tolkningen ikke utelukker den andre; $ {2\cdot3 =2+2+2} $ og $ {2\cdot3=3+3} $ er to uttrykk med samme verdi.\regv

\reg[Ganging som gjentatt addisjon \label{ganggjad2}]{
	Ganging med et positivt heltal kan uttrykkes som gjentatt addisjon.
}
\eks[1]{\vsb \vs
	\alg{
		4+4+4 &= 4\cdot 3=3+3+3+3 \vn
		8+8 &=8\cdot 2=2+2+2+2+2+2+2 \vn
		1+1+1+1+1&= 1\cdot5 =5
	}
}
\info{Merk}{
	At ganging med positive heltal kan uttrykkes som gjentatt addisjon, utelukker ikke andre uttrykk. Det er ikke feil å skrive at $ {2\cdot 3=1+5} $.
}

\section{\del \label{Divisjon}}
\sym{:} er \outl{divisjonstegnet}.
I praksis har divisjon tre forskjellige betydninger, her eksemplifisert ved regnestykket $ 12:3 $:\regv

\reg[Divisjon sine tre betydninger]{ \vs
	\begin{itemize}
		\item \textbf{Inndeling av mengder} \\
		$ 12:3 = \text{''Antallet i hver gruppe når 12 deles inn i 3 like}$\\
		\hspace{1.6cm}store grupper'' 
		\item \textbf{Antall ganger} \\
		$ 12:3=\text{''Antall ganger 3 går på 12''} $
		\item \textbf{Omvendt operasjon av multiplikasjon}\\
		$ 12:3=\text{''Tallet man må gange 3 med for å få 12''} $
	\end{itemize}
} \regv

\spr{ \label{sprakdiv}
	Et divisjonsstykke består av en \outl{dividend}\index{dividend}, en \outl{divisor}\index{divisor} og en\\ \outl{kvotient}\index{kvotient}.	
	I divisjonstykket
	\[  {12:3=4} \]
	er $ 12 $ dividenden, $ 3 $ er divisoren og $ 4 $ er kvotienten.\vsk
	
	Vanlige måter å uttale $ 12:3 $ på er
	\begin{itemize}
		\item ''12 delt med/på 3''
		\item ''12 dividert med/på 3''
		\item ''12 på 3''
	\end{itemize}
	I noen sammenhenger blir $ {12:3} $ kalt ''\outl{forholdet}\index{forhold} mellom 12 og 3''. Da er 4 \outl{forholdstallet}\index{forholdstal}. \vsk
	
	Ofte brukes \sym{$ / $} i steden for \sym{$:$}, spesielt i programmeringsspråk.
}
\newpage
\subsection*{Divisjon av mengder}
Regnestykket $ {12:3} $ forteller oss at vi skal dele 12 inn i 3 like store \\grupper:
\fig{del1}
Vi ser at hver gruppe inneholder 4 ruter, dette betyr at
\[ 12:3=4 \]


\subsection*{Antall ganger}
\fig{del2}
3 går 4 ganger på 12, altså er $ 12:3=4 $.


\subsection*{Omvendt operasjon av multiplikasjon}
Vi har sett at hvis vi deler 12 inn i 3 like grupper, får vi 4 i hver gruppe. Altså er $ 12:3=4$. Om vi legger sammen igjen disse gruppene, får vi naturligvis 12: 
\fig{del1b}
Men dette er det samme som å gange 4 med 3. Altså;
om vi vet at $ {4\cdot 3=12} $, så vet vi også at $ {12:3=4} $. I tillegg vet vi da at $ {12:4=3} $. 
\fig{del1a}


\eks[1]{Siden $ 6\cdot 3 = 18  $, er\vs
	\alg{
		18:6= 3\vn
		18:3=6
	}
}
\eks[2]{
	Siden $ 5\cdot 7 = 35  $, er\vs
	\alg{
		35:5= 7\vn
		35:7=5
	}
}

\end{document}