\documentclass[english, 11 pt, class=article, crop=false]{standalone}
\input{../preamb_bm}

\begin{document}
	
\opgt

\op{opgnegfyllinn}
\opgeks{
	8 har retning mot høgre og lengde 8. \\
	$ -7 $ har retning mot venstre og lengde 7.
} \vsk

Fyll inn ordene som mangler.
\abc{
	\item 9 har retning mot ... og lengde ... .
	\item 4 har retning mot ... og lengde ... .
	\item -3 har retning mot ... og lengde ... .
	\item 12 har retning mot ... og lengde ... .
	\item -11 har retning mot ... og lengde ... .
	\item -25 har retning mot ... og lengde ... .
}
	
	\op{opgnegplas}
	\fig{opgneg1}
	Tegn av tallinja over og plasser tallene. \os
	\abch{
		\item $ 3 $
		\item $ -4 $
		\item $ -8 $
		\item $ 7 $
		\item $ -3 $
		\item $ 5 $
		\item $ -5 $
	}
	
	\op{opgnegogpos}
	Se tilbake på svarene fra \opr{opgnegplas}. Skriv ned hvilke av tallene som er\os
	\abch{
		\item positive tall
		\item negative tall
	}
	
	\op{opgneglengde}
	Finn lengden til tallet.\os
	\abch{
		\item 5
		\item $ -6 $
		\item $ 7 $
		\item $ -8 $
		\item 9 
		\item $ -10 $
	}
	
	\nes
	\op{opgnegad}
	Regn ut. \os
	\abch{
		\item $ 8+(-7) $
		\item $ 12+(-5) $
		\item $ 9+(-3) $
		\item $ 7+(-7) $
	} \vsk
	
	\abchs{5}{
		\item $ -5+8 $
		\ \ \ \item $ -9+10 $
		\ \ \ `\item $ -1+11 $
		\item $ -4+9 $
	}
	
	\op{opgnegad2}
	Regn ut. \os
	\abch{
		\item $ 3+(-19) $
		\item $ 7+(-15) $
		\item $ -20+(-3) $
		\item $ 7+(-7) $
	} \vsk
	
	\abchs{
		\item $ 3+(-19) $
		\item $ 7+(-15) $
		\item $ -20+(-3) $
		\item $ 7+(-7) $
	}
	
\end{document}

