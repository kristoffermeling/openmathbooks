\documentclass[english,hidelinks,pdftex, 11 pt, class=report,crop=false]{standalone}
\usepackage[T1]{fontenc}
\usepackage[utf8]{luainputenc}
\usepackage{geometry}
\geometry{verbose,paperwidth=16.4 cm, paperheight=29cm, inner=2.05cm, outer=2.05 cm, bmargin=2cm, tmargin=1.8cm}
\usepackage{amsmath}
\usepackage{amssymb}
\usepackage{esint}
\usepackage{babel}
\usepackage{tabu}


\begin{document}
\begin{itemize}
\item \textbf{Eksempel 2.}
\item La oss no rekne ut $ 204,6-93,7  $. 
\item Me starte da med 93,7 og plusse oss opp te 204,6.
\item Først legg me te 0,3 og fer 94,0
\item Så legg me te 0,6 og fer 94,6.
\item Så legg me te 6 og fer 100,6
\item så legg me te 4 og fer 104,6
\item Så legg me te 100 og fer 204,6.
\item Da e me framme på 204,6.
\item No legg me sammen 0.3, 0.6, 6, 4 og 100, som e 110,9.
\item Altså e $ 204,6-93,7=110,9 $.
\end{itemize}
\begin{itemize}
	\item \textbf{Eksempel 3}
	\item Me ska no sjå på det samme reknestykket som i Eksempel 2 for å vise kordane ein kan rekne på kortare måta etterkvart som ein he blitt godt trent.
	\item me starte da med 93,7 og legg på 0,9. Da fer me 94,6.
	\item Så legg me på 110 og fer 204,6.
	\item $ 0,9+110=110,9 $, altså e $ 204,6-93,7=110,9 $.
\end{itemize}
\end{document}