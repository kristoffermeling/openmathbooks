\documentclass[english,hidelinks,pdftex, 11 pt, class=report,crop=false]{standalone}
\usepackage[T1]{fontenc}
\usepackage[utf8]{luainputenc}
\usepackage{geometry}
\geometry{verbose,paperwidth=16.4 cm, paperheight=29cm, inner=2.05cm, outer=2.05 cm, bmargin=2cm, tmargin=1.8cm}
\usepackage{amsmath}
\usepackage{amssymb}
\usepackage{esint}
\usepackage{babel}
\usepackage{tabu}


\begin{document}
\begin{itemize}
\item  Å utvide brøkar.
\item På bildet ser me brøken $ \dfrac{3}{5} $. Me skal no sjå kordan me kan utvide denna brøken. Å utvide ein brøk betyr at me endre på talla i brøken, uten at \textit{verdien} til brøken endre se.
\item Me lage no ei linje som går tvers over firkanten. Firkanten vår e da delt inn i 10 bita, og 6 av dei e blå. Altså he me brøken $ \dfrac{6}{10} $. Men \textit{området} som e blått e like stort i $ \dfrac{3}{5} $ som i $ \dfrac{6}{10} $, og det betyr at dessa to brøkan he samme verdi.
\item Det me he gjort reint matematisk e å \begin{itemize}
	\item starte med $ \dfrac{3}{5} $
	\item deretter gange med 2 over og under brøkstreken
	\item som gir $ \dfrac{6}{10} $. 
	\item Da sei me at me he utvide $ \dfrac{3}{5} $ med 2.
\end{itemize}
\end{itemize}
\end{document}