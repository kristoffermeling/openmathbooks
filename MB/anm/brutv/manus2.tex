\documentclass[english,hidelinks,pdftex, 11 pt, class=report,crop=false]{standalone}
\usepackage[T1]{fontenc}
\usepackage[utf8]{luainputenc}
\usepackage{geometry}
\geometry{verbose,paperwidth=16.4 cm, paperheight=29cm, inner=2.05cm, outer=2.05 cm, bmargin=2cm, tmargin=1.8cm}
\usepackage{amsmath}
\usepackage{amssymb}
\usepackage{esint}
\usepackage{babel}
\usepackage{tabu}


\begin{document}
\begin{itemize}
\item La oss sjå på eit nytt eksempel ved å rekne ut $ \frac{9}{4} +\frac{7}{6}$.
\item Sia brøkan he 4 og 6 som nevnar, sjekke me om dei he eit felles tall i gangetabellen. Og det he dei, for $3\cdot 4 $ blir 12 og $ 2\cdot 6 $ blir 12. Dette betyr at 12 kan være nevnar for begge brøkan.
\item Så da starte me med å utvide. For at $ \frac{9}{4}  $ skal ha 12 i nevnar, må vi utvide med 3, og for at $ \frac{7}{6} $ skal ha 12 i nevnar, må me utvide med 2.
\item Da fer me $ \frac{27}{12}+\frac{14}{12} $, som e $ \frac{41}{12} $
\end{itemize}
\end{document}