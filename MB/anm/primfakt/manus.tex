\documentclass[english,hidelinks,pdftex, 11 pt, class=report,crop=false]{standalone}
\usepackage[T1]{fontenc}
\usepackage[utf8]{luainputenc}
\usepackage{geometry}
\geometry{verbose,paperwidth=16.4 cm, paperheight=29cm, inner=2.05cm, outer=2.05 cm, bmargin=2cm, tmargin=1.8cm}
\usepackage{amsmath}
\usepackage{amssymb}
\usepackage{esint}
\usepackage{babel}
\usepackage{tabu}
\usepackage{lmodern}


\begin{document}
\begin{itemize}
	\item Primtalsfaktorisering
	\item Eksempel 1
	\item Me ska no primtallsfaktorisere 84. Detta innebær å skrive 84 som eit gangestykke der berre primtal e med.
	\item For å primtallsfaktorisere e me avhengig av å vite ka tall som e primtall, i vårt eksempel nøye vi oss med å skrive opp dei 4 første primtalla; 2, 3, 5 og 7. 
	\item No starte me ved å sjekke om $ 84 $ e delelig med 2. Det e det, sia $ 84:2 = 42 $. Detta skriv me inn i ein tabell.
	\item Så sjekke me om 42 e delelig med 2. Det e det, sia $ 42:2=21 $. Detta skriv me au inn i tabellen vår. 
	\item Så sjekke me om 21 e delelig med 2. Det e det ikkje, sia $ 21:2 =10,5$. 
	\item Derfor gjeng me no over te å sjekke om 21 e delelig med 3. Det e det, sia $ 21:3=7 $. Detta skriv me au inn i tabellen vår.
	\item No legg me merke te at 7 e eit primtall, nåkkå som betyr at me e ferdig med å dele. 84 kan me no skrive som eit gangestykke som består av primtalla me delte med, i tillegg til primtallet me endte opp med.
\end{itemize}

\end{document}