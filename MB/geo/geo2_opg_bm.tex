\documentclass[english, 11 pt, class=article, crop=false]{standalone}
\usepackage[T1]{fontenc}
%\renewcommand*\familydefault{\sfdefault} % For dyslexia-friendly text
\usepackage{lmodern} % load a font with all the characters
\usepackage{geometry}
\geometry{verbose,paperwidth=16.1 cm, paperheight=24 cm, inner=2.3cm, outer=1.8 cm, bmargin=2cm, tmargin=1.8cm}
\setlength{\parindent}{0bp}
\usepackage{import}
\usepackage[subpreambles=false]{standalone}
\usepackage{amsmath}
\usepackage{amssymb}
\usepackage{esint}
\usepackage{babel}
\usepackage{tabu}
\makeatother
\makeatletter

\usepackage{titlesec}
\usepackage{ragged2e}
\RaggedRight
\raggedbottom
\frenchspacing

% Norwegian names of figures, chapters, parts and content
\addto\captionsenglish{\renewcommand{\figurename}{Figur}}
\makeatletter
\addto\captionsenglish{\renewcommand{\chaptername}{Kapittel}}
\addto\captionsenglish{\renewcommand{\partname}{Del}}


\usepackage{graphicx}
\usepackage{float}
\usepackage{subfig}
\usepackage{placeins}
\usepackage{cancel}
\usepackage{framed}
\usepackage{wrapfig}
\usepackage[subfigure]{tocloft}
\usepackage[font=footnotesize,labelfont=sl]{caption} % Figure caption
\usepackage{bm}
\usepackage[dvipsnames, table]{xcolor}
\definecolor{shadecolor}{rgb}{0.105469, 0.613281, 1}
\colorlet{shadecolor}{Emerald!15} 
\usepackage{icomma}
\makeatother
\usepackage[many]{tcolorbox}
\usepackage{multicol}
\usepackage{stackengine}

\usepackage{esvect} %For vectors with capital letters

% For tabular
\usepackage{array}
\usepackage{multirow}
\usepackage{longtable} %breakable table

% Ligningsreferanser
\usepackage{mathtools}
\mathtoolsset{showonlyrefs}

% index
\usepackage{imakeidx}
\makeindex[title=Indeks]

%Footnote:
\usepackage[bottom, hang, flushmargin]{footmisc}
\usepackage{perpage} 
\MakePerPage{footnote}
\addtolength{\footnotesep}{2mm}
\renewcommand{\thefootnote}{\arabic{footnote}}
\renewcommand\footnoterule{\rule{\linewidth}{0.4pt}}
\renewcommand{\thempfootnote}{\arabic{mpfootnote}}

%colors
\definecolor{c1}{cmyk}{0,0.5,1,0}
\definecolor{c2}{cmyk}{1,0.25,1,0}
\definecolor{n3}{cmyk}{1,0.,1,0}
\definecolor{neg}{cmyk}{1,0.,0.,0}

% Lister med bokstavar
\usepackage[inline]{enumitem}

\newcounter{rg}
\numberwithin{rg}{chapter}
\newcommand{\reg}[2][]{\begin{tcolorbox}[boxrule=0.3 mm,arc=0mm,colback=blue!3] {\refstepcounter{rg}\phantomsection \large \textbf{\therg \;#1} \vspace{5 pt}}\newline #2  \end{tcolorbox}\vspace{-5pt}}

\newcommand\alg[1]{\begin{align} #1 \end{align}}

\newcommand\eks[2][]{\begin{tcolorbox}[boxrule=0.3 mm,arc=0mm,enhanced jigsaw,breakable,colback=green!3] {\large \textbf{Eksempel #1} \vspace{5 pt}\\} #2 \end{tcolorbox}\vspace{-5pt} }

\newcommand{\st}[1]{\begin{tcolorbox}[boxrule=0.0 mm,arc=0mm,enhanced jigsaw,breakable,colback=yellow!12]{ #1} \end{tcolorbox}}

\newcommand{\spr}[1]{\begin{tcolorbox}[boxrule=0.3 mm,arc=0mm,enhanced jigsaw,breakable,colback=yellow!7] {\large \textbf{Språkboksen} \vspace{5 pt}\\} #1 \end{tcolorbox}\vspace{-5pt} }

\newcommand{\sym}[1]{\colorbox{blue!15}{#1}}

\newcommand{\info}[2]{\begin{tcolorbox}[boxrule=0.3 mm,arc=0mm,enhanced jigsaw,breakable,colback=cyan!6] {\large \textbf{#1} \vspace{5 pt}\\} #2 \end{tcolorbox}\vspace{-5pt} }

\newcommand\algv[1]{\vspace{-11 pt}\begin{align*} #1 \end{align*}}

\newcommand{\regv}{\vspace{5pt}}
\newcommand{\mer}{\textsl{Merk}: }
\newcommand{\mers}[1]{{\footnotesize \mer #1}}
\newcommand\vsk{\vspace{11pt}}
\newcommand\vs{\vspace{-11pt}}
\newcommand\vsb{\vspace{-16pt}}
\newcommand\sv{\vsk \textbf{Svar} \vspace{4 pt}\\}
\newcommand\br{\\[5 pt]}
\newcommand{\figp}[1]{../fig/#1}
\newcommand\algvv[1]{\vs\vs\begin{align*} #1 \end{align*}}
\newcommand{\y}[1]{$ {#1} $}
\newcommand{\os}{\\[5 pt]}
\newcommand{\prbxl}[2]{
\parbox[l][][l]{#1\linewidth}{#2
	}}
\newcommand{\prbxr}[2]{\parbox[r][][l]{#1\linewidth}{
		\setlength{\abovedisplayskip}{5pt}
		\setlength{\belowdisplayskip}{5pt}	
		\setlength{\abovedisplayshortskip}{0pt}
		\setlength{\belowdisplayshortskip}{0pt} 
		\begin{shaded}
			\footnotesize	#2 \end{shaded}}}

\renewcommand{\cfttoctitlefont}{\Large\bfseries}
\setlength{\cftaftertoctitleskip}{0 pt}
\setlength{\cftbeforetoctitleskip}{0 pt}

\newcommand{\bs}{\\[3pt]}
\newcommand{\vn}{\\[6pt]}
\newcommand{\fig}[1]{\begin{figure}
		\centering
		\includegraphics[]{\figp{#1}}
\end{figure}}

\newcommand{\figc}[2]{\begin{figure}
		\centering
		\includegraphics[]{\figp{#1}}
		\caption{#2}
\end{figure}}

\newcommand{\sectionbreak}{\clearpage} % New page on each section

\newcommand{\nn}[1]{
\begin{equation}
	#1
\end{equation}
}

% Equation comments
\newcommand{\cm}[1]{\llap{\color{blue} #1}}

\newcommand\fork[2]{\begin{tcolorbox}[boxrule=0.3 mm,arc=0mm,enhanced jigsaw,breakable,colback=yellow!7] {\large \textbf{#1 (forklaring)} \vspace{5 pt}\\} #2 \end{tcolorbox}\vspace{-5pt} }
 
%colors
\newcommand{\colr}[1]{{\color{red} #1}}
\newcommand{\colb}[1]{{\color{blue} #1}}
\newcommand{\colo}[1]{{\color{orange} #1}}
\newcommand{\colc}[1]{{\color{cyan} #1}}
\definecolor{projectgreen}{cmyk}{100,0,100,0}
\newcommand{\colg}[1]{{\color{projectgreen} #1}}

% Methods
\newcommand{\metode}[2]{
	\textsl{#1} \\[-8pt]
	\rule{#2}{0.75pt}
}

%Opg
\newcommand{\abc}[1]{
	\begin{enumerate}[label=\alph*),leftmargin=18pt]
		#1
	\end{enumerate}
}
\newcommand{\abcs}[2]{
	\begin{enumerate}[label=\alph*),start=#1,leftmargin=18pt]
		#2
	\end{enumerate}
}
\newcommand{\abcn}[1]{
	\begin{enumerate}[label=\arabic*),leftmargin=18pt]
		#1
	\end{enumerate}
}
\newcommand{\abch}[1]{
	\hspace{-2pt}	\begin{enumerate*}[label=\alph*), itemjoin=\hspace{1cm}]
		#1
	\end{enumerate*}
}
\newcommand{\abchs}[2]{
	\hspace{-2pt}	\begin{enumerate*}[label=\alph*), itemjoin=\hspace{1cm}, start=#1]
		#2
	\end{enumerate*}
}

% Oppgaver
\newcommand{\opgt}{\phantomsection \addcontentsline{toc}{section}{Oppgaver} \section*{Oppgaver for kapittel \thechapter}\vs \setcounter{section}{1}}
\newcounter{opg}
\numberwithin{opg}{section}
\newcommand{\op}[1]{\vspace{15pt} \refstepcounter{opg}\large \textbf{\color{blue}\theopg} \vspace{2 pt} \label{#1} \\}
\newcommand{\ekspop}[1]{\vsk\textbf{Gruble \thechapter.#1}\vspace{2 pt} \\}
\newcommand{\nes}{\stepcounter{section}
	\setcounter{opg}{0}}
\newcommand{\opr}[1]{\vspace{3pt}\textbf{\ref{#1}}}
\newcommand{\oeks}[1]{\begin{tcolorbox}[boxrule=0.3 mm,arc=0mm,colback=white]
		\textit{Eksempel: } #1	  
\end{tcolorbox}}
\newcommand\opgeks[2][]{\begin{tcolorbox}[boxrule=0.1 mm,arc=0mm,enhanced jigsaw,breakable,colback=white] {\footnotesize \textbf{Eksempel #1} \\} \footnotesize #2 \end{tcolorbox}\vspace{-5pt} }
\newcommand{\rknut}{
Rekn ut.
}

%License
\newcommand{\lic}{\textit{Matematikken sine byggesteinar by Sindre Sogge Heggen is licensed under CC BY-NC-SA 4.0. To view a copy of this license, visit\\ 
		\net{http://creativecommons.org/licenses/by-nc-sa/4.0/}{http://creativecommons.org/licenses/by-nc-sa/4.0/}}}

%referances
\newcommand{\net}[2]{{\color{blue}\href{#1}{#2}}}
\newcommand{\hrs}[2]{\hyperref[#1]{\color{blue}\textsl{#2 \ref*{#1}}}}
\newcommand{\rref}[1]{\hrs{#1}{regel}}
\newcommand{\refkap}[1]{\hrs{#1}{kapittel}}
\newcommand{\refsec}[1]{\hrs{#1}{seksjon}}

\newcommand{\mb}{\net{https://sindrsh.github.io/FirstPrinciplesOfMath/}{MB}}


%line to seperate examples
\newcommand{\linje}{\rule{\linewidth}{1pt} }

\usepackage{datetime2}
%%\usepackage{sansmathfonts} for dyslexia-friendly math
\usepackage[]{hyperref}


\newcommand{\note}{Merk}
\newcommand{\notesm}[1]{{\footnotesize \textsl{\note:} #1}}
\newcommand{\ekstitle}{Eksempel }
\newcommand{\sprtitle}{Språkboksen}
\newcommand{\expl}{forklaring}

\newcommand{\vedlegg}[1]{\refstepcounter{vedl}\section*{Vedlegg \thevedl: #1}  \setcounter{vedleq}{0}}

\newcommand\sv{\vsk \textbf{Svar} \vspace{4 pt}\\}

%references
\newcommand{\reftab}[1]{\hrs{#1}{tabell}}
\newcommand{\rref}[1]{\hrs{#1}{regel}}
\newcommand{\dref}[1]{\hrs{#1}{definisjon}}
\newcommand{\refkap}[1]{\hrs{#1}{kapittel}}
\newcommand{\refsec}[1]{\hrs{#1}{seksjon}}
\newcommand{\refdsec}[1]{\hrs{#1}{delseksjon}}
\newcommand{\refved}[1]{\hrs{#1}{vedlegg}}
\newcommand{\eksref}[1]{\textsl{#1}}
\newcommand\fref[2][]{\hyperref[#2]{\textsl{figur \ref*{#2}#1}}}
\newcommand{\refop}[1]{{\color{blue}Oppgave \ref{#1}}}
\newcommand{\refops}[1]{{\color{blue}oppgave \ref{#1}}}
\newcommand{\refgrubs}[1]{{\color{blue}gruble \ref{#1}}}

\newcommand{\openmathser}{\openmath\,-\,serien}

% Exercises
\newcommand{\opgt}{\newpage \phantomsection \addcontentsline{toc}{section}{Oppgaver} \section*{Oppgaver for kapittel \thechapter}\vs \setcounter{section}{1}}


% Sequences and series
\newcommand{\sumarrek}{Summen av en aritmetisk rekke}
\newcommand{\sumgerek}{Summen av en geometrisk rekke}
\newcommand{\regnregsum}{Regneregler for summetegnet}

% Trigonometry
\newcommand{\sincoskomb}{Sinus og cosinus kombinert}
\newcommand{\cosfunk}{Cosinusfunksjonen}
\newcommand{\trid}{Trigonometriske identiteter}
\newcommand{\deravtri}{Den deriverte av de trigonometriske funksjonene}
% Solutions manual
\newcommand{\selos}{Se løsningsforslag.}
\newcommand{\se}[1]{Se eksempel på side \pageref{#1}}

%Vectors
\newcommand{\parvek}{Parallelle vektorer}
\newcommand{\vekpro}{Vektorproduktet}
\newcommand{\vekproarvol}{Vektorproduktet som areal og volum}


% 3D geometries
\newcommand{\linrom}{Linje i rommet}
\newcommand{\avstplnpkt}{Avstand mellom punkt og plan}


% Integral
\newcommand{\bestminten}{Bestemt integral I}
\newcommand{\anfundteo}{Analysens fundamentalteorem}
\newcommand{\intuf}{Integralet av utvalge funksjoner}
\newcommand{\bytvar}{Bytte av variabel}
\newcommand{\intvol}{Integral som volum}
\newcommand{\andordlindif}{Andre ordens lineære differensialligninger}



\begin{document}
	
\opgt

\op{geoopgformlvink}
Trekantene er formlike. Bestem verdien til $ \angle ACB $. \vs
\fig{geoopg2}

\op{geoopgsamsv}
Trekantene er formlike. Finn de tre parene med samsvarende sider.
\fig{geoopg3} 

\op{geofinnlen}
Trekantene er formlike. Finn lengden til $ EF $ og lengden til $ DF $.
\fig{geoopg4}
\newpage
\op{geofinnlen2}
Trekantene er formlike. Finn lengden til $ AC $ og lengden til $ DF $.
\fig{geoopg5}

\op{opggeofinnforml}
Finn alle formlike trekanter definert av $ A $, $ B $, $ C $ og $ D $.
\fig{opggeofinnforml}

\op{opggeoarforhold}
$ \triangle ABC $ og $ \triangle DEF $ er formlike.
\abc{
\item Hva er forholdet mellom arealet til $ \triangle DEF $ og arealet til $ \triangle ABC $ hvis $ h_1=2 $ og $ h_2=6 $?
\item Gitt et tall $ a $, og at $ h_2=a h_1 $. Uttrykk forholdet mellom arealet til $ \triangle DEF $ og arealet til $ \triangle ABC $ ved $ a $.
}
\fig{opggeoarforh}

\op{opggeovolkjegl}
En kjegle har radius 10 og høgde 4.
\abc{
\item Finn grunnflaten til kjeglen.
\item Finn volumet til kjeglen.
}

\op{opggepvolforhold} \vs
\abc{
	\item En kule har radius $ 2 $ og en annen kule har radius $ 6 $. Hva er forholdet mellom volumet til den største kula og volumet til den minste kula?
	\item En kule har radius $ r $ og en annen kule har radius $ ar $, hvor $ {a>1} $. Hva er forholdet mellom volumet til den største kula og volumet til den minste kula?
}

\newpage

\grubop{opggeo263}
\fig{opggeo263}
Finn lengden til den røde linja.

\grubop{opgegeotrapes}
\fig{opggeotrapes}
Finn arealet til det grønne området.

\grubop{GV21D1opg12}
(GV21D1) \os
Figuren under viser en regulær\footnote{I regulære mangekanter har alle sidene lik lengde.} sekskant. Bestem hvor mange grader $  v $ er.
\fig{opggv21d1opg12}

\grubop{opglikbmidtn}
Gitt en likebeint trekant $\triangle {ABC} $ hvor $ AC=BC $. Vis at halveringslinja\footnote{Definisjonen av halveringslinja til en vinkel og midtnormalen til ei linje finner du i \tmen.} til $ \angle ACB $ er midtnormalen til $ AB $.

\grubop{opggeoeqlheight}
$ \triangle ABC $ er likesidet og har sidelengde $ s $. 
\fig{opggeoeqlheight}
\abc{
\item Vis at i en trekant med vinklene $ 30^\circ $, $ 60^\circ $, $ 90^\circ $, så er den lengste siden dobbelt så lang som den korteste siden.
\item Vis at høgda i trekanten er $ \frac{\sqrt{3}}{2}s $.
}


\grubop{opg306090}
Gitt $ \triangle ABC $ hvor $ \angle BAC=90^\circ $, $ \angle ACB=60^\circ $ og $ \angle CBA=30^\circ $. Vis at $ BC=2AC $.

\newpage
\grubop{opgdoublear}
Vis at det doble arealet til $ \triangle ABC $ er gitt som
\[ AE\cdot BD + CE\cdot AD \]
\fig{opggeoqrst}


\grubop{opgmedian}
En \outl{median} i en trekant er et linjestykke som går fra et hjørne til midten av den motstående siden. 
\fig{geoopg1}
Gitt en vilkårlig trekant $ \triangle ABC $ med medianer $ AE $, $ BF $ og  $ CD $. 
\abc{
\item Vis at $ AE $, $ BF $ og $ CD $ skjærer hverandre i samme punkt ($ G $ på figuren).
\item Vis at
\[ \frac{GC}{DG}=\frac{GB}{FG}=\frac{GA}{EG}=2 \]
}
\mers{Oppgave b) er nok lettere enn oppgave a).}

\newpage
\grubop{opgtresirkar}
De tre sirklene har radius $ 2 $, og $ A $, $ B $ og $ C $ ligger på linje.
Finn arealet til det røde området.
\fig{opggeotresirk}
{\footnotesize Hint: Her kan du nok få bruk for at arealet til en sektor med vinkel $ v $ utgjør $ \frac{a}{360^\circ} $ av arealet til sirkelen med samme radius.}


\grubop{opggeolikar}
De fargede områdene utgjør et kvadrat, og $ F $, $ G $, $ H $ og $ I $ er de respektive midpunktene på sidene til dette kvadratet.\os

Vi at arealet til det blåfargede området er det samme som arealet til det grønnfargede området.
\fig{opggeolikar}

\grubop{opggeosirkfirk}
Kvadratet har sidelengde 4. Finn radien til sirkelen.
\fig{opggeosirkfirk}

\newpage
\grubop{opggeosin15inv}
\abc{
\item Vis at $ \frac{a}{b}=\sqrt{2}+\sqrt{6} $.\\
\fig{opggeosin15inv}
\mers{For å løse denne oppgaven er det mulig (men ikke nødvendigvis) du vil få bruk for \textit{abc}-formelen, som du finner i \tmen.}
\item $ AD=BC $. Bestem verdien til $ \angle A $.
\fig{opggeo1530}
}
\newpage
\grubop{opggeotrekulik}
\mers{Denne oppgaven tar for seg resultater som intuitivt virker helt opplagte, men som kan være krevende å bevise. 
}
\abc{
\item Vis at hvis $ AC=BC $, er $ \angle A =\angle B$.
\fig{opggeotrekulikisc}
\mers{Vi har tidligere erklært at en likebeint trekant har to vinkler som er like store, men strengt tatt kan vi ikke bare gå ut ifra at det er slik.
}
\item Vis at hvis $ AC>BC $, er $ \angle B>\angle C $.
\fig{opggeotrekulikvink}
\item Gitt $ \triangle ABC $, hvor $ AB $ er den lengste siden. Vis at når $ AB $ er grunnlinje, ligger høgden inni trekanten.
\item I figuren under er $ c $ den lengste siden i trekanten. 
\fig{opggeotrekulik}
Bevis at  
\[ c>a+b\qquad,\qquad b+c>a\qquad,\qquad a+c>b  \]
\mers{Disse tre ulikhetene samlet kalles gjerne \outl{trekantulikheten}.
}

}


\end{document}

