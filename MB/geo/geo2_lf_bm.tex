\documentclass[english, 11 pt, class=article, crop=false]{standalone}
\usepackage[T1]{fontenc}
%\renewcommand*\familydefault{\sfdefault} % For dyslexia-friendly text
\usepackage{lmodern} % load a font with all the characters
\usepackage{geometry}
\geometry{verbose,paperwidth=16.1 cm, paperheight=24 cm, inner=2.3cm, outer=1.8 cm, bmargin=2cm, tmargin=1.8cm}
\setlength{\parindent}{0bp}
\usepackage{import}
\usepackage[subpreambles=false]{standalone}
\usepackage{amsmath}
\usepackage{amssymb}
\usepackage{esint}
\usepackage{babel}
\usepackage{tabu}
\makeatother
\makeatletter

\usepackage{titlesec}
\usepackage{ragged2e}
\RaggedRight
\raggedbottom
\frenchspacing

% Norwegian names of figures, chapters, parts and content
\addto\captionsenglish{\renewcommand{\figurename}{Figur}}
\makeatletter
\addto\captionsenglish{\renewcommand{\chaptername}{Kapittel}}
\addto\captionsenglish{\renewcommand{\partname}{Del}}


\usepackage{graphicx}
\usepackage{float}
\usepackage{subfig}
\usepackage{placeins}
\usepackage{cancel}
\usepackage{framed}
\usepackage{wrapfig}
\usepackage[subfigure]{tocloft}
\usepackage[font=footnotesize,labelfont=sl]{caption} % Figure caption
\usepackage{bm}
\usepackage[dvipsnames, table]{xcolor}
\definecolor{shadecolor}{rgb}{0.105469, 0.613281, 1}
\colorlet{shadecolor}{Emerald!15} 
\usepackage{icomma}
\makeatother
\usepackage[many]{tcolorbox}
\usepackage{multicol}
\usepackage{stackengine}

\usepackage{esvect} %For vectors with capital letters

% For tabular
\usepackage{array}
\usepackage{multirow}
\usepackage{longtable} %breakable table

% Ligningsreferanser
\usepackage{mathtools}
\mathtoolsset{showonlyrefs}

% index
\usepackage{imakeidx}
\makeindex[title=Indeks]

%Footnote:
\usepackage[bottom, hang, flushmargin]{footmisc}
\usepackage{perpage} 
\MakePerPage{footnote}
\addtolength{\footnotesep}{2mm}
\renewcommand{\thefootnote}{\arabic{footnote}}
\renewcommand\footnoterule{\rule{\linewidth}{0.4pt}}
\renewcommand{\thempfootnote}{\arabic{mpfootnote}}

%colors
\definecolor{c1}{cmyk}{0,0.5,1,0}
\definecolor{c2}{cmyk}{1,0.25,1,0}
\definecolor{n3}{cmyk}{1,0.,1,0}
\definecolor{neg}{cmyk}{1,0.,0.,0}

% Lister med bokstavar
\usepackage[inline]{enumitem}

\newcounter{rg}
\numberwithin{rg}{chapter}
\newcommand{\reg}[2][]{\begin{tcolorbox}[boxrule=0.3 mm,arc=0mm,colback=blue!3] {\refstepcounter{rg}\phantomsection \large \textbf{\therg \;#1} \vspace{5 pt}}\newline #2  \end{tcolorbox}\vspace{-5pt}}

\newcommand\alg[1]{\begin{align} #1 \end{align}}

\newcommand\eks[2][]{\begin{tcolorbox}[boxrule=0.3 mm,arc=0mm,enhanced jigsaw,breakable,colback=green!3] {\large \textbf{Eksempel #1} \vspace{5 pt}\\} #2 \end{tcolorbox}\vspace{-5pt} }

\newcommand{\st}[1]{\begin{tcolorbox}[boxrule=0.0 mm,arc=0mm,enhanced jigsaw,breakable,colback=yellow!12]{ #1} \end{tcolorbox}}

\newcommand{\spr}[1]{\begin{tcolorbox}[boxrule=0.3 mm,arc=0mm,enhanced jigsaw,breakable,colback=yellow!7] {\large \textbf{Språkboksen} \vspace{5 pt}\\} #1 \end{tcolorbox}\vspace{-5pt} }

\newcommand{\sym}[1]{\colorbox{blue!15}{#1}}

\newcommand{\info}[2]{\begin{tcolorbox}[boxrule=0.3 mm,arc=0mm,enhanced jigsaw,breakable,colback=cyan!6] {\large \textbf{#1} \vspace{5 pt}\\} #2 \end{tcolorbox}\vspace{-5pt} }

\newcommand\algv[1]{\vspace{-11 pt}\begin{align*} #1 \end{align*}}

\newcommand{\regv}{\vspace{5pt}}
\newcommand{\mer}{\textsl{Merk}: }
\newcommand{\mers}[1]{{\footnotesize \mer #1}}
\newcommand\vsk{\vspace{11pt}}
\newcommand\vs{\vspace{-11pt}}
\newcommand\vsb{\vspace{-16pt}}
\newcommand\sv{\vsk \textbf{Svar} \vspace{4 pt}\\}
\newcommand\br{\\[5 pt]}
\newcommand{\figp}[1]{../fig/#1}
\newcommand\algvv[1]{\vs\vs\begin{align*} #1 \end{align*}}
\newcommand{\y}[1]{$ {#1} $}
\newcommand{\os}{\\[5 pt]}
\newcommand{\prbxl}[2]{
\parbox[l][][l]{#1\linewidth}{#2
	}}
\newcommand{\prbxr}[2]{\parbox[r][][l]{#1\linewidth}{
		\setlength{\abovedisplayskip}{5pt}
		\setlength{\belowdisplayskip}{5pt}	
		\setlength{\abovedisplayshortskip}{0pt}
		\setlength{\belowdisplayshortskip}{0pt} 
		\begin{shaded}
			\footnotesize	#2 \end{shaded}}}

\renewcommand{\cfttoctitlefont}{\Large\bfseries}
\setlength{\cftaftertoctitleskip}{0 pt}
\setlength{\cftbeforetoctitleskip}{0 pt}

\newcommand{\bs}{\\[3pt]}
\newcommand{\vn}{\\[6pt]}
\newcommand{\fig}[1]{\begin{figure}
		\centering
		\includegraphics[]{\figp{#1}}
\end{figure}}

\newcommand{\figc}[2]{\begin{figure}
		\centering
		\includegraphics[]{\figp{#1}}
		\caption{#2}
\end{figure}}

\newcommand{\sectionbreak}{\clearpage} % New page on each section

\newcommand{\nn}[1]{
\begin{equation}
	#1
\end{equation}
}

% Equation comments
\newcommand{\cm}[1]{\llap{\color{blue} #1}}

\newcommand\fork[2]{\begin{tcolorbox}[boxrule=0.3 mm,arc=0mm,enhanced jigsaw,breakable,colback=yellow!7] {\large \textbf{#1 (forklaring)} \vspace{5 pt}\\} #2 \end{tcolorbox}\vspace{-5pt} }
 
%colors
\newcommand{\colr}[1]{{\color{red} #1}}
\newcommand{\colb}[1]{{\color{blue} #1}}
\newcommand{\colo}[1]{{\color{orange} #1}}
\newcommand{\colc}[1]{{\color{cyan} #1}}
\definecolor{projectgreen}{cmyk}{100,0,100,0}
\newcommand{\colg}[1]{{\color{projectgreen} #1}}

% Methods
\newcommand{\metode}[2]{
	\textsl{#1} \\[-8pt]
	\rule{#2}{0.75pt}
}

%Opg
\newcommand{\abc}[1]{
	\begin{enumerate}[label=\alph*),leftmargin=18pt]
		#1
	\end{enumerate}
}
\newcommand{\abcs}[2]{
	\begin{enumerate}[label=\alph*),start=#1,leftmargin=18pt]
		#2
	\end{enumerate}
}
\newcommand{\abcn}[1]{
	\begin{enumerate}[label=\arabic*),leftmargin=18pt]
		#1
	\end{enumerate}
}
\newcommand{\abch}[1]{
	\hspace{-2pt}	\begin{enumerate*}[label=\alph*), itemjoin=\hspace{1cm}]
		#1
	\end{enumerate*}
}
\newcommand{\abchs}[2]{
	\hspace{-2pt}	\begin{enumerate*}[label=\alph*), itemjoin=\hspace{1cm}, start=#1]
		#2
	\end{enumerate*}
}

% Oppgaver
\newcommand{\opgt}{\phantomsection \addcontentsline{toc}{section}{Oppgaver} \section*{Oppgaver for kapittel \thechapter}\vs \setcounter{section}{1}}
\newcounter{opg}
\numberwithin{opg}{section}
\newcommand{\op}[1]{\vspace{15pt} \refstepcounter{opg}\large \textbf{\color{blue}\theopg} \vspace{2 pt} \label{#1} \\}
\newcommand{\ekspop}[1]{\vsk\textbf{Gruble \thechapter.#1}\vspace{2 pt} \\}
\newcommand{\nes}{\stepcounter{section}
	\setcounter{opg}{0}}
\newcommand{\opr}[1]{\vspace{3pt}\textbf{\ref{#1}}}
\newcommand{\oeks}[1]{\begin{tcolorbox}[boxrule=0.3 mm,arc=0mm,colback=white]
		\textit{Eksempel: } #1	  
\end{tcolorbox}}
\newcommand\opgeks[2][]{\begin{tcolorbox}[boxrule=0.1 mm,arc=0mm,enhanced jigsaw,breakable,colback=white] {\footnotesize \textbf{Eksempel #1} \\} \footnotesize #2 \end{tcolorbox}\vspace{-5pt} }
\newcommand{\rknut}{
Rekn ut.
}

%License
\newcommand{\lic}{\textit{Matematikken sine byggesteinar by Sindre Sogge Heggen is licensed under CC BY-NC-SA 4.0. To view a copy of this license, visit\\ 
		\net{http://creativecommons.org/licenses/by-nc-sa/4.0/}{http://creativecommons.org/licenses/by-nc-sa/4.0/}}}

%referances
\newcommand{\net}[2]{{\color{blue}\href{#1}{#2}}}
\newcommand{\hrs}[2]{\hyperref[#1]{\color{blue}\textsl{#2 \ref*{#1}}}}
\newcommand{\rref}[1]{\hrs{#1}{regel}}
\newcommand{\refkap}[1]{\hrs{#1}{kapittel}}
\newcommand{\refsec}[1]{\hrs{#1}{seksjon}}

\newcommand{\mb}{\net{https://sindrsh.github.io/FirstPrinciplesOfMath/}{MB}}


%line to seperate examples
\newcommand{\linje}{\rule{\linewidth}{1pt} }

\usepackage{datetime2}
%%\usepackage{sansmathfonts} for dyslexia-friendly math
\usepackage[]{hyperref}


\newcommand{\note}{Merk}
\newcommand{\notesm}[1]{{\footnotesize \textsl{\note:} #1}}
\newcommand{\ekstitle}{Eksempel }
\newcommand{\sprtitle}{Språkboksen}
\newcommand{\expl}{forklaring}

\newcommand{\vedlegg}[1]{\refstepcounter{vedl}\section*{Vedlegg \thevedl: #1}  \setcounter{vedleq}{0}}

\newcommand\sv{\vsk \textbf{Svar} \vspace{4 pt}\\}

%references
\newcommand{\reftab}[1]{\hrs{#1}{tabell}}
\newcommand{\rref}[1]{\hrs{#1}{regel}}
\newcommand{\dref}[1]{\hrs{#1}{definisjon}}
\newcommand{\refkap}[1]{\hrs{#1}{kapittel}}
\newcommand{\refsec}[1]{\hrs{#1}{seksjon}}
\newcommand{\refdsec}[1]{\hrs{#1}{delseksjon}}
\newcommand{\refved}[1]{\hrs{#1}{vedlegg}}
\newcommand{\eksref}[1]{\textsl{#1}}
\newcommand\fref[2][]{\hyperref[#2]{\textsl{figur \ref*{#2}#1}}}
\newcommand{\refop}[1]{{\color{blue}Oppgave \ref{#1}}}
\newcommand{\refops}[1]{{\color{blue}oppgave \ref{#1}}}
\newcommand{\refgrubs}[1]{{\color{blue}gruble \ref{#1}}}

\newcommand{\openmathser}{\openmath\,-\,serien}

% Exercises
\newcommand{\opgt}{\newpage \phantomsection \addcontentsline{toc}{section}{Oppgaver} \section*{Oppgaver for kapittel \thechapter}\vs \setcounter{section}{1}}


% Sequences and series
\newcommand{\sumarrek}{Summen av en aritmetisk rekke}
\newcommand{\sumgerek}{Summen av en geometrisk rekke}
\newcommand{\regnregsum}{Regneregler for summetegnet}

% Trigonometry
\newcommand{\sincoskomb}{Sinus og cosinus kombinert}
\newcommand{\cosfunk}{Cosinusfunksjonen}
\newcommand{\trid}{Trigonometriske identiteter}
\newcommand{\deravtri}{Den deriverte av de trigonometriske funksjonene}
% Solutions manual
\newcommand{\selos}{Se løsningsforslag.}
\newcommand{\se}[1]{Se eksempel på side \pageref{#1}}

%Vectors
\newcommand{\parvek}{Parallelle vektorer}
\newcommand{\vekpro}{Vektorproduktet}
\newcommand{\vekproarvol}{Vektorproduktet som areal og volum}


% 3D geometries
\newcommand{\linrom}{Linje i rommet}
\newcommand{\avstplnpkt}{Avstand mellom punkt og plan}


% Integral
\newcommand{\bestminten}{Bestemt integral I}
\newcommand{\anfundteo}{Analysens fundamentalteorem}
\newcommand{\intuf}{Integralet av utvalge funksjoner}
\newcommand{\bytvar}{Bytte av variabel}
\newcommand{\intvol}{Integral som volum}
\newcommand{\andordlindif}{Andre ordens lineære differensialligninger}




\begin{document}
\footnotesize

\grubr{opgGV21D1opg12} \\
En regulær sekskant kan deles in i seks kongruente, likesidete trekanter. Dette betyr at $ \angle C=120^\circ$. Da $ \triangle ABC $ er likebeint, er derfor
\alg{
2\angle BAC +120^\circ &= 180^\circ \\
\angle BAC &= 30
}
\fig{opggv21d1opg12los}

\grubr{opglikbmidtn}
\fig{opglikbmidtnlos}
Vi lar $ D $ være punktet der halveringslinja til $ \angle ACB $ skjærer $ AB $. $ \triangle DAC\cong \triangle DBC $ fordi de har $ CD $ felles og $ AC=BC $ (trekantene oppfyller altså vilkår iii for formlikhet, og må da være kongruente). Følgelig er $ \angle BDA=\angle ADC $, og da er $ 2\angle DBA=180^\circ $. Altså er $ \angle DBA=90^\circ $, og da $ AD=BD $, ligger $ DC $ på midtnormalen til $ AB $.
\newpage
\grubr{opggeoeqlheight}
\fig{opggeoeqlheightlos}
\abc{
\item Da $ \triangle ABC $ er likesidet, er $ D $ midpunktet på $ AB $. Dermed er 
\nn{
AD=DB=\frac{AB}{2}=\frac{s}{2}
}
$ \triangle ACD $ er en trekant med vinkler lik $ 30^\circ $, $ 60^\circ $ og $ 90^\circ $ og $ AC=2AD $. Altså er den lengste siden dobbelt så lang som den korteste.
\item Av \pyt på $ \triangle ADC $ har vi at
\alg{
CD^2 &= AC^2 - AD^2 \\
&= s^2+\left(\frac{1}{2}\right)^2 \\
&= \frac{3}{4}s^2
}
}
Altså er 
\[ CD=\sqrt{\frac{3}{4}s^2}=\frac{\sqrt{3}}{2}s \]

\grubr{opggeovispyt}
\fig{opggeovispyt}
\abc{
\item $ \triangle CDA \sim \triangle BCA $ fordi begge er rettvinklede og de har $ \angle BAC $ felles. Dermed er
\alg{
AD&=\frac{AC}{AB}AC = \frac{b^2}{c}
}
\item $ \triangle BDA \sim \triangle BCA $ fordi begge er rettvinklede og de har $ \angle CBA $ felles. Dermed er
\alg{
	DB&=\frac{BC}{AB}BC = \frac{a^2}{c}
}
\item Vi har at
\alg{
c&=AD+DB \\
c &= \frac{b^2}{c}+ \frac{a^2}{c} \br
c^2 &= b^2+a^2
}
}

\grubr{opggeotresirk}
\fig{opggeotresirklos}
$ \triangle ABD $ er likesidet fordi $ AD=AB=BD $, og har dermed areal lik $ \frac{1}{2}\cdot \frac{\sqrt{3}}{2}AB^2=\sqrt{3} $. Da $ \angle B=60^\circ $, utgjør den grønne sektoren $ \frac{1}{6} $ av sirklenes areal, følgelig er arealet til den grønne sektoren $ \frac{1}{6}\cdot \pi\cdot2^2=\frac{2\pi}{3} $. Vi har at
\alg{
\text{areal til grønt og blått område}&=2\cdot\text{areal til grønt område}-A_{\triangle ABD} \\
&=\frac{4\pi}{3}-\sqrt{3}
}
Videre har vi at
\alg{
\text{areal til rødt omrdåde}&=\text{areal til sirkel}-4\cdot\text{areal til grønt og blått område} \\
&= 4\pi-4\left(\frac{4\pi}{3}-\sqrt{3}\right) \\
&= 4\sqrt{3}-\frac{4}{3}\pi
}

\grubr{opggeoqrst}
\fig{opggeoqrst_lf}	
$ {\triangle EFC \sim \triangle DFB} $ fordi begge er rettvinklede, og $ {\angle CFE = \angle BFD}$ (de er toppvinkler). Dermed har vi at
\begin{equation}\label{opggeoqrsteq1}
	\frac{EF}{CE}=\frac{FD}{BD} 
\end{equation}
Videre er
\begin{equation}\label{opggeoqrsteq2}
	EF+FD= AD-AE
\end{equation}
Ved å løse likningssettet vi får av \eqref{opggeoqrsteq1} og \eqref{opggeoqrsteq2}, med hensyn på $ EF $ og $ ED $, får vi at
\[ 
EF = \frac{AD-AE}{CE+BD}CE\qquad,\qquad  FD=\frac{AD-AE}{CE+BD}BD
\]
Det doble arealet til $ \triangle ABC $ er gitt som
\begin{multline*}
(AE+EF)CE+(AD-FD)BD \\=\left(AE+\frac{AD-AE}{CE+BD}CE\right)CE+\left(AD-\frac{AD-AE}{CE+BD}BC\right)BD	
\end{multline*}
\alg{
&=\frac{1}{CE+BD}\left[\left(AE\cdot BD+AD\cdot CE\right)CE+\left(AD\cdot CE+AE\cdot BD\right)BD\right]\br
&=AD\cdot CE+ AE\cdot BD
}

\newpage
\grubr{opggeolikar}
\fig{opggeolikarlos}
Av å legge merke til trekanter med grunnlinje og høgde av lik lengde, finner vi at
\alg{
A_{\triangle AFE}=A_{\triangle FBE} &&
A_{\triangle AIE}=A_{\triangle EDI}\vn
A_{\triangle BCE}=A_{\triangle GCE}&&
A_{\triangle HDE}=A_{\triangle HCE}
}
Følgelig er
\alg{
(A_{\triangle AFE} + A_{\triangle AIE})+(A_{\triangle BCE}+A_{\triangle HDE}) &= (A_{\triangle FBE}+A_{\triangle EDI})+ (A_{\triangle GCE}+A_{\triangle HCE}) \\
A_{\square AFEI} + A_{\square GCHE} &= A_{\square FBGE} + A_{\square DIEH}
}
Altså er arealet til det blåfargede området er det samme som arealet til det grønnfargede området.\vsk


\grubr{opggeosirkfirk}\\
Vi lar $ r $ være radien til sirkelen. Vi har at $ AS=ES=r $, $ AF=2 $, og at $ FS=EF-SE=4-r $. Av \pyt\ med hensyn på $ \triangle AFS $ er
\alg{
AS^2&=AF^2+SF^2 \br
r^2&=2^2+(4-r)^2 \\
r^2&=4+16-8r+r^2 \\
8r &= 20 \\
r&= \frac{5}{2}
} 
\fig{opggeosirkfirk_lf}
\newpage

\grubr{opggeo1530}
\abc{ 
	\item \mbox{}
	\textbf{Alternativ 1}
	\fig{opggeosin15invlos3}
	Med hensyn på vinkelsummen i $ \triangle ABC $ har vi at $ \angle ACB= 90-15^\circ=75^\circ$. Vi lar $ D $ være punktet på $ AB $ slik at $ \angle ACD=15^\circ $. Da er $ \angle CDA =75^\circ$ og $ \angle DCE=60^\circ $. Videre lar vi $ E $ være punktet på $ BC $ slik at $ CD=CE $, da er $ \triangle CDE $ likesidet. Vi setter $ s=CD $. Med hensyn på vinkelsummen i $ \triangle CBD $ er $ \angle BDC=180^\circ-15^\circ-60^\circ=105^\circ $, og da er $ \angle FDE=45^\circ $. Altså er $ \triangle DFE $ rettvinklet og likebeint, som betyr at $ DF=\frac{s}{\sqrt{2}} $. Altså er
	\[ CF = CG+GF=\frac{\sqrt{3}}{2}s+\frac{s}{2} \]
	Vi uttrykker det doble arealet til $ \triangle DFC $ på to måter:
	\begin{flalign*}
	&&DF\cdot CA &= GD\cdot CF \\
	&&\frac{s}{\sqrt{2}} b &=\frac{s}{2}\left(\frac{\sqrt{3}}{2}s+\frac{s}{2}\right) &&(s\neq0) \\
	&&4b&=s(\sqrt{2}+\sqrt{6}) \\
	&& s&= \frac{4b}{\sqrt{2}+\sqrt{6}}
	\end{flalign*}
	Da $ \triangle ABC\sim \triangle BFE $, er
	\alg{
	\frac{BC}{AC}&=\frac{BE}{EF} \br 
	\frac{a}{b}&=\frac{a-s}{\frac{s}{\sqrt{2}}}\\
	sa-a\sqrt{2}&=-bs\sqrt{2}\br
	\frac{a}{b} &= s\frac{\sqrt{2}}{\sqrt{2}b-s}
	}
	Altså er
	\[ \frac{a}{b}=\frac{4b}{\sqrt{2}+\sqrt{6}}\cdot\frac{\sqrt{2}}{\sqrt{2}b-\frac{4b}{\sqrt{2}+\sqrt{6}}}=\sqrt{2}+\sqrt{6} \]
	\newpage
	\textbf{Alternativ 2}
	\fig{opggeosin15invlos2}
	Med hensyn på vinkelsummen i $ \triangle ABC $ har vi at $ \angle ACB= 90-15^\circ=75^\circ$. Vi lar $ D $ være punktet på $ AB $ slik at $ \angle ACD=15^\circ $. Da er $ \angle CDA =75^\circ$ og $ \angle DCE=60^\circ $, og dermed er $ \triangle CDE $ en $ 30^\circ $, $ 60^\circ $, $ 90^\circ $ trekant. Vi setter $ s=CE $ og $ c=AB $. Da er $ DE=\frac{\sqrt{3}}{2}s $ og $ CE=\frac{s}{2} $. $ \triangle ABC \sim \triangle ACD \sim \triangle EBD$ fordi alle er rettvinklede og har en vinkel lik $ 15^\circ $. Dermed er
	\alg{
	CD\cdot AB&=BC\cdot AC \\
	cs &= ab
	}
	Videre har vi at
	\alg{
	\frac{AB}{AC}&=\frac{EB}{DE}  \br
	\frac{c}{b}&=\frac{a-\frac{s}{2}}{\frac{\sqrt{3}}{2}s} \br
	c&=\frac{2ab-s}{\sqrt{3}s} \\
	\sqrt{3}c &= 2c-b 
	}
	Altså er
	\[ c= \frac{b}{2-\sqrt{3}}=b(2+\sqrt{3}) \]
	Av \pyt\ med hensyn på $ \triangle ABC $ er
	\alg{
	a^2 &=b^2+c^2 \\
	a^2 &= b^2(2+\sqrt{3})^2+b^2 \\
	\frac{a^2}{b^2}&= 8+4\sqrt{3}
	} 
	Da $ (\sqrt{2}+\sqrt{6})^2=8+4\sqrt{3} $, er
	\[ \frac{a}{b}=\sqrt{2}+\sqrt{6} \]
	\newpage
	\textbf{Alternativ 3}
	\fig{opggeosin15invlos}
	Med hensyn på vinkelsummen i $ \triangle ABC $ har vi at $ \angle ACB= 90-15^\circ=75^\circ$. Vi lar $ D $ være punktet på $ AB $ slik at $ \angle ACD=15^\circ $. Da er $ \angle CDA =75^\circ$ og $ \angle DCE=60^\circ $. Videre lar vi $ E $ være punktet på $ BC $ slik at $ CD=CE $, da er $ \triangle CDE $ likesidet. Vi setter $ s=CD $, og $ c=AB $. $ \triangle ABC\sim \triangle ACD $ fordi begge er rettvinklede, og $ \angle ACD=\angle ABC$. Dermed er
	\alg{
		AD=AC\frac{AC}{AB}=\frac{b^2}{c} \vn
		s=BC\frac{AC}{AB}=\frac{ab}{c}
	}
	Med hensyn på vinkelsummen i $ \triangle CBD $ er $ \angle BDC=180^\circ-15^\circ-60^\circ=105^\circ $, og da er $ \angle FDE=45^\circ $. Altså er $ \triangle DFE $ rettvinklet og likebeint, som betyr at $ DF=FE=\frac{s}{\sqrt{2}} $. Da $ \triangle ABC\sim FBE$, er $ \triangle ACD\sim \triangle FBE $, og dermed er
	\begin{flalign*}
		&&	EF\cdot CD &= AD\cdot EB \\
		&&	\frac{1}{\sqrt{2}}\left(\frac{ab}{c}\right)^2&=\frac{b^2}{c}\left(a-\frac{ab}{c}\right) &&(a,b\neq0) \\
		&& a&=c\sqrt{2}-b\sqrt{2}
	\end{flalign*}
	Av \pyt\ med hensyn på $ \triangle ABC$ har vi at $ c^2=a^2-b^2 $, og følgelig er
	\alg{
		a&=\sqrt{2}\sqrt{a^2-b^2}-b\sqrt{2} \\
		a+b\sqrt{2}&=\sqrt{2}\sqrt{a^2-b^2} \\
		a^2+2ab\sqrt{2}+2b^2&= 2(a^2-b^2) \\
		-a^2+2ab\sqrt{2}+4b^2&=0
	}
	Av \textit{abc}-formelen har vi at
	\alg{
		a&=\frac{-2b\sqrt{2}\pm \sqrt{8b^2+16b^2}}{-2} \\
		&= \left(\sqrt{2}\mp \sqrt{6}\right)b
	}
	Vi forkaster den negative løsningen for $ a $, og får at
	\nn{
		\frac{a}{b}=\sqrt{2}+\sqrt{6}
	}
\newpage
	\item \mbox{}
	
	\fig{opggeo1530los}
	$ A_{\triangle DBC}=A_{\triangle ADC} $ fordi med henholdsvis $ DB $ og $ AD $ som grunnlinje har de lik høgde, og $ DB=AD $. Altså er $ AF\cdot DC = EB\cdot DC $, og da er $ AF=EB $. Videre er $ \triangle DAF\cong \triangle DBE $ fordi begge er rettvinklede $ \angle ADF =\angle BDE$ (de er toppvinkler), og $ AD=DB $. Vi setter $ x=DE $, $ a=EB $ og $ b=AC $. Da $ \triangle BCE $ er en $ 30^\circ $, $ 60^\circ $, $ 90^\circ $ trekant, er $ EC=\sqrt{3}a $ og $ BC=2a $. Da $ \triangle BGC $ er en $ 45^\circ $, $ 45^\circ $ , $ 90^\circ $ trekant, er $ GB=\frac{2}{\sqrt{2}}a $. Da $ A_{\triangle ABC}=2A_{\triangle DBC} $, har vi at
	\alg{
		b\cdot \frac{2}{\sqrt{2}}a &= 2(\sqrt{3}a+x)\cdot a \\
		b &= \sqrt{2}(\sqrt{3}a+x)
	}
	Av løsningen i oppgave a) har vi at $ {AC=(\sqrt{2}+\sqrt{6})AF}$, og dermed er $ b=a\sqrt{2}(\sqrt{3}+1)$. Altså er $ x=a $, som betyr at $ \triangle AFD $ er en $ 45^\circ $, $ 45^\circ $, $ 90^\circ $ trekant. Ved å betrakte vinkelsummen i $ \triangle CAF $, finner vi da at
	\alg{
		\angle DAC &=180^\circ-15^\circ-90^\circ-45^\circ \\
		&=30^\circ
	}
	\textbf{Alternativ metode for å vise at \boldmath $ x=a $} \os
	Av Pytagoras' setning på $ \triangle ACF $ har vi at
	\alg{
		AC^2 &= FC^2 + AF^2 \\
		2(\sqrt{3}a+x)^2 &= (\sqrt{3}a+2x)^2 + a^2 \\
		x^2 &= a^2
	}
}
\newpage
\grubr{opggeotrekulik}
\abc{
\item 
\textbf{Alternativ 1}
\fig{opggeotrekulikisclosp}
Vi lar $ \triangle A'B'C' $ være en speilet utgave av $ \triangle ABC $. Da $ \angle C=\angle C' $, $ \frac{BC}{AB}=\frac{B'C'}{AB} $ og $ \frac{AC}{B'C'}=\frac{BC}{A'C'} $, har vi av vilkår (iii) i \rref{vilkform} at $ \triangle ABC\sim \triangle BA'C' $. Mer spesifikt betyr dette at $ AC $ er den samsvarende siden til $ B'C' $, som betyr at $ \angle B =\angle A'=\angle A $.\vsk

\textbf{Alternativ 2}
\fig{opggeotrekulikisclos}
Vi kan alltids konsturere en rettvinklet trekant $ \triangle A'D'C' $ hvor $ 2AD'=AB $ og $ A'C'=AC $. Ved å la $ B' $ være $ A' $ speilet om $ C'D' $, har vi at $ \angle A' = \angle B $ og $ B'C'=A'C $. Dermed har $ \triangle ABC $ og $ \triangle A'B'C' $ parvis like lange sider, og er derfor kongruente. Da $ AB $ er den samsvarende siden til $ A'B' $, er $ BC $ den samsvarende siden enten til $ B'C' $ eller til $ A'C' $. Uansett hvilke to av disse det er, har vi at $ \angle A= \angle A'=\angle B' $, og tilsvarende er $ \angle B=\angle A'=\angle B' $. 
\newpage
\item Vi plasserer $ D $ på forlengelsen av $ CB $ slik at $ CD=CA $. Av oppgave a) er da $ \angle DAC=\angle D $, som betyr at 
\begin{equation}\label{opggeotrekulikbac}
	\angle BAC<\angle D \qquad, \qquad \angle D-\angle BAC>0
\end{equation}
Videre er $ \angle C=180^\circ-2\angle D $, og da er
\begin{equation}\label{opggeotrekulikvb}
	B=180^\circ-\angle C-\angle BAC= 2\angle D-\angle BAC
\end{equation}
Av \eqref{opggeotrekulikbac} og \eqref{opggeotrekulikvb} har vi at
\[ \angle B> \angle D \]
Dermed er 
\[ \angle BAC<\angle D<\angle B \]
\fig{opggeotrekulikvinklos} 
\item Hvis $ CD $ ligger utenfor $ \triangle ABC $, har vi av \pyt\ at
\[ (AB+DB)^2=AC^2-CD^2 \]
Dette betyr at 
\alg{
(AB+DB)^2 &< AC^2 \\
AB^2 &< AC^2 \\
AB &< AC 
}
Da $ AB $ er den lengste siden i $ \triangle ABC $, er dette en selvmotsigelse, og dermed må $ CD $ ligge inni trekanten.
\fig{opggeotrekulikh}
\item At $ a+c>b $ og at $ b+c>a $ følger direkte av at $ c $ er den største lengden. Av oppgave $ c) $ vet vi at $ CD $ ligger inni $ \triangle ABC $, som vist i figuren under.
\fig{opggeotrekuliklos}
Av \pyt\ har vi at
\[ b^2=AD^2+h^2\qquad,\qquad a^2=BD^2+h^2 \]
Som betyr at
\[ b>AD \qquad,\qquad a>BD \]
Da $ c=AD+DB $, er dermed
\[ c<b+a \]
}

\end{document}

