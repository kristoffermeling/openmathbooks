\documentclass[english, 11 pt, class=article, crop=false]{standalone}
\usepackage[T1]{fontenc}
%\renewcommand*\familydefault{\sfdefault} % For dyslexia-friendly text
\usepackage{lmodern} % load a font with all the characters
\usepackage{geometry}
\geometry{verbose,paperwidth=16.1 cm, paperheight=24 cm, inner=2.3cm, outer=1.8 cm, bmargin=2cm, tmargin=1.8cm}
\setlength{\parindent}{0bp}
\usepackage{import}
\usepackage[subpreambles=false]{standalone}
\usepackage{amsmath}
\usepackage{amssymb}
\usepackage{esint}
\usepackage{babel}
\usepackage{tabu}
\makeatother
\makeatletter

\usepackage{titlesec}
\usepackage{ragged2e}
\RaggedRight
\raggedbottom
\frenchspacing

% Norwegian names of figures, chapters, parts and content
\addto\captionsenglish{\renewcommand{\figurename}{Figur}}
\makeatletter
\addto\captionsenglish{\renewcommand{\chaptername}{Kapittel}}
\addto\captionsenglish{\renewcommand{\partname}{Del}}


\usepackage{graphicx}
\usepackage{float}
\usepackage{subfig}
\usepackage{placeins}
\usepackage{cancel}
\usepackage{framed}
\usepackage{wrapfig}
\usepackage[subfigure]{tocloft}
\usepackage[font=footnotesize,labelfont=sl]{caption} % Figure caption
\usepackage{bm}
\usepackage[dvipsnames, table]{xcolor}
\definecolor{shadecolor}{rgb}{0.105469, 0.613281, 1}
\colorlet{shadecolor}{Emerald!15} 
\usepackage{icomma}
\makeatother
\usepackage[many]{tcolorbox}
\usepackage{multicol}
\usepackage{stackengine}

\usepackage{esvect} %For vectors with capital letters

% For tabular
\usepackage{array}
\usepackage{multirow}
\usepackage{longtable} %breakable table

% Ligningsreferanser
\usepackage{mathtools}
\mathtoolsset{showonlyrefs}

% index
\usepackage{imakeidx}
\makeindex[title=Indeks]

%Footnote:
\usepackage[bottom, hang, flushmargin]{footmisc}
\usepackage{perpage} 
\MakePerPage{footnote}
\addtolength{\footnotesep}{2mm}
\renewcommand{\thefootnote}{\arabic{footnote}}
\renewcommand\footnoterule{\rule{\linewidth}{0.4pt}}
\renewcommand{\thempfootnote}{\arabic{mpfootnote}}

%colors
\definecolor{c1}{cmyk}{0,0.5,1,0}
\definecolor{c2}{cmyk}{1,0.25,1,0}
\definecolor{n3}{cmyk}{1,0.,1,0}
\definecolor{neg}{cmyk}{1,0.,0.,0}

% Lister med bokstavar
\usepackage[inline]{enumitem}

\newcounter{rg}
\numberwithin{rg}{chapter}
\newcommand{\reg}[2][]{\begin{tcolorbox}[boxrule=0.3 mm,arc=0mm,colback=blue!3] {\refstepcounter{rg}\phantomsection \large \textbf{\therg \;#1} \vspace{5 pt}}\newline #2  \end{tcolorbox}\vspace{-5pt}}

\newcommand\alg[1]{\begin{align} #1 \end{align}}

\newcommand\eks[2][]{\begin{tcolorbox}[boxrule=0.3 mm,arc=0mm,enhanced jigsaw,breakable,colback=green!3] {\large \textbf{Eksempel #1} \vspace{5 pt}\\} #2 \end{tcolorbox}\vspace{-5pt} }

\newcommand{\st}[1]{\begin{tcolorbox}[boxrule=0.0 mm,arc=0mm,enhanced jigsaw,breakable,colback=yellow!12]{ #1} \end{tcolorbox}}

\newcommand{\spr}[1]{\begin{tcolorbox}[boxrule=0.3 mm,arc=0mm,enhanced jigsaw,breakable,colback=yellow!7] {\large \textbf{Språkboksen} \vspace{5 pt}\\} #1 \end{tcolorbox}\vspace{-5pt} }

\newcommand{\sym}[1]{\colorbox{blue!15}{#1}}

\newcommand{\info}[2]{\begin{tcolorbox}[boxrule=0.3 mm,arc=0mm,enhanced jigsaw,breakable,colback=cyan!6] {\large \textbf{#1} \vspace{5 pt}\\} #2 \end{tcolorbox}\vspace{-5pt} }

\newcommand\algv[1]{\vspace{-11 pt}\begin{align*} #1 \end{align*}}

\newcommand{\regv}{\vspace{5pt}}
\newcommand{\mer}{\textsl{Merk}: }
\newcommand{\mers}[1]{{\footnotesize \mer #1}}
\newcommand\vsk{\vspace{11pt}}
\newcommand\vs{\vspace{-11pt}}
\newcommand\vsb{\vspace{-16pt}}
\newcommand\sv{\vsk \textbf{Svar} \vspace{4 pt}\\}
\newcommand\br{\\[5 pt]}
\newcommand{\figp}[1]{../fig/#1}
\newcommand\algvv[1]{\vs\vs\begin{align*} #1 \end{align*}}
\newcommand{\y}[1]{$ {#1} $}
\newcommand{\os}{\\[5 pt]}
\newcommand{\prbxl}[2]{
\parbox[l][][l]{#1\linewidth}{#2
	}}
\newcommand{\prbxr}[2]{\parbox[r][][l]{#1\linewidth}{
		\setlength{\abovedisplayskip}{5pt}
		\setlength{\belowdisplayskip}{5pt}	
		\setlength{\abovedisplayshortskip}{0pt}
		\setlength{\belowdisplayshortskip}{0pt} 
		\begin{shaded}
			\footnotesize	#2 \end{shaded}}}

\renewcommand{\cfttoctitlefont}{\Large\bfseries}
\setlength{\cftaftertoctitleskip}{0 pt}
\setlength{\cftbeforetoctitleskip}{0 pt}

\newcommand{\bs}{\\[3pt]}
\newcommand{\vn}{\\[6pt]}
\newcommand{\fig}[1]{\begin{figure}
		\centering
		\includegraphics[]{\figp{#1}}
\end{figure}}

\newcommand{\figc}[2]{\begin{figure}
		\centering
		\includegraphics[]{\figp{#1}}
		\caption{#2}
\end{figure}}

\newcommand{\sectionbreak}{\clearpage} % New page on each section

\newcommand{\nn}[1]{
\begin{equation}
	#1
\end{equation}
}

% Equation comments
\newcommand{\cm}[1]{\llap{\color{blue} #1}}

\newcommand\fork[2]{\begin{tcolorbox}[boxrule=0.3 mm,arc=0mm,enhanced jigsaw,breakable,colback=yellow!7] {\large \textbf{#1 (forklaring)} \vspace{5 pt}\\} #2 \end{tcolorbox}\vspace{-5pt} }
 
%colors
\newcommand{\colr}[1]{{\color{red} #1}}
\newcommand{\colb}[1]{{\color{blue} #1}}
\newcommand{\colo}[1]{{\color{orange} #1}}
\newcommand{\colc}[1]{{\color{cyan} #1}}
\definecolor{projectgreen}{cmyk}{100,0,100,0}
\newcommand{\colg}[1]{{\color{projectgreen} #1}}

% Methods
\newcommand{\metode}[2]{
	\textsl{#1} \\[-8pt]
	\rule{#2}{0.75pt}
}

%Opg
\newcommand{\abc}[1]{
	\begin{enumerate}[label=\alph*),leftmargin=18pt]
		#1
	\end{enumerate}
}
\newcommand{\abcs}[2]{
	\begin{enumerate}[label=\alph*),start=#1,leftmargin=18pt]
		#2
	\end{enumerate}
}
\newcommand{\abcn}[1]{
	\begin{enumerate}[label=\arabic*),leftmargin=18pt]
		#1
	\end{enumerate}
}
\newcommand{\abch}[1]{
	\hspace{-2pt}	\begin{enumerate*}[label=\alph*), itemjoin=\hspace{1cm}]
		#1
	\end{enumerate*}
}
\newcommand{\abchs}[2]{
	\hspace{-2pt}	\begin{enumerate*}[label=\alph*), itemjoin=\hspace{1cm}, start=#1]
		#2
	\end{enumerate*}
}

% Oppgaver
\newcommand{\opgt}{\phantomsection \addcontentsline{toc}{section}{Oppgaver} \section*{Oppgaver for kapittel \thechapter}\vs \setcounter{section}{1}}
\newcounter{opg}
\numberwithin{opg}{section}
\newcommand{\op}[1]{\vspace{15pt} \refstepcounter{opg}\large \textbf{\color{blue}\theopg} \vspace{2 pt} \label{#1} \\}
\newcommand{\ekspop}[1]{\vsk\textbf{Gruble \thechapter.#1}\vspace{2 pt} \\}
\newcommand{\nes}{\stepcounter{section}
	\setcounter{opg}{0}}
\newcommand{\opr}[1]{\vspace{3pt}\textbf{\ref{#1}}}
\newcommand{\oeks}[1]{\begin{tcolorbox}[boxrule=0.3 mm,arc=0mm,colback=white]
		\textit{Eksempel: } #1	  
\end{tcolorbox}}
\newcommand\opgeks[2][]{\begin{tcolorbox}[boxrule=0.1 mm,arc=0mm,enhanced jigsaw,breakable,colback=white] {\footnotesize \textbf{Eksempel #1} \\} \footnotesize #2 \end{tcolorbox}\vspace{-5pt} }
\newcommand{\rknut}{
Rekn ut.
}

%License
\newcommand{\lic}{\textit{Matematikken sine byggesteinar by Sindre Sogge Heggen is licensed under CC BY-NC-SA 4.0. To view a copy of this license, visit\\ 
		\net{http://creativecommons.org/licenses/by-nc-sa/4.0/}{http://creativecommons.org/licenses/by-nc-sa/4.0/}}}

%referances
\newcommand{\net}[2]{{\color{blue}\href{#1}{#2}}}
\newcommand{\hrs}[2]{\hyperref[#1]{\color{blue}\textsl{#2 \ref*{#1}}}}
\newcommand{\rref}[1]{\hrs{#1}{regel}}
\newcommand{\refkap}[1]{\hrs{#1}{kapittel}}
\newcommand{\refsec}[1]{\hrs{#1}{seksjon}}

\newcommand{\mb}{\net{https://sindrsh.github.io/FirstPrinciplesOfMath/}{MB}}


%line to seperate examples
\newcommand{\linje}{\rule{\linewidth}{1pt} }

\usepackage{datetime2}
%%\usepackage{sansmathfonts} for dyslexia-friendly math
\usepackage[]{hyperref}


\newcommand{\note}{Merk}
\newcommand{\notesm}[1]{{\footnotesize \textsl{\note:} #1}}
\newcommand{\ekstitle}{Eksempel }
\newcommand{\sprtitle}{Språkboksen}
\newcommand{\expl}{forklaring}

\newcommand{\vedlegg}[1]{\refstepcounter{vedl}\section*{Vedlegg \thevedl: #1}  \setcounter{vedleq}{0}}

\newcommand\sv{\vsk \textbf{Svar} \vspace{4 pt}\\}

%references
\newcommand{\reftab}[1]{\hrs{#1}{tabell}}
\newcommand{\rref}[1]{\hrs{#1}{regel}}
\newcommand{\dref}[1]{\hrs{#1}{definisjon}}
\newcommand{\refkap}[1]{\hrs{#1}{kapittel}}
\newcommand{\refsec}[1]{\hrs{#1}{seksjon}}
\newcommand{\refdsec}[1]{\hrs{#1}{delseksjon}}
\newcommand{\refved}[1]{\hrs{#1}{vedlegg}}
\newcommand{\eksref}[1]{\textsl{#1}}
\newcommand\fref[2][]{\hyperref[#2]{\textsl{figur \ref*{#2}#1}}}
\newcommand{\refop}[1]{{\color{blue}Oppgave \ref{#1}}}
\newcommand{\refops}[1]{{\color{blue}oppgave \ref{#1}}}
\newcommand{\refgrubs}[1]{{\color{blue}gruble \ref{#1}}}

\newcommand{\openmathser}{\openmath\,-\,serien}

% Exercises
\newcommand{\opgt}{\newpage \phantomsection \addcontentsline{toc}{section}{Oppgaver} \section*{Oppgaver for kapittel \thechapter}\vs \setcounter{section}{1}}


% Sequences and series
\newcommand{\sumarrek}{Summen av en aritmetisk rekke}
\newcommand{\sumgerek}{Summen av en geometrisk rekke}
\newcommand{\regnregsum}{Regneregler for summetegnet}

% Trigonometry
\newcommand{\sincoskomb}{Sinus og cosinus kombinert}
\newcommand{\cosfunk}{Cosinusfunksjonen}
\newcommand{\trid}{Trigonometriske identiteter}
\newcommand{\deravtri}{Den deriverte av de trigonometriske funksjonene}
% Solutions manual
\newcommand{\selos}{Se løsningsforslag.}
\newcommand{\se}[1]{Se eksempel på side \pageref{#1}}

%Vectors
\newcommand{\parvek}{Parallelle vektorer}
\newcommand{\vekpro}{Vektorproduktet}
\newcommand{\vekproarvol}{Vektorproduktet som areal og volum}


% 3D geometries
\newcommand{\linrom}{Linje i rommet}
\newcommand{\avstplnpkt}{Avstand mellom punkt og plan}


% Integral
\newcommand{\bestminten}{Bestemt integral I}
\newcommand{\anfundteo}{Analysens fundamentalteorem}
\newcommand{\intuf}{Integralet av utvalge funksjoner}
\newcommand{\bytvar}{Bytte av variabel}
\newcommand{\intvol}{Integral som volum}
\newcommand{\andordlindif}{Andre ordens lineære differensialligninger}




\begin{document}
\section{Python}
\mers{Denne teksten bygger på teksten om Python i \am}.
\subsection{NumPy}
I Python kan man importere det som kalles \outl{bibliotek} for å få tilgang til enda flere typer objekter, funksjoner og liknende. NumPy er et bibliotek som inneholder typen \pymet{numpy.ndarray}. Denne typen har mange fellestrekk med en liste, men skiller seg ut ved at den inneholder et bestemt antall elemener. Dette gjør blant annet at prossesser som bruker NumPy-arrays istedenfor lister går raskere, og at NumPy-arrays egner seg bedre til regneoperasjoner.\vsk

For å lage NumPy-arrays må vi importere NumPy-biblioteket: \regv
\pythonut{array1.py}{
	[1 2] \newline
	[0 1 2 3]\newline
	[0. 0. 0.] \newline
	[ 2.  5.  8. 11.]
	
}
\info{Merk}{
	Til forskjell fra lister, er elementene skilt bare med mellomrom når de printes.
}

\newpage
\subsubsection{Klassisek regnearter}
Regneoperasjoner mellom NumPy-arrays blir utført elementvis:
\pythonut{arrayopr.py}{
	[12 24] \newline
	[20 80]
}
\subsubsection{Vektoroperasjoner}
NumPy-arrays fungerer ypperlig til å representere vektorer, og har innebygde metoder for å finne skalarprodukt, kryssprodukt og determinanter:\regv

\pythonut{skalkryss.py}{
	-33 \newline
	[-5  1 17] \newline
	17.0
}	
	
\section{Newtons metode}
Gitt en funskjon $ f(x) $ si at vi ønsker å finne et tall $ a $ slik at $ {f(a)=0} $. Ved \outl{Newtons metode} gjør vi denne antakelsen for å en tilnærming $ a $: \regv
\st{
La $ x_1 $ være skjæringspunktet mellom $ x $-aksen og tangenten til $ f $ i $ x_0 $. Vi antar da at $ |x_1-a|<|x_0-a| $. Sagt med ord antar vi at $ x_1 $ gir en bedre tilnærming for $ a $ enn det $ x_0 $ gjør.
}
Siden $ x_1 $ er skjæringspunktet mellom $ x $-aksen og tangenten til $ f $ i $ x_0 $, har vi  at\footnote{Se oppgave??}
\alg{
f'(x_0)(x_1-x_0)+f(x_0)&=0 \\
f'(x_0)x_1 &= f'(x_0)x_0-f(x_0) \\
x_1 &= x_0-\frac{f(x_0)}{f'(x_0)}
}
\begin{figure}
\subfloat[]{
\includegraphics{\figp{newt1a}}
}
\subfloat[]{
	\includegraphics{\figp{newt1b}}
}
\end{figure}
La $ x_2 $ være skjæringspunktet mellom $ x $-aksen og tangenten til $ f $ i $ x_1 $. Ved å gjenta prosedyren vi brukte for å finne $ x_1 $, kan vi finne $ x_2 $, som vi antar er en enda bedre tilnærming for $ a $ enn $ x_1 $.
Prosedyren kan vi gjenta fram til vi har funnet en $ x $-verdi som gir en tilstrekkelig\footnote{Hva som er en \textit{tilstrekkelig tilnærming} er det opp til oss selv å bestemme.} tilnærming til $ a$.
\newpage
\reg[Newtons metode]{
Gitt en funskjon $ f(x) $ si at vi ønsker å finne et tall $ a $ slik at $ {f(a)=0} $. Gitt $ x $-verdiene $ x_{n} $ og $ x_{n+1} $ for $ {n\in\mathbb{N}} $. Ved å bruke formelen
\nn{
x_{n+1} = x_n-\frac{f(x_n)}{f'(x_n)} 
}
antas det at $ x_{n+1} $ gir en bedre tilnærming for $ a $ enn $ x_{n} $.
}
\spr{
\outl{Newtons metode} kalles også \outl{Newton-Rhapsos metode}.
}
\info{Når er tilmærmingen god nok?}{
Newtons metode beskriver en iterasjonsprossess som man håper at nærmer seg en verdi. Hvis meotden lykkes, vil $ x_{n+1}$ og $ x_n $ etterhvert være veldig like, og slik kan en grense for hvor liten $ {|x_{n+1}-x_n|} $ kan være fungere som et godt mål for når iterasjonsprossessen skal stoppe.
}
\section{Trapesmetoden} \label{Trapesmetoden}
Gitt en funksjone $ f(x) $. Integralet $ \int_a^b f \,dx $ kan vi tilnærme ved å 
\st{
\begin{enumerate}
	\item Dele intervallet $ [a, b] $ inn i mindre intervall. Disse kaller vi \outl{delintervall}.
	\item Finne en tilnærmet verdi for integralet av $ f $ på hvert\\ delintervall.
	\item Summere verdiene fra punkt 2.
\end{enumerate}
}\regv

I \fref[a]{trapmetfig} har vi 3 like store delintervaller. Hvis vi setter $ {a=x_0} $ og $ {\Delta x=\frac{b-a}{3}} $, betyr dette at 
\alg{
x_1&=x_0+\Delta x & x_2&=x_0+2\Delta x & x_3&=x_3+3\Delta x=b
}
En tilnæret verdi for $ \int_{a}^{x_1}f\,dx $ får vi ved å finne arealet til trapeset med hjørner (husk at $ x_0=a $)
\alg{
(x_0, 0) && (x_1, 0) && (x_1, f(x_1)) && (x_0, f(a))
}
Dette arealet er gitt ved uttrykket
\[ \frac{1}{2}(x_1-x_0)\left[f(x_0)+f(x_1)\right]=
\frac{\Delta x}{2}\left[f(x_0)-f(x_1)\right] \]
Ved å tilnærme integralet for hvert delintervall på denne måten, kan vi skrive
\[ \int_{a}^{b} f\,dx \approx \frac{\Delta x}{2}\sum_{i=0}^{2} \left[f(x_i)+f(x_{i+1})\right] \]
\begin{figure}
	\centering
	\subfloat[Tilnærming med 3\\ \hspace{0.55cm}delintervaller.]{\includegraphics{\figp{trapmet}}}\qquad
	\subfloat[Tilnærming med 20\\ \hspace{0.55cm}delintervaller]{\includegraphics{\figp{trapmetb}}}
	\caption{\label{trapmetfig}}
\end{figure}

\reg[Trapesmetoden \label{trapmet}]{
Gitt en integrerbar funksjon $ f $. En tilnærmet verdi for $ \int_{a}^{b} f\,dx $ er da gitt som
\begin{equation}\label{trapmeteq}
	\int_{a}^{b} f\,dx \approx \frac{\Delta x}{2}\sum_{i=0}^{n}\left[f(x_i)+f(x_{i+1})\right]
\end{equation}
hvor
\algv{
n&\in \mathbb{\hat{N}}\vn
a&=x_0 \vn
b&=x_n \vn
\Delta x &= \frac{b-a}{n+1}\vn
x_{n+1}&=x_n+i\Delta x
}

}
\info{\note}{
Slik \rref{trapmet} er formulert, vil $ {[a, b]} $ være delt inn i $ {n+1} $ delintervaller.
}

\end{document}