\documentclass[english, 11 pt, class=article, crop=false]{standalone}

\newcommand{\note}{Merk}
\newcommand{\notesm}[1]{{\footnotesize \textsl{\note:} #1}}
\newcommand{\ekstitle}{Eksempel }
\newcommand{\sprtitle}{Språkboksen}
\newcommand{\expl}{forklaring}

\newcommand{\vedlegg}[1]{\refstepcounter{vedl}\section*{Vedlegg \thevedl: #1}  \setcounter{vedleq}{0}}

\newcommand\sv{\vsk \textbf{Svar} \vspace{4 pt}\\}

%references
\newcommand{\reftab}[1]{\hrs{#1}{tabell}}
\newcommand{\rref}[1]{\hrs{#1}{regel}}
\newcommand{\dref}[1]{\hrs{#1}{definisjon}}
\newcommand{\refkap}[1]{\hrs{#1}{kapittel}}
\newcommand{\refsec}[1]{\hrs{#1}{seksjon}}
\newcommand{\refdsec}[1]{\hrs{#1}{delseksjon}}
\newcommand{\refved}[1]{\hrs{#1}{vedlegg}}
\newcommand{\eksref}[1]{\textsl{#1}}
\newcommand\fref[2][]{\hyperref[#2]{\textsl{figur \ref*{#2}#1}}}
\newcommand{\refop}[1]{{\color{blue}Oppgave \ref{#1}}}
\newcommand{\refops}[1]{{\color{blue}oppgave \ref{#1}}}
\newcommand{\refgrubs}[1]{{\color{blue}gruble \ref{#1}}}

\newcommand{\openmathser}{\openmath\,-\,serien}

% Exercises
\newcommand{\opgt}{\newpage \phantomsection \addcontentsline{toc}{section}{Oppgaver} \section*{Oppgaver for kapittel \thechapter}\vs \setcounter{section}{1}}


% Sequences and series
\newcommand{\sumarrek}{Summen av en aritmetisk rekke}
\newcommand{\sumgerek}{Summen av en geometrisk rekke}
\newcommand{\regnregsum}{Regneregler for summetegnet}

% Trigonometry
\newcommand{\sincoskomb}{Sinus og cosinus kombinert}
\newcommand{\cosfunk}{Cosinusfunksjonen}
\newcommand{\trid}{Trigonometriske identiteter}
\newcommand{\deravtri}{Den deriverte av de trigonometriske funksjonene}
% Solutions manual
\newcommand{\selos}{Se løsningsforslag.}
\newcommand{\se}[1]{Se eksempel på side \pageref{#1}}

%Vectors
\newcommand{\parvek}{Parallelle vektorer}
\newcommand{\vekpro}{Vektorproduktet}
\newcommand{\vekproarvol}{Vektorproduktet som areal og volum}


% 3D geometries
\newcommand{\linrom}{Linje i rommet}
\newcommand{\avstplnpkt}{Avstand mellom punkt og plan}


% Integral
\newcommand{\bestminten}{Bestemt integral I}
\newcommand{\anfundteo}{Analysens fundamentalteorem}
\newcommand{\intuf}{Integralet av utvalge funksjoner}
\newcommand{\bytvar}{Bytte av variabel}
\newcommand{\intvol}{Integral som volum}
\newcommand{\andordlindif}{Andre ordens lineære differensialligninger}


\usepackage[T1]{fontenc}
%\renewcommand*\familydefault{\sfdefault} % For dyslexia-friendly text
\usepackage{lmodern} % load a font with all the characters
\usepackage{geometry}
\geometry{verbose,paperwidth=16.1 cm, paperheight=24 cm, inner=2.3cm, outer=1.8 cm, bmargin=2cm, tmargin=1.8cm}
\setlength{\parindent}{0bp}
\usepackage{import}
\usepackage[subpreambles=false]{standalone}
\usepackage{amsmath}
\usepackage{amssymb}
\usepackage{esint}
\usepackage{babel}
\usepackage{tabu}
\makeatother
\makeatletter

\usepackage{titlesec}
\usepackage{ragged2e}
\RaggedRight
\raggedbottom
\frenchspacing

% Norwegian names of figures, chapters, parts and content
\addto\captionsenglish{\renewcommand{\figurename}{Figur}}
\makeatletter
\addto\captionsenglish{\renewcommand{\chaptername}{Kapittel}}
\addto\captionsenglish{\renewcommand{\partname}{Del}}


\usepackage{graphicx}
\usepackage{float}
\usepackage{subfig}
\usepackage{placeins}
\usepackage{cancel}
\usepackage{framed}
\usepackage{wrapfig}
\usepackage[subfigure]{tocloft}
\usepackage[font=footnotesize,labelfont=sl]{caption} % Figure caption
\usepackage{bm}
\usepackage[dvipsnames, table]{xcolor}
\definecolor{shadecolor}{rgb}{0.105469, 0.613281, 1}
\colorlet{shadecolor}{Emerald!15} 
\usepackage{icomma}
\makeatother
\usepackage[many]{tcolorbox}
\usepackage{multicol}
\usepackage{stackengine}

\usepackage{esvect} %For vectors with capital letters

% For tabular
\usepackage{array}
\usepackage{multirow}
\usepackage{longtable} %breakable table

% Ligningsreferanser
\usepackage{mathtools}
\mathtoolsset{showonlyrefs}

% index
\usepackage{imakeidx}
\makeindex[title=Indeks]

%Footnote:
\usepackage[bottom, hang, flushmargin]{footmisc}
\usepackage{perpage} 
\MakePerPage{footnote}
\addtolength{\footnotesep}{2mm}
\renewcommand{\thefootnote}{\arabic{footnote}}
\renewcommand\footnoterule{\rule{\linewidth}{0.4pt}}
\renewcommand{\thempfootnote}{\arabic{mpfootnote}}

%colors
\definecolor{c1}{cmyk}{0,0.5,1,0}
\definecolor{c2}{cmyk}{1,0.25,1,0}
\definecolor{n3}{cmyk}{1,0.,1,0}
\definecolor{neg}{cmyk}{1,0.,0.,0}

% Lister med bokstavar
\usepackage[inline]{enumitem}

\newcounter{rg}
\numberwithin{rg}{chapter}
\newcommand{\reg}[2][]{\begin{tcolorbox}[boxrule=0.3 mm,arc=0mm,colback=blue!3] {\refstepcounter{rg}\phantomsection \large \textbf{\therg \;#1} \vspace{5 pt}}\newline #2  \end{tcolorbox}\vspace{-5pt}}

\newcommand\alg[1]{\begin{align} #1 \end{align}}

\newcommand\eks[2][]{\begin{tcolorbox}[boxrule=0.3 mm,arc=0mm,enhanced jigsaw,breakable,colback=green!3] {\large \textbf{Eksempel #1} \vspace{5 pt}\\} #2 \end{tcolorbox}\vspace{-5pt} }

\newcommand{\st}[1]{\begin{tcolorbox}[boxrule=0.0 mm,arc=0mm,enhanced jigsaw,breakable,colback=yellow!12]{ #1} \end{tcolorbox}}

\newcommand{\spr}[1]{\begin{tcolorbox}[boxrule=0.3 mm,arc=0mm,enhanced jigsaw,breakable,colback=yellow!7] {\large \textbf{Språkboksen} \vspace{5 pt}\\} #1 \end{tcolorbox}\vspace{-5pt} }

\newcommand{\sym}[1]{\colorbox{blue!15}{#1}}

\newcommand{\info}[2]{\begin{tcolorbox}[boxrule=0.3 mm,arc=0mm,enhanced jigsaw,breakable,colback=cyan!6] {\large \textbf{#1} \vspace{5 pt}\\} #2 \end{tcolorbox}\vspace{-5pt} }

\newcommand\algv[1]{\vspace{-11 pt}\begin{align*} #1 \end{align*}}

\newcommand{\regv}{\vspace{5pt}}
\newcommand{\mer}{\textsl{Merk}: }
\newcommand{\mers}[1]{{\footnotesize \mer #1}}
\newcommand\vsk{\vspace{11pt}}
\newcommand\vs{\vspace{-11pt}}
\newcommand\vsb{\vspace{-16pt}}
\newcommand\sv{\vsk \textbf{Svar} \vspace{4 pt}\\}
\newcommand\br{\\[5 pt]}
\newcommand{\figp}[1]{../fig/#1}
\newcommand\algvv[1]{\vs\vs\begin{align*} #1 \end{align*}}
\newcommand{\y}[1]{$ {#1} $}
\newcommand{\os}{\\[5 pt]}
\newcommand{\prbxl}[2]{
\parbox[l][][l]{#1\linewidth}{#2
	}}
\newcommand{\prbxr}[2]{\parbox[r][][l]{#1\linewidth}{
		\setlength{\abovedisplayskip}{5pt}
		\setlength{\belowdisplayskip}{5pt}	
		\setlength{\abovedisplayshortskip}{0pt}
		\setlength{\belowdisplayshortskip}{0pt} 
		\begin{shaded}
			\footnotesize	#2 \end{shaded}}}

\renewcommand{\cfttoctitlefont}{\Large\bfseries}
\setlength{\cftaftertoctitleskip}{0 pt}
\setlength{\cftbeforetoctitleskip}{0 pt}

\newcommand{\bs}{\\[3pt]}
\newcommand{\vn}{\\[6pt]}
\newcommand{\fig}[1]{\begin{figure}
		\centering
		\includegraphics[]{\figp{#1}}
\end{figure}}

\newcommand{\figc}[2]{\begin{figure}
		\centering
		\includegraphics[]{\figp{#1}}
		\caption{#2}
\end{figure}}

\newcommand{\sectionbreak}{\clearpage} % New page on each section

\newcommand{\nn}[1]{
\begin{equation}
	#1
\end{equation}
}

% Equation comments
\newcommand{\cm}[1]{\llap{\color{blue} #1}}

\newcommand\fork[2]{\begin{tcolorbox}[boxrule=0.3 mm,arc=0mm,enhanced jigsaw,breakable,colback=yellow!7] {\large \textbf{#1 (forklaring)} \vspace{5 pt}\\} #2 \end{tcolorbox}\vspace{-5pt} }
 
%colors
\newcommand{\colr}[1]{{\color{red} #1}}
\newcommand{\colb}[1]{{\color{blue} #1}}
\newcommand{\colo}[1]{{\color{orange} #1}}
\newcommand{\colc}[1]{{\color{cyan} #1}}
\definecolor{projectgreen}{cmyk}{100,0,100,0}
\newcommand{\colg}[1]{{\color{projectgreen} #1}}

% Methods
\newcommand{\metode}[2]{
	\textsl{#1} \\[-8pt]
	\rule{#2}{0.75pt}
}

%Opg
\newcommand{\abc}[1]{
	\begin{enumerate}[label=\alph*),leftmargin=18pt]
		#1
	\end{enumerate}
}
\newcommand{\abcs}[2]{
	\begin{enumerate}[label=\alph*),start=#1,leftmargin=18pt]
		#2
	\end{enumerate}
}
\newcommand{\abcn}[1]{
	\begin{enumerate}[label=\arabic*),leftmargin=18pt]
		#1
	\end{enumerate}
}
\newcommand{\abch}[1]{
	\hspace{-2pt}	\begin{enumerate*}[label=\alph*), itemjoin=\hspace{1cm}]
		#1
	\end{enumerate*}
}
\newcommand{\abchs}[2]{
	\hspace{-2pt}	\begin{enumerate*}[label=\alph*), itemjoin=\hspace{1cm}, start=#1]
		#2
	\end{enumerate*}
}

% Oppgaver
\newcommand{\opgt}{\phantomsection \addcontentsline{toc}{section}{Oppgaver} \section*{Oppgaver for kapittel \thechapter}\vs \setcounter{section}{1}}
\newcounter{opg}
\numberwithin{opg}{section}
\newcommand{\op}[1]{\vspace{15pt} \refstepcounter{opg}\large \textbf{\color{blue}\theopg} \vspace{2 pt} \label{#1} \\}
\newcommand{\ekspop}[1]{\vsk\textbf{Gruble \thechapter.#1}\vspace{2 pt} \\}
\newcommand{\nes}{\stepcounter{section}
	\setcounter{opg}{0}}
\newcommand{\opr}[1]{\vspace{3pt}\textbf{\ref{#1}}}
\newcommand{\oeks}[1]{\begin{tcolorbox}[boxrule=0.3 mm,arc=0mm,colback=white]
		\textit{Eksempel: } #1	  
\end{tcolorbox}}
\newcommand\opgeks[2][]{\begin{tcolorbox}[boxrule=0.1 mm,arc=0mm,enhanced jigsaw,breakable,colback=white] {\footnotesize \textbf{Eksempel #1} \\} \footnotesize #2 \end{tcolorbox}\vspace{-5pt} }
\newcommand{\rknut}{
Rekn ut.
}

%License
\newcommand{\lic}{\textit{Matematikken sine byggesteinar by Sindre Sogge Heggen is licensed under CC BY-NC-SA 4.0. To view a copy of this license, visit\\ 
		\net{http://creativecommons.org/licenses/by-nc-sa/4.0/}{http://creativecommons.org/licenses/by-nc-sa/4.0/}}}

%referances
\newcommand{\net}[2]{{\color{blue}\href{#1}{#2}}}
\newcommand{\hrs}[2]{\hyperref[#1]{\color{blue}\textsl{#2 \ref*{#1}}}}
\newcommand{\rref}[1]{\hrs{#1}{regel}}
\newcommand{\refkap}[1]{\hrs{#1}{kapittel}}
\newcommand{\refsec}[1]{\hrs{#1}{seksjon}}

\newcommand{\mb}{\net{https://sindrsh.github.io/FirstPrinciplesOfMath/}{MB}}


%line to seperate examples
\newcommand{\linje}{\rule{\linewidth}{1pt} }

\usepackage{datetime2}
%%\usepackage{sansmathfonts} for dyslexia-friendly math
\usepackage[]{hyperref}


\begin{document}

\section{Størrelser, enheter og prefikser}
Det vi kan måle og uttrykke med tall, kaller vi \textit{størrelser}. En størrelse består gjerne av både en verdi og en \textit{enhet}, og i denne seksjonen skal vi se på disse fire enhetene:
\tbs
\begin{center}
	\begin{tabular}{c|c|c}
		\textbf{enhet} & \textbf{forkortelse} &\textbf{enhet for}\\ \hline
		meter & m &lengde\\\hline
		gram & g &masse\\\hline
		sekund & s & tid\\\hline 
		liter & L & volum
	\end{tabular}
\end{center}\tbs
Noen ganger har vi veldig store eller veldig små størrelser, for eksempel er det ca 40\,075\,000\,m rundt ekvator! For så store tall er det vanlig å bruke en \textit{prefiks}. Da kan vi skrive at det er ca 40\,075 km rundt ekvator. Her står 'km' for 'kilometer', og 'kilo' betyr '1\,000'. Så 1\,000\enh{meter}  er altså 1\enh{kilometer} . Her er prefiksene man oftest\footnote{Unntaket er 'deka', som er en veldig lite brukt prefiks, men vi har tatt den med fordi den kompletterer tallmønsteret.} møter på i hverdagen:\tbs
\begin{center}
	\begin{tabular}{c|c|r}
		\textbf{prefiks} & \textbf{forkortelse}&\textbf{verdi} \\ \hline
		kilo & k & 1\,000\phantom{000\;}\\\hline
		hekto & h & 100\phantom{000\;}\\\hline
		deka & da & 10\phantom{000\;}\\\hline
		desi & d & 0,1\phantom{0\,\;}\\\hline
		centi & c & 0,01\phantom{\,\;}\\\hline
		milli & m & 0,001\\\hline		
	\end{tabular}
\end{center}
Bruker vi denne tabellen i kombinasjon med enhetene kan vi for eksempel se at\vs
\alg{
	1000\enh{g}&= 1\enh{kg} \\
	0,1 \enh{m} &= 1\enh{dm} \\
	1\enh{s} &= 1000 \enh{ms} \\
	0,01 \enh{L} &= 1\enh{cL} 
}
Enda ryddigere kan vi få det hvis vi lager en vannrett tabell (se neste side) med meter, gram eller liter lagt til i midten\footnote{Legg merke til at 'meter', 'gram', 'sekund' og 'liter' er \textsl{enheter}, mens 'kilo', 'hekto' osv. er \textsl{tall}. Det kan derfor virke litt rart å sette dem opp i samme tabell, men for vårt formål fungerer det helt fint.}. 
\newpage
\reg[\ompref \label{ompref}]{Når vi skal endre prefikser kan vi bruke denne tabellen:
	\begin{center}
		\begin{tabular}{|c|c|c|c|c|c|c|c}
			kilo &
			hekto &
			deka & m/g/L &
			desi & 
			centi & 
			milli & 		
		\end{tabular}
	\end{center}
	Komma må flyttes like mange ganger som antall ruter vi må flytte oss fra opprinnelig prefiks til ny prefiks.\vsk
	
	{\footnotesize For lengde brukes også enheten 'mil' (1 mil $ = $ 10\,000\,m). Denne kan legges på til venstre for 'kilo'.}
}\regv

\spr{
	En (eventuell) \outl{prefiks} og en \outl{enhet} utgjør en \outl{benevning}. For eksempel har 9\enh{km} benevningen 'km', mens 9\enh{m} har benevningen 'm'. Disse størrelsene har forskjellige benevning, men begge har 'm' som enhet.
} \regv

\eks[1]{
	Skriv om 23,4\,mL til antall L.
	
	\sv
	Vi skriver tabellen vår med L i midten, og legger merke til at vi må \textsl{tre ruter til venstre} for å komme oss fra mL til L:
	\begin{center}
		\begin{tabular}{|c|c|c|c|c|c|c|c}
			kilo &
			hekto &
			deka & \color{blue}L &
			desi & 
			centi & 
			\color{red} milli & 		
		\end{tabular}
	\end{center}
	Det betyr at vi må flytte kommaet vårt tre plasser til venstre for å gjøre om mL til L:
	\[ 23,4\enh{mL}=0,0234\enh{L} \]
}
\newpage
\eks[2]{
	Skriv om 30\,hg til antall cg.
	
	\sv
	Vi skriver tabellen vår med 'g' i midten, og legger merke til at vi må \textbf{fire ruter til høyre} for å komme oss fra \colr{hg} til \colb{cg}:
	\begin{center}
		\begin{tabular}{|c|c|c|c|c|c|c|c}
			kilo &
			\color{red}hekto &
			deka & g &
			desi & 
			\color{blue}centi & 
			milli & 		
		\end{tabular}
	\end{center}
	Dét betyr at vi må flytte kommaet vårt fire plasser til høyre for å gjøre om 'hg' til 'cg':\vs
	\[ 30\enh{mg}=300\,000\enh{cg} \]
}\regv

\eks[3]{
	Skriv om 2,7\enh{s} til antall \enh{ms}.
	
	\sv
	Vi skriver tabellen vår med 's' i midten, og legger merke til at vi må \textbf{tre ruter til høgre} for å komme oss fra \colr{s} til \colb{ms}:
	\begin{center}
		\begin{tabular}{|c|c|c|c|c|c|c|c}
			kilo &
			hekto &
			deka &\color{red} s &
			desi & 
			centi & 
			\color{blue}milli & 		
		\end{tabular}
	\end{center}
	Dét betyr at vi må flytte kommaet vårt tre plasser til høgre for å gjøre om 's' til 'ms': \vs
	\[ 2,7\enh{s}=2\,700\enh{ms} \]
}\regv

\eks[4]{
	Gjør om 12\,500\,dm til antall mil.
	
	\sv
	Vi skriver tabellen vår med m i midten, legger til 'mil', og merker oss at vi må \textbf{fem ruter til venstre} for å komme oss fra \colb{dm} til \colr{mil}:
	\begin{center}
		\begin{tabular}{|c|c|c|c|c|c|c|c|c}
			\color{blue}mil &kilo &
			hekto &
			deka & m &
			\color{red} desi & 
			centi & 
			milli & 		
		\end{tabular}
	\end{center}
	Dét betyr at vi må flytte kommaet vårt fem plasser til høyre for å gjøre om mil til dm:
	\[ 12\,500\enh{dm}=0,125\enh{mil} \]
}\vsk

\fork{\ref{ompref} \ompref}{
Omgjøring av prefikser handler om å gange/dele med 10, 100 osv. (\mb) \vsk

La oss som første eksempel skrive om $ 3,452\enh{km} $ til antall meter. Vi har at
\algv{
3,452\enh{km}&= 3,452\cdot1000 \enh{m} \\
&=3\,452\enh{m}
}
La oss som andre eksempel skrive om 47\enh{mm} til antall meter. Vi har at
\algv{
47\enh{mm}&=47\cdot\frac{1}{1000} \enh{m} \\
&= (47:1000) \enh{m}\\
&=0,047\enh{m}
}
}
\section{Regning med størrelser}
\mers{I eksemplene til denne seksjonen bruker vi areal- og volumformler som du finner i \mb.}\os

I utregninger behandler vi benevninger på samme måte som vi behandler variabler i algebra\footnote{Se kapitlet om algebra i \mb.}. På samme måte som at $ a+a=2a $, er altså $ 1\enh{cm}+1\enh{cm}=2\enh{cm} $ og på samme måte som at $ 2a\cdot 3a=6a^2$, er $ 2\enh{cm}\cdot 3\enh{cm}=6\enh{cm}^2 $.\regv

\eks[1]{
\fig{armedstorl}
\abc{
\item Finn omkretsen til rektangelet.
\item Finn arealet til rektangelet.
} \vs
\sv \vs
\abc{
\item Omkretsen til rektangelet er
\[ 7\enh{cm}+2\enh{cm}+7\enh{cm}+2\enh{cm}=18\enh{cm} \]
\item Arealet til rektangelet er
\[ 7\enh{cm}\cdot 2\enh{cm}=14\enh{cm}^2 \]
}

} \regv
\eks[3]{
	En sylinder har radius $ 4\enh{m} $ og høgde $ 2\enh{m} $. Finn volumet til\\ sylinderen.
	
	\sv \vs \vs
	\alg{
		\text{grunnflaten til sylinderen}&=\pi\cdot \left(4\enh{cm}\right)^2 
		= 16 \pi \enh{cm}^2 \vn
		\text{volumet til sylinderen}&= 16\pi \enh{cm}^2 \cdot2\enh{cm} 
		= 32\pi \enh{cm}^3
	}
} 
\section{Proporsjonale størrelser \label{Propstorl}}
Si at det koster $ 10\enh{kr}$ for $ 0,5\enh{kg} $ poteter. Hvis det er slik at denne prisen gjelder også hvis vi ønsker å kjøpe $ 0,5\enh{kg} $ poteter mer, koster det $ 20\enh{kr} $ for $ 1\enh{kg} $ poteter. Hvis prisen gjelder også hvis vi ønsker å kjøpe $ 0,5\enh{kg} $ poteter mer enn dette, koster det $ 30\enh{kr} $ for $ 1,5\enh{kg} $ poteter. Antall kroner og antall kilogram poteter for hver av tilfellene kan vi sette opp i en tabell: \vs
\begin{center}
	\begin{tabular}{|l|c|c|c|}
		\hline
\textbf{kr} & 10  & 20 & 30 \\ \hline
\textbf{kg} & 0,5 & 1\,& 1,5 \\ \hline
	\end{tabular}
\end{center}
La oss videre dele prisen med vekten for hvert av tilfellene:
\alg{
\frac{10\enh{kr}}{0,5\enh{kg}}&=20\enh{kr/kg} & \frac{20\enh{kr}}{1\enh{kg}}&=20\enh{kr/kg} & \frac{30\enh{kr}}{1,5}&=20\enh{kr/kg}
}
Vi ser nå at forholdet mellom prisen og vekten er det samme for alle tilfellene. Da sier vi at prisen og vekten er\footnote{Se også \refved{Funkvedl} i \mb.} \outl{proporsjonale størrelser}. Av disse to størrelsene har vi også ''lagd'' en ny størrelse, med benevning\footnote{Vi kunne også skrevet $ \frac{\enh{kr}}{\enh{kg}} $, men i dette tilfellet er det vanligst å bruke \sym{/} som divisjonstegn.} 'kr/kg'. Dette uttaler vi ''kroner per kilogram'', og denne størrelsen blir gjerne kalt \outl{kiloprisen}. I dette tilfellet er kiloprisen 20\,kr/kg. Da kiloprisen er forholdet mellom to proporsjonale størrelser, kalles den en \outl{proporsjonalitetskonstant.} \regv
\regdef[Proporsjonale størrelser \label{prop}]{
\begin{equation}\label{propeq}
	\text{proporsjonalitetskonstant}=\frac{\text{en størrelse}}{\text{en annen størrelse}}
\end{equation}
} \regv

Anvendt matematikk er full av størrelser som er proporsjonalitetskonstanter, og i definisjonsboksene under finner du et utvalg av disse. Legg merke til at formlene som vises matematisk sett er de samme som \eqref{propeq}, det er bare navnene på størrelsene og enhetene som er forskjellige.
\newpage
\regdef[Kilopris]{
\outl{Kilopris} gir forholdet mellom en pris (i kr) og en vekt \\(i kilogram)
\begin{equation}\label{kilopriseq}
	\text{kilopris}=\frac{\text{pris}}{\text{vekt}}
\end{equation}
Alternativt kan vi skrive
\begin{equation}\label{kiloprisalteq}
	\text{pris}=\text{kilopris}\cdot \text{vekt}
\end{equation}
Benevningen for kilopris er 'kr/kg'.
}
\regdef[Literpris]{
	\outl{Literpris} gir forholdet mellom en pris (i kr) og et volum \\(i liter)
	\begin{equation}\label{literpriseq}
		\text{litepris}=\frac{\text{pris}}{\text{volum}}
	\end{equation}
	Alternativt kan vi skrive
	\begin{equation}\label{literprisalteq}
		\text{pris}=\text{literpris}\cdot \text{volum}
	\end{equation}
	Benevningen for literpris er 'kr/L'.
}
\regdef[Fart \label{fart}]{
\outl{Fart} gir forholdet mellom en lengde og en tid.
\begin{equation}\label{lengdeeq}
	\text{fart}=\frac{\text{lengde}}{\text{tid}}
\end{equation}
Alternativt kan vi skrive
\begin{equation}\label{lengdealteq}
	\text{lengde}=\text{fart}\cdot\text{tid}
\end{equation}
Vanlige benevninger for fart er\footnote{'h' står for 'time' ('hour' på engelsk).} 'km/h' og 'm/s'
}
\newpage
\regdef[Tetthet]{
En \outl{tetthet} gir forholdet mellom en vekt og et volum.
\begin{equation}\label{tettheteq}
	\text{tetthet}=\frac{\text{vekt}}{\text{volum}}
\end{equation}
Alternativt kan vi skrive
\begin{equation}\label{tetthetalteq}
	\text{vekt}=\text{tetthet}\cdot\text{volum}
\end{equation}
Vanlige benevninger for tetthet er er 'kg/m$ ^3 $' og 'g/cm$ ^3 $'	
}
\regdef[Effekt]{
	\outl{Effekt} gir forholdet mellom energi og tid
	\begin{equation}\label{effekteq}
		\text{effekt}=\frac{\text{energi}}{\text{tid}}
	\end{equation}
	Alternativt kan vi skrive
	\begin{equation}\label{Effektalteq}
		\text{energi}=\text{effekt}\cdot\text{tid}
	\end{equation}
	Vanlige benevninger for effekt er 'J/s' og 'kWh/s'. 'J' står for energieneheten 'Joule'. 'J/s' er det samme som 'W', som står for 'Watt'. I 'kWh' står 'k' for 'kilo', 'W' for 'Watt' og 'h' for 'time'.
}
\info{\note}{
I \eqref{kilopriseq}\,-\,\eqref{tettheteq} er det antatt at størrelsene på venstre side av likningen er \textit{konstant}, men det er ikke alltid slik. Lar du en stein falle fra en høgde, vil den åpenbart ikke ha den samme farten hele tiden. Ved å dele lengden den har falt med tiden det tok, vil du finne \textsl{hvilken fart ballen ville hatt dersom den skulle kommet seg like langt på den samme tiden, dersom farten var den samme hele tiden.}
}
\section{Regning med forskjellige benevninger \label{regnmforbenvn}}
Når vi skal utføre regneoperasjoner med størrelser som har benevning, er det helt avgjørende at vi passer på at benevningene som er involvert er de samme. \regv

\eks[1]{
Regn ut $ 5\enh{km}+4\,000\enh{m} $.

\sv
Her må vi enten gjøre om 5\enh{km} til antall m eller 4\,000\enh{m} til antall km før vi kan legge sammen verdiene. Vi velger å gjøre om $ 5\enh{km} $ til antall m:
\[ \text{5\enh{km}=5\,000\enh{m}}\]
Nå har vi at
\alg{
5\enh{km}+4\,000\enh{m} &= 5\,000\enh{m}+4\,000\enh{m} \\
&= 9\,000\enh{m}
}
}
\info{Tips}{
I mange utregninger kan enheter føre til at uttrykkene blir litt rotete. Hvis du er helt sikker på at alle benevningene er like, kan du med fordel skrive utregninger uten benevning. I \textsl{Eksempel 1} over kunne vi da regnet ut
\[ 5\,000+4\,000=9\,000 \]
Men merk at i et endelig svar \textsl{må} vi ha med benevning:
\[ 5\enh{km}+4\,000\enh{m}= 9\,000\enh{m} \]
}
\newpage
\eks[2]{
	Bruk ligning \eqref{lengdealteq} til å svare på spørsmålene.
	\abc{
		\item Hvor langt kjører en bil som holder farten 50\enh{km/h} i 3 timer?
		\item Hvor langt kjører en bil som holder farten  90\enh{km/h} i 45 minutt?
	}
	
	\sv
	\abc{
		\item I ligning \eqref{lengdealteq} er nå  farten $ 50 $ og tiden $ 3 $, og da er
		\[ \text{strekning}=50\cdot3=150 \]
		Altså har bilen kjørt 150\enh{km} 
		\item Her har vi to forskjellige enheter for tid involvert; 'timer' ('h') og 'minutt' ('min'). Da må vi enten gjøre om farten til antall 'km/min' eller tiden til antall 'h'. Vi velger å gjøre om antall 'min' til antall 'h'\footnote{Husk at $ 60\enh{min}=1\enh{h}. $}:
		\alg{
			45\enh{minutt}&=\frac{45}{60}\enh{timer} \br
			&=\frac{3}{4}\enh{timer}
		}
	I ligning \eqref{lengdealteq} er nå farten 90 og tiden $ \dfrac{3}{4} $, og da er
	\[ \text{strekning}=90\cdot\frac{3}{4}=67.5\]
	Altså har bilen kjørt 67.5\enh{km}.
	}
}
\newpage
\eks[3]{
	Bruk ligning \eqref{kilopriseq} til å svare på spørsmålene.
	\abc{
		\item 10\enh{kg} tomater koster 35\enh{kr}. Hva er kiloprisen til tomatene?
		\item Safran går for å være verdens dyreste krydder, 5\enh{g} kan koste 600\enh{kr}. Hva er da kiloprisen på safran?
	}
	\sv
	\abc{
		\item I ligning \eqref{kilopriseq} er nå prisen 35 og vekten 10, og da er
		\[ \text{kilopris}=\frac{35}{10}=3,5 \]
		Altså er kiloprisen på tomater 3,5\enh{kr/kg}
		\item Her har vi to forskjellige enheter for vekt involvert; kg og gram. Vi gjør om antall g til antall kg (se \rref{ompref}):
		\[ 5\enh{g}=0,005\enh{kg} \]
			I ligning \eqref{kilopriseq} er nå prisen 600 og vekten 0,005, og da er
		\[ \text{kilopris}=\frac{600}{0,005}=120\,000 \]
		Altså koster safran 120\,000\enh{kr/kg}.
	}

}

\end{document}


