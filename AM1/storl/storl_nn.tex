\documentclass[english, 11 pt, class=article, crop=false]{standalone}
\usepackage[T1]{fontenc}
\usepackage[utf8]{luainputenc}
\usepackage{lmodern} % load a font with all the characters
\usepackage{geometry}
\geometry{verbose,paperwidth=16.1 cm, paperheight=24 cm, inner=2.3cm, outer=1.8 cm, bmargin=2cm, tmargin=1.8cm}
\setlength{\parindent}{0bp}
\usepackage{import}
\usepackage[subpreambles=false]{standalone}
\usepackage{amsmath}
\usepackage{amssymb}
\usepackage{esint}
\usepackage{babel}
\usepackage{tabu}
\makeatother
\makeatletter

\usepackage{titlesec}
\usepackage{ragged2e}
\RaggedRight
\raggedbottom
\frenchspacing

% Norwegian names of figures, chapters, parts and content
\addto\captionsenglish{\renewcommand{\figurename}{Figur}}
\makeatletter
\addto\captionsenglish{\renewcommand{\chaptername}{Kapittel}}
\addto\captionsenglish{\renewcommand{\partname}{Del}}

\addto\captionsenglish{\renewcommand{\contentsname}{Innhald}}

\usepackage{graphicx}
\usepackage{float}
\usepackage{subfig}
\usepackage{placeins}
\usepackage{cancel}
\usepackage{framed}
\usepackage{wrapfig}
\usepackage[subfigure]{tocloft}
\usepackage[font=footnotesize]{caption} % Figure caption
\usepackage{bm}
\usepackage[dvipsnames, table]{xcolor}
\definecolor{shadecolor}{rgb}{0.105469, 0.613281, 1}
\colorlet{shadecolor}{Emerald!15} 
\usepackage{icomma}
\makeatother
\usepackage[many]{tcolorbox}
\usepackage{multicol}
\usepackage{stackengine}

% For tabular
\addto\captionsenglish{\renewcommand{\tablename}{Tabell}}
\usepackage{array}
\usepackage{multirow}
\usepackage{longtable} %breakable table

% Ligningsreferanser
\usepackage{mathtools}
\mathtoolsset{showonlyrefs}

% index
\usepackage{imakeidx}
\makeindex[title=Indeks]

%Footnote:
\usepackage[bottom, hang, flushmargin]{footmisc}
\usepackage{perpage} 
\MakePerPage{footnote}
\addtolength{\footnotesep}{2mm}
\renewcommand{\thefootnote}{\arabic{footnote}}
\renewcommand\footnoterule{\rule{\linewidth}{0.4pt}}
\renewcommand{\thempfootnote}{\arabic{mpfootnote}}

%colors
\definecolor{c1}{cmyk}{0,0.5,1,0}
\definecolor{c2}{cmyk}{1,0.25,1,0}
\definecolor{n3}{cmyk}{1,0.,1,0}
\definecolor{neg}{cmyk}{1,0.,0.,0}

% Lister med bokstavar
\usepackage[inline]{enumitem}

\newcounter{rg}
\numberwithin{rg}{chapter}
\newcommand{\reg}[2][]{\begin{tcolorbox}[boxrule=0.3 mm,arc=0mm,colback=blue!3] {\refstepcounter{rg}\phantomsection \large \textbf{\therg \;#1} \vspace{5 pt}}\newline #2  \end{tcolorbox}\vspace{-5pt}}

\newcommand{\regg}[2]{\begin{tcolorbox}[boxrule=0.3 mm,arc=0mm,colback=blue!3] {\large \textbf{#1} \vspace{5 pt}}\newline #2  \end{tcolorbox}\vspace{-5pt}}

\newcommand\alg[1]{\begin{align} #1 \end{align}}

\newcommand\eks[2][]{\begin{tcolorbox}[boxrule=0.3 mm,arc=0mm,enhanced jigsaw,breakable,colback=green!3] {\large \textbf{Eksempel #1} \vspace{5 pt}\\} #2 \end{tcolorbox}\vspace{-5pt} }

\newcommand{\st}[1]{\begin{tcolorbox}[boxrule=0.0 mm,arc=0mm,enhanced jigsaw,breakable,colback=yellow!12]{ #1} \end{tcolorbox}}

\newcommand{\spr}[1]{\begin{tcolorbox}[boxrule=0.3 mm,arc=0mm,enhanced jigsaw,breakable,colback=yellow!7] {\large \textbf{Språkboksen} \vspace{5 pt}\\} #1 \end{tcolorbox}\vspace{-5pt} }

\newcommand{\sym}[1]{\colorbox{blue!15}{#1}}

\newcommand{\info}[2]{\begin{tcolorbox}[boxrule=0.3 mm,arc=0mm,enhanced jigsaw,breakable,colback=cyan!6] {\large \textbf{#1} \vspace{5 pt}\\} #2 \end{tcolorbox}\vspace{-5pt} }

\newcommand\algv[1]{\vspace{-11 pt}\begin{align*} #1 \end{align*}}

\newcommand{\regv}{\vspace{5pt}}
\newcommand{\mer}{\textsl{Merk}: }
\newcommand\vsk{\vspace{11pt}}
\newcommand\vs{\vspace{-11pt}}
\newcommand\vsabc{\vspace{-5pt}}
\newcommand\vsb{\vspace{-16pt}}
\newcommand\sv{\vsk \textbf{Svar:} \vspace{4 pt}\\}
\newcommand\br{\\[5 pt]}
\newcommand{\fpath}[1]{../fig/#1}
\newcommand\algvv[1]{\vs\vs\begin{align*} #1 \end{align*}}
\newcommand{\y}[1]{$ {#1} $}
\newcommand{\os}{\\[5 pt]}
\newcommand{\prbxl}[2]{
	\parbox[l][][l]{#1\linewidth}{#2
}}
\newcommand{\prbxr}[2]{\parbox[r][][l]{#1\linewidth}{
		\setlength{\abovedisplayskip}{5pt}
		\setlength{\belowdisplayskip}{5pt}	
		\setlength{\abovedisplayshortskip}{0pt}
		\setlength{\belowdisplayshortskip}{0pt} 
		\begin{shaded}
			\footnotesize	#2 \end{shaded}}}

\newcommand{\fgbxr}[2]{
	\parbox[r][][l]{#1\linewidth}{#2
}}

\renewcommand{\cfttoctitlefont}{\Large\bfseries}
\setlength{\cftaftertoctitleskip}{0 pt}
\setlength{\cftbeforetoctitleskip}{0 pt}

\newcommand{\bs}{\\[3pt]}
\newcommand{\vn}{\\[6pt]}
\newcommand{\fig}[1]{\begin{figure}
		\centering
		\includegraphics[]{\fpath{#1}}
\end{figure}}


\newcommand{\sectionbreak}{\clearpage} % New page on each section

% Section comment
\newcommand{\rmerk}[1]{
\rule{\linewidth}{1pt}
#1 \\[-4pt]
\rule{\linewidth}{1pt}
}

% Equation comments
\newcommand{\cm}[1]{\llap{\color{blue} #1}}

\newcommand\fork[2]{\begin{tcolorbox}[boxrule=0.3 mm,arc=0mm,enhanced jigsaw,breakable,colback=yellow!7] {\large \textbf{#1 (forklaring)} \vspace{5 pt}\\} #2 \end{tcolorbox}\vspace{-5pt} }


%%% Rule boxes %%%
\newcommand{\gangdestihundre}{Å gonge desimaltall med 10, 100 osv.}
\newcommand{\delmedtihundre}{Deling med 10, 100, 1\,000 osv.}
\newcommand{\ompref}{Omgjering av prefiksar}


%License
\newcommand{\lic}{\textit{Anvend matematikk for grunnskule og VGS by Sindre Sogge Heggen is licensed under CC BY-NC-SA 4.0. To view a copy of this license, visit\\ 
		\net{http://creativecommons.org/licenses/by-nc-sa/4.0/}{http://creativecommons.org/licenses/by-nc-sa/4.0/}}}

%references
\newcommand{\net}[2]{{\color{blue}\href{#1}{#2}}}
\newcommand{\hrs}[2]{\hyperref[#1]{\color{blue}\textsl{#2 \ref*{#1}}}}
\newcommand{\rref}[1]{\hrs{#1}{Regel}}
\newcommand{\refkap}[1]{\hrs{#1}{Kapittel}}
\newcommand{\refsec}[1]{\hrs{#1}{Seksjon}}
\newcommand{\refdsec}[1]{\hrs{#1}{Delseksjon}}

\newcommand{\colr}[1]{{\color{red} #1}}
\newcommand{\colb}[1]{{\color{blue} #1}}
\newcommand{\colo}[1]{{\color{orange} #1}}
\newcommand{\colc}[1]{{\color{cyan} #1}}
\definecolor{projectgreen}{cmyk}{100,0,100,0}
\newcommand{\colg}[1]{{\color{projectgreen} #1}}

\newcommand{\mb}{\net{https://sindrsh.github.io/FirstPrinciplesOfMath/}{MB}}
\newcommand{\enh}[1]{\,\textrm{#1}}

\newcommand{\metode}[2]{
\textsl{#1} \\[-8pt]
\rule{#2}{0.75pt}
}

\newcommand{\linje}{\rule{\linewidth}{1pt} }

% Opg
\newcommand{\abc}[1]{
\begin{enumerate}[label=\alph*),leftmargin=18pt]
#1
\end{enumerate}
}
\newcommand{\abcs}[2]{
	\begin{enumerate}[label=\alph*),start=#1,leftmargin=18pt]
		#2
	\end{enumerate}
}

\newcommand{\abch}[1]{
	\hspace{-2pt}	\begin{enumerate*}[label=\alph*), itemjoin=\hspace{1cm}]
		#1
	\end{enumerate*}
}

\newcommand{\abchs}[2]{
	\hspace{-2pt}	\begin{enumerate*}[label=\alph*), itemjoin=\hspace{1cm}, start=#1]
		#2
	\end{enumerate*}
}

\newcommand{\abcn}[1]{
	\begin{enumerate}[label=\arabic*),leftmargin=20pt]
		#1
	\end{enumerate}
}


\newcommand{\opgt}{
\newpage
\phantomsection \addcontentsline{toc}{section}{Oppgaver} \section*{Oppgaver for kapittel \thechapter}\vs \setcounter{section}{1}}
\newcounter{opg}
\numberwithin{opg}{section}
\newcommand{\op}[1]{\vspace{15pt} \refstepcounter{opg}\large \textbf{\color{blue}\theopg} \vspace{2 pt} \label{#1} \\}
\newcommand{\oprgn}[1]{\vspace{15pt} \refstepcounter{opg}\large \textbf{\color{blue}\theopg\;(regneark)} \vspace{2 pt} \label{#1} \\}
\newcommand{\oppr}[1]{\vspace{15pt} \refstepcounter{opg}\large \textbf{\color{blue}\theopg\;(programmering)} \vspace{2 pt} \label{#1} \\}
\newcommand{\ekspop}{\vsk\textbf{Gruble \thechapter}\vspace{2 pt} \\}
\newcommand{\nes}{\stepcounter{section}
	\setcounter{opg}{0}}
\newcommand{\opr}[1]{\vspace{3pt}\textbf{\ref{#1}}}
\newcommand{\tbs}{\vspace{5pt}}

%Vedlegg
\newcounter{vedl}
\newcommand{\vedlg}[1]{\refstepcounter{vedl}\phantomsection\section*{G.\thevedl\;#1}  \addcontentsline{toc}{section}{G.\thevedl\; #1} }
\newcommand{\nreqvd}{\refstepcounter{vedleq}\tag{\thevedl \thevedleq}}

\newcounter{vedlE}
\newcommand{\vedle}[1]{\refstepcounter{vedlE}\phantomsection\section*{E.\thevedlE\;#1}  \addcontentsline{toc}{section}{E.\thevedlE\; #1} }

\newcounter{opge}
\numberwithin{opge}{part}
\newcommand{\ope}[1]{\vspace{15pt} \refstepcounter{opge}\large \textbf{\color{blue}E\theopge} \vspace{2 pt} \label{#1} \\}

%Excel og GGB:

\newcommand{\g}[1]{\begin{center} {\tt #1} \end{center}}
\newcommand{\gv}[1]{\begin{center} \vspace{-11 pt} {\tt #1}  \end{center}}
\newcommand{\cmds}[2]{{\tt #1}\\[-3pt]
	#2}


\usepackage{datetime2}
\usepackage[]{hyperref}

\begin{document}

\section{Størrelsar, einingar og prefiks}
Det vi kan måle og uttrykke med tall, kaller vi \textit{størrelsar}. Ein størrelse består gjerne av både ein verdi og ei \textit{eining}, og i denne seksjonen skal vi sjå på desse tre einingane:
\tbs
\begin{center}
	\begin{tabular}{c|c|c}
		\textbf{eining} & \textbf{forkorting} &\textbf{eining for}\\ \hline
		meter & m &lengde\\\hline
		gram & g &masse\\\hline
		liter & L & volum
	\end{tabular}
\end{center}\tbs
Nokre gongar har vi veldig store eller veldig små størrelsar, for eksempel er det ca. 40\,075\,000\,m rundt ekvator! For så store tall er det vanleg å bruke ein \textit{prefiks}. Da kan vi skrive at det er ca. 40\,075 km rundt ekvator. Her står 'km' for 'kilometer', og 'kilo' betyr '1\,000'. Så 1\,000\enh{meter}  er altså 1\enh{kilometer} . Her er prefiksane ein oftast\footnote{Unntaket er 'deka', som er ein veldig lite brukt prefiks, men vi har tatt den med fordi den kompletterer tallmønsteret.} møter på i kvardagen:
\begin{center}
	\begin{tabular}{c|c|r}
		\textbf{prefiks} & \textbf{forkortelse}&\textbf{verdi} \\ \hline
		kilo & k & 1\,000\phantom{000\;}\\\hline
		hekto & h & 100\phantom{000\;}\\\hline
		deka & da & 10\phantom{000\;}\\\hline
		desi & d & 0,1\phantom{0\,\;}\\\hline
		centi & c & 0,01\phantom{\,\;}\\\hline
		milli & m & 0,001\\\hline		
	\end{tabular}
\end{center}
Bruker vi denne tabellen i kombinasjon med einingane kan vi for eksempel sjå at\vs
\alg{
	1000\enh{g}&= 1\enh{kg} \\
	0,1 \enh{m} &= 1\enh{dm} \\
	0,01 \enh{L} &= 1\enh{cL}
}
Enda ryddigare kan vi få det viss vi lager ein vannrett tabell (se neste side) med meter, gram eller liter lagt til i midten\footnote{Legg merke til at 'meter', 'gram' og 'liter' er \textsl{einingar}, mens 'kilo', 'hekto' osv. er \textsl{tal}. Det kan derfor verke litt rart å sette dei opp i samme tabell, men for vårt formål fungerer det heilt fint.}. 
\newpage
\reg[\ompref \label{ompref}]{Når vi skal endre prefiksar kan vi bruke denne tabellen:
	\begin{center}
		\begin{tabular}{|c|c|c|c|c|c|c|c}
			kilo &
			hekto &
			deka & m/g/L &
			desi & 
			centi & 
			milli & 		
		\end{tabular}
	\end{center}
	Komma må flyttast like mange gongar som antal ruter vi må flytte oss frå opprinnelig prefiks til ny prefiks.\vsk
	
	{\footnotesize For lengde brukes også eininga 'mil' (1 mil $ = $ 10\,000\,m). Denne kan leggast på til venstre for 'kilo'.}
}
\eks[1]{
	Skriv om 23,4\,mL til antall 'L'.
	
	\sv
	Vi skriv tabellen vår med L i midten, og legg merke til at vi må \textsl{tre ruter til venstre} for å komme oss fra 'mL' til 'L':
	\begin{center}
		\begin{tabular}{|c|c|c|c|c|c|c|c}
			kilo &
			hekto &
			deka & \color{blue}L &
			desi & 
			centi & 
			\color{red} milli & 		
		\end{tabular}
	\end{center}
	Det betyr at vi må flytte kommaet vårt tre plassar til venstre for å gjere om mL til L:
	\[ 23,4\enh{mL}=0,0234\enh{L} \]
}
\eks[2]{
	Skriv om 30\,hg til antall 'cg'.
	
	\sv
	Vi skriv tabellen vår med 'g' i midten og legg merke til at vi må \textsl{fire ruter til høyre} for å komme oss fra 'hg' til 'cg':
	\begin{center}
		\begin{tabular}{|c|c|c|c|c|c|c|c}
			kilo &
			\color{red}hekto &
			deka & g &
			desi & 
			\color{blue}centi & 
			milli & 		
		\end{tabular}
	\end{center}
	Dét betyr at vi må flytte kommaet vårt fire plassar til høyre for å gjere om 'hg' til 'cg':
	\[ 30\enh{mg}=300\,000\enh{cg} \]
}
\newpage
\eks[3]{
	Gjør om 12\,500\,dm til antall 'mil'.
	
	\sv
	Vi skriv tabellen vår med m i midten, legg til 'mil', og merker oss at vi må \textsl{fem ruter til høyre} for å komme oss fra hg til cg:
	\begin{center}
		\begin{tabular}{|c|c|c|c|c|c|c|c|c}
			\color{blue}mil &kilo &
			hekto &
			deka & m &
			\color{red} desi & 
			centi & 
			milli & 		
		\end{tabular}
	\end{center}
	Dét betyr at vi må flytte kommaet vårt fem plassar til høyre for å gjere om 'mil' til 'dm':
	\[ 12\,500\enh{dm}=0,125\enh{mil} \]
}\vsk

\fork{\ref{ompref} \ompref}{
Omgjering av prefikser handlar om å gange/dele med 10, 100 osv. (se \refsec{Ganging} og \refsec{Divisjon}) \vsk

La oss som første eksempel skrive om $ 3,452\enh{km} $ til antall 'meter'. Vi har at
\algv{
3,452\enh{km}&= 3,452\cdot1000 \enh{m} \\
&=3\,452\enh{m}
}
La oss som andre eksempel skrive om 47\enh{mm} til antall 'meter'. Vi har at
\algv{
47\enh{mm}&=47\cdot\frac{1}{1000} \enh{m} \\
&= (47:1000) \enh{m}\\
&=0,047\enh{m}
}
}
\section{Regning med forskjellige nemningar \label{regnmforbenvn}}
En (eventuell) prefiks og ei eining utgjer ei \textit{nemning}. For eksempel, 9\enh{km} har nemninga 'km', mens 9\enh{m} har nemninga 'm'. Når vi skal utføre rekneoperasjoner med størrelsar som har nemning, er det heilt avgjerande at vi passar på at nemningane som er involvert er dei same. \regv

\eks[1]{
Regn ut $ 5\enh{km}+4\,000\enh{m} $.

\sv
Her må vi enten gjere om 5\enh{km} til antall m eller 4\,000\enh{m} til antall km før vi kan legge sammen verdiene. Vi velger å gjere om $ 5\enh{km} $ til antall m:
\[ \text{5\enh{km}=5\,000\enh{m}}\]
Nå har vi at
\alg{
5\enh{km}+4\,000\enh{m} &= 5\,000\enh{m}+4\,000\enh{m} \\
&= 9\,000\enh{m}
}
}
\info{Tips}{
I mange utregninger kan eininger føre til at uttrykkene blir litt rotete. Hvis du er helt sikker på at alle benevningene er like, kan du med fordel skrive utregninger uten benevning. I \textsl{Eksempel 1} over kunne vi da regnet ut
\[ 5\,000+4\,000=9\,000 \]
Men merk at i et endelig svar \textsl{må} vi ha med benevning:
\[ 5\enh{km}+4\,000\enh{m}= 9\,000\enh{m} \]
}
\newpage
\eks[2]{
	Hvis du kjører med konstant fart, er strekningen du har kjørt etter ein viss tid gitt ved formelen
	\[ \text{strekning}=\text{fart}\cdot \text{tid} \]
	
	\abc{
		\item Hvor langt kjører ein bil som holder farten 50\enh{km/h} i 3 timer?
		\item Hvor langt kjører ein bil som holder farten  90\enh{km/h} i 45 minutt?
	}
	
	\sv
	\abc{
		\item I formelen er nå  farten $ 50 $ og tiden $ 3 $, og da er
		\[ \text{strekning}=50\cdot3=150 \]
		Altså har bilen kjørt 150\enh{km} 
		\item Her har vi to forskjellige eininger for tid involvert; timer og minutt. Da må vi enten gjere om farten til km/min eller tiden til timer. Vi velger å gjere om minutt til timer:
		\alg{
			45\enh{minutt}&=\frac{45}{60}\enh{timer} \br
			&=\frac{3}{4}\enh{timer}
		}
	}
	I formelen er nå farten 90 og tiden $ \dfrac{3}{4} $, og da er
	\[ \text{strekning}=90\cdot\frac{3}{4}=67.5\]
	Altså har bilen kjørt 67.5\enh{km}.
}
\newpage
\eks[3]{
	\textit{Kiloprisen} til ein vare er hva ein vare koster per kg. Kilopris er gitt ved formelen
	\[ \text{kilopris}=\frac{\text{pris}}{\text{vekt}} \]
	\abc{
		\item 10\enh{kg} tomater koster 35\enh{kr}. Hva er kiloprisen til tomatene?
		\item Safran går for å være verdens dyreste krydder, 5\enh{g} kan koste 600\enh{kr}. Hva er da kiloprisen på safran?
	}
	\sv
	\abc{
		\item I formelen er nå prisen 35 og vekten 10, og da er
		\[ \text{kilopris}=\frac{35}{10}=3,5 \]
		Altså er kiloprisen på tomater 3,5\enh{kr/kg}
		\item Her har vi to forskjellige eininger for vekt involvert; kg og gram. Vi gjør om antall g til antall kg (se ??):
		\[ 5\enh{g}=0,005\enh{kg} \]
	}
	I formelen vår er nå prisen 600 og vekten 0,005, og da er
	\[ \text{kilopris}=\frac{600}{0,005}=120\,000 \]
	Altså koster safran 120\,000\enh{kr/kg}.
}

\end{document}


