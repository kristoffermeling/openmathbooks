\documentclass[english, 11 pt, class=article, crop=false]{standalone}
\usepackage[T1]{fontenc}
%\renewcommand*\familydefault{\sfdefault} % For dyslexia-friendly text
\usepackage{lmodern} % load a font with all the characters
\usepackage{geometry}
\geometry{verbose,paperwidth=16.1 cm, paperheight=24 cm, inner=2.3cm, outer=1.8 cm, bmargin=2cm, tmargin=1.8cm}
\setlength{\parindent}{0bp}
\usepackage{import}
\usepackage[subpreambles=false]{standalone}
\usepackage{amsmath}
\usepackage{amssymb}
\usepackage{esint}
\usepackage{babel}
\usepackage{tabu}
\makeatother
\makeatletter

\usepackage{titlesec}
\usepackage{ragged2e}
\RaggedRight
\raggedbottom
\frenchspacing

% Norwegian names of figures, chapters, parts and content
\addto\captionsenglish{\renewcommand{\figurename}{Figur}}
\makeatletter
\addto\captionsenglish{\renewcommand{\chaptername}{Kapittel}}
\addto\captionsenglish{\renewcommand{\partname}{Del}}


\usepackage{graphicx}
\usepackage{float}
\usepackage{subfig}
\usepackage{placeins}
\usepackage{cancel}
\usepackage{framed}
\usepackage{wrapfig}
\usepackage[subfigure]{tocloft}
\usepackage[font=footnotesize,labelfont=sl]{caption} % Figure caption
\usepackage{bm}
\usepackage[dvipsnames, table]{xcolor}
\definecolor{shadecolor}{rgb}{0.105469, 0.613281, 1}
\colorlet{shadecolor}{Emerald!15} 
\usepackage{icomma}
\makeatother
\usepackage[many]{tcolorbox}
\usepackage{multicol}
\usepackage{stackengine}

\usepackage{esvect} %For vectors with capital letters

% For tabular
\usepackage{array}
\usepackage{multirow}
\usepackage{longtable} %breakable table

% Ligningsreferanser
\usepackage{mathtools}
\mathtoolsset{showonlyrefs}

% index
\usepackage{imakeidx}
\makeindex[title=Indeks]

%Footnote:
\usepackage[bottom, hang, flushmargin]{footmisc}
\usepackage{perpage} 
\MakePerPage{footnote}
\addtolength{\footnotesep}{2mm}
\renewcommand{\thefootnote}{\arabic{footnote}}
\renewcommand\footnoterule{\rule{\linewidth}{0.4pt}}
\renewcommand{\thempfootnote}{\arabic{mpfootnote}}

%colors
\definecolor{c1}{cmyk}{0,0.5,1,0}
\definecolor{c2}{cmyk}{1,0.25,1,0}
\definecolor{n3}{cmyk}{1,0.,1,0}
\definecolor{neg}{cmyk}{1,0.,0.,0}

% Lister med bokstavar
\usepackage[inline]{enumitem}

\newcounter{rg}
\numberwithin{rg}{chapter}
\newcommand{\reg}[2][]{\begin{tcolorbox}[boxrule=0.3 mm,arc=0mm,colback=blue!3] {\refstepcounter{rg}\phantomsection \large \textbf{\therg \;#1} \vspace{5 pt}}\newline #2  \end{tcolorbox}\vspace{-5pt}}

\newcommand\alg[1]{\begin{align} #1 \end{align}}

\newcommand\eks[2][]{\begin{tcolorbox}[boxrule=0.3 mm,arc=0mm,enhanced jigsaw,breakable,colback=green!3] {\large \textbf{Eksempel #1} \vspace{5 pt}\\} #2 \end{tcolorbox}\vspace{-5pt} }

\newcommand{\st}[1]{\begin{tcolorbox}[boxrule=0.0 mm,arc=0mm,enhanced jigsaw,breakable,colback=yellow!12]{ #1} \end{tcolorbox}}

\newcommand{\spr}[1]{\begin{tcolorbox}[boxrule=0.3 mm,arc=0mm,enhanced jigsaw,breakable,colback=yellow!7] {\large \textbf{Språkboksen} \vspace{5 pt}\\} #1 \end{tcolorbox}\vspace{-5pt} }

\newcommand{\sym}[1]{\colorbox{blue!15}{#1}}

\newcommand{\info}[2]{\begin{tcolorbox}[boxrule=0.3 mm,arc=0mm,enhanced jigsaw,breakable,colback=cyan!6] {\large \textbf{#1} \vspace{5 pt}\\} #2 \end{tcolorbox}\vspace{-5pt} }

\newcommand\algv[1]{\vspace{-11 pt}\begin{align*} #1 \end{align*}}

\newcommand{\regv}{\vspace{5pt}}
\newcommand{\mer}{\textsl{Merk}: }
\newcommand{\mers}[1]{{\footnotesize \mer #1}}
\newcommand\vsk{\vspace{11pt}}
\newcommand\vs{\vspace{-11pt}}
\newcommand\vsb{\vspace{-16pt}}
\newcommand\sv{\vsk \textbf{Svar} \vspace{4 pt}\\}
\newcommand\br{\\[5 pt]}
\newcommand{\figp}[1]{../fig/#1}
\newcommand\algvv[1]{\vs\vs\begin{align*} #1 \end{align*}}
\newcommand{\y}[1]{$ {#1} $}
\newcommand{\os}{\\[5 pt]}
\newcommand{\prbxl}[2]{
\parbox[l][][l]{#1\linewidth}{#2
	}}
\newcommand{\prbxr}[2]{\parbox[r][][l]{#1\linewidth}{
		\setlength{\abovedisplayskip}{5pt}
		\setlength{\belowdisplayskip}{5pt}	
		\setlength{\abovedisplayshortskip}{0pt}
		\setlength{\belowdisplayshortskip}{0pt} 
		\begin{shaded}
			\footnotesize	#2 \end{shaded}}}

\renewcommand{\cfttoctitlefont}{\Large\bfseries}
\setlength{\cftaftertoctitleskip}{0 pt}
\setlength{\cftbeforetoctitleskip}{0 pt}

\newcommand{\bs}{\\[3pt]}
\newcommand{\vn}{\\[6pt]}
\newcommand{\fig}[1]{\begin{figure}
		\centering
		\includegraphics[]{\figp{#1}}
\end{figure}}

\newcommand{\figc}[2]{\begin{figure}
		\centering
		\includegraphics[]{\figp{#1}}
		\caption{#2}
\end{figure}}

\newcommand{\sectionbreak}{\clearpage} % New page on each section

\newcommand{\nn}[1]{
\begin{equation}
	#1
\end{equation}
}

% Equation comments
\newcommand{\cm}[1]{\llap{\color{blue} #1}}

\newcommand\fork[2]{\begin{tcolorbox}[boxrule=0.3 mm,arc=0mm,enhanced jigsaw,breakable,colback=yellow!7] {\large \textbf{#1 (forklaring)} \vspace{5 pt}\\} #2 \end{tcolorbox}\vspace{-5pt} }
 
%colors
\newcommand{\colr}[1]{{\color{red} #1}}
\newcommand{\colb}[1]{{\color{blue} #1}}
\newcommand{\colo}[1]{{\color{orange} #1}}
\newcommand{\colc}[1]{{\color{cyan} #1}}
\definecolor{projectgreen}{cmyk}{100,0,100,0}
\newcommand{\colg}[1]{{\color{projectgreen} #1}}

% Methods
\newcommand{\metode}[2]{
	\textsl{#1} \\[-8pt]
	\rule{#2}{0.75pt}
}

%Opg
\newcommand{\abc}[1]{
	\begin{enumerate}[label=\alph*),leftmargin=18pt]
		#1
	\end{enumerate}
}
\newcommand{\abcs}[2]{
	\begin{enumerate}[label=\alph*),start=#1,leftmargin=18pt]
		#2
	\end{enumerate}
}
\newcommand{\abcn}[1]{
	\begin{enumerate}[label=\arabic*),leftmargin=18pt]
		#1
	\end{enumerate}
}
\newcommand{\abch}[1]{
	\hspace{-2pt}	\begin{enumerate*}[label=\alph*), itemjoin=\hspace{1cm}]
		#1
	\end{enumerate*}
}
\newcommand{\abchs}[2]{
	\hspace{-2pt}	\begin{enumerate*}[label=\alph*), itemjoin=\hspace{1cm}, start=#1]
		#2
	\end{enumerate*}
}

% Oppgaver
\newcommand{\opgt}{\phantomsection \addcontentsline{toc}{section}{Oppgaver} \section*{Oppgaver for kapittel \thechapter}\vs \setcounter{section}{1}}
\newcounter{opg}
\numberwithin{opg}{section}
\newcommand{\op}[1]{\vspace{15pt} \refstepcounter{opg}\large \textbf{\color{blue}\theopg} \vspace{2 pt} \label{#1} \\}
\newcommand{\ekspop}[1]{\vsk\textbf{Gruble \thechapter.#1}\vspace{2 pt} \\}
\newcommand{\nes}{\stepcounter{section}
	\setcounter{opg}{0}}
\newcommand{\opr}[1]{\vspace{3pt}\textbf{\ref{#1}}}
\newcommand{\oeks}[1]{\begin{tcolorbox}[boxrule=0.3 mm,arc=0mm,colback=white]
		\textit{Eksempel: } #1	  
\end{tcolorbox}}
\newcommand\opgeks[2][]{\begin{tcolorbox}[boxrule=0.1 mm,arc=0mm,enhanced jigsaw,breakable,colback=white] {\footnotesize \textbf{Eksempel #1} \\} \footnotesize #2 \end{tcolorbox}\vspace{-5pt} }
\newcommand{\rknut}{
Rekn ut.
}

%License
\newcommand{\lic}{\textit{Matematikken sine byggesteinar by Sindre Sogge Heggen is licensed under CC BY-NC-SA 4.0. To view a copy of this license, visit\\ 
		\net{http://creativecommons.org/licenses/by-nc-sa/4.0/}{http://creativecommons.org/licenses/by-nc-sa/4.0/}}}

%referances
\newcommand{\net}[2]{{\color{blue}\href{#1}{#2}}}
\newcommand{\hrs}[2]{\hyperref[#1]{\color{blue}\textsl{#2 \ref*{#1}}}}
\newcommand{\rref}[1]{\hrs{#1}{regel}}
\newcommand{\refkap}[1]{\hrs{#1}{kapittel}}
\newcommand{\refsec}[1]{\hrs{#1}{seksjon}}

\newcommand{\mb}{\net{https://sindrsh.github.io/FirstPrinciplesOfMath/}{MB}}


%line to seperate examples
\newcommand{\linje}{\rule{\linewidth}{1pt} }

\usepackage{datetime2}
%%\usepackage{sansmathfonts} for dyslexia-friendly math
\usepackage[]{hyperref}

\geometry{verbose,paperwidth=21 cm, paperheight=29.7 cm, inner=2.3cm, outer=1.8 cm, bmargin=2cm, tmargin=1.8cm}

\begin{document}
\begin{center}
	\begin{tabular}{p{10.5cm} | c | c |} 
	\textbf{Kompetansemål 5. klasse} & \textbf{MB} & \textbf{AM}\\ \hline
\shortstack[l]{\\utforske og forklare sammenhenger mellom brøker,\\ desimaltall og prosent og bruke det i hoderegning} &\shortstack{1 \\4} &\shortstack{4} \\ \hline
	
\shortstack[l]{\\beskrive brøk som del av en hel, som del av en mengde\\ og som tall på tallinjen og vurdere og navngi størrelsene} &\shortstack{1\\4} &\shortstack{4} \\ \hline

\shortstack[l]{\\representere brøker på ulike måter og oversette\\ mellom de ulike representasjonene
} &\shortstack{1 \\4} &\shortstack{4} \\ \hline

\shortstack[l]{\\utvikle og bruke ulike strategier for regning med positive \\tall og brøk og forklare tenkemåtene sine
} &\shortstack{1\\4} &\shortstack{4} \\ \hline	

\shortstack[l]{\\formulere og løse problemer fra egen hverdag\\ som har med brøk å gjøre
} &\shortstack{1\\4} &\shortstack{4} \\ \hline

\shortstack[l]{\\diskutere tilfeldighet og sannsynlighet i spill og \\praktiske situasjoner og knytte det til brøk
} &\shortstack{} &\shortstack{7} \\ \hline

\shortstack[l]{\\løse ligninger og ulikheter gjennom logiske resonnementer og\\ forklare hva det vil si at et tall er en løsning på en ligning
} &\shortstack{8} &\shortstack{} \\ \hline

\shortstack[l]{\\lage og løse oppgaver i regneark som \\omhandler personlig økonomi
} &\shortstack{} &\shortstack{6\\E} \\ \hline

\shortstack[l]{\\formulere og løse problemer fra egen hverdag\\ som har med tid å gjøre
} &\shortstack{} &\shortstack{} \\ \hline

\shortstack[l]{\\lage og programmere algoritmer med bruk av \\variabler, vilkår og løkker
} &\shortstack{} &\shortstack{} \\ \hline
\end{tabular}
\end{center}

\begin{center}
	\begin{tabular}{p{10.5cm} | c | c} 
		\textbf{Kompetansemål 6. klasse} & \textbf{MB} & \textbf{AM}\\ \hline
		\shortstack[l]{\\utforske, navngi og plassere desimaltall \\på tallinjen
		} &\shortstack{1} &\shortstack{} \\ \hline
	
	\shortstack[l]{\\utforske strategier for regning med desimaltall og \\sammenligne med regnestrategier for hele tall
	} &\shortstack{} &\shortstack{1} \\ \hline

\shortstack[l]{\\formulere og løse problemer fra sin egen hverdag som har med \\desimaltall, brøk og prosent å gjøre, og forklare egne\\ tenkemåter
} &\shortstack{} &\shortstack{7\\4} \\ \hline

\shortstack[l]{\\beskrive egenskaper ved og minimumsdefinisjoner av to- og \\tredimensjonale figurer og forklare hvilke egenskaper figurene \\har felles, og hvilke egenskaper som skiller dem fra hverandre
} &\shortstack{} &\shortstack{} \\ \hline

\shortstack[l]{\\utforske og beskrive symmetri i mønstre og utføre \\kongruensavbildninger med og uten koordinatsystem
} &\shortstack{} &\shortstack{} \\ \hline

\shortstack[l]{\\måle radius, diameter og omkrets i sirkler og utforske\\ og argumentere for sammenhengen
} &\shortstack{10} &\shortstack{} \\ \hline

\shortstack[l]{\\utforske mål for areal og volum i praktiske situasjoner \\og representere dem på ulike måter
} &\shortstack{10} &\shortstack{3} \\ \hline

\shortstack[l]{\\bruke ulike strategier for å regne ut areal og omkrets \\og utforske sammenhenger mellom disse
} &\shortstack{6} &\shortstack{3} \\ \hline

\shortstack[l]{\\bruke variabler og formler til å uttrykke sammenhenger\\ i praktiske situasjoner
} &\shortstack{6} &\shortstack{3 \\5} \\ \hline
	\end{tabular}
\end{center}


\begin{center}
	\begin{tabular}{p{10.5cm} | c | c |} 
		\textbf{Kompetansemål 7. klasse} & \textbf{MB} & \textbf{AM}\\ \hline
		\shortstack[l]{\\utvikle og bruke hensiktsmessige strategier i regning med brøk,\\ desimaltall og prosent og forklare tenkemåtene sine
		} &\shortstack{1 \\4} &\shortstack{4} \\ \hline
	
	\shortstack[l]{\\representere og bruke brøk, desimaltall og prosent på ulike \\måter og utforske de matematiske sammenhengene mellom \\disse representasjonsformene
	} &\shortstack{1 \\4} &\shortstack{4} \\ \hline

	\shortstack[l]{\\utforske negative tall i praktiske situasjoner
} &\shortstack{5} &\shortstack{} \\ \hline

	\shortstack[l]{\\bruke tallinje i regning med positive og negative tall
} &\shortstack{5} &\shortstack{} \\ \hline

	\shortstack[l]{\\bruke sammensatte regneuttrykk til å beskrive\\ og utføre utregninger
} &\shortstack{1} &\shortstack{3} \\ \hline

	\shortstack[l]{\\bruke ulike strategier for å løse lineære ligninger og ulikheter \\og vurdere om løsninger er gyldige
} &\shortstack{} &\shortstack{8} \\ \hline

	\shortstack[l]{\\ utforske og bruke hensiktsmessige sentralmål i egne og andres\\ statistiske undersøkelser
} &\shortstack{} &\shortstack{2} \\ \hline

	\shortstack[l]{\\ logge, sortere, presentere og lese data i tabeller og diagrammer\\ og begrunne valget av framstilling
} &\shortstack{} &\shortstack{2} \\ \hline

	\shortstack[l]{\\ lage og vurdere budsjett og regnskap ved å bruke regneark \\med cellereferanser og formler
} &\shortstack{} &\shortstack{6} \\ \hline

\shortstack[l]{\\ bruke programmering til å utforske data i tabeller og datasett
} &\shortstack{} &\shortstack{} \\ \hline	
	\end{tabular}	
\end{center}

\begin{center}
	\begin{tabular}{p{10.5cm} | c | c |} 
		\textbf{Kompetansemål 8. klasse} & \textbf{MB} & \textbf{AM}\\ \hline
		\shortstack[l]{\\ bruke potenser og kvadratrøtter i utforsking og problemløsing\\ og argumentere for framgangsmåter og resultater
		} &\shortstack{4} &\shortstack{4} \\ \hline
		
		\shortstack[l]{\\utvikle og kommunisere strategiar for hovudrekning \\i utrekningar
		} &\shortstack{1} &\shortstack{} \\ \hline
		
		\shortstack[l]{\\ utforske og beskrive primtalsfaktorisering \\og bruke det i brøkrekning
		} &\shortstack{4} &\shortstack{} \\ \hline
		
		\shortstack[l]{\\ utforske algebraiske reknereglar
		} &\shortstack{7} &\shortstack{} \\ \hline
		
		\shortstack[l]{\\ beskrive og generalisere mønster med eigne ord og algebraisk
		} &\shortstack{} &\shortstack{9} \\ \hline
		
		\shortstack[l]{\\lage og løyse problem som omhandlar samansette måleiningar
		} &\shortstack{} &\shortstack{1} \\ \hline
		
		\shortstack[l]{\\ lage og forklare rekneuttrykk med tal, variablar og konstantar \\knytte til praktiske situasjonar
		} &\shortstack{} &\shortstack{5} \\ \hline
		
		\shortstack[l]{\\ lage, løyse og forklare likningar knytte til praktiske situasjonar
		} &\shortstack{} &\shortstack{5} \\ \hline
		
		\shortstack[l]{\\ utforske, forklare og samanlikne funksjonar knytte til\\ praktiske situasjonar
		} &\shortstack{} &\shortstack{5} \\ \hline
		
		\shortstack[l]{\\ representere funksjonar på ulike måtar og vise samanhengar\\ mellom representasjonane
		} &\shortstack{9} &\shortstack{5} \\ \hline	

		\shortstack[l]{\\ utforske korleis algoritmar kan skapast, testast\\ og forbetrast ved hjelp av programmering
} &\shortstack{} &\shortstack{} \\ \hline		
	\end{tabular}	
\end{center}


\begin{center}
	\begin{tabular}{p{10.5cm} | c | c |} 
		\textbf{Kompetansemål 9. klasse} & \textbf{MB} & \textbf{AM}\\ \hline
		\shortstack[l]{\\ beskrive, forklare og presentere strukturer og utviklinger\\ i geometriske mønstre og i tallmønstre
		} &\shortstack{9} &\shortstack{} \\ \hline
		
		\shortstack[l]{\\ utforske egenskapene ved ulike polygoner og forklare\\ begrepene formlikhet og kongruens
		} &\shortstack{6 \\ 10} &\shortstack{} \\ \hline
		
		\shortstack[l]{\\ utforske, beskrive og argumentere for sammenhenger mellom\\ sidelengdene i trekanter
		} &\shortstack{1} &\shortstack{} \\ \hline
		
		\shortstack[l]{\\ utforske og argumentere for hvordan det å endre forutsetninger \\i geometriske problemstillinger påvirker løsninger
		} &\shortstack{10} &\shortstack{1} \\ \hline
		
		\shortstack[l]{\\ utforske og argumentere for formler for areal og volum av\\ tredimensjonale figurer
		} &\shortstack{1} &\shortstack{5} \\ \hline
		
		\shortstack[l]{\\ tolke og kritisk vurdere statistiske \\framstillinger fra mediene og lokalsamfunnet
		} &\shortstack{} &\shortstack{2} \\ \hline
		
		\shortstack[l]{\\ finne og diskutere sentralmål og spredningsmål i reelle datasett
		} &\shortstack{} &\shortstack{2} \\ \hline
		
		\shortstack[l]{\\ utforske og argumentere for hvordan framstillinger av tall og \\data kan brukes for å fremme ulike synspunkter
		} &\shortstack{} &\shortstack{2\\4} \\ \hline
		
		\shortstack[l]{\\ beregne og vurdere sannsynlighet i statistikk og spill
		} &\shortstack{} &\shortstack{7} \\ \hline
		
		\shortstack[l]{\\ simulere utfall i tilfeldige forsøk og beregne sannsynligheten \\for at noe skal inntreffe, ved å bruke programmering
		} &\shortstack{} &\shortstack{} \\ \hline	
	\end{tabular}	
\end{center}

\begin{center}
	\begin{tabular}{p{10.5cm} | c | c |} 
		\textbf{Kompetansemål 10. klasse} & \textbf{MB} & \textbf{AM}\\ \hline
		\shortstack[l]{\\ utforske og generalisere multiplikasjon av\\ polynomer algebraisk og geometrisk
		} &\shortstack{3\\7} &\shortstack{} \\ \hline
		
		\shortstack[l]{\\ utforske og sammenligne egenskaper ved ulike funksjoner \\ved å bruke digitale verktøy
		} &\shortstack{} &\shortstack{G} \\ \hline
		
		\shortstack[l]{\\ lage, løse og forklare ligningssett knyttet til praktiske \\situasjoner
		} &\shortstack{} &\shortstack{} \\ \hline
		
		\shortstack[l]{\\ utforske sammenhengen mellom konstant prosentvis endring,\\ vekstfaktor og eksponentialfunksjoner
		} &\shortstack{} &\shortstack{4} \\ \hline
		
		\shortstack[l]{\\ hente ut og tolke relevant informasjon fra tekster om\\ kjøp og salg og ulike typer lån og bruke det til å formulere \\og løse problemer
		} &\shortstack{} &\shortstack{6} \\ \hline
		
		\shortstack[l]{\\ planlegge, utføre og presentere et utforskende arbeid\\ knyttet til personlig økonomi
		} &\shortstack{} &\shortstack{6} \\ \hline
		
		\shortstack[l]{\\ bruke funksjoner i modellering og argumentere\\ for framgangsmåter og resultater
		} &\shortstack{9} &\shortstack{5} \\ \hline
		
		\shortstack[l]{\\ modellere situasjoner knyttet til reelle datasett, presentere\\ resultatene og argumentere for at modellene er gyldige
		} &\shortstack{} &\shortstack{2\\5} \\ \hline
		
		\shortstack[l]{\\ utforske matematiske egenskaper og sammenhenger ved\\ å bruke programmering
		} &\shortstack{} &\shortstack{} \\ \hline
		
	\end{tabular}	
\end{center}


\end{document}





