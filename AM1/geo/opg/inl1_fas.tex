\documentclass[english, 11 pt, class=article, crop=false]{standalone}
\usepackage[T1]{fontenc}
%\renewcommand*\familydefault{\sfdefault} % For dyslexia-friendly text
\usepackage{lmodern} % load a font with all the characters
\usepackage{geometry}
\geometry{verbose,paperwidth=16.1 cm, paperheight=24 cm, inner=2.3cm, outer=1.8 cm, bmargin=2cm, tmargin=1.8cm}
\setlength{\parindent}{0bp}
\usepackage{import}
\usepackage[subpreambles=false]{standalone}
\usepackage{amsmath}
\usepackage{amssymb}
\usepackage{esint}
\usepackage{babel}
\usepackage{tabu}
\makeatother
\makeatletter

\usepackage{titlesec}
\usepackage{ragged2e}
\RaggedRight
\raggedbottom
\frenchspacing

% Norwegian names of figures, chapters, parts and content
\addto\captionsenglish{\renewcommand{\figurename}{Figur}}
\makeatletter
\addto\captionsenglish{\renewcommand{\chaptername}{Kapittel}}
\addto\captionsenglish{\renewcommand{\partname}{Del}}


\usepackage{graphicx}
\usepackage{float}
\usepackage{subfig}
\usepackage{placeins}
\usepackage{cancel}
\usepackage{framed}
\usepackage{wrapfig}
\usepackage[subfigure]{tocloft}
\usepackage[font=footnotesize,labelfont=sl]{caption} % Figure caption
\usepackage{bm}
\usepackage[dvipsnames, table]{xcolor}
\definecolor{shadecolor}{rgb}{0.105469, 0.613281, 1}
\colorlet{shadecolor}{Emerald!15} 
\usepackage{icomma}
\makeatother
\usepackage[many]{tcolorbox}
\usepackage{multicol}
\usepackage{stackengine}

\usepackage{esvect} %For vectors with capital letters

% For tabular
\usepackage{array}
\usepackage{multirow}
\usepackage{longtable} %breakable table

% Ligningsreferanser
\usepackage{mathtools}
\mathtoolsset{showonlyrefs}

% index
\usepackage{imakeidx}
\makeindex[title=Indeks]

%Footnote:
\usepackage[bottom, hang, flushmargin]{footmisc}
\usepackage{perpage} 
\MakePerPage{footnote}
\addtolength{\footnotesep}{2mm}
\renewcommand{\thefootnote}{\arabic{footnote}}
\renewcommand\footnoterule{\rule{\linewidth}{0.4pt}}
\renewcommand{\thempfootnote}{\arabic{mpfootnote}}

%colors
\definecolor{c1}{cmyk}{0,0.5,1,0}
\definecolor{c2}{cmyk}{1,0.25,1,0}
\definecolor{n3}{cmyk}{1,0.,1,0}
\definecolor{neg}{cmyk}{1,0.,0.,0}

% Lister med bokstavar
\usepackage[inline]{enumitem}

\newcounter{rg}
\numberwithin{rg}{chapter}
\newcommand{\reg}[2][]{\begin{tcolorbox}[boxrule=0.3 mm,arc=0mm,colback=blue!3] {\refstepcounter{rg}\phantomsection \large \textbf{\therg \;#1} \vspace{5 pt}}\newline #2  \end{tcolorbox}\vspace{-5pt}}

\newcommand\alg[1]{\begin{align} #1 \end{align}}

\newcommand\eks[2][]{\begin{tcolorbox}[boxrule=0.3 mm,arc=0mm,enhanced jigsaw,breakable,colback=green!3] {\large \textbf{Eksempel #1} \vspace{5 pt}\\} #2 \end{tcolorbox}\vspace{-5pt} }

\newcommand{\st}[1]{\begin{tcolorbox}[boxrule=0.0 mm,arc=0mm,enhanced jigsaw,breakable,colback=yellow!12]{ #1} \end{tcolorbox}}

\newcommand{\spr}[1]{\begin{tcolorbox}[boxrule=0.3 mm,arc=0mm,enhanced jigsaw,breakable,colback=yellow!7] {\large \textbf{Språkboksen} \vspace{5 pt}\\} #1 \end{tcolorbox}\vspace{-5pt} }

\newcommand{\sym}[1]{\colorbox{blue!15}{#1}}

\newcommand{\info}[2]{\begin{tcolorbox}[boxrule=0.3 mm,arc=0mm,enhanced jigsaw,breakable,colback=cyan!6] {\large \textbf{#1} \vspace{5 pt}\\} #2 \end{tcolorbox}\vspace{-5pt} }

\newcommand\algv[1]{\vspace{-11 pt}\begin{align*} #1 \end{align*}}

\newcommand{\regv}{\vspace{5pt}}
\newcommand{\mer}{\textsl{Merk}: }
\newcommand{\mers}[1]{{\footnotesize \mer #1}}
\newcommand\vsk{\vspace{11pt}}
\newcommand\vs{\vspace{-11pt}}
\newcommand\vsb{\vspace{-16pt}}
\newcommand\sv{\vsk \textbf{Svar} \vspace{4 pt}\\}
\newcommand\br{\\[5 pt]}
\newcommand{\figp}[1]{../fig/#1}
\newcommand\algvv[1]{\vs\vs\begin{align*} #1 \end{align*}}
\newcommand{\y}[1]{$ {#1} $}
\newcommand{\os}{\\[5 pt]}
\newcommand{\prbxl}[2]{
\parbox[l][][l]{#1\linewidth}{#2
	}}
\newcommand{\prbxr}[2]{\parbox[r][][l]{#1\linewidth}{
		\setlength{\abovedisplayskip}{5pt}
		\setlength{\belowdisplayskip}{5pt}	
		\setlength{\abovedisplayshortskip}{0pt}
		\setlength{\belowdisplayshortskip}{0pt} 
		\begin{shaded}
			\footnotesize	#2 \end{shaded}}}

\renewcommand{\cfttoctitlefont}{\Large\bfseries}
\setlength{\cftaftertoctitleskip}{0 pt}
\setlength{\cftbeforetoctitleskip}{0 pt}

\newcommand{\bs}{\\[3pt]}
\newcommand{\vn}{\\[6pt]}
\newcommand{\fig}[1]{\begin{figure}
		\centering
		\includegraphics[]{\figp{#1}}
\end{figure}}

\newcommand{\figc}[2]{\begin{figure}
		\centering
		\includegraphics[]{\figp{#1}}
		\caption{#2}
\end{figure}}

\newcommand{\sectionbreak}{\clearpage} % New page on each section

\newcommand{\nn}[1]{
\begin{equation}
	#1
\end{equation}
}

% Equation comments
\newcommand{\cm}[1]{\llap{\color{blue} #1}}

\newcommand\fork[2]{\begin{tcolorbox}[boxrule=0.3 mm,arc=0mm,enhanced jigsaw,breakable,colback=yellow!7] {\large \textbf{#1 (forklaring)} \vspace{5 pt}\\} #2 \end{tcolorbox}\vspace{-5pt} }
 
%colors
\newcommand{\colr}[1]{{\color{red} #1}}
\newcommand{\colb}[1]{{\color{blue} #1}}
\newcommand{\colo}[1]{{\color{orange} #1}}
\newcommand{\colc}[1]{{\color{cyan} #1}}
\definecolor{projectgreen}{cmyk}{100,0,100,0}
\newcommand{\colg}[1]{{\color{projectgreen} #1}}

% Methods
\newcommand{\metode}[2]{
	\textsl{#1} \\[-8pt]
	\rule{#2}{0.75pt}
}

%Opg
\newcommand{\abc}[1]{
	\begin{enumerate}[label=\alph*),leftmargin=18pt]
		#1
	\end{enumerate}
}
\newcommand{\abcs}[2]{
	\begin{enumerate}[label=\alph*),start=#1,leftmargin=18pt]
		#2
	\end{enumerate}
}
\newcommand{\abcn}[1]{
	\begin{enumerate}[label=\arabic*),leftmargin=18pt]
		#1
	\end{enumerate}
}
\newcommand{\abch}[1]{
	\hspace{-2pt}	\begin{enumerate*}[label=\alph*), itemjoin=\hspace{1cm}]
		#1
	\end{enumerate*}
}
\newcommand{\abchs}[2]{
	\hspace{-2pt}	\begin{enumerate*}[label=\alph*), itemjoin=\hspace{1cm}, start=#1]
		#2
	\end{enumerate*}
}

% Oppgaver
\newcommand{\opgt}{\phantomsection \addcontentsline{toc}{section}{Oppgaver} \section*{Oppgaver for kapittel \thechapter}\vs \setcounter{section}{1}}
\newcounter{opg}
\numberwithin{opg}{section}
\newcommand{\op}[1]{\vspace{15pt} \refstepcounter{opg}\large \textbf{\color{blue}\theopg} \vspace{2 pt} \label{#1} \\}
\newcommand{\ekspop}[1]{\vsk\textbf{Gruble \thechapter.#1}\vspace{2 pt} \\}
\newcommand{\nes}{\stepcounter{section}
	\setcounter{opg}{0}}
\newcommand{\opr}[1]{\vspace{3pt}\textbf{\ref{#1}}}
\newcommand{\oeks}[1]{\begin{tcolorbox}[boxrule=0.3 mm,arc=0mm,colback=white]
		\textit{Eksempel: } #1	  
\end{tcolorbox}}
\newcommand\opgeks[2][]{\begin{tcolorbox}[boxrule=0.1 mm,arc=0mm,enhanced jigsaw,breakable,colback=white] {\footnotesize \textbf{Eksempel #1} \\} \footnotesize #2 \end{tcolorbox}\vspace{-5pt} }
\newcommand{\rknut}{
Rekn ut.
}

%License
\newcommand{\lic}{\textit{Matematikken sine byggesteinar by Sindre Sogge Heggen is licensed under CC BY-NC-SA 4.0. To view a copy of this license, visit\\ 
		\net{http://creativecommons.org/licenses/by-nc-sa/4.0/}{http://creativecommons.org/licenses/by-nc-sa/4.0/}}}

%referances
\newcommand{\net}[2]{{\color{blue}\href{#1}{#2}}}
\newcommand{\hrs}[2]{\hyperref[#1]{\color{blue}\textsl{#2 \ref*{#1}}}}
\newcommand{\rref}[1]{\hrs{#1}{regel}}
\newcommand{\refkap}[1]{\hrs{#1}{kapittel}}
\newcommand{\refsec}[1]{\hrs{#1}{seksjon}}

\newcommand{\mb}{\net{https://sindrsh.github.io/FirstPrinciplesOfMath/}{MB}}


%line to seperate examples
\newcommand{\linje}{\rule{\linewidth}{1pt} }

\usepackage{datetime2}
%%\usepackage{sansmathfonts} for dyslexia-friendly math
\usepackage[]{hyperref}


\newcounter{inl}
\numberwithin{inl}{chapter}
\newcommand{\inl}{\vspace{15pt} \refstepcounter{inl} \textbf{Oppgave \theinl} \vspace{2 pt}\\}
\renewcommand\theinl{\arabic{inl}}

\begin{document}
{\textbf{\huge Innlevering for kapittel 1-4}}\\ \vspace{12 pt}

{\textbf{\Large Løsningsforslag}}\\ \vspace{12 pt}

	
\inl \vs \vs
\alg{
2^2(5-7\cdot2)+11-3\cdot6  &= 4(5-14)+11-18 \\
&= 4(-9)+11-18 \\
&= -36-7\\
&= -43
}

\inl 
\textbf{a)} \vs
\alg{
4x &= 20-x  \\
4x+x &= 20 \\
5x &= 20 \\
\frac{\cancel{5}x}{\cancel{5}}&=\frac{20}{5}	\\
x &= 4
}

\textbf{b)} \vs
\alg{
	\frac{x}{5}+7 &= 9 \\
	\frac{x}{\cancel{5}}\cdot5 +7\cdot5 &= 9\cdot5 \\
	x +35 &= 45 \\
	x&= 45-35 \\
	x &= 10
}

\textbf{c)} \vs
\alg{
2(3x-4)&=-8+5x \\
2\cdot3x-2\cdot4&= -8+5x \\
6x-8 &= -8+5x \\
6x-5x &= -8+8\\
x &= 0 
}

\textbf{d)} Vi ganger med 15 fordi 15 er delelig med både 5 og 3. Da kvitter vi oss med brøkene i ligningen. \os

\textsl{Obs!} Vi kan bare gjøre dette hvis vi har en ligning, \textsl{ikke} hvis vi har et regnestykke med brøk (som f. eks i oppgave 3c).
\alg{
\frac{1}{5}(x-4)&=\frac{1}{3}(4x+2) \\
15\cdot\frac{1}{5}(x-4)&= \frac{1}{3}(3x+2) \\
3(x-4)&= 5(4x+2) \\
3\cdot x-3\cdot4 &= 5\cdot4x+5\cdot2 \\
3x-12 &= 20x +10 \\
3x-20x &= 10+12 \\
-17x &= 22 \\
\frac{\bcancel{-17}x}{\bcancel{-17}}&= \frac{22}{-17} \br
x &= -\frac{22}{17}
}
(Hvis vi har en brøk med ett negativt tall, er det valig å skrive minustegnet midt på og foran brøken).

\inl
\textbf{a)} \vs \alg{
\frac{3}{2}\cdot\frac{7}{4} &= \frac{3\cdot7}{2\cdot4} \\
&= \frac{21}{8}
}
\textbf{b)} \vs \alg{
\frac{5}{3}:\frac{4}{9}&= \frac{5}{3}\cdot\frac{9}{4} \br &=\frac{5\cdot9}{3\cdot4} \br &= \frac{45}{12}
}
\textsl{Obs!} Brøken kan forkores til $ \frac{15}{4} $. \vsk

\textbf{c)} 4, 2 og 5 kan vi alle gange med et heltall for å få 20. Derfor skriver vi om brøkene slik at de alle har 20 som nevner: 
\alg{
\frac{7}{4}+\frac{1}{2}-\frac{4}{5} &= \frac{7}{4}\cdot\frac{5}{5}+\frac{1}{2}\cdot\frac{10}{10}-\frac{4}{5}\cdot\frac{4}{4} \br
&= \frac{35}{20}+\frac{10}{20}-\frac{16}{20} \br
&= \frac{29}{20}
}

\inl
\textbf{a)}\vs \alg{
\text{1\% av 300}&=\frac{300}{100} \\
&= 3
}
\textbf{b)}\vs \alg{
15\% &= \frac{15}{100} \\
&= 0,15
}

\textbf{c)} Vi ganger 15 med 1\% av 300: \alg{
15\cdot 3 &= 45
}
15\% av 300 er altså 45.\vsk

\textbf{d)} Vi deler 15 med 1\% av 300:
\[ \frac{15}{3}=5 \]
15 utgjør 5\% av 300.

\inl
\textbf{a)} Barna skal ha én tredel $\left( \frac{1}{3}\right) $ av én halvpart $\left( \frac{1}{2}\right) $, altså:
\[ \frac{1}{2}\cdot\frac{1}{3}=\frac{1}{6} \]

\textbf{b)} \[ 6\,000\,000\cdot \frac{1}{6}=1\,000\,000 \]
De får 1 million hver.
\newpage
\inl
\textbf{a)} Prisen blir satt ned, altså redusert, med 40\%. Prosentfaktoren til reduksjonen er 0,4. Fordi prisen blir redusert er vekstfaktoren da:
\[ 1-0,4=0,6 \]

\textsl{Obs!} Hvis du liker det bedre kan du regne slik: Om vi tar bort 40\% sitter vi igjen med 60\%. Prosentfaktoren til 60\% er 0,6 og kalles da også vekstfaktoren (!).\vsk

\textbf{b)} Originalprisen finner vi ved å dele den nye prisen med vekstfaktoren:
\[ \frac{900}{0,6}=1500 \]
\obs Dette er det samme som $ \frac{900\cdot100}{60} $.

\inl
Når varen ble satt ned med 50\% var vekstfaktoren 0,5. Prisen før dette skjedde finner vi ved å dele 350 med 0,5. (\textsl{Tips:} Å dele med 0,5 er det samme som å gange med 2.)
\[ \frac{350}{0,5}=700 \]
Når varen ble satt ned med 25\% var vekstfaktren 0,75\%. Originalprisen finner vi ved å dele 700 med 0,75:
\[ \frac{700}{0,75}\approx 933,33 \]
Originalprisen var altså ca 933 kr. 

\end{document}

