\documentclass[english, 11 pt, class=article, crop=false]{standalone}
\usepackage[T1]{fontenc}
\usepackage[utf8]{luainputenc}
\usepackage{lmodern} % load a font with all the characters
\usepackage{geometry}
\geometry{verbose,paperwidth=16.1 cm, paperheight=24 cm, inner=2.3cm, outer=1.8 cm, bmargin=2cm, tmargin=1.8cm}
\setlength{\parindent}{0bp}
\usepackage{import}
\usepackage[subpreambles=false]{standalone}
\usepackage{amsmath}
\usepackage{amssymb}
\usepackage{esint}
\usepackage{babel}
\usepackage{tabu}
\makeatother
\makeatletter

\usepackage{titlesec}
\usepackage{ragged2e}
\RaggedRight
\raggedbottom
\frenchspacing

% Norwegian names of figures, chapters, parts and content
\addto\captionsenglish{\renewcommand{\figurename}{Figur}}
\makeatletter
\addto\captionsenglish{\renewcommand{\chaptername}{Kapittel}}
\addto\captionsenglish{\renewcommand{\partname}{Del}}

\addto\captionsenglish{\renewcommand{\contentsname}{Innhald}}

\usepackage{graphicx}
\usepackage{float}
\usepackage{subfig}
\usepackage{placeins}
\usepackage{cancel}
\usepackage{framed}
\usepackage{wrapfig}
\usepackage[subfigure]{tocloft}
\usepackage[font=footnotesize]{caption} % Figure caption
\usepackage{bm}
\usepackage[dvipsnames, table]{xcolor}
\definecolor{shadecolor}{rgb}{0.105469, 0.613281, 1}
\colorlet{shadecolor}{Emerald!15} 
\usepackage{icomma}
\makeatother
\usepackage[many]{tcolorbox}
\usepackage{multicol}
\usepackage{stackengine}

% For tabular
\addto\captionsenglish{\renewcommand{\tablename}{Tabell}}
\usepackage{array}
\usepackage{multirow}
\usepackage{longtable} %breakable table

% Ligningsreferanser
\usepackage{mathtools}
\mathtoolsset{showonlyrefs}

% index
\usepackage{imakeidx}
\makeindex[title=Indeks]

%Footnote:
\usepackage[bottom, hang, flushmargin]{footmisc}
\usepackage{perpage} 
\MakePerPage{footnote}
\addtolength{\footnotesep}{2mm}
\renewcommand{\thefootnote}{\arabic{footnote}}
\renewcommand\footnoterule{\rule{\linewidth}{0.4pt}}
\renewcommand{\thempfootnote}{\arabic{mpfootnote}}

%colors
\definecolor{c1}{cmyk}{0,0.5,1,0}
\definecolor{c2}{cmyk}{1,0.25,1,0}
\definecolor{n3}{cmyk}{1,0.,1,0}
\definecolor{neg}{cmyk}{1,0.,0.,0}

% Lister med bokstavar
\usepackage[inline]{enumitem}

\newcounter{rg}
\numberwithin{rg}{chapter}
\newcommand{\reg}[2][]{\begin{tcolorbox}[boxrule=0.3 mm,arc=0mm,colback=blue!3] {\refstepcounter{rg}\phantomsection \large \textbf{\therg \;#1} \vspace{5 pt}}\newline #2  \end{tcolorbox}\vspace{-5pt}}

\newcommand{\regg}[2]{\begin{tcolorbox}[boxrule=0.3 mm,arc=0mm,colback=blue!3] {\large \textbf{#1} \vspace{5 pt}}\newline #2  \end{tcolorbox}\vspace{-5pt}}

\newcommand\alg[1]{\begin{align} #1 \end{align}}

\newcommand\eks[2][]{\begin{tcolorbox}[boxrule=0.3 mm,arc=0mm,enhanced jigsaw,breakable,colback=green!3] {\large \textbf{Eksempel #1} \vspace{5 pt}\\} #2 \end{tcolorbox}\vspace{-5pt} }

\newcommand{\st}[1]{\begin{tcolorbox}[boxrule=0.0 mm,arc=0mm,enhanced jigsaw,breakable,colback=yellow!12]{ #1} \end{tcolorbox}}

\newcommand{\spr}[1]{\begin{tcolorbox}[boxrule=0.3 mm,arc=0mm,enhanced jigsaw,breakable,colback=yellow!7] {\large \textbf{Språkboksen} \vspace{5 pt}\\} #1 \end{tcolorbox}\vspace{-5pt} }

\newcommand{\sym}[1]{\colorbox{blue!15}{#1}}

\newcommand{\info}[2]{\begin{tcolorbox}[boxrule=0.3 mm,arc=0mm,enhanced jigsaw,breakable,colback=cyan!6] {\large \textbf{#1} \vspace{5 pt}\\} #2 \end{tcolorbox}\vspace{-5pt} }

\newcommand\algv[1]{\vspace{-11 pt}\begin{align*} #1 \end{align*}}

\newcommand{\regv}{\vspace{5pt}}
\newcommand{\mer}{\textsl{Merk}: }
\newcommand\vsk{\vspace{11pt}}
\newcommand\vs{\vspace{-11pt}}
\newcommand\vsabc{\vspace{-5pt}}
\newcommand\vsb{\vspace{-16pt}}
\newcommand\sv{\vsk \textbf{Svar:} \vspace{4 pt}\\}
\newcommand\br{\\[5 pt]}
\newcommand{\fpath}[1]{../fig/#1}
\newcommand\algvv[1]{\vs\vs\begin{align*} #1 \end{align*}}
\newcommand{\y}[1]{$ {#1} $}
\newcommand{\os}{\\[5 pt]}
\newcommand{\prbxl}[2]{
	\parbox[l][][l]{#1\linewidth}{#2
}}
\newcommand{\prbxr}[2]{\parbox[r][][l]{#1\linewidth}{
		\setlength{\abovedisplayskip}{5pt}
		\setlength{\belowdisplayskip}{5pt}	
		\setlength{\abovedisplayshortskip}{0pt}
		\setlength{\belowdisplayshortskip}{0pt} 
		\begin{shaded}
			\footnotesize	#2 \end{shaded}}}

\newcommand{\fgbxr}[2]{
	\parbox[r][][l]{#1\linewidth}{#2
}}

\renewcommand{\cfttoctitlefont}{\Large\bfseries}
\setlength{\cftaftertoctitleskip}{0 pt}
\setlength{\cftbeforetoctitleskip}{0 pt}

\newcommand{\bs}{\\[3pt]}
\newcommand{\vn}{\\[6pt]}
\newcommand{\fig}[1]{\begin{figure}
		\centering
		\includegraphics[]{\fpath{#1}}
\end{figure}}


\newcommand{\sectionbreak}{\clearpage} % New page on each section

% Section comment
\newcommand{\rmerk}[1]{
\rule{\linewidth}{1pt}
#1 \\[-4pt]
\rule{\linewidth}{1pt}
}

% Equation comments
\newcommand{\cm}[1]{\llap{\color{blue} #1}}

\newcommand\fork[2]{\begin{tcolorbox}[boxrule=0.3 mm,arc=0mm,enhanced jigsaw,breakable,colback=yellow!7] {\large \textbf{#1 (forklaring)} \vspace{5 pt}\\} #2 \end{tcolorbox}\vspace{-5pt} }


%%% Rule boxes %%%
\newcommand{\gangdestihundre}{Å gonge desimaltall med 10, 100 osv.}
\newcommand{\delmedtihundre}{Deling med 10, 100, 1\,000 osv.}
\newcommand{\ompref}{Omgjering av prefiksar}


%License
\newcommand{\lic}{\textit{Anvend matematikk for grunnskule og VGS by Sindre Sogge Heggen is licensed under CC BY-NC-SA 4.0. To view a copy of this license, visit\\ 
		\net{http://creativecommons.org/licenses/by-nc-sa/4.0/}{http://creativecommons.org/licenses/by-nc-sa/4.0/}}}

%references
\newcommand{\net}[2]{{\color{blue}\href{#1}{#2}}}
\newcommand{\hrs}[2]{\hyperref[#1]{\color{blue}\textsl{#2 \ref*{#1}}}}
\newcommand{\rref}[1]{\hrs{#1}{Regel}}
\newcommand{\refkap}[1]{\hrs{#1}{Kapittel}}
\newcommand{\refsec}[1]{\hrs{#1}{Seksjon}}
\newcommand{\refdsec}[1]{\hrs{#1}{Delseksjon}}

\newcommand{\colr}[1]{{\color{red} #1}}
\newcommand{\colb}[1]{{\color{blue} #1}}
\newcommand{\colo}[1]{{\color{orange} #1}}
\newcommand{\colc}[1]{{\color{cyan} #1}}
\definecolor{projectgreen}{cmyk}{100,0,100,0}
\newcommand{\colg}[1]{{\color{projectgreen} #1}}

\newcommand{\mb}{\net{https://sindrsh.github.io/FirstPrinciplesOfMath/}{MB}}
\newcommand{\enh}[1]{\,\textrm{#1}}

\newcommand{\metode}[2]{
\textsl{#1} \\[-8pt]
\rule{#2}{0.75pt}
}

\newcommand{\linje}{\rule{\linewidth}{1pt} }

% Opg
\newcommand{\abc}[1]{
\begin{enumerate}[label=\alph*),leftmargin=18pt]
#1
\end{enumerate}
}
\newcommand{\abcs}[2]{
	\begin{enumerate}[label=\alph*),start=#1,leftmargin=18pt]
		#2
	\end{enumerate}
}

\newcommand{\abch}[1]{
	\hspace{-2pt}	\begin{enumerate*}[label=\alph*), itemjoin=\hspace{1cm}]
		#1
	\end{enumerate*}
}

\newcommand{\abchs}[2]{
	\hspace{-2pt}	\begin{enumerate*}[label=\alph*), itemjoin=\hspace{1cm}, start=#1]
		#2
	\end{enumerate*}
}

\newcommand{\abcn}[1]{
	\begin{enumerate}[label=\arabic*),leftmargin=20pt]
		#1
	\end{enumerate}
}


\newcommand{\opgt}{
\newpage
\phantomsection \addcontentsline{toc}{section}{Oppgaver} \section*{Oppgaver for kapittel \thechapter}\vs \setcounter{section}{1}}
\newcounter{opg}
\numberwithin{opg}{section}
\newcommand{\op}[1]{\vspace{15pt} \refstepcounter{opg}\large \textbf{\color{blue}\theopg} \vspace{2 pt} \label{#1} \\}
\newcommand{\oprgn}[1]{\vspace{15pt} \refstepcounter{opg}\large \textbf{\color{blue}\theopg\;(regneark)} \vspace{2 pt} \label{#1} \\}
\newcommand{\oppr}[1]{\vspace{15pt} \refstepcounter{opg}\large \textbf{\color{blue}\theopg\;(programmering)} \vspace{2 pt} \label{#1} \\}
\newcommand{\ekspop}{\vsk\textbf{Gruble \thechapter}\vspace{2 pt} \\}
\newcommand{\nes}{\stepcounter{section}
	\setcounter{opg}{0}}
\newcommand{\opr}[1]{\vspace{3pt}\textbf{\ref{#1}}}
\newcommand{\tbs}{\vspace{5pt}}

%Vedlegg
\newcounter{vedl}
\newcommand{\vedlg}[1]{\refstepcounter{vedl}\phantomsection\section*{G.\thevedl\;#1}  \addcontentsline{toc}{section}{G.\thevedl\; #1} }
\newcommand{\nreqvd}{\refstepcounter{vedleq}\tag{\thevedl \thevedleq}}

\newcounter{vedlE}
\newcommand{\vedle}[1]{\refstepcounter{vedlE}\phantomsection\section*{E.\thevedlE\;#1}  \addcontentsline{toc}{section}{E.\thevedlE\; #1} }

\newcounter{opge}
\numberwithin{opge}{part}
\newcommand{\ope}[1]{\vspace{15pt} \refstepcounter{opge}\large \textbf{\color{blue}E\theopge} \vspace{2 pt} \label{#1} \\}

%Excel og GGB:

\newcommand{\g}[1]{\begin{center} {\tt #1} \end{center}}
\newcommand{\gv}[1]{\begin{center} \vspace{-11 pt} {\tt #1}  \end{center}}
\newcommand{\cmds}[2]{{\tt #1}\\[-3pt]
	#2}


\usepackage{datetime2}
\usepackage[]{hyperref}

\begin{document}
\section{Symmetri}

\begin{figure}
	\centering
	\includegraphics[scale=0.19]{\fpath{sym}}\;
	\includegraphics[scale=0.16]{\fpath{symb}}\;
	\includegraphics[scale=0.16]{\fpath{symc}}
	\caption*{Bilder henta fra \net{https://freesvg.org}{freesvg.org}.}
\end{figure}
Mange figurar kan delast inn i minst to delar der den éine delen berre er ei forskyvd, speilvend eller rotert utgåve av den andre. Dette kallast \textit{symmetri}\index{symmetri}. Dei tre komande regelboksane definerer dei tre variantene for symmetri, men merk dette: Symmetri blir som regel intuitivt forstått ved å studere figurar, men er omstendeleg å skildre med ord. Her vil det derfor, for mange, vere ein fordel å hoppe rett til eksempla. \vsk

\reg[Translasjonssymmetri (parallellforskyvning)]{
Ein symmetri der minst to deler er forskyvde utgåver av kvarandre kallast en \textit{translasjonssymmetri}. \vsk

Når ei form forskyvast, blir kvart punkt på forma flytta langs den samme \vs \text{vektoren}\footnote{Ein vektor er eit linjestykke med retning.}.
}
\eks[1]{
Figuren under viser ein translasjonssymmetri som består av to sommerfuglar.
\begin{figure}
	\centering
	\subfloat{\includegraphics[scale=0.2]{\fpath{btfly0}}}\quad
	\subfloat{\includegraphics[scale=0.2]{\fpath{btfly0}}}
	\caption*{Bilde henta fra \net{https://freesvg.org}{freesvg.org}.}
\end{figure}
}
\newpage
\eks[2]{
Under visast $ \triangle ABC $ og ein blå vektor.
\fig{trans1a}
Under visast $ \triangle ABC $ forskyvd med den blå vektoren. 
\fig{trans1}
}
\reg[Speiling]{Ein symmetri der minst to delar er vende utgåver av kvarandre kallast ein \textit{speilingssymmetri} og har minst éin \textit{symmetrilinje} (\textit{symmetriakse}).\vsk

Når eit punkt speilast, blir det forskjyvd vinkelrett på symmetrilinja, fram til det nye og det opprinnelege punktet har samme avstand til symmetrilinja.
} 
\newpage
\eks[1]{
Sommerfuglen er ein speilsymmetri, med den raude linja som symmetrilinje.
\begin{figure}
	\centering
	\includegraphics[scale=0.3]{\fpath{btfly}}
\end{figure}
}
\eks[2]{
	Den raude linja og den blå linja er begge symmetrilinjer til det grøne rektangelet.
	\fig{sym2}
}
\eks[3]{
Under visast ei form laga av punkta $ A, B, C, D, E $ og $ F $, og denne forma speila om den blå linja.
\fig{sym3}
}
\reg[Rotasjonssymmetri]{
Ein symmetri der minst to delar er ei rotert utgåve av kvarandre kallast ein \textit{rotasjonssymmetri} og har alltid eit tilhørande \textit{rotasjonspunkt} og ein \textit{rotasjonsvinkel}. \vsk

Når eit punkt roterast vil det nye og det opprinnelege punktet
\begin{itemize}
	\item ligge langs den same sirkelbogen, som har sentrum i rotasjonspunktet. 
	\item med rotasjonspunktet som toppunkt danne rotasjonsvinkelen.
\end{itemize} 
Viss rotasjonsvinkelen er eit positivt tal, vil det nye punktet forflyttast langs sirkelbogen \textsl{mot} klokka. Hvis rotasjonsvinkelen er eit negativt tall, vil det nye punktet forflyttast langs sirkelbogen \textsl{med} klokka.
}
\eks[1]{
Mønsteret under er rotasjonssymmetrisk. Rotasjonssenteret er i midten av figuren og rotasjonsvinkelen er $ 120^\circ $
\begin{figure}
	\centering
	\includegraphics[scale=0.2]{\fpath{rot0}}
	\caption*{Bilde henta fra \net{https://freesvg.org}{freesvg.org}.}
\end{figure}
}
\newpage
\eks[2]{
Figuren under viser $ \triangle ABC $ rotert $ 80^\circ $ om rotasjonspunktet $ P $.
\fig{rot1}
Da er
\[ PA='PA \quad,\quad PB=PB'\quad,\quad PC=PC' \]
og
\[ \angle APA'=\angle BPB'=\angle CPC'=80^\circ \]
} \vsk

\spr{
Ei form som er ei forskyvd, speilvend eller rotert utgåve av ei anna form, kallast ei \textit{kongruensavbilding}.
}

\section{Omkrets, areal og volum med einigar}
\mer \textsl{I eksempla til denne seksjonen bruker vi areal- og volumformlar som du finn i \mb.} \vsk

Når vi måler lengder med linjal eller liknande, må vi passe på å ta med nemningane i svaret vårt. \regv

\eks[1]{ \vs
\begin{figure}
	\centering
	\includegraphics[scale=0.04]{\fpath{2t5}}
\end{figure}
\alg{
	\text{omkretsen til rektangelet} &= 5\enh{cm}+2\enh{cm}+5\enh{cm}+2\enh{cm} \\
	&= 14\enh{cm}
} \vs

\prbxl{0.65}{\alg{
		\text{arealet til rektangelet}&=2\enh{cm}\cdot5\enh{cm} \\
		&= 2\cdot 5\enh{cm}^2\\
		&= 10\enh{cm}^2
}}
\prbxr{0.3}{Vi skriv 'cm$ ^2 $' fordi vi har gonga sammen 2 lengder som vi har målt i 'cm'.}
}
\newpage
\eks[2]{
Ein sylinder har radius $ 4\enh{m} $ og høgde $ 2\enh{m} $. Finn volumet til sylinderen.

\sv
Så lenge vi er sikre på at størrelsane vår har same nemning (i dette tilfellet 'm'), kan vi først rekne uten størrelser:
\alg{
\text{grunnflate til sylinderen}&=\pi\cdot 4^2 \\
&= 16 \pi
}
\alg{
\text{volumet til sylinderen}&= 16\pi \cdot2 \\
&= 32\pi
}
Vi har her ganget sammen tre lengder (to faktorer lik 4\enh{m} og én faktor lik $ 2\enh{m} $) med meter som enhet, altså er volumet til sylinderen $ 32\pi\enh{m}^3 $.
} \vsk

\info{Merk}{
Når vi finn volumet til gjenstandar, måler vi gjerne lengder som høgde, breidde, radius og liknande. Desse lengdene har eininga 'meter'. Men i det daglege oppgir vi gjerne volum med eininga 'liter'. Da er det verd å ha med seg at
\[ 1\enh{L} = 1\enh{dm}^3 \]
}
\end{document}


