\documentclass[english, 11 pt, class=article, crop=false]{standalone}

% note
\newcommand{\note}{Note}
\newcommand{\notesm}[1]{{\footnotesize \textsl{\note:} #1}}
\newcommand{\selos}{See the solutions manual.}

\newcommand{\texandasy}{The text is written in \LaTeX\ and the figures are made with the aid of Asymptote.}

\newcommand{\ekstitle}{Example }
\newcommand{\sprtitle}{The language box}
\newcommand{\expl}{explanation}

%%% SECTION HEADLINES %%%

% Our numbers
\newcommand{\likteikn}{The equal sign}
\newcommand{\talsifverd}{Numbers, digits and values}
\newcommand{\koordsys}{Coordinate systems}

% Calculations
\newcommand{\adi}{Addition}
\newcommand{\sub}{Subtraction}
\newcommand{\gong}{Multiplication}
\newcommand{\del}{Division}

%Factorization and order of operations
\newcommand{\fak}{Factorization}
\newcommand{\rrek}{Order of operations}

%Fractions
\newcommand{\brgrpr}{Introduction}
\newcommand{\brvu}{Values, expanding and simplifying}
\newcommand{\bradsub}{Addition and subtraction}
\newcommand{\brgngheil}{Fractions multiplied by integers}
\newcommand{\brdelheil}{Fractions divided by integers}
\newcommand{\brgngbr}{Fractions multiplied by fractions}
\newcommand{\brkans}{Cancelation of fractions}
\newcommand{\brdelmbr}{Division by fractions}
\newcommand{\Rasjtal}{Rational numbers}

%Negative numbers
\newcommand{\negintro}{Introduction}
\newcommand{\negrekn}{The elementary operations}
\newcommand{\negmeng}{Negative numbers as amounts}

%Calculation methods
\newcommand{\delmedtihundre}{Deling med 10, 100, 1\,000 osv.}

% Geometry 1
\newcommand{\omgr}{Terms}
\newcommand{\eignsk}{Attributes of triangles and quadrilaterals}
\newcommand{\omkr}{Perimeter}
\newcommand{\area}{Area}

%Algebra 
\newcommand{\algintro}{Introduction}
\newcommand{\pot}{Powers}
\newcommand{\irrasj}{Irrational numbers}

%Equations
\newcommand{\ligintro}{Introduction}
\newcommand{\liglos}{Solving with the elementary operations}
\newcommand{\ligloso}{Solving with elementary operations summarized}

%Functions
\newcommand{\fintro}{Introduction}
\newcommand{\lingraf}{Linear functions and graphs}

%Geometry 2
\newcommand{\geoform}{Formulas of area and perimeter}
\newcommand{\kongogsim}{Congruent and similar triangles}
\newcommand{\geofork}{Explanations}

% Names of rules
\newcommand{\adkom}{Addition is commutative}
\newcommand{\gangkom}{Multiplication is commutative}
\newcommand{\brdef}{Fractions as rewriting of division}
\newcommand{\brtbr}{Fractions multiplied by fractions}
\newcommand{\delmbr}{Fractions divided by fractions}
\newcommand{\gangpar}{Distributive law}
\newcommand{\gangparsam}{Paranthesis multiplied together}
\newcommand{\gangmnegto}{Multiplication by negative numbers I}
\newcommand{\gangmnegtre}{Multiplication by negative numbers II}
\newcommand{\konsttre}{Unique construction of triangles}
\newcommand{\kongtre}{Congruent triangles}
\newcommand{\topv}{Vertical angles}
\newcommand{\trisum}{The sum of angles in a triangle}
\newcommand{\firsum}{The sum of angles in a quadrilateral}
\newcommand{\potgang}{Multiplication by powers}
\newcommand{\potdivpot}{Division by powers}
\newcommand{\potanull}{The special case of \boldmath $a^0$}
\newcommand{\potneg}{Powers with negative exponents}
\newcommand{\potbr}{Fractions as base}
\newcommand{\faktgr}{Factors as base}
\newcommand{\potsomgrunn}{Powers as base}
\newcommand{\arsirk}{The area of a circle}
\newcommand{\artrap}{The area of a trapezoid}
\newcommand{\arpar}{The area of a parallelogram}
\newcommand{\pyt}{Pythagoras's theorem}
\newcommand{\forform}{Ratios in similar triangles}
\newcommand{\vilkform}{Terms of similar triangles}
\newcommand{\omkrsirk}{The perimeter of a circle (and the value of $ \bm \pi $)}
\newcommand{\artri}{The area of a triangle}
\newcommand{\arrekt}{The area of a rectangle}
\newcommand{\liknflyt}{Moving terms across the equal sign}
\newcommand{\funklin}{Linear functions}


\usepackage[T1]{fontenc}
%\renewcommand*\familydefault{\sfdefault} % For dyslexia-friendly text
\usepackage{lmodern} % load a font with all the characters
\usepackage{geometry}
\geometry{verbose,paperwidth=16.1 cm, paperheight=24 cm, inner=2.3cm, outer=1.8 cm, bmargin=2cm, tmargin=1.8cm}
\setlength{\parindent}{0bp}
\usepackage{import}
\usepackage[subpreambles=false]{standalone}
\usepackage{amsmath}
\usepackage{amssymb}
\usepackage{esint}
\usepackage{babel}
\usepackage{tabu}
\makeatother
\makeatletter

\usepackage{titlesec}
\usepackage{ragged2e}
\RaggedRight
\raggedbottom
\frenchspacing

% Norwegian names of figures, chapters, parts and content
\addto\captionsenglish{\renewcommand{\figurename}{Figur}}
\makeatletter
\addto\captionsenglish{\renewcommand{\chaptername}{Kapittel}}
\addto\captionsenglish{\renewcommand{\partname}{Del}}


\usepackage{graphicx}
\usepackage{float}
\usepackage{subfig}
\usepackage{placeins}
\usepackage{cancel}
\usepackage{framed}
\usepackage{wrapfig}
\usepackage[subfigure]{tocloft}
\usepackage[font=footnotesize,labelfont=sl]{caption} % Figure caption
\usepackage{bm}
\usepackage[dvipsnames, table]{xcolor}
\definecolor{shadecolor}{rgb}{0.105469, 0.613281, 1}
\colorlet{shadecolor}{Emerald!15} 
\usepackage{icomma}
\makeatother
\usepackage[many]{tcolorbox}
\usepackage{multicol}
\usepackage{stackengine}

\usepackage{esvect} %For vectors with capital letters

% For tabular
\usepackage{array}
\usepackage{multirow}
\usepackage{longtable} %breakable table

% Ligningsreferanser
\usepackage{mathtools}
\mathtoolsset{showonlyrefs}

% index
\usepackage{imakeidx}
\makeindex[title=Indeks]

%Footnote:
\usepackage[bottom, hang, flushmargin]{footmisc}
\usepackage{perpage} 
\MakePerPage{footnote}
\addtolength{\footnotesep}{2mm}
\renewcommand{\thefootnote}{\arabic{footnote}}
\renewcommand\footnoterule{\rule{\linewidth}{0.4pt}}
\renewcommand{\thempfootnote}{\arabic{mpfootnote}}

%colors
\definecolor{c1}{cmyk}{0,0.5,1,0}
\definecolor{c2}{cmyk}{1,0.25,1,0}
\definecolor{n3}{cmyk}{1,0.,1,0}
\definecolor{neg}{cmyk}{1,0.,0.,0}

% Lister med bokstavar
\usepackage[inline]{enumitem}

\newcounter{rg}
\numberwithin{rg}{chapter}
\newcommand{\reg}[2][]{\begin{tcolorbox}[boxrule=0.3 mm,arc=0mm,colback=blue!3] {\refstepcounter{rg}\phantomsection \large \textbf{\therg \;#1} \vspace{5 pt}}\newline #2  \end{tcolorbox}\vspace{-5pt}}

\newcommand\alg[1]{\begin{align} #1 \end{align}}

\newcommand\eks[2][]{\begin{tcolorbox}[boxrule=0.3 mm,arc=0mm,enhanced jigsaw,breakable,colback=green!3] {\large \textbf{Eksempel #1} \vspace{5 pt}\\} #2 \end{tcolorbox}\vspace{-5pt} }

\newcommand{\st}[1]{\begin{tcolorbox}[boxrule=0.0 mm,arc=0mm,enhanced jigsaw,breakable,colback=yellow!12]{ #1} \end{tcolorbox}}

\newcommand{\spr}[1]{\begin{tcolorbox}[boxrule=0.3 mm,arc=0mm,enhanced jigsaw,breakable,colback=yellow!7] {\large \textbf{Språkboksen} \vspace{5 pt}\\} #1 \end{tcolorbox}\vspace{-5pt} }

\newcommand{\sym}[1]{\colorbox{blue!15}{#1}}

\newcommand{\info}[2]{\begin{tcolorbox}[boxrule=0.3 mm,arc=0mm,enhanced jigsaw,breakable,colback=cyan!6] {\large \textbf{#1} \vspace{5 pt}\\} #2 \end{tcolorbox}\vspace{-5pt} }

\newcommand\algv[1]{\vspace{-11 pt}\begin{align*} #1 \end{align*}}

\newcommand{\regv}{\vspace{5pt}}
\newcommand{\mer}{\textsl{Merk}: }
\newcommand{\mers}[1]{{\footnotesize \mer #1}}
\newcommand\vsk{\vspace{11pt}}
\newcommand\vs{\vspace{-11pt}}
\newcommand\vsb{\vspace{-16pt}}
\newcommand\sv{\vsk \textbf{Svar} \vspace{4 pt}\\}
\newcommand\br{\\[5 pt]}
\newcommand{\figp}[1]{../fig/#1}
\newcommand\algvv[1]{\vs\vs\begin{align*} #1 \end{align*}}
\newcommand{\y}[1]{$ {#1} $}
\newcommand{\os}{\\[5 pt]}
\newcommand{\prbxl}[2]{
\parbox[l][][l]{#1\linewidth}{#2
	}}
\newcommand{\prbxr}[2]{\parbox[r][][l]{#1\linewidth}{
		\setlength{\abovedisplayskip}{5pt}
		\setlength{\belowdisplayskip}{5pt}	
		\setlength{\abovedisplayshortskip}{0pt}
		\setlength{\belowdisplayshortskip}{0pt} 
		\begin{shaded}
			\footnotesize	#2 \end{shaded}}}

\renewcommand{\cfttoctitlefont}{\Large\bfseries}
\setlength{\cftaftertoctitleskip}{0 pt}
\setlength{\cftbeforetoctitleskip}{0 pt}

\newcommand{\bs}{\\[3pt]}
\newcommand{\vn}{\\[6pt]}
\newcommand{\fig}[1]{\begin{figure}
		\centering
		\includegraphics[]{\figp{#1}}
\end{figure}}

\newcommand{\figc}[2]{\begin{figure}
		\centering
		\includegraphics[]{\figp{#1}}
		\caption{#2}
\end{figure}}

\newcommand{\sectionbreak}{\clearpage} % New page on each section

\newcommand{\nn}[1]{
\begin{equation}
	#1
\end{equation}
}

% Equation comments
\newcommand{\cm}[1]{\llap{\color{blue} #1}}

\newcommand\fork[2]{\begin{tcolorbox}[boxrule=0.3 mm,arc=0mm,enhanced jigsaw,breakable,colback=yellow!7] {\large \textbf{#1 (forklaring)} \vspace{5 pt}\\} #2 \end{tcolorbox}\vspace{-5pt} }
 
%colors
\newcommand{\colr}[1]{{\color{red} #1}}
\newcommand{\colb}[1]{{\color{blue} #1}}
\newcommand{\colo}[1]{{\color{orange} #1}}
\newcommand{\colc}[1]{{\color{cyan} #1}}
\definecolor{projectgreen}{cmyk}{100,0,100,0}
\newcommand{\colg}[1]{{\color{projectgreen} #1}}

% Methods
\newcommand{\metode}[2]{
	\textsl{#1} \\[-8pt]
	\rule{#2}{0.75pt}
}

%Opg
\newcommand{\abc}[1]{
	\begin{enumerate}[label=\alph*),leftmargin=18pt]
		#1
	\end{enumerate}
}
\newcommand{\abcs}[2]{
	\begin{enumerate}[label=\alph*),start=#1,leftmargin=18pt]
		#2
	\end{enumerate}
}
\newcommand{\abcn}[1]{
	\begin{enumerate}[label=\arabic*),leftmargin=18pt]
		#1
	\end{enumerate}
}
\newcommand{\abch}[1]{
	\hspace{-2pt}	\begin{enumerate*}[label=\alph*), itemjoin=\hspace{1cm}]
		#1
	\end{enumerate*}
}
\newcommand{\abchs}[2]{
	\hspace{-2pt}	\begin{enumerate*}[label=\alph*), itemjoin=\hspace{1cm}, start=#1]
		#2
	\end{enumerate*}
}

% Oppgaver
\newcommand{\opgt}{\phantomsection \addcontentsline{toc}{section}{Oppgaver} \section*{Oppgaver for kapittel \thechapter}\vs \setcounter{section}{1}}
\newcounter{opg}
\numberwithin{opg}{section}
\newcommand{\op}[1]{\vspace{15pt} \refstepcounter{opg}\large \textbf{\color{blue}\theopg} \vspace{2 pt} \label{#1} \\}
\newcommand{\ekspop}[1]{\vsk\textbf{Gruble \thechapter.#1}\vspace{2 pt} \\}
\newcommand{\nes}{\stepcounter{section}
	\setcounter{opg}{0}}
\newcommand{\opr}[1]{\vspace{3pt}\textbf{\ref{#1}}}
\newcommand{\oeks}[1]{\begin{tcolorbox}[boxrule=0.3 mm,arc=0mm,colback=white]
		\textit{Eksempel: } #1	  
\end{tcolorbox}}
\newcommand\opgeks[2][]{\begin{tcolorbox}[boxrule=0.1 mm,arc=0mm,enhanced jigsaw,breakable,colback=white] {\footnotesize \textbf{Eksempel #1} \\} \footnotesize #2 \end{tcolorbox}\vspace{-5pt} }
\newcommand{\rknut}{
Rekn ut.
}

%License
\newcommand{\lic}{\textit{Matematikken sine byggesteinar by Sindre Sogge Heggen is licensed under CC BY-NC-SA 4.0. To view a copy of this license, visit\\ 
		\net{http://creativecommons.org/licenses/by-nc-sa/4.0/}{http://creativecommons.org/licenses/by-nc-sa/4.0/}}}

%referances
\newcommand{\net}[2]{{\color{blue}\href{#1}{#2}}}
\newcommand{\hrs}[2]{\hyperref[#1]{\color{blue}\textsl{#2 \ref*{#1}}}}
\newcommand{\rref}[1]{\hrs{#1}{regel}}
\newcommand{\refkap}[1]{\hrs{#1}{kapittel}}
\newcommand{\refsec}[1]{\hrs{#1}{seksjon}}

\newcommand{\mb}{\net{https://sindrsh.github.io/FirstPrinciplesOfMath/}{MB}}


%line to seperate examples
\newcommand{\linje}{\rule{\linewidth}{1pt} }

\usepackage{datetime2}
%%\usepackage{sansmathfonts} for dyslexia-friendly math
\usepackage[]{hyperref}



\begin{document}
\section{Introduction}
In a \outl{undersøkelse} we collect information. This information is often words or numbers, and is called \outl{data}. A collection of data is called a \outl{data set}. \vsk

For example, say you ask two people whether they like caviar. The one answers ''yes'', the other ''no''. Then ''yes'' and ''no'' are the data (answers) you have collected, and $\{\text{''yes''}, \text{''no''}\} $ is your data set. \vsk

Roughly speaking, statistics involves two things;  \textsl{presenting} and \textsl{interpreting} data sets. For both purposes we have some terms which we will study in the following sections, helped by different examples of studies. These examples are found on page \pageref{undersok}. \vsk

There are no universal laws telling you how to present or interpret data sets, however, you should follow these two guidelines: \regv

\st{\begin{itemize}
	\item Let it always be clear exactly what you have studied, and what data you have collected.
	\item Always be aware the methods you use to interpret the data.
\end{itemize}}  \vspace{9pt}

\spr{
Persons participating in a survey where they are asked to answer questions are called \outl{respondents}.
}
\newpage
\label{undersok}
\st{
\textbf{Survey 1} \os
	10 persones tested how many seconds they could hold their breath. These were the results (in seconds):
	\[ 47\quad124\quad 61\quad 38\quad 97\quad 84\quad 101\quad79\quad 56\quad 40 \]
}  \vspace{9pt}
\st{
\textbf{Survey 2} \os
15 persons were asked how many apples they eat during a week. The answers were these:
\[ 7\quad 4\quad 5\quad 4\quad 1\quad 0\quad 6\quad 5\quad 4\quad 8\quad1\quad6\quad8\quad0\quad 14 \] 
}  \vspace{9pt}
\st{\textbf{Survey 3}\os
300 persones where asked to name their favorite animal.
\begin{itemize}
	\item 46 persons answered tiger
	\item 23 persons answered lion
	\item 17 persons answered crocodile
	\item 91 persons answered dog
	\item 72 persons answered cat
	\item 51 persons answered other animals
\end{itemize}} \vspace{9pt}
\st{
\textbf{Survey 4} \os
Mobile phones with smart-functions (app-based) came to the Norwegian market in 2009. The table\footnote{Numbers imported from \net{https://www.medienorge.uib.no/statistikk/medium/ikt/405}{medienorge.uib.no}.} below shows the total sale of mobile phones during the time period 2009-2014, and the share with and without smart-functions. The numbers express the amount of 1\,000 phones.
\begin{center}
	\begin{tabular}{l|r|r|r|r|r|r|r|r|r|r|} 
	\textbf{År} &\textbf{2009}&\textbf{2010} &\textbf{2011} &\textbf{2012} &\textbf{2013}&\textbf{2014} \\ \hline
	totalt& 2\,365 & 2\,500 &2\,250&2\,200&	2\,400&2\,100\\
	\phantom{---}wtho. sm.f. & 1\,665 & 1\,250 &790&300&240&147\\
	\phantom{---}wth. sm.f. &700&1\,250&	1\,460&	1\,900&	2\,160 &1\,953
	\end{tabular}
\end{center}
}


\section{Ways of presenting}
When presenting data sets, it should be easy to see for others what we have found. This can be accomplished by using frequency tables, bar charts, sector graphs, or line graphs.

\subsection{Frequency table}
In a \outl{frequency table} the data set are organized in a table showing the amount of times each unique answer appears. This amount is called the \outl{frequency}.\regv

\eks[- Survey 2]{ \label{frkvtbund2}	
In this survey we have two 0's, two 1's, three 4's, two 5's, two 6's, one 7, two 8's and one 14. In a frequency table we then write
\begin{center}
	\begin{tabular}{|c|c|}
		\hline
		amount of apples & frequency\\ \hline
		0 & 2 \\
		1 & 2 \\
		4 & 3 \\
		5 & 2\\
		6 & 2 \\
		7 & 1 \\
		8 & 2 \\
		14& 1 \\ \hline
	\end{tabular}
\end{center}}
\newpage
\subsection{Søylediagram (stolpediagram)}
In a bar chart the frequencies are represented by bars. \\[8pt]
\st{ \label{soylund2}
\textbf{Survey 2}
\fig{stat1}
} \vsk

\st{
\textbf{Survey 3}
\fig{stat2}
}
\newpage
\subsection{Sektordiagram (kakediagram)}
In a sector graph the frequencies are represented by sectors in a circle. \regv
\st{  \label{sektorund2}
\textbf{Survey 2}
\fig{stat4}
} \vsk

\st{ \label{sektorundsk3}
\textbf{Survey 3}	
	\fig{stat3}
} \vsk

\info{Making a sector graph from scratch
}{
There are a lot of software available that generates sector graphs. However, if you were to make one from scratch, you will need basic knowledge of angles and fraction shares (see \mb).}
\newpage
\subsection{Linjediagram}
In a line graph the data is represented as points in a coordinate system, with lines drawn between the points. Line graphs is typically used for describing evolving data.
\regv
\st{
\textbf{Survey 4} \\
\fig{stat5}
}
\section{Interpretation; central tendencies}
In a data set there are often answers which are totally or approximately equal, and which reoccur. This means we can tell something about things that apply to the many; a \outl{central tendeny}. The most common measures of central tendencies are the mode, the mean and the median.
\subsection{Mode}
\reg[Mode]{
The \outl{mode} is the value occurring the most in the data set.
} \regv
\eks[- Survey 2]{
In this dataset 4 has the highest occurrence (three), so 4 is the mode.
} \regv

\info{Multiple modes}{
If multiple values have the highest occurance in the data set, the data set has multiple modes.
}
\newpage
\subsection{Mean}
When a data set includes numbers, we can find the sum of their values. When raising the question what the \outl{mean} is, we ask this: \os
\textsl{''If all the numbers had equal value, and the sum were still the same, what would the value be?''}\os

The question is answered by the aid of division\footnote{See \mb, side 23.}: \regv

\reg[Gjennomsnitt \label{mean}]{ \vs
\[ \text{mean}=\frac{\text{sum of the values of the data set}}{\text{amount of values}} \]
} \regv
\spr{
The mean, as defined here, is also called the \outl{average}. Also, there are multiple types of means. More specifically, the mean from \rref{mean} is  called the \outl{arithmetic mean}.
}\vsk 

\eks[- Survey 1]{
We sum the values from the data set, and divide by the amount of values:
\small
\alg{
\text{mean}&= \frac{47+124+ 61+ 38+ 97+ 84+ 101+79+ 56+ 40}{10} \\
&= \frac{727}{10}\br
&=72.7
} \normalsize
Hence, the 10 participators held their breath for 72.7 seconds on average. 
}
\newpage
\eks[- Survey 2]{  \label{gjsnund2}
\metode{Method 1}{0.4\linewidth}\\
\footnotesize
\alg{
\text{mean}&=\frac{7+ 4+ 5+ 4+ 1+ 0+ 6+ 5+ 4+ 8+1+6+8+0+ 14}{15} \br
&=\frac{73}{15}\br
&\approx 4.87
}  \vsk

\normalsize
\metode{Method 2}{0.4\linewidth} \os
We expand our frequency table \pageref{frkvtbund2} to find the sum of the values from the data set. (we have also included the sum of the frequencies):
\begin{center}
	\begin{tabular}{|c|c|c|}
		\hline
		Amount of apples & Frequency&$ \text{amount}\cdot \text{frequency} $\\ \hline
		0 & 2 &$ 0\cdot2=\phantom{0}0 $ \\
		1 & 2 &$ 1\cdot2=\phantom{0}2 $\\
		4 & 3 &$ 4\cdot3=12 $\\
		5 & 2 &$ 5\cdot2=10 $\\
		6 & 2 &$ 6\cdot2=12 $\\
		7 & 1 &$ 7\cdot1=14 $\\
		8 & 1 &$ 8\cdot2=16 $\\
		14& 1 &$ 14\cdot1=14\phantom{0} $\\ \hline
		 \textbf{sum}&15& \qquad\quad73\\ \hline
	\end{tabular}
\end{center}
Now
\algv{
\text{mean}&=\frac{73}{15} \\
&\approx 4.87
}
Hence, on average, the 15 respondents ate 4.87 apples a week.
} \regv

\eks[- Survey 4]{
{\footnotesize (Calculations omitted. The values are rounded off to the nearest one.)}
\begin{itemize}
	\item Mean of total sale of mobiles: 2302
	\item Mean of sale of mobiles without smart-functions: 732
	\item Mean of sale of mobiles with smart-functions: 1570
\end{itemize}
} \vsk

\info{Equal distribution}{
Note that the mean is about equal distribution. If we have 4 rectangles with respective lengths 1, 6, 3 and 2, their joint length is $ 1+6+3+2=12 $.
\fig{meana}
Therefore, their mean length is $ {\frac{12}{4}=3} $. Thus, if we could reshape the rectangles so that their lengths were equal, with their joint length unchanged, they would each have length 3.
\fig{meanb}
}
\subsubsection{The average rate of change}
Say you go for a jog, and measure your speed three times. Also, say that the data set you end up with is
\[ 10\enh{m/s}\qquad10\enh{m/s} \qquad10\enh{m/s}\]
Then your \outl{average speed} was
\[ \frac{10+10+10}{3}\enh{m/s}=10\enh{m/s} \]
In other words; if your speed is the same alle the time\footnote{In other words, your speed is \textsl{constant}.}, this speed is also you average speed. Consequently, the formula for the speed from \hrs{fart}{Definition} is also the formula for the average speed. Alternatively stated, it is the formula for the average rate og change of length per time.\regv

\regdef[Average rate of change]{
If we \textsl{assume} or \textsl{hold} two quantities to be proportional, the proportionality constant from \eqref{propeq} is called the \outl{averate rate of change}.
} \regv

\eks[- Survey 4]{ \label{gjsnittperund4}
\begin{itemize}
	\item For the years 2009 and 2010, the difference of smartphones sold to the difference of years is
	\[ \frac{1\,260-700}{2010-2009}=\frac{550}{1}=550  \]
	Therefore, between 2009 and 2010 the sale of smartphones have \textsl{increased} by 550\,000 smartphones per year.
	\item For the years 2010 and 2014, the difference of smartphones sold to the difference of years is
\[ \frac{1\,953-1\,250}{2014-2010}=\frac{703}{4}=175,75  \]
	Therefore, between 2010 and 2014 the sale of smartphones have \textsl{increased} by ca. 176\,000 smartphones per year.	
\item For the years 2013 and 2014, the difference of smartphones sold to the difference of years is
\[ \frac{1\,953-2\,160}{2014-2013}=\frac{-207}{1}=-207  \]
	Therefore, between 2013 and 2014 the sale of smartphones have \textsl{decreased} by ca. 207 000 smartphones per year.		
\end{itemize}
}
\newpage
\info{The slope of the line through two points}{
Given a function $ f(x) $. In \mb\ we have seen that the slope of the line through the points $ (a, f(a)) $ og $ (b, f(b)) $ is
\[ \frac{f(b)-f(a)}{b-a} \]
\fig{lintopunkt}
Comparing this expression with the calculations made on page \pageref{gjsnittperund4}, we realize that the expressions are, in general, identical. Hence, the slope of a line through two points yields the average rate of change between the two points.
}
\subsection{Median}
\reg[Median \label{median}]{The \textit{median} is the number that ends up in the middle when the data set is arranged in an increasing order.\vsk

If the data set has an even amount of values, the median equals the mean of the two values in the middle (after the arranging).} \regv

\eks[- Survey 1]{
	We arrange the data set in an increasing order:
	\[ 38\quad40\quad47\quad 56\quad \colr{61}\quad\colr{79}\quad 84\quad97\quad 101\quad 124  \]
	The two numbers in the middle are 61 and 79. The mean of these is
	\[ \frac{61+79}{2}=70 \]
	Hence, the median is 70.
} 
\eks[- Survey 2]{
We arrange the data set in an increasing order:
\[0\quad0\quad 1\quad 1\quad 4\quad 4\quad 4\quad  \colr{5}\quad 5\quad  6\quad6\quad  7\quad8\quad8 \quad14\]
The value in the middle is 5, and thus the median is 5.
}

\eks[- Survey 4]{
{\footnotesize
(Calculations omitted. The values are rounded off to the nearest one).
}
\begin{itemize}
	\item The median for the total sale of mobiles: 2307
	\item The median for the sale of mobiles without smart-functions: 545
	\item The median for the sale of mobiles with smart-functions: 1570
\end{itemize}
}
\section{Interpretation; variations}
Often there will be large differences between collected data. The collective term for various type of differences is \outl{variation}. The most common measures of variation are variation width, quartile width, variance and standard deviation.
\subsection{Variation width}
\reg[Variasjonsbredde]{
The difference between the data with the largest value and the data with the smallest value yields the \outl{variation width}. 
} \regv
\eks[Survey 1]{
Svaret med henholdsvis høgest og lavest verdi er 124 og 38. Altså er
\[ \text{variasjonsbredde}=124-38=86 \]
} \regv
\eks[- Survey 2]{
	Svaret med henholdsvis høgest og lavest verdi er 14 og 0. Altså er
	\[ \text{variasjonsbredde}=14-0=14 \]
}\regv
\eks[- Survey 4]{
	\begin{itemize}
		\item Variasjonsbredde for mobiler totalt:
		\[ 2\,500-2\,100=400  \]
		\item Variasjonsbredde for mobiler uten smartfunksjoner:
		\[ 1\,665-147=518 \]
		\item Variasjonsbredde for mobiler med smartfunksjoner:
		\[ 2\,160-700=1460 \]
	\end{itemize}
}
\subsection{Kvartilbredde}
\reg[Kvartilbredde og øvre og nedre kvartil]{
Kvartilbredden til et datasett kan finnes på følgende måte:
\begin{enumerate}
	\item Ranger datasettet fra høgest til lavest verdi.
	\item Skill det rangerte datasettet på midten, slik at to nye sett oppstår. (Viss det er oddetalls antall verdier i datasettet, utelates medianen).
	\item Finn de respektive medianene i de to nye settene.
	\item Finn differansen mellom medianene fra punkt 3.
\end{enumerate}
Om medianene fra punkt 3: Den med høgest verdi kalles \outl{øvre kvartil} og den med lavest verdi kalles \outl{nedre kvartil}.
} \regv

\st{
\textbf{Survey 1} \os
\begin{enumerate}
	\item $ 38\quad40\quad47\quad 56\quad 61\quad79\quad 84\quad97\quad 101\quad 124 $
	\item $ \color{blue}38\quad40\quad47\quad 56\quad 61 \quad \color{red}79\quad 84\quad97\quad 101\quad 124 $
	\item Medianen i det blå settet er 47 (nedre kvartil) og medianen i det røde settet er 97 (øvre kvartil).
	\[ \color{blue}38\quad40\quad{\color{black}47}\quad 56\quad 61 \qquad\quad \color{red}79\quad 84\quad{\color{black}97}\quad 101\quad 124 \] \vs \vs
	\item $ \text{Kvartilbredde}=97-47=50 $
\end{enumerate}
}
\st{
\textbf{Survey 2}\os
\begin{enumerate}
	\item $ 0\quad0\quad 1\quad 1\quad 4\quad 4\quad 4\quad  5\quad 5\quad  6\quad6\quad  7\quad8\quad8 \quad14 $
	\item $ \color{blue}0\quad0\quad 1\quad 1\quad 4\quad 4\quad 4\quad  {\color{black}5}\quad \color{red} 5\quad  6\quad6\quad  7\quad8\quad8 \quad14 $
	\item Medianen i det blå settet er 1 (nedre kvartil) og medianen i det røde settet er 7 (øvre kvartil).
	\[ \color{blue}0\quad0\quad 1\quad {\color{black}1}\quad 4\quad 4\quad 4\qquad\quad \color{red} 5\quad  6\quad6\quad  {\color{black}7}\quad8\quad8 \quad14\] \vs \vs
	\item $ \text{Kvartilbredde}=7-1=6 $
\end{enumerate}
}
\st{
\textbf{Survey 4} \os
(Utregning utelatt)
\begin{itemize}
	\item For mobiler totalt er kvartilbredden: 200 
	\item For mobiler uten smartfunksjoner er kvartilbredden: 1010
	\item For mobiler med smartfunksjoner er kvartilbredden: 703
\end{itemize}
} \vsk

\spr{
Nedre kvartil, medianen og øvre kvartil blir også kalt henholdsvis \outl{1. kvartil}, \outl{2. kvartil} og \outl{3. kvartil}.
}
\newpage
\subsection{Avvik, varians og standardavvik}
\reg[Varians]{
Differansen mellom en verdi og gjennomsnittet i et datasett kalles \outl{avviket} til verdien. \vsk
	
Variansen til et datasett kan finnes på følgende måte:	
\begin{enumerate}
	\item Kvadrer avviket til hver verdi i datasettet, og summer disse.
	\item Divider med antall verdier i datasettet.
\end{enumerate} 

\outl{Standardavviket} er kvadratroten av variansen.
}

\eks{ \label{vareks}
Gitt datasettet 
\[\color{blue} 2\quad 5\quad 9\quad 7\quad 7 \]
Da har vi at
\[ \text{gjennomsnitt}=\frac{\colb{2+5+9+7+7}}{5}=\colr{6} \]
Og videre er
\alg{
\text{variansen}&= \frac{(\colb{2}-\colr{6})^2+(\colb{5}-\colr{6})^2+(\colb{9}-\colr{6})^2+(\colb{7}-\colr{6})^2+(\colb{7}-\colr{6})^2}{5} \\ 
&= 5
}
Da er $ \text{standardavviket}=\sqrt{5}\approx2,23 $.
} \vsk
\st{
\textbf{Survey 1} \os
(Utregning utelatt)\os
Variansen er 754,01. Standardavviket er $ \sqrt{754,01}\approx 27,46 $
}
\newpage
\st{
\textbf{Survey 2} \os
Gjennomsnittet fant vi på side \pageref{gjsnund2}. Vi utvider frekvenstabellen vår fra side \pageref{frkvtbund2}:
\begin{center}
	\renewcommand{\arraystretch}{2}
	\begin{tabular}{|c|c|c|}
		\hline
		antall epler & frekvens& frekvens $ \cdot $ kvadrert avvik \\ \hline
		0 & 2 &$2\cdot \left(0-\frac{73}{15}\right)^2 $ \\
		1 & 2 &$2\cdot \left(1-\frac{73}{15}\right)^2 $\\
		4 & 3 &$3\cdot \left(4-\frac{73}{15}\right)^2 $\\
		5 & 2&$2\cdot \left(5-\frac{73}{15}\right)^2 $\\
		6 & 2 &$2\cdot \left(6-\frac{73}{15}\right)^2 $\\
		7 & 1 &$1\cdot \left(7-\frac{73}{15}\right)^2 $\\
		8 & 2 &$2\cdot \left(8-\frac{73}{15}\right)^2 $\\
		14& 1 &$1\cdot \left(9-\frac{73}{15}\right)^2 $\\ \hline 
		sum& 15 & $ 189,7\bar{3} $ \\ \hline
	\end{tabular}
\end{center}
Altså er variansen
\[ \frac{189,7\bar{3}}{15}\approx 12,65 \]
Da er standardavviket $ \sqrt{12,65}\approx3.57 $
}
\st{
	\textbf{Survey 4} \os
	(Utregning utelatt)
	\begin{itemize}
		\item For mobiler totalt er variansen 17\,781,25 og standardavviket ca. $ 133,4 $.
		\item For mobiler uten smartfunksjoner er variansen $ 318\,848.\bar{3} $ og standardavviket ca. $ 17,87 $
		\item For mobiler med smartfunksjoner er variansen $245\,847.91\bar{6} $ og standardavviket ca. 495,83.
	\end{itemize}
}
\info{Hvorfor innebærer variansen kvadrering?}{
La oss se hva som skjer hvis vi gjentar utregningen fra \textsl{Eksempel} på side \pageref{vareks}, men uten å kvadrere:
\begin{multline}
	\frac{(\colb{2}-\colr{6})+(\colb{5}-\colr{6})+(\colb{9}-\colr{6})+(\colb{7}-\colr{6})+(\colb{7}-\colr{6})}{5}\\=\frac{\colb{2+5+9+7+7}}{5} - \colr{6}
\end{multline}
Men brøken $ \frac{\colb{2+5+9+7+7}}{5} $ er jo per definisjon gjennomsnittet til datasettet, og dermed blir uttrykket over lik 0. Dette vil gjelde for alle datasett, så i denne sammenhengen gir ikke tallet 0 noen ytterligere informasjon. Om vi derimot kvadrerer avvikene, unngår vi et uttrykk som alltid blir lik 0.
}
\end{document}

