\documentclass[english, 11 pt, class=article, crop=false]{standalone}
\usepackage[T1]{fontenc}
%\renewcommand*\familydefault{\sfdefault} % For dyslexia-friendly text
\usepackage{lmodern} % load a font with all the characters
\usepackage{geometry}
\geometry{verbose,paperwidth=16.1 cm, paperheight=24 cm, inner=2.3cm, outer=1.8 cm, bmargin=2cm, tmargin=1.8cm}
\setlength{\parindent}{0bp}
\usepackage{import}
\usepackage[subpreambles=false]{standalone}
\usepackage{amsmath}
\usepackage{amssymb}
\usepackage{esint}
\usepackage{babel}
\usepackage{tabu}
\makeatother
\makeatletter

\usepackage{titlesec}
\usepackage{ragged2e}
\RaggedRight
\raggedbottom
\frenchspacing

% Norwegian names of figures, chapters, parts and content
\addto\captionsenglish{\renewcommand{\figurename}{Figur}}
\makeatletter
\addto\captionsenglish{\renewcommand{\chaptername}{Kapittel}}
\addto\captionsenglish{\renewcommand{\partname}{Del}}


\usepackage{graphicx}
\usepackage{float}
\usepackage{subfig}
\usepackage{placeins}
\usepackage{cancel}
\usepackage{framed}
\usepackage{wrapfig}
\usepackage[subfigure]{tocloft}
\usepackage[font=footnotesize,labelfont=sl]{caption} % Figure caption
\usepackage{bm}
\usepackage[dvipsnames, table]{xcolor}
\definecolor{shadecolor}{rgb}{0.105469, 0.613281, 1}
\colorlet{shadecolor}{Emerald!15} 
\usepackage{icomma}
\makeatother
\usepackage[many]{tcolorbox}
\usepackage{multicol}
\usepackage{stackengine}

\usepackage{esvect} %For vectors with capital letters

% For tabular
\usepackage{array}
\usepackage{multirow}
\usepackage{longtable} %breakable table

% Ligningsreferanser
\usepackage{mathtools}
\mathtoolsset{showonlyrefs}

% index
\usepackage{imakeidx}
\makeindex[title=Indeks]

%Footnote:
\usepackage[bottom, hang, flushmargin]{footmisc}
\usepackage{perpage} 
\MakePerPage{footnote}
\addtolength{\footnotesep}{2mm}
\renewcommand{\thefootnote}{\arabic{footnote}}
\renewcommand\footnoterule{\rule{\linewidth}{0.4pt}}
\renewcommand{\thempfootnote}{\arabic{mpfootnote}}

%colors
\definecolor{c1}{cmyk}{0,0.5,1,0}
\definecolor{c2}{cmyk}{1,0.25,1,0}
\definecolor{n3}{cmyk}{1,0.,1,0}
\definecolor{neg}{cmyk}{1,0.,0.,0}

% Lister med bokstavar
\usepackage[inline]{enumitem}

\newcounter{rg}
\numberwithin{rg}{chapter}
\newcommand{\reg}[2][]{\begin{tcolorbox}[boxrule=0.3 mm,arc=0mm,colback=blue!3] {\refstepcounter{rg}\phantomsection \large \textbf{\therg \;#1} \vspace{5 pt}}\newline #2  \end{tcolorbox}\vspace{-5pt}}

\newcommand\alg[1]{\begin{align} #1 \end{align}}

\newcommand\eks[2][]{\begin{tcolorbox}[boxrule=0.3 mm,arc=0mm,enhanced jigsaw,breakable,colback=green!3] {\large \textbf{Eksempel #1} \vspace{5 pt}\\} #2 \end{tcolorbox}\vspace{-5pt} }

\newcommand{\st}[1]{\begin{tcolorbox}[boxrule=0.0 mm,arc=0mm,enhanced jigsaw,breakable,colback=yellow!12]{ #1} \end{tcolorbox}}

\newcommand{\spr}[1]{\begin{tcolorbox}[boxrule=0.3 mm,arc=0mm,enhanced jigsaw,breakable,colback=yellow!7] {\large \textbf{Språkboksen} \vspace{5 pt}\\} #1 \end{tcolorbox}\vspace{-5pt} }

\newcommand{\sym}[1]{\colorbox{blue!15}{#1}}

\newcommand{\info}[2]{\begin{tcolorbox}[boxrule=0.3 mm,arc=0mm,enhanced jigsaw,breakable,colback=cyan!6] {\large \textbf{#1} \vspace{5 pt}\\} #2 \end{tcolorbox}\vspace{-5pt} }

\newcommand\algv[1]{\vspace{-11 pt}\begin{align*} #1 \end{align*}}

\newcommand{\regv}{\vspace{5pt}}
\newcommand{\mer}{\textsl{Merk}: }
\newcommand{\mers}[1]{{\footnotesize \mer #1}}
\newcommand\vsk{\vspace{11pt}}
\newcommand\vs{\vspace{-11pt}}
\newcommand\vsb{\vspace{-16pt}}
\newcommand\sv{\vsk \textbf{Svar} \vspace{4 pt}\\}
\newcommand\br{\\[5 pt]}
\newcommand{\figp}[1]{../fig/#1}
\newcommand\algvv[1]{\vs\vs\begin{align*} #1 \end{align*}}
\newcommand{\y}[1]{$ {#1} $}
\newcommand{\os}{\\[5 pt]}
\newcommand{\prbxl}[2]{
\parbox[l][][l]{#1\linewidth}{#2
	}}
\newcommand{\prbxr}[2]{\parbox[r][][l]{#1\linewidth}{
		\setlength{\abovedisplayskip}{5pt}
		\setlength{\belowdisplayskip}{5pt}	
		\setlength{\abovedisplayshortskip}{0pt}
		\setlength{\belowdisplayshortskip}{0pt} 
		\begin{shaded}
			\footnotesize	#2 \end{shaded}}}

\renewcommand{\cfttoctitlefont}{\Large\bfseries}
\setlength{\cftaftertoctitleskip}{0 pt}
\setlength{\cftbeforetoctitleskip}{0 pt}

\newcommand{\bs}{\\[3pt]}
\newcommand{\vn}{\\[6pt]}
\newcommand{\fig}[1]{\begin{figure}
		\centering
		\includegraphics[]{\figp{#1}}
\end{figure}}

\newcommand{\figc}[2]{\begin{figure}
		\centering
		\includegraphics[]{\figp{#1}}
		\caption{#2}
\end{figure}}

\newcommand{\sectionbreak}{\clearpage} % New page on each section

\newcommand{\nn}[1]{
\begin{equation}
	#1
\end{equation}
}

% Equation comments
\newcommand{\cm}[1]{\llap{\color{blue} #1}}

\newcommand\fork[2]{\begin{tcolorbox}[boxrule=0.3 mm,arc=0mm,enhanced jigsaw,breakable,colback=yellow!7] {\large \textbf{#1 (forklaring)} \vspace{5 pt}\\} #2 \end{tcolorbox}\vspace{-5pt} }
 
%colors
\newcommand{\colr}[1]{{\color{red} #1}}
\newcommand{\colb}[1]{{\color{blue} #1}}
\newcommand{\colo}[1]{{\color{orange} #1}}
\newcommand{\colc}[1]{{\color{cyan} #1}}
\definecolor{projectgreen}{cmyk}{100,0,100,0}
\newcommand{\colg}[1]{{\color{projectgreen} #1}}

% Methods
\newcommand{\metode}[2]{
	\textsl{#1} \\[-8pt]
	\rule{#2}{0.75pt}
}

%Opg
\newcommand{\abc}[1]{
	\begin{enumerate}[label=\alph*),leftmargin=18pt]
		#1
	\end{enumerate}
}
\newcommand{\abcs}[2]{
	\begin{enumerate}[label=\alph*),start=#1,leftmargin=18pt]
		#2
	\end{enumerate}
}
\newcommand{\abcn}[1]{
	\begin{enumerate}[label=\arabic*),leftmargin=18pt]
		#1
	\end{enumerate}
}
\newcommand{\abch}[1]{
	\hspace{-2pt}	\begin{enumerate*}[label=\alph*), itemjoin=\hspace{1cm}]
		#1
	\end{enumerate*}
}
\newcommand{\abchs}[2]{
	\hspace{-2pt}	\begin{enumerate*}[label=\alph*), itemjoin=\hspace{1cm}, start=#1]
		#2
	\end{enumerate*}
}

% Oppgaver
\newcommand{\opgt}{\phantomsection \addcontentsline{toc}{section}{Oppgaver} \section*{Oppgaver for kapittel \thechapter}\vs \setcounter{section}{1}}
\newcounter{opg}
\numberwithin{opg}{section}
\newcommand{\op}[1]{\vspace{15pt} \refstepcounter{opg}\large \textbf{\color{blue}\theopg} \vspace{2 pt} \label{#1} \\}
\newcommand{\ekspop}[1]{\vsk\textbf{Gruble \thechapter.#1}\vspace{2 pt} \\}
\newcommand{\nes}{\stepcounter{section}
	\setcounter{opg}{0}}
\newcommand{\opr}[1]{\vspace{3pt}\textbf{\ref{#1}}}
\newcommand{\oeks}[1]{\begin{tcolorbox}[boxrule=0.3 mm,arc=0mm,colback=white]
		\textit{Eksempel: } #1	  
\end{tcolorbox}}
\newcommand\opgeks[2][]{\begin{tcolorbox}[boxrule=0.1 mm,arc=0mm,enhanced jigsaw,breakable,colback=white] {\footnotesize \textbf{Eksempel #1} \\} \footnotesize #2 \end{tcolorbox}\vspace{-5pt} }
\newcommand{\rknut}{
Rekn ut.
}

%License
\newcommand{\lic}{\textit{Matematikken sine byggesteinar by Sindre Sogge Heggen is licensed under CC BY-NC-SA 4.0. To view a copy of this license, visit\\ 
		\net{http://creativecommons.org/licenses/by-nc-sa/4.0/}{http://creativecommons.org/licenses/by-nc-sa/4.0/}}}

%referances
\newcommand{\net}[2]{{\color{blue}\href{#1}{#2}}}
\newcommand{\hrs}[2]{\hyperref[#1]{\color{blue}\textsl{#2 \ref*{#1}}}}
\newcommand{\rref}[1]{\hrs{#1}{regel}}
\newcommand{\refkap}[1]{\hrs{#1}{kapittel}}
\newcommand{\refsec}[1]{\hrs{#1}{seksjon}}

\newcommand{\mb}{\net{https://sindrsh.github.io/FirstPrinciplesOfMath/}{MB}}


%line to seperate examples
\newcommand{\linje}{\rule{\linewidth}{1pt} }

\usepackage{datetime2}
%%\usepackage{sansmathfonts} for dyslexia-friendly math
\usepackage[]{hyperref}


% note
\newcommand{\note}{Note}
\newcommand{\notesm}[1]{{\footnotesize \textsl{\note:} #1}}
\newcommand{\selos}{See the solutions manual.}

\newcommand{\texandasy}{The text is written in \LaTeX\ and the figures are made with the aid of Asymptote.}

\newcommand{\ekstitle}{Example }
\newcommand{\sprtitle}{The language box}
\newcommand{\expl}{explanation}

%%% SECTION HEADLINES %%%

% Our numbers
\newcommand{\likteikn}{The equal sign}
\newcommand{\talsifverd}{Numbers, digits and values}
\newcommand{\koordsys}{Coordinate systems}

% Calculations
\newcommand{\adi}{Addition}
\newcommand{\sub}{Subtraction}
\newcommand{\gong}{Multiplication}
\newcommand{\del}{Division}

%Factorization and order of operations
\newcommand{\fak}{Factorization}
\newcommand{\rrek}{Order of operations}

%Fractions
\newcommand{\brgrpr}{Introduction}
\newcommand{\brvu}{Values, expanding and simplifying}
\newcommand{\bradsub}{Addition and subtraction}
\newcommand{\brgngheil}{Fractions multiplied by integers}
\newcommand{\brdelheil}{Fractions divided by integers}
\newcommand{\brgngbr}{Fractions multiplied by fractions}
\newcommand{\brkans}{Cancelation of fractions}
\newcommand{\brdelmbr}{Division by fractions}
\newcommand{\Rasjtal}{Rational numbers}

%Negative numbers
\newcommand{\negintro}{Introduction}
\newcommand{\negrekn}{The elementary operations}
\newcommand{\negmeng}{Negative numbers as amounts}

%Calculation methods
\newcommand{\delmedtihundre}{Deling med 10, 100, 1\,000 osv.}

% Geometry 1
\newcommand{\omgr}{Terms}
\newcommand{\eignsk}{Attributes of triangles and quadrilaterals}
\newcommand{\omkr}{Perimeter}
\newcommand{\area}{Area}

%Algebra 
\newcommand{\algintro}{Introduction}
\newcommand{\pot}{Powers}
\newcommand{\irrasj}{Irrational numbers}

%Equations
\newcommand{\ligintro}{Introduction}
\newcommand{\liglos}{Solving with the elementary operations}
\newcommand{\ligloso}{Solving with elementary operations summarized}

%Functions
\newcommand{\fintro}{Introduction}
\newcommand{\lingraf}{Linear functions and graphs}

%Geometry 2
\newcommand{\geoform}{Formulas of area and perimeter}
\newcommand{\kongogsim}{Congruent and similar triangles}
\newcommand{\geofork}{Explanations}

% Names of rules
\newcommand{\adkom}{Addition is commutative}
\newcommand{\gangkom}{Multiplication is commutative}
\newcommand{\brdef}{Fractions as rewriting of division}
\newcommand{\brtbr}{Fractions multiplied by fractions}
\newcommand{\delmbr}{Fractions divided by fractions}
\newcommand{\gangpar}{Distributive law}
\newcommand{\gangparsam}{Paranthesis multiplied together}
\newcommand{\gangmnegto}{Multiplication by negative numbers I}
\newcommand{\gangmnegtre}{Multiplication by negative numbers II}
\newcommand{\konsttre}{Unique construction of triangles}
\newcommand{\kongtre}{Congruent triangles}
\newcommand{\topv}{Vertical angles}
\newcommand{\trisum}{The sum of angles in a triangle}
\newcommand{\firsum}{The sum of angles in a quadrilateral}
\newcommand{\potgang}{Multiplication by powers}
\newcommand{\potdivpot}{Division by powers}
\newcommand{\potanull}{The special case of \boldmath $a^0$}
\newcommand{\potneg}{Powers with negative exponents}
\newcommand{\potbr}{Fractions as base}
\newcommand{\faktgr}{Factors as base}
\newcommand{\potsomgrunn}{Powers as base}
\newcommand{\arsirk}{The area of a circle}
\newcommand{\artrap}{The area of a trapezoid}
\newcommand{\arpar}{The area of a parallelogram}
\newcommand{\pyt}{Pythagoras's theorem}
\newcommand{\forform}{Ratios in similar triangles}
\newcommand{\vilkform}{Terms of similar triangles}
\newcommand{\omkrsirk}{The perimeter of a circle (and the value of $ \bm \pi $)}
\newcommand{\artri}{The area of a triangle}
\newcommand{\arrekt}{The area of a rectangle}
\newcommand{\liknflyt}{Moving terms across the equal sign}
\newcommand{\funklin}{Linear functions}



\begin{document}

\section{Brøkdeler av helheter \label{brkdlavhel}} 
In \mb, we have seen how fractions are defined by a division of 1. In everyday use we use fractions to describe division of wholes.\vs
\begin{figure}
	\centering
	\subfloat[]{\includegraphics{\figp{br1}}}\qquad\qquad
	\subfloat[]{\includegraphics{\figp{br1a}}}\qquad \qquad
	\subfloat[]{\includegraphics{\figp{br1b}}}
\end{figure}
\begin{center}
	\begin{enumerate}[label=({\alph*})]
		\item The whole is 8 boxes. $ \frac{7}{8} $ of the boxes are blue. 
		\item The whole is a square. $ \frac{1}{4} $ of the square is red.
		\item The whole is 5 circles. $ \frac{3}{5} $ of the circles are black.
	\end{enumerate}
\end{center}
\subsection*{Fractions of numbers}
Say that the rectangle below has value $ 12 $. 
\fig{br2}
When we say ''$\frac{\colb{2}}{\colc{3}}$ of \colr{12}'', we intend to
\st{\begin{enumerate}[label=\alph*)]
	\item distribute \colr{12} into \colc{3} equal groups.
	\item find how much \colb{2} of these groups make up.
\end{enumerate}}
We have
\begin{enumerate}[label=\alph*)]
	\item $ 12 $ distributed into 3 equal groups is $ 12:3=4 $.
	\fig{br2a}
	\item 2 groups with value 4 make up $ 2\cdot4=8 $.
	\fig{br2b}
\end{enumerate}
Hence
\[ \frac{2}{3}\text{ av } 12= 8 \]
\newpage
To find $ \frac{2}{3} $ of 12, we divided 12 by 3, and multiplied the quotient by 2. This is the same as multiplying $ 12 $ with $ \frac{2}{3} $ (see \mb).\regv

\reg[The fraction of a number \label{brokdelavtall}]{
To find the fraction of a number, we multiply the fraction by the number.
\[ \frac{a}{b} \text{ av } c=\frac{a}{b}\cdot c \]
}
\eks[1]{
Find $ \frac{2}{5} $ of 15.

\sv \vsb
\[ \frac{2}{5}\text{ av } 15=\frac{2}{5}\cdot 15= 6\]
}
\eks[2]{
	Find $ \frac{7}{9} $ of $ \frac{5}{6} $.
	
	\sv \vsb
	\[ \frac{7}{9}\text{ av } \frac{5}{6}=\frac{7}{9}\cdot \frac{5}{6}= \frac{35}{54}\]
} \regv 
\spr{
Parts of a whole is also called \outl{shares}.
}
\section{Prosent} \index{prosent}
\parbox[l][][l]{0.65\linewidth}{
Fractions are excellent to describe shares because they quickly give an impression of the size. It is for example easy to see (approximately) how much $ \frac{3}{5} $ or $ \frac{7}{12} $ make up of a cake. However, sometimes it is preferable to quickly determine what share makes up the \textsl{most}, and that is easiest to decide if the fractions have the same denominator.}
\parbox[r][][l]{0.3\linewidth}{
	\begin{figure}
		\centering
		\includegraphics[scale=0.1]{\figp{kake}}
\end{figure}} \\[12pt]
When shares are stated in everyday life, they are often expressed as fractions with denominator 100. Fractions with this denominator are so frequently used that they got their own name. \regv

\reg[Percentage \label{prosenttall}]{ \vs
\[ a\% = \frac{a}{100} \]
}
\spr{
\sym{\%} is pronounced \textit{per cent}. The word origins from the latin \textit{per centum}, meaning \textit{per hundreds}.
}
\eks[1]{ \vs
\[ 43\%=\frac{43}{100} \]
}
\eks[2]{ \vs
\[ 12,7\% = \frac{12,7}{100} \]
\notesm{A bit unfamiliar maybe, but there is nothing wrong with the numerator (or the denominator) being a decimal number.}
}

\newpage
\eks[3]{
	Find the value of \os 
	\abch{
		\item 12\%
		\item 19,6\%
		\item 149\%
	}

	\sv
	{\footnotesize (See \mb\ for calculations involving division by 100.)}
	\abc{
		\item $ 12\%=\dfrac{12}{100}=0,12 $
		\item $ 19,6\%=\dfrac{19,6}{100}=0,196 $
		\item $ 149\% =\dfrac{149}{100}=1,49 $
	}
}

\eks[4]{
Write the fraction as percentage.\os
\textbf{a)} $ \dfrac{34}{100} $\\[12pt]
\textbf{b)} $ \dfrac{203}{100} $

\sv \vsk

\textbf{a)} $ \dfrac{34}{100}=34\% $\\[12pt]
\textbf{b)} $ \dfrac{203}{100}=203\% $
}

\eks[5]{
	Find 50\% of 800.
	
	\sv
	By \rref{brokdelavtall} and \rref{prosenttall},
	\[ 50\% \text{ av } 800=\frac{50}{100}\cdot 800=400 \]
}
\newpage
\eks[6]{
	Find 2\% av 7.4. 
	
	\sv \vsb 
	\[ 2\%\text{ of }7.4= \frac{2}{100}\cdot 7.4=0.148 \]
} \vsk
\info{Tip}{Dividing by 100 being straight forward, we can express percentages as decimal numbers when performing calculations. In the preceding \textsl{Example 6} we could have written the following:
\[ 2\% \text{ av } 7.4 = 0.02\cdot 7.4 =0.148 \]
}
\newpage
\subsection*{Percentage shares}
What percentage does 15 make up of 20?\vsk

15 equals $ \frac{15}{20} $ of 20, so the answer to the question becomes apparent if we expand $ \frac{15}{20} $ to a fraction with denominator 100. Since $ 20\cdot\frac{100}{20}=100 $, we expand our fraction with $ \frac{100}{20}=5 $:
\alg{
\frac{15\cdot5}{20\cdot 5}= \frac{75}{100}
}
So, 15 makes up 75\% of 20. We could have got 75 directly writing
\[ 15\cdot \frac{100}{20}=75 \]
\reg[The percentage \boldmath $ a $ makes up of $ b $ \label{proaavb}]{
	\vs
	\begin{equation}
		\text{the percentage \textit{a} makes up of \textit{b}}=a\cdot \frac{100}{b}
	\end{equation}
}
\eks[1]{
What percentage does \colb{340} make up of \colc{400}? 

\sv \vsb
\[ \colb{340} \cdot \frac{100}{\colc{400}}=85 \]
340 makes up 85\% of 400.
}
\eks[2]{
What percentage does 119 make up of 500?

\sv \vsb
\[ 119\cdot \frac{100}{500}=23,8 \]
119 makes up 23.8\% of 500.
}
\newpage
\info{Tip}{
Since multiplying by 100 is an easy task, we can omit it from out calculation. In \textsl{Example 2} above we could have written
\[ \frac{119}{500}=0.238 \]
So, 119 makes up 23.8\% of 500. (That is, we arrive at our final answer by simply moving the decimal separator two places, $ 0.238\cdot100=23.8 $.)
}


\subsection{Percentage change; increase or reduction \label{Proendring}}
\subsubsection{Increase} \label{prookning}
The phrase ''200 increased by 30\%'' implies the following:
\st{
200 added with 30\% av 200.
}
Therefore
\algv{
\text{200 increased with 30\%} &=200+200\cdot 30\% \\
&=200+60 \\
&= 260
}
We note that 200 is present in both our terms in the above equation, so, according to the distributive law\footnote{See \mb.},
\alg{
\text{200 increased by 30\%}&=200+200\cdot30\% \\
&= 200\cdot(1+30\%) \\
&= 200\cdot(100\%+30\%) \\
&=200\cdot 130\% 
}
Consequently,
\[ \text{200 økt med 30\% = 130\% av 200} \] 
\newpage
\subsubsection{Reduction} \label{proredusering}
The phrase ''Reduce 200 by 60\%'' implies the following:
\st{60\% of 200 subtracted from 200}
Thus
\algv{
\text{200 reduced by 60\%} &= 200-200\cdot 60\% \\
&= 200-120 \\
&= 80
}
Here as well we notice the presence of 200 in both terms:
\alg{
\text{200 reduced by 30\%} &= 200-200\cdot60\% \\
&= 200\cdot(1-60\%) \\
&= 200\cdot 40\%
}
Hence
\[ \text{200 reduced by 60\%}= \text{40\% of 200} \]

\subsubsection{Percentage change summary}
\reg[Percentage change]{\vs
	\begin{itemize}
		\item When a quantity is reduced by $ a $\%, we end up with $ (100\% - a\%) $ of the quantity.
		
		\item When a quantity is increased by $ a $\%, we end up with $ (100\% + a\%) $ of the quantity. 
	\end{itemize}
}
\eks[1]{ \label{vekstfakteks}
	What is \colb{210} reduced by \colr{70}\%?
	
	\sv
	$ 100\%-\colr{70}\%=\colc{30}\% $, so
	\alg{
		\colb{210}\text{ reduced by } \colr{70}\% &=\colc{30}\% \text{ of } \colb{210} \br &=\frac{\colc{30}}{100}\cdot\colb{210}\br
		&=63 
	}
}
\eks[2]{
	What is 208.9 increased by 124.5\%?
	
	\sv
	
	$ 100\%+124.5\%=224.5\% $, so
	\alg{
		208.9 \text{ increased by } 124.5 &= 224.5\% \text{ of } 208.9 \br
		&=\frac{224.5}{100}\cdot208.9
	}
}
\spr{
\textit{Discount} is an amount of money subtracted from a price when an offer is given. This is also called a \textit{cut price}. Discount is usually expressed as an amount of money or as a percentage of the price. \vsk

\net{https://www.skatteetaten.no/bedrift-og-organisasjon/avgifter/mva/slik-fungerer-mva/}{{\textit{Value added tax}}} (VAT) is a fee added to the price of merchandise and goods. Value added tax is usually expressed as a percentage of the price.
} \vsk


\eks[3]{
\vsb
\parbox[l][][l]{0.75\linewidth}{
	In a shop a shirt first cost 500\enh{kr}, but is now saled with 40\% discount. \os

	What is the new price of the shirt?}
\parbox[r][][l]{0.2\linewidth}{
	\begin{figure}
		\centering
		\includegraphics[scale=0.3]{\figp{sale}}
\end{figure}} \\[-10pt]
\sv
\mers{The currency is omitted from the calculations}\os

If we were to pay full price, we would pay 100\% of 500. But if we get 40\% discount, we only need to pay $100\%-40\%=60\%$ of 500:
	\alg{
	\text{60\% of 500}&=\frac{60}{100}\cdot500 \br
	&= 300 
}
Therefore, with the discount the shirt costs 300\enh{kr}.
}
\newpage
\eks[4]{
\parbox[l][][l]{0.485\linewidth}{
	The picture says that the price of a headset is 999.20\enh{kr} with VAT \textsl{excluded}, and 1\,249 with VAT \textsl{included}. For headsets the VAT is 25\% of the price. \os
	
	Examine whether the price with VAT included is correct.
}\quad
\parbox[r][][l]{0.55\linewidth}{
	{\vspace{4pt}
		\includegraphics[scale=0.3]{\figp{peltor}}}}
		
\sv 

\mers{The currency is omitted from the calculations}\os

When VAT is included, we must pay $ 100\%+25\% $ of 999.20: 
\alg{
	\text{125\% of 999.20}&=\frac{125}{100}\cdot999.20\br
	&= 1249
}
Hence, we have to pay 1249\enh{kr}, which is also stated in the picture.
}
\subsection{Change factor}
On page \pageref{prookning} we increased 200 by 30\%, resulting in 130\% of 200. In that case we say that the \textit{change factor} is 1.3. On page \pageref{proredusering} we reduced 200 by 60\%, resulting in 40\% of 200. In that case the \textit{change factor} is 0.40.\regv

\reg[Change factor I \label{vekstfaktordef}]{
When a quantity is changed by $ a\% $, the \outl{change factor} is the value of $ {100\% \pm a\%} $.\vsk

\sym{$ + $} is used when increasing, and \sym{$ - $} is used when reducing.
}
\newpage
\eks[1]{
A quantity is increased by 15\%. What is the change factor?

\sv
$ 100\%+15\% =115\% $, so the change factor is 1.15.
}
\eks[2]{
A quantity is reduced by 19,7\%. What is the change factor?

\sv
$ 100\%-19,7\%=80.3\% $, so the change factor is 0.803
} \vsk

Let us look back at \textsl{Example 1} on page \pageref{vekstfakteks}, where 210 was reduced by 70\%. Then the change factor is 0.3. Also,
\[ 0.3\cdot210=63 \]
Therefore, to find the value of 210 reduced by 70\%, we can multiply 210 with the change factor (explain to yourself why!). \regv

\reg[Percentage change using change factor \label{vekstfaktendr}]{ \vs
\[ \text{changed original value}=\text{change factor}\cdot \text{original value} \]	
}

\eks[1]{ 
A merchandise worth 1\,000\enh{kr} is discounted with 20\%.
\abc{
\item What is the change factor?
\item Find the new price.
}

\sv  \vs
\abc{
	\item Since there is 20\% discount, we have to pay $ 100\%-20\%= 80\% $	of the original price. Therefore, the change factor is 0.8. 

\item We have
\[ 0.8\cdot1000  = 800 \]
Hence, the new price is 800\enh{kr} .
}
}
\newpage
\eks[2]{A chocolate costs 9.80\enh{kr}, VAT excluded. The VAT on food products is 15\%.
	\abc{
\item What is the change factor when finding the price of the chocolate with VAT included?	
\item What is the price of the chocolate with VAT included?	
}
	
	\sv
\abc{
\item With 15\% VAT included, we have to pay
$ 100\%+15\%= 115\% $
of the price with VAT excluded. Thus, the change factor is 1.15.
\item
\[ 1.15\cdot 9.90=12.25 \]
The chocolate costs 12.25\enh{kr} with VAT included.
}
} \vsk
By rewriting the equation in\footnote{See \refkap{LigningerAM} regarding how to rewrite equations.}  \rref{vekstfaktendr}, we get an alternative formula for the change factor: \regv

\reg[Change factor II \label{vekstfaktfrm}]{ \vs
\[ \text{change factor}=\frac{\text{changed original value}}{\text{original value}} \]
}
\subsubsection{Finding the percentage change}
When seeking to determine a percentage change, it is important to remember it is about a percentage \textsl{of a whole}, which is the reference for the calculations. This whole is the original value. \vsk

Let us as an example say that Jakob earned 10\,000\enh{kr} in 2019, and 12\,000\enh{kr} in 2020. Then we can raise the question ''What was the percentage change of Jakob's salary from 2019 to 2020?''. \vsk

The question refer to the salary from 2019, which means that 10\,000 is our original value. Two ways of finding the percentage change of salary are whe following:
\begin{itemize}
\item Jakob's salary changed from 10\,000 to 12\,000, a change of $12\,000-10\,000= 2\,000 $. Moreover, (see \rref{proaavb})
\alg{
\text{the percentage 2\,000 makes up of 10\,000}&=2\,000\cdot\frac{100}{10\,000} \\
&=20
}
Thus, from 2019 to 2020 Jakob's salary increased by 20\%. 
\item 
We have
\alg{
\frac{12\,000}{10\,000}=1,2
}
Hence, from 2019 to 2020 the change factor of Jakob's was 1,2 (see \rref{vekstfaktendr}). This change factor corresponds to a 20\% increase (see \rref{vekstfaktordef}), which means his salary increased by 20\% during the period.
\end{itemize}
\reg[Percentage change \label{proendra}]{ \vs
\[ \text{percantage change}=\frac{\text{changed original value}-\text{original value}}{\text{original value}}\cdot100 \]
A positive/negative percentage change corresponds to a percentage increase/reduction. 
} 
\info{Comment}{
The looks of \rref{proendra} can be a bit intimidating, and is not necessarily easy to remember. However, if you have reached a deep understanding of the topics covered in \refdsec{Proendring}, you already know how to find a percentage change through a step by step approach without using  \rref{proendra}. In the following examples we will show both a step by step approach, and the use of \rref{proendra}.} 
\newpage
\eks[1]{
In 2019, a football team had 20 players. I 2020 the team had 12 players. What percentage of players in 2019 had quit in 2020?

\sv	

Firstly, we note that it is the number of players in 2019 that is our original value.\vsk

\metode{Method 1; step by step approach}{0.6\linewidth} \os
The football team went from 20 to 12 players, so $ {20-12=8} $ players quit. We have
\[ \text{the percentage 4 makes up of 20}=8\cdot\frac{100}{20}=40 \]
Hence, 40\% of the players from 2019 had quit in 2020. \vsk \vsk

\metode{Method 2; using \rref{proendra}}{0.6\linewidth} \os
We have
\alg{
\text{percentage change}&=\frac{12-20}{20}\cdot100\br
&=-\frac{8}{20}\cdot 100 \br
&=-40
}
Hence, 40\% of the players from 2019 had quit in 2020. \vsk

\mers{Players quitting involves  a \textsl{reduction} in number of players. Therefore, we expected the percentage change to be a negative number.}
} \vsk

Applying \rref{vekstfaktfrm} and \rref{proendra}, we get another formula\footnote{You are challenged to arrive at this formula in \grubr{opgbrokvisproendrb}.} for the\\ percentage change:\regv

\reg[Percentage change II \label{proendrb}]{\vs
	\[ \text{percentage change}=100\left(\text{change factor}-1\right) \]
	
}
\newpage
\eks[1]{
	In 2019 your income was 12\,000\enh{kr}, and in 2020 your income was 10\,000\enh{kr}. Find you percentage change of income, with the income of 2019 as reference.
	
	\sv
	In this case, 12\,000 is our original value. Then, by \rref{vekstfaktfrm},
\alg{
	\text{change factor}&= \frac{10\,000}{12\,000}\\
	&= 0.8
}
Hence 
\algv{
\text{percentage change} &=100(0.8-1) \\
&= 100(-0.2) \\
&= -20
}
Therefore, the income was \textsl{reduced} with 20\% in 2020 compared to the income of 2019.
}\vsk

\info{\note}{Both \rref{proendra} and \rref{proendrb} provides formulas that can be used to find percentage changes. Which of these rules to choose is just a question of preference.
} 
\subsection{Percentage point} \vspace{-20pt}
\prbxl{0.65}{In many situations, we speak of multiple quantities at the same time, and when we use the term \textit{percentage} our sentences can turn out both long and ambiguous if we speak of different original values (references). In order to help simplify such sentences, we have the term  \outl{percentage point}.}
\fgbxr{0.25}{\begin{figure}
		\centering
		\includegraphics[scale=0.3]{\figp{sunglasses}}
\end{figure}} 
\prbxl{0.65}{
Say a pair of sunglasses were first sold with a 30\% discount of the original price, and a while after that with a 80\% discount of the original price. In that case we say that the discount increased by 50 \textit{percentage points}.
} \qquad
\prbxr{0.25}{
$ 80\%-30\%=50\% $ 
} \vsk

\textsl{Why can't we say that the discount increased by 50\%?}\vsk

Say that the original price of the sunglasses were 1\,000\enh{kr}.
30\% discount of 1\,000\enh{kr} yields a cut price of 300\enh{kr}. 80\% discount of 1000\enh{kr} yields a cut price of 800\enh{kr}. However, if we increase 300 by 50\%, we get $ {300\cdot1.5=450} $, which is not equal to 800. The reason for this disagreement is that we have two different references:\regv

\st{''The discount was first 30\%, and then it increased by 50 percentage points. Then the discount was 80\%.'' \vsk

\textit{Explanation:} 'The discount' is a quantity we calculate with the original price as reference. When we say ''percentage point'', we imply that \textbf{the original price is still the reference} in calculations. When the price is 1\,000\enh{kr}, we start with a discount of $ {1\,000\enh{kr}\cdot0.3=300\enh{kr}} $. When we add 50 \textsl{percentage points} to the discount, we cut off an additional 50\% of the original price, that is $ {1\,000\enh{kr}\cdot0.5=500\enh{kr}} $. In total that is a cut price of 800\enh{kr}, which makes up $ 80\% $ of the original price.
} \regv

\st{''The discount was first 30\%, and then it increased by 50\%. Then the discount was 45\%.''\vsk

\textit{Explanation:} 'The discount'' is a quantity we calculate with the original price as a reference, but when calculating the increased discount, \textbf{the original discount is the reference}. Since the original price is 1\,000\enh{kr}, we start with a discount of 300\enh{kr}. Moreover,
\[ 300\enh{kr} \text{ increased by } 50\%=300\enh{kr}\cdot1,5=450\enh{kr} \]
and
\[ \text{the percentage 450 makes up of 1\,000}=\frac{450}{100}=45 \]
Hence, the new discount is 45\%.
} 
\newpage
In the two (yellow) foregoing textboxes, we calculated the increased discounts using the original price of the sunglasses (1\,000\enh{kr}). In general, this is not necessary:
\begin{itemize}
	\item The discount was first 30\%, and then it increased by 50 percentage points. Then the discount was
	\[ 30\%+50\%= 80\% \]
	\item The discount was first 30\%, and then it increased by 50\%. Then the discount was
	\[ 30\%\cdot 1.5 =45\% \]
\end{itemize}
\reg[Percentage points versus percentage change \label{propoeng}]{
$ a\% $ increased/reduced by $ b $ percentage points $ = a\%\pm b\% $.\vsk

$ a\% $ increased/reduced by $ b\% $ $ = $ $ a\%\cdot(1\pm b\%) $
}
\info{\note}{
The second equation in \rref{propoeng} is identical to the equation in \rref{vekstfaktendr}.
}
\eks{
	One day, 5\% of the students in a school were absent. The day after 7.5\%, of the students were absent.
\abc{
\item By how many percentage points did the absence increase?
\item By how many percentages did the absence increase?
}
	
	\sv
\abc{
\item $ {7.5\%-5\%=2.5\%} $, so the absence increased by 2.5 percentage points. \vsk

\item The question is how much the difference of the absence, that is 2.5\%, makes up of 5\%. By \rref{proaavb},	
\alg{
	\text{the percentage 2.5\% makes up of 5\%}&=2.5\%\cdot \frac{100}{5\%} \\
	&= 50
}
Hence, the absence increased by 50\%.
}
}
\info{\note}{
In the preceding \textsl{Example 1}, asking the question ''By how many percentage points did the absence increase?'' is the same as
asking the question ''How many percentage does the difference of absent students make up of the total number of students?''.
}
\newpage
\subsection{Gjentatt prosentvis endring}
What is the pattern when a quantity is changed by the same percentage multiple times? As an example, let us start with 2000 and perform a 10\% increase 3 consecutive times (see \rref{vekstfaktendr}): 
\alg{
	\text{value after 1st change}&=\quad\;\mathclap{\overbrace{2000}^{\text{original value}}}\quad\cdot1.10=2\,200	\\
	\text{value after 2nd change}&=\overbrace{2\,000\cdot1.10}^{\text{2\,200}}\cdot1,10=2\,420 \\
	\text{value after 3rd change}&=\overbrace{2\,420\cdot1.10\cdot1,10}^{\text{2\,420}}\cdot1,10=2\,662 
}
The intermediate calculations we have performed may seem a bit \\ unnecessary, but if we exploit the conventions of powers\footnote{See \mb} a pattern appears:
\algv{
	\text{value after 1st change}=2\,000\cdot1,10^1=2\,200 \\		
	\text{value after 2nd change}=2\,000\cdot1,10^2=2\,420 \\	
	\text{value after 3rd change}=2\,000\cdot1,10^3=2\,662 
}
\reg[Repeated change \label{progjen}]{\vs \vs
	\[ \text{new value}=\text{original value}\cdot \text{change factor}^{\text{number of changes}} \]
}
\eks[1]{
Find the new value when 10\,000 is increased by 20\% 6 consecutive times.

\sv
The change factor is $ 1.2 $, so
\alg{
\text{new value} &= 10\,000\cdot 1.2^6\\
&=29\,859,84 
}
}
\newpage
\eks[2]{
	Marion has bought herself a new car worth 300\,000\enh{kr}, and she expects the value of the car will decrease by 12\% each year the next four years. If so, what is the car worth in four years?
	
	\sv
	Since the annual reduction is 12\%, the change factor is 0.88. The original value is 300\,000, and the number of changes is 4:
	\[ 300\,000\cdot0.88^4\approx179\,908 \]
	Therefore, Marion expects her car to be worth approximately 179\,908\enh{kr} in four years.
}

\section{Ratio}
\prbxl{0.7}{
The \outl{ratio} between two quantities refers to the one quantity divided by the other. If we for example have 1 red circles and 5 blue circles, we say that
}\qquad
\parbox[r][][l]{0.2\linewidth}{
\fig{bolle3}
}
\st{\small \vs
	\[ \text{the ratio of the amount of red circles to the amount of blue circles}=\frac{1}{5} \]}\regv
\prbxl{0.51}{
We can also (of course) write the ratio as $ {1:5} $, with the value being
	\[ 1:5=0.2 \]}\qquad
\prbxr{0.4}{Whether we write a ratio as a fraction or as a division depends on the situation.}

\reg[Forhold]{\vs
	\[\text{ratio of \textit{a} to \textit{b}}= \frac{a}{b} \]
}
\eks[1]{
	In a class there are 10 handball players and 5 football players.
	\abc{
\item What is the ratio of the amount of handball players to the amount of football players?

\item What is the ratio of the amount of football players to the amount of handball players?
} 
\sv
\abc{
\item The ratio of the amount of handball players to the amount of football players is
\[ \frac{10}{5}=2 \]

\item The ratio of the amount of football players to the amount of handball players is
\[  \frac{5}{10}=0.5 \]
}
}

\subsection{Scale}
In \mb, we have studied similar triangles. The fact that the ratio of corresponding sides are equal are also valid for a lot of shapes, such as squares, circles, prisms, spheres etc. From this it follows that small drawings (models) can give us information about real-sized quantities. The number that holds this information is called the \outl{scale}.\regv

\reg[Scale \label{maalstk}]{ \vs
\[ \text{scale}=\frac{\text{a length in a model}}{\text{the corresponding real-sized length}
} \]
}
\eks[1]{
In a drawing of a house, a wall is 6\enh{cm}. The real size of the wall is 12\enh{m}. \os 

What is the scale of the drawing?

\sv
First we must ensure that the lengts have the same unit\footnote{See \refsec{regnmforbenvn}.}. We convert 12\enh{m} into an amount of 'cm':
\[ 12\enh{m}=1200\enh{cm} \]
Now
\algv{
\text{scale}&=\frac{6\enh{cm}}{12\enh{cm}} \br
&= \frac{6}{12}
}
Also, we should reduce the fraction
\[ \text{scale}=\frac{1}{6} \]
}
\newpage
\info{Tip}{
The scale of a map is almost always given as a fraction with numerator $ 1 $. In that case one can make the following rules:
\begin{tcolorbox}[boxrule=0.3 mm,arc=0mm,colback=white] \alg{
		\text{real-sized length}&=\text{length on map}\cdot \text{denominator of the scale} \vn
		\text{length on map}&=\frac{\text{real-sized length}}{\text{denominator of the scale}}
}
\end{tcolorbox}
}
\newpage
\eks[2]{
	The below map has scale $ {1:25\,000} $. 
	\abc{
		\item The direct route (the blue) between Helland and Vike is  10.4\enh{cm} on the map. What is the real distance between Helland and Vike?
		\item The real length of the Tresfjord bridge is approximately 1300\enh{m}. How long is the Tresfjord bridge on the map?
	}
	\fig{vikves}
	\sv
	\abc{
		\item $ \text{Real distance between Helland and Vike}=10.4\enh{cm}\cdot 25\,000 $ \\
		$ \phantom{''''Real distance between Helland and Vike}=260\,000\enh{cm} $\\
		Hence, the real distance between Helland and Vike is 2.6\enh{km}.
		\item $ \text{Map length of the Tresfjord bridge}=\frac{1\,300\enh{m}}{25\,000}=0.052 \enh{m}$\\
		Hence, the map length of the Tresfjord bridge is 5.2\enh{cm}.
	}
}
\newpage
\subsection{Mixing ratio}
In many situations we want to mix two (or more) to a fitting ratio. \\[3pt]

\prbxl{0.6}{ If you read the symbol ''2 + 5'' on a bottle of juice syrup, it means you are supposed to mix syrup and water to the ratio $ {2:5} $. So, if we pour 2\enh{dL} syrup in a can, we must add 5\enh{dL} water in order to make the juice with the correct mixing ratio.}\qquad
\prbxr{0.3}{If you mix juice syrup with water, you get juice :-)}
\vsk

Sometimes we don't care \textsl{how much} we are mixing, as long as the mixing ratio is correct. For example, we can mix two full buckets of juice syrup with five full buckets of water, knowing that our mixing ratio is correct, even if we don't know the volume of the bucket. When we only care about the mixing ratio, we use the term \outl{part}. Then we read ''2 + 5'' on the bottle of juice syrup as ''2 parts of syrup to 5 parts of water''. Hence, the juice includes $ {2+5}=7 $ parts:\vspace{3pt}
\fig{forh_eng}
Consequently, 1 part makes up $ \frac{1}{7} $ of the juice, the syrup makes up $ \frac{2}{7} $ of the juice, and the water makes up $ \frac{5}{7} $ of the juice.
\newpage
\reg[Parts in a mixing ratio]{A mix with ratio $ {a:b} $ includes $ {a+b} $ parts.
	\begin{itemize}
		\item 1 part makes up $ \frac{1}{a+b} $ of the mix.
		\item $ a $ makes up $ \frac{a}{a+b} $ of the mix.
		\item $ b $ makes up $ \frac{b}{a+b} $ of the mix.
	\end{itemize}
}
\eks[1]{
	A can with volume 21\enh{dL} is filled with a juice where the mixing ratio of syrup to water is $ {2:5} $. 
	\abc{
	\item How much water is it in the can?
	\item How much syrup is it in the can?
	}

	
	\sv
	\abc{
	\item In total, the juice includes $ {2+5=7} $ parts. Since 5 of these are water,
	\algv{
		\text{amount of water}&=\frac{5}{7}\text{ av 21\enh{dL}} \br
		&= \frac{5\cdot21}{7}\enh{dL}\br
		&= 15\enh{dL}
	}
	\item We could solve this exercise in a similar way as that of exercise a), but it is better to note that if we have 15\enh{dL} water in a total of 21\enh{dL}, then we have $ (21-15)\enh{dL}=6\enh{dL} $ syrup.
	}
	
}
\newpage
\eks[2]{
	In a paint bucket green and red paint is mixed in the ratio ${ 3:7} $, and the volume of this mix is 5\enh{L}. You wish to change te mixing ratio into $ 3:11 $.\os
	How much red paint do you have to add?
	
	\sv
	The mix includes \y{3+7=10} parts. Since there is 5\enh{L} in total,
	\alg{
		\text{1 part}&=\frac{1}{10} \text{ of 5\enh{L}} \br
		&= \frac{1\cdot5}{10}\enh{L}\br
		&= 0.5 \enh{L}
	}
	When there is 7 parts red paint but we wish 11, we must add 4 parts. So the volume of red paint needed is
	\[ 4\cdot0,5 \enh{L}=2\enh{L} \]
	We must add 2\enh{L} red paint in order to get a mixing ratio of $ {3:11} $.
}
\eks[3]{
In a juice the ratio of syrup to water is $ {3:5} $.\os
	
	How many parts syrup and/or water du you have to add in order to make the ratio $ {1:4} $?
	
	\sv
	The fraction we seek, $ \frac{1}{4} $, can be written as a fraction with the same numerator the original ratio (that is $ \frac{3}{5} $):
	\[ \frac{1}{4}=\frac{1\cdot3}{4\cdot3}=\frac{3}{12} \]
	In the original ratio there are 3 parts syrup and 5 parts water. If this is to be changed into 3 parts syrup and 12 parts water, we must add 7 parts water.
}

\newpage

\end{document}


