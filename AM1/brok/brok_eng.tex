\documentclass[english, 11 pt, class=article, crop=false]{standalone}
\usepackage[T1]{fontenc}
%\renewcommand*\familydefault{\sfdefault} % For dyslexia-friendly text
\usepackage{lmodern} % load a font with all the characters
\usepackage{geometry}
\geometry{verbose,paperwidth=16.1 cm, paperheight=24 cm, inner=2.3cm, outer=1.8 cm, bmargin=2cm, tmargin=1.8cm}
\setlength{\parindent}{0bp}
\usepackage{import}
\usepackage[subpreambles=false]{standalone}
\usepackage{amsmath}
\usepackage{amssymb}
\usepackage{esint}
\usepackage{babel}
\usepackage{tabu}
\makeatother
\makeatletter

\usepackage{titlesec}
\usepackage{ragged2e}
\RaggedRight
\raggedbottom
\frenchspacing

% Norwegian names of figures, chapters, parts and content
\addto\captionsenglish{\renewcommand{\figurename}{Figur}}
\makeatletter
\addto\captionsenglish{\renewcommand{\chaptername}{Kapittel}}
\addto\captionsenglish{\renewcommand{\partname}{Del}}


\usepackage{graphicx}
\usepackage{float}
\usepackage{subfig}
\usepackage{placeins}
\usepackage{cancel}
\usepackage{framed}
\usepackage{wrapfig}
\usepackage[subfigure]{tocloft}
\usepackage[font=footnotesize,labelfont=sl]{caption} % Figure caption
\usepackage{bm}
\usepackage[dvipsnames, table]{xcolor}
\definecolor{shadecolor}{rgb}{0.105469, 0.613281, 1}
\colorlet{shadecolor}{Emerald!15} 
\usepackage{icomma}
\makeatother
\usepackage[many]{tcolorbox}
\usepackage{multicol}
\usepackage{stackengine}

\usepackage{esvect} %For vectors with capital letters

% For tabular
\usepackage{array}
\usepackage{multirow}
\usepackage{longtable} %breakable table

% Ligningsreferanser
\usepackage{mathtools}
\mathtoolsset{showonlyrefs}

% index
\usepackage{imakeidx}
\makeindex[title=Indeks]

%Footnote:
\usepackage[bottom, hang, flushmargin]{footmisc}
\usepackage{perpage} 
\MakePerPage{footnote}
\addtolength{\footnotesep}{2mm}
\renewcommand{\thefootnote}{\arabic{footnote}}
\renewcommand\footnoterule{\rule{\linewidth}{0.4pt}}
\renewcommand{\thempfootnote}{\arabic{mpfootnote}}

%colors
\definecolor{c1}{cmyk}{0,0.5,1,0}
\definecolor{c2}{cmyk}{1,0.25,1,0}
\definecolor{n3}{cmyk}{1,0.,1,0}
\definecolor{neg}{cmyk}{1,0.,0.,0}

% Lister med bokstavar
\usepackage[inline]{enumitem}

\newcounter{rg}
\numberwithin{rg}{chapter}
\newcommand{\reg}[2][]{\begin{tcolorbox}[boxrule=0.3 mm,arc=0mm,colback=blue!3] {\refstepcounter{rg}\phantomsection \large \textbf{\therg \;#1} \vspace{5 pt}}\newline #2  \end{tcolorbox}\vspace{-5pt}}

\newcommand\alg[1]{\begin{align} #1 \end{align}}

\newcommand\eks[2][]{\begin{tcolorbox}[boxrule=0.3 mm,arc=0mm,enhanced jigsaw,breakable,colback=green!3] {\large \textbf{Eksempel #1} \vspace{5 pt}\\} #2 \end{tcolorbox}\vspace{-5pt} }

\newcommand{\st}[1]{\begin{tcolorbox}[boxrule=0.0 mm,arc=0mm,enhanced jigsaw,breakable,colback=yellow!12]{ #1} \end{tcolorbox}}

\newcommand{\spr}[1]{\begin{tcolorbox}[boxrule=0.3 mm,arc=0mm,enhanced jigsaw,breakable,colback=yellow!7] {\large \textbf{Språkboksen} \vspace{5 pt}\\} #1 \end{tcolorbox}\vspace{-5pt} }

\newcommand{\sym}[1]{\colorbox{blue!15}{#1}}

\newcommand{\info}[2]{\begin{tcolorbox}[boxrule=0.3 mm,arc=0mm,enhanced jigsaw,breakable,colback=cyan!6] {\large \textbf{#1} \vspace{5 pt}\\} #2 \end{tcolorbox}\vspace{-5pt} }

\newcommand\algv[1]{\vspace{-11 pt}\begin{align*} #1 \end{align*}}

\newcommand{\regv}{\vspace{5pt}}
\newcommand{\mer}{\textsl{Merk}: }
\newcommand{\mers}[1]{{\footnotesize \mer #1}}
\newcommand\vsk{\vspace{11pt}}
\newcommand\vs{\vspace{-11pt}}
\newcommand\vsb{\vspace{-16pt}}
\newcommand\sv{\vsk \textbf{Svar} \vspace{4 pt}\\}
\newcommand\br{\\[5 pt]}
\newcommand{\figp}[1]{../fig/#1}
\newcommand\algvv[1]{\vs\vs\begin{align*} #1 \end{align*}}
\newcommand{\y}[1]{$ {#1} $}
\newcommand{\os}{\\[5 pt]}
\newcommand{\prbxl}[2]{
\parbox[l][][l]{#1\linewidth}{#2
	}}
\newcommand{\prbxr}[2]{\parbox[r][][l]{#1\linewidth}{
		\setlength{\abovedisplayskip}{5pt}
		\setlength{\belowdisplayskip}{5pt}	
		\setlength{\abovedisplayshortskip}{0pt}
		\setlength{\belowdisplayshortskip}{0pt} 
		\begin{shaded}
			\footnotesize	#2 \end{shaded}}}

\renewcommand{\cfttoctitlefont}{\Large\bfseries}
\setlength{\cftaftertoctitleskip}{0 pt}
\setlength{\cftbeforetoctitleskip}{0 pt}

\newcommand{\bs}{\\[3pt]}
\newcommand{\vn}{\\[6pt]}
\newcommand{\fig}[1]{\begin{figure}
		\centering
		\includegraphics[]{\figp{#1}}
\end{figure}}

\newcommand{\figc}[2]{\begin{figure}
		\centering
		\includegraphics[]{\figp{#1}}
		\caption{#2}
\end{figure}}

\newcommand{\sectionbreak}{\clearpage} % New page on each section

\newcommand{\nn}[1]{
\begin{equation}
	#1
\end{equation}
}

% Equation comments
\newcommand{\cm}[1]{\llap{\color{blue} #1}}

\newcommand\fork[2]{\begin{tcolorbox}[boxrule=0.3 mm,arc=0mm,enhanced jigsaw,breakable,colback=yellow!7] {\large \textbf{#1 (forklaring)} \vspace{5 pt}\\} #2 \end{tcolorbox}\vspace{-5pt} }
 
%colors
\newcommand{\colr}[1]{{\color{red} #1}}
\newcommand{\colb}[1]{{\color{blue} #1}}
\newcommand{\colo}[1]{{\color{orange} #1}}
\newcommand{\colc}[1]{{\color{cyan} #1}}
\definecolor{projectgreen}{cmyk}{100,0,100,0}
\newcommand{\colg}[1]{{\color{projectgreen} #1}}

% Methods
\newcommand{\metode}[2]{
	\textsl{#1} \\[-8pt]
	\rule{#2}{0.75pt}
}

%Opg
\newcommand{\abc}[1]{
	\begin{enumerate}[label=\alph*),leftmargin=18pt]
		#1
	\end{enumerate}
}
\newcommand{\abcs}[2]{
	\begin{enumerate}[label=\alph*),start=#1,leftmargin=18pt]
		#2
	\end{enumerate}
}
\newcommand{\abcn}[1]{
	\begin{enumerate}[label=\arabic*),leftmargin=18pt]
		#1
	\end{enumerate}
}
\newcommand{\abch}[1]{
	\hspace{-2pt}	\begin{enumerate*}[label=\alph*), itemjoin=\hspace{1cm}]
		#1
	\end{enumerate*}
}
\newcommand{\abchs}[2]{
	\hspace{-2pt}	\begin{enumerate*}[label=\alph*), itemjoin=\hspace{1cm}, start=#1]
		#2
	\end{enumerate*}
}

% Oppgaver
\newcommand{\opgt}{\phantomsection \addcontentsline{toc}{section}{Oppgaver} \section*{Oppgaver for kapittel \thechapter}\vs \setcounter{section}{1}}
\newcounter{opg}
\numberwithin{opg}{section}
\newcommand{\op}[1]{\vspace{15pt} \refstepcounter{opg}\large \textbf{\color{blue}\theopg} \vspace{2 pt} \label{#1} \\}
\newcommand{\ekspop}[1]{\vsk\textbf{Gruble \thechapter.#1}\vspace{2 pt} \\}
\newcommand{\nes}{\stepcounter{section}
	\setcounter{opg}{0}}
\newcommand{\opr}[1]{\vspace{3pt}\textbf{\ref{#1}}}
\newcommand{\oeks}[1]{\begin{tcolorbox}[boxrule=0.3 mm,arc=0mm,colback=white]
		\textit{Eksempel: } #1	  
\end{tcolorbox}}
\newcommand\opgeks[2][]{\begin{tcolorbox}[boxrule=0.1 mm,arc=0mm,enhanced jigsaw,breakable,colback=white] {\footnotesize \textbf{Eksempel #1} \\} \footnotesize #2 \end{tcolorbox}\vspace{-5pt} }
\newcommand{\rknut}{
Rekn ut.
}

%License
\newcommand{\lic}{\textit{Matematikken sine byggesteinar by Sindre Sogge Heggen is licensed under CC BY-NC-SA 4.0. To view a copy of this license, visit\\ 
		\net{http://creativecommons.org/licenses/by-nc-sa/4.0/}{http://creativecommons.org/licenses/by-nc-sa/4.0/}}}

%referances
\newcommand{\net}[2]{{\color{blue}\href{#1}{#2}}}
\newcommand{\hrs}[2]{\hyperref[#1]{\color{blue}\textsl{#2 \ref*{#1}}}}
\newcommand{\rref}[1]{\hrs{#1}{regel}}
\newcommand{\refkap}[1]{\hrs{#1}{kapittel}}
\newcommand{\refsec}[1]{\hrs{#1}{seksjon}}

\newcommand{\mb}{\net{https://sindrsh.github.io/FirstPrinciplesOfMath/}{MB}}


%line to seperate examples
\newcommand{\linje}{\rule{\linewidth}{1pt} }

\usepackage{datetime2}
%%\usepackage{sansmathfonts} for dyslexia-friendly math
\usepackage[]{hyperref}


% note
\newcommand{\note}{Note}
\newcommand{\notesm}[1]{{\footnotesize \textsl{\note:} #1}}
\newcommand{\selos}{See the solutions manual.}

\newcommand{\texandasy}{The text is written in \LaTeX\ and the figures are made with the aid of Asymptote.}

\newcommand{\ekstitle}{Example }
\newcommand{\sprtitle}{The language box}
\newcommand{\expl}{explanation}

%%% SECTION HEADLINES %%%

% Our numbers
\newcommand{\likteikn}{The equal sign}
\newcommand{\talsifverd}{Numbers, digits and values}
\newcommand{\koordsys}{Coordinate systems}

% Calculations
\newcommand{\adi}{Addition}
\newcommand{\sub}{Subtraction}
\newcommand{\gong}{Multiplication}
\newcommand{\del}{Division}

%Factorization and order of operations
\newcommand{\fak}{Factorization}
\newcommand{\rrek}{Order of operations}

%Fractions
\newcommand{\brgrpr}{Introduction}
\newcommand{\brvu}{Values, expanding and simplifying}
\newcommand{\bradsub}{Addition and subtraction}
\newcommand{\brgngheil}{Fractions multiplied by integers}
\newcommand{\brdelheil}{Fractions divided by integers}
\newcommand{\brgngbr}{Fractions multiplied by fractions}
\newcommand{\brkans}{Cancelation of fractions}
\newcommand{\brdelmbr}{Division by fractions}
\newcommand{\Rasjtal}{Rational numbers}

%Negative numbers
\newcommand{\negintro}{Introduction}
\newcommand{\negrekn}{The elementary operations}
\newcommand{\negmeng}{Negative numbers as amounts}

%Calculation methods
\newcommand{\delmedtihundre}{Deling med 10, 100, 1\,000 osv.}

% Geometry 1
\newcommand{\omgr}{Terms}
\newcommand{\eignsk}{Attributes of triangles and quadrilaterals}
\newcommand{\omkr}{Perimeter}
\newcommand{\area}{Area}

%Algebra 
\newcommand{\algintro}{Introduction}
\newcommand{\pot}{Powers}
\newcommand{\irrasj}{Irrational numbers}

%Equations
\newcommand{\ligintro}{Introduction}
\newcommand{\liglos}{Solving with the elementary operations}
\newcommand{\ligloso}{Solving with elementary operations summarized}

%Functions
\newcommand{\fintro}{Introduction}
\newcommand{\lingraf}{Linear functions and graphs}

%Geometry 2
\newcommand{\geoform}{Formulas of area and perimeter}
\newcommand{\kongogsim}{Congruent and similar triangles}
\newcommand{\geofork}{Explanations}

% Names of rules
\newcommand{\adkom}{Addition is commutative}
\newcommand{\gangkom}{Multiplication is commutative}
\newcommand{\brdef}{Fractions as rewriting of division}
\newcommand{\brtbr}{Fractions multiplied by fractions}
\newcommand{\delmbr}{Fractions divided by fractions}
\newcommand{\gangpar}{Distributive law}
\newcommand{\gangparsam}{Paranthesis multiplied together}
\newcommand{\gangmnegto}{Multiplication by negative numbers I}
\newcommand{\gangmnegtre}{Multiplication by negative numbers II}
\newcommand{\konsttre}{Unique construction of triangles}
\newcommand{\kongtre}{Congruent triangles}
\newcommand{\topv}{Vertical angles}
\newcommand{\trisum}{The sum of angles in a triangle}
\newcommand{\firsum}{The sum of angles in a quadrilateral}
\newcommand{\potgang}{Multiplication by powers}
\newcommand{\potdivpot}{Division by powers}
\newcommand{\potanull}{The special case of \boldmath $a^0$}
\newcommand{\potneg}{Powers with negative exponents}
\newcommand{\potbr}{Fractions as base}
\newcommand{\faktgr}{Factors as base}
\newcommand{\potsomgrunn}{Powers as base}
\newcommand{\arsirk}{The area of a circle}
\newcommand{\artrap}{The area of a trapezoid}
\newcommand{\arpar}{The area of a parallelogram}
\newcommand{\pyt}{Pythagoras's theorem}
\newcommand{\forform}{Ratios in similar triangles}
\newcommand{\vilkform}{Terms of similar triangles}
\newcommand{\omkrsirk}{The perimeter of a circle (and the value of $ \bm \pi $)}
\newcommand{\artri}{The area of a triangle}
\newcommand{\arrekt}{The area of a rectangle}
\newcommand{\liknflyt}{Moving terms across the equal sign}
\newcommand{\funklin}{Linear functions}



\begin{document}

\section{Brøkdeler av helheter \label{brkdlavhel}} 
In \mb, we have seen how fractions are defined by a division of 1. In everyday use we use fractions to describe division of wholes.\vs
\begin{figure}
	\centering
	\subfloat[]{\includegraphics{\figp{br1}}}\qquad\qquad
	\subfloat[]{\includegraphics{\figp{br1a}}}\qquad \qquad
	\subfloat[]{\includegraphics{\figp{br1b}}}
\end{figure}
\begin{center}
	\begin{enumerate}[label=({\alph*})]
		\item The whole is 8 boxes. $ \frac{7}{8} $ of the boxes are blue. 
		\item The whole is a square. $ \frac{1}{4} $ of the square is red.
		\item The whole is 5 circles. $ \frac{3}{5} $ of the circles are black.
	\end{enumerate}
\end{center}
\subsection*{Fractions of numbers}
Say that the rectangle below has value $ 12 $. 
\fig{br2}
When we say ''$\frac{\colb{2}}{\colc{3}}$ of \colr{12}'', we intend to
\st{\begin{enumerate}[label=\alph*)]
	\item distribute \colr{12} into \colc{3} equal groups.
	\item find how much \colb{2} of these groups make up.
\end{enumerate}}
We have
\begin{enumerate}[label=\alph*)]
	\item $ 12 $ distributed into 3 equal groups is $ 12:3=4 $.
	\fig{br2a}
	\item 2 groups with value 4 make up $ 2\cdot4=8 $.
	\fig{br2b}
\end{enumerate}
Hence
\[ \frac{2}{3}\text{ av } 12= 8 \]
\newpage
To find $ \frac{2}{3} $ of 12, we divided 12 by 3, and multiplied the quotient by 2. This is the same as multiplying $ 12 $ with $ \frac{2}{3} $ (see \mb).\regv

\reg[The fraction of a number \label{brokdelavtall}]{
To find the fraction of a number, we multiply the fraction by the number.
\[ \frac{a}{b} \text{ av } c=\frac{a}{b}\cdot c \]
}
\eks[1]{
Find $ \frac{2}{5} $ of 15.

\sv \vsb
\[ \frac{2}{5}\text{ av } 15=\frac{2}{5}\cdot 15= 6\]
}
\eks[2]{
	Find $ \frac{7}{9} $ of $ \frac{5}{6} $.
	
	\sv \vsb
	\[ \frac{7}{9}\text{ av } \frac{5}{6}=\frac{7}{9}\cdot \frac{5}{6}= \frac{35}{54}\]
} \regv 
\spr{
Parts of a whole is also called \outl{shares}.
}
\section{Prosent} \index{prosent}
\parbox[l][][l]{0.65\linewidth}{
Fractions are excellent to describe shares because they quickly give an impression of the size. It is for example easy to see (approximately) how much $ \frac{3}{5} $ or $ \frac{7}{12} $ make up of a cake. However, sometimes it is preferable to quickly determine what share makes up the \textsl{most}, and that is easiest to decide if the fractions have the same denominator.}
\parbox[r][][l]{0.3\linewidth}{
	\begin{figure}
		\centering
		\includegraphics[scale=0.1]{\figp{kake}}
\end{figure}} \\[12pt]
When shares are stated in everyday life, they are often expressed as fractions with denominator 100. Fractions with this denominator are so frequently used that they got their own name. \regv

\reg[Percentage \label{prosenttall}]{ \vs
\[ a\% = \frac{a}{100} \]
}
\spr{
\sym{\%} is pronounced \textit{per cent}. The word origins from the latin \textit{per centum}, meaning \textit{per hundreds}.
}
\eks[1]{ \vs
\[ 43\%=\frac{43}{100} \]
}
\eks[2]{ \vs
\[ 12,7\% = \frac{12,7}{100} \]
\mers{A bit unfamiliar maybe, but there is nothing wrong with the numerator (or the denominator) being a decimal number.}
}

\newpage
\eks[3]{
	Find the value of \os 
	\abch{
		\item 12\%
		\item 19,6\%
		\item 149\%
	}

	\sv
	{\footnotesize (See \mb\ for calculations involving division by 100.)}
	\abc{
		\item $ 12\%=\dfrac{12}{100}=0,12 $
		\item $ 19,6\%=\dfrac{19,6}{100}=0,196 $
		\item $ 149\% =\dfrac{149}{100}=1,49 $
	}
}

\eks[4]{
Write the fraction as percentage.\os
\textbf{a)} $ \dfrac{34}{100} $\\[12pt]
\textbf{b)} $ \dfrac{203}{100} $

\sv \vsk

\textbf{a)} $ \dfrac{34}{100}=34\% $\\[12pt]
\textbf{b)} $ \dfrac{203}{100}=203\% $
}

\eks[5]{
	Find 50\% of 800.
	
	\sv
	By \rref{brokdelavtall} and \rref{prosenttall},
	\[ 50\% \text{ av } 800=\frac{50}{100}\cdot 800=400 \]
}
\newpage
\eks[6]{
	Find 2\% av 7.4. 
	
	\sv \vsb 
	\[ 2\%\text{ of }7.4= \frac{2}{100}\cdot 7.4=0.148 \]
} \vsk
\info{Tip}{Dividing by 100 being straight forward, we can express percentages as decimal numbers when performing calculations. In \textsl{Example 6} above we could have written the following:
\[ 2\% \text{ av } 7.4 = 0.02\cdot 7.4 =0.148 \]
}
\newpage
\subsection*{Percentage shares}
What percentage does 15 make up of 20?\vsk

15 equals $ \frac{15}{20} $ of 20, so the answer to the question becomes apparent if we expand $ \frac{15}{20} $ to a fraction with denominator 100. Since $ 20\cdot\frac{100}{20}=100 $, we expand our fraction with $ \frac{100}{20}=5 $:
\alg{
\frac{15\cdot5}{20\cdot 5}= \frac{75}{100}
}
So, 15 makes up 75\% of 20. We could have got 75 directly writing
\[ 15\cdot \frac{100}{20}=75 \]
\reg[The percentage \boldmath $ a $ makes up of $ b $ \label{proaavb}]{
	\vs
	\begin{equation}
		\text{the percentage \textit{a} makes up of \textit{b}}=a\cdot \frac{100}{b}
	\end{equation}
}
\eks[1]{
What percentage does \colb{340} make up of \colc{400}? 

\sv \vsb
\[ \colb{340} \cdot \frac{100}{\colc{400}}=85 \]
340 makes up 85\% of 400.
}
\eks[2]{
What percentage does 119 make up of 500?

\sv \vsb
\[ 119\cdot \frac{100}{500}=23,8 \]
119 makes up 23.8\% of 500.
}
\newpage
\info{Tip}{
Since multiplying by 100 is an easy task, we can omit it from out calculation. In \textsl{Example 2} above we could have written
\[ \frac{119}{500}=0.238 \]
So, 119 makes up 23.8\% of 500. (That is, we arrive at our final answer by simply moving the decimal separator two places, $ 0.238\cdot100=23.8 $.)
}


\subsection{Percentage change; increase or reduction \label{Proendring}}
\subsubsection{Økning} \label{prookning}
The phrase ''200 increased by 30\%'' implies the following:
\st{
200 added with 30\% av 200.
}
Therefore
\algv{
\text{200 increased with 30\%} &=200+200\cdot 30\% \\
&=200+60 \\
&= 260
}
We note that 200 is present in both our terms in the above equation, so, according to the distributive law\footnote{See \mb.},
\alg{
\text{200 increased by 30\%}&=200+200\cdot30\% \\
&= 200\cdot(1+30\%) \\
&= 200\cdot(100\%+30\%) \\
&=200\cdot 130\% 
}
Consequently,
\[ \text{200 økt med 30\% = 130\% av 200} \] 
\newpage
\subsubsection{Reduction} \label{proredusering}
The phrase ''Reduce 200 by 60\%'' implies the following:
\st{60\% of 200 subtracted from 200}
Thus
\algv{
\text{200 reduced by 60\%} &= 200-200\cdot 60\% \\
&= 200-120 \\
&= 80
}
Here as well we notice the presence of 200 in both terms:
\alg{
\text{200 reduced by 30\%} &= 200-200\cdot60\% \\
&= 200\cdot(1-60\%) \\
&= 200\cdot 40\%
}
Hence
\[ \text{200 reduced by 60\%}= \text{40\% of 200} \]

\subsubsection{Percentage change summary}
\reg[Percentage change]{\vs
	\begin{itemize}
		\item When a quantity is reduced by $ a $\%, we end up with $ (100\% - a\%) $ of the quantity.
		
		\item When a quantity is increased by $ a $\%, we end up with $ (100\% + a\%) $ of the quantity. 
	\end{itemize}
}
\eks[1]{ \label{vekstfakteks}
	What is \colb{210} reduced by \colr{70}\%?
	
	\sv
	$ 100\%-\colr{70}\%=\colc{30}\% $, so
	\alg{
		\colb{210}\text{ reduced by } \colr{70}\% &=\colc{30}\% \text{ of } \colb{210} \br &=\frac{\colc{30}}{100}\cdot\colb{210}\br
		&=63 
	}
}
\eks[2]{
	What is 208.9 increased by 124.5\%?
	
	\sv
	
	$ 100\%+124.5\%=224.5\% $, so
	\alg{
		208.9 \text{ increased by } 124.5 &= 224.5\% \text{ of } 208.9 \br
		&=\frac{224.5}{100}\cdot208.9
	}
}
\spr{
\textit{Discount} is an amount of money subtracted from a price when an offer is given. This is also called a \textit{cut price}. Discount is usually expressed as an amount of money or as a percentage of the price. \vsk

\net{https://www.skatteetaten.no/bedrift-og-organisasjon/avgifter/mva/slik-fungerer-mva/}{{\textit{Value added tax}}} (VAT) is a fee added to the price of merchandise and goods. Value added tax is usually expressed as a percentage of the price.
} \vsk


\eks[3]{
\vsb
\parbox[l][][l]{0.75\linewidth}{
	In a shop a shirt first cost 500\enh{kr}, but is now saled with 40\% discount. \os

	What is the new price of the shirt?}
\parbox[r][][l]{0.2\linewidth}{
	\begin{figure}
		\centering
		\includegraphics[scale=0.3]{\figp{sale}}
\end{figure}} \\[-10pt]
\sv
\mers{The currency is omitted from the calculations}\os

If we were to pay full price, we would pay 100\% of 500. But if we get 40\% discount, we only need to pay $100\%-40\%=60\%$ of 500:
	\alg{
	\text{60\% of 500}&=\frac{60}{100}\cdot500 \br
	&= 300 
}
Therefore, with the discount the shirt costs 300\enh{kr}.
}
\newpage
\eks[4]{
\parbox[l][][l]{0.485\linewidth}{
	The picture says that the price of a headset is 999.20\enh{kr} with VAT \textsl{excluded}, and 1\,249 with VAT \textsl{included}. For headsets the VAT is 25\% of the price. \os
	
	Examine whether the price with VAT included is correct.
}\quad
\parbox[r][][l]{0.55\linewidth}{
	{\vspace{4pt}
		\includegraphics[scale=0.3]{\figp{peltor}}}}
		
\sv 

\mers{The currency is omitted from the calculations}\os

When VAT is included, we must pay $ 100\%+25\% $ of 999.20: 
\alg{
	\text{125\% of 999.20}&=\frac{125}{100}\cdot999.20\br
	&= 1249
}
Hence, we have to pay 1249\enh{kr}, which is also stated in the picture.
}
\subsection{Change factor}
Onf page \pageref{prookning} we increased 200 by 30\%, resulting in 130\% of 200. In that case we say that the \textit{change factor} is 1.3. On page \pageref{proredusering} we reduced 200 by 60\%, resulting in 40\% of 200. In that case the \textit{change factor} is 0.40.\regv

\reg[Vekstfaktor I \label{vekstfaktordef}]{
When a quantity is changed by $ a\% $, the \outl{change factor} is the value of $ {100\% \pm a\%} $.\vsk

\sym{$ + $} is used when increasing, and \sym{$ - $} is used when decreasing.
}
\newpage
\eks[1]{
A quantity is increased by 15\%. What is the change factor?

\sv
$ 100\%+15\% =115\% $, so the change factor is 1.15.
}
\eks[2]{
A quantity is reduced by 19,7\%. What is the change factor?

\sv
$ 100\%-19,7\%=80,3\% $, so the change factor is 0.803
} \vsk

Let us look back at \textsl{Example 1} on page \pageref{vekstfakteks}, where 210 was reduced by 70\%. Then the change factor is 0.3. Also,
\[ 0.3\cdot210=63 \]
Therefore, to find the value of 210 reduced by 70\%, we can multiply 210 with the change factor (explain to yourself why!). \regv

\reg[Percentage change using change factor \label{vekstfaktendr}]{ \vs
\[ \text{changed original value}=\text{change factor}\cdot \text{original value} \]	
}

\eks[1]{ 
En vare verd 1\,000\enh{kr}  er rabattert med 20\%.
\abc{
\item Hva er vekstfaktoren?
\item Finn den nye prisen.
}

\sv  \vs
\abc{
	\item Siden det er 20\% rabbatt, må vi betale $ 100\%-20\%= 80\% $	av originalprisen. Vekstfaktoren er derfor 0,8. 

\item Vi har at
\[ 0,8\cdot1000  = 800 \]
Den nye prisen er altså 800\enh{kr} .
}
}
\newpage
\eks[2]{En sjokolade koster 9,80\enh{kr} , ekskludert mva. På matvarer er det 15\% mva.
	\abc{
\item Hva er vekstfaktoren?	
\item Hva koster sjokoladen inkludert mva.?	
}
	
	\sv
\abc{
\item Med 15\% i tillegg må vi betale
$ 100\%+15\%= 115\% $
av prisen eksludert mva. Vekstfaktoren er derfor 1,15.
\item
\[ 1,15\cdot 9.90=12,25 \]
Sjokoladen koster 12,25\enh{kr}  inkludert mva.
}
} \vsk
Vi kan også omksrive likningen\footnote{Se \refkap{LigningerAM} for hvordan skrive om likninger.} fra \rref{vekstfaktendr} for å få et uttrykk for vekstfaktoren: \regv

\reg[Vekstfaktor II \label{vekstfaktfrm}]{ \vs
\[ \text{vekstfaktor}=\frac{\text{endret originalverdi}}{\text{originalverdi}} \]
}
\subsubsection{Å finne den prosentvise endringen}
Når man skal finne en prosentvis endring, er det viktig å være klar over at det er snakk om prosent \textsl{av} en helhet. Denne helheten man har som utgangspunkt er den originale verdien. \vsk

La oss som et eksempel si at Jakob tjente 10\,000\enh{kr} i 2019 og 12\,000\enh{kr} i 2020. Vi kan da stille spørsmålet ''Hvor mye endret lønnen til Jakob seg med fra 2019 til 2020, i prosent?''. \vsk

Spørsmålet tar utgangspunkt i lønnen fra 2019, dette betyr at 10\,000 er vår originale verdi. To måter å finne den prosentvise endringen på er disse (vi tar ikke med 'kr' i utregningene):
\begin{itemize}
\item Lønnen til Jakob endret seg fra 10\,000 til 12\,000, en forandring på $12\,000-10\,000= 2\,000 $. Videre er (se \rref{proaavb})
\alg{
\text{antall prosent 2\,000 utgjør av 10\,000}&=2\,000\cdot\frac{100}{10\,000} \\
&=20
}
Fra 2019 til 2020 økte altså lønnen til Jakob med 20\%. 
\item 
Vi har at
\alg{
\frac{12\,000}{10\,000}=1,2
}
Fra 2019 til 2020 økte altså lønnen til Jakob med en vekstfaktor lik 1,2 (se \rref{vekstfaktendr}). Denne vekstfaktoren tilsvarer en endring lik 20\% (se \rref{vekstfaktordef}). Det betyr at lønnen økte med 20\%.
\end{itemize}
\reg[Prosentvis endring I \label{proendra}]{ \vs
\[ \text{prosentvis endring}=\frac{\text{endret originalverdi}-\text{originalverdi}}{\text{originalverdi}}\cdot100 \]
Hvis 'prosentvis endring' er et positivt/negativt tall, er det snakk om en prosentvis økning/reduksjon. 
} 
\info{Kommentar}{\rref{proendra} kan se litt voldsom ut, og er ikke nødvendigvis så lett å huske. Hvis du virkelig har forstått \refdsec{Proendring}, kan du uten å bruke \rref{proendra} finne prosentvise endringer trinnvis. I påfølgende eksempel viser vi både en trinnvis løsningsmetode og en metode ved bruk av formel.} 
\newpage
\eks[1]{
I 2019 hadde et fotballag 20 medlemmer. I 2020 hadde laget 12 medlemmer. Hvor mange prosent av medlemmene fra 2019 hadde sluttet i 2020?

\sv	

Vi starter med å merke oss at det er medlemstallet fra 2019 som er originalverdien vår.\vsk

\metode{Metode 1; trinnvis metode}{0.6\linewidth} \os
Fotballaget gikk fra å ha 20 til 12 medlemmer, altså var det $ 20-12=8 $ som sluttet. Vi har at
\[ \text{antall prosent 4 utgjør av 20}=8\cdot\frac{100}{20}=40 \]
I 2020 hadde altså 40\% av medlemmene fra 2019 sluttet. \vsk \vsk

\metode{Metode 2; formel}{0.6\linewidth} \os
Vi har at
\alg{
\text{prosentvis endring}&=\frac{12-20}{20}\cdot100\br
&=-\frac{8}{20}\cdot 100 \br
&=-40
}
I 2020 hadde altså 40\% av medlemmene fra 2019 sluttet. \vsk

{\footnotesize \mer At medlemmer slutter, innebærer en \textsl{reduksjon} i medlemstall. Vi forventet derfor at 'prosentvis endring' skulle være et negativt tall.}
} \regv

\reg[Prosentvis endring II \label{proendrb}]{\vs
	\[ \text{prosentvis endring}=100\left(\text{vekstfaktor}-1\right) \]
	
}
\newpage
\info{Merk}{\rref{proendra} og \rref{proendrb} gir begge formler som kan brukes til å finne prosentvise endringer. Her er det opp til en selv å velge hvilken man liker best.
} 
\eks[1]{
	I 2019 tjente du 12\,000\enh{kr} og i 2020 tjente du 10\,000\enh{kr}. Beskriv endringen i din inntekt, med inntekten i 2019 som utganspunkt.
	
	\sv
	Her er 12\,000 vår originalveri. Av \rref{vekstfaktfrm} har vi da at
\alg{
	\text{vekstfaktor}&= \frac{10\,000}{12\,000}\\
	&= 0,8
}
Dermed er 
\algv{
\text{prosentvis endring} &=100(0,8-1) \\
&= 100(-0,2) \\
&= -20
}
Altså er lønnen \textsl{redusert} med 20\% i 2020 sammenliknet med lønnen i 2019.
}
\subsection{Prosentpoeng} \vspace{-20pt}
\prbxl{0.65}{Ofte snakker vi om mange størrelser samtidig, og når man da bruker prosent-ordet kan setninger bli veldig lange og knotete hvis man også snakker om forskjellige originalverdier (utgangspunkt). For å forenkle setningene, har vi begrepet \textit{prosentpoeng}.}
\fgbxr{0.25}{\begin{figure}
		\centering
		\includegraphics[scale=0.3]{\figp{sunglasses}}
\end{figure}} 
\prbxl{0.65}{
Tenk at et par solbriller først ble solgt med 30\% rabatt av originalprisen, og etter det med 80\% rabatt av originalprisen. Da sier vi at rabatten har økt med 50 \textit{prosentpoeng}.
} \qquad
\prbxr{0.25}{
$ 80\%-30\%=50\% $ 
} \vsk

\textsl{Hvorfor kan vi ikke si at rabatten har økt med 50\%?}\vsk

Tenk at solbrillene hadde originalpris 1\,000\enh{kr}.
30\% rabatt på 1\,000\enh{kr} tilsvarer 300\enh{kr} i rabatt. 80\% rabatt på 1000\enh{kr} tilsvarer 800\enh{kr} i rabatt. Men hvis vi øker 300 med 50\%, får vi $ 300\cdot1,5=450 $, og det er ikke det samme som 800! Saken er at vi har to forskjellige originalverdier som utgangspunkt:\regv

\st{''Rabatten var først 30\%, så økte rabatten med 50 prosentpoeng. Da ble rabatten 80\%.'' \vsk

\textit{Forklaring:} ''Rabatten'' er en størrelse vi regner ut ifra orignalprisen til solbrillene. Når vi sier ''prosentpoeng'' viser vi til at \textbf{originalprisen fortsatt er utgangspunktet} for den kommende prosentregningen. Når prisen er 1\,000\enh{kr}, starter vi med $ {1\,000\enh{kr}\cdot0,3=300\enh{kr}} $  i rabatt. Når vi legger til 50 \textsl{prosentpoeng}, legger vi til 50\% av originalprisen, altså $ {1\,000\enh{kr}\cdot0,5=500\enh{kr}} $. Totalt blir det 800\,kr i rabatt, som utgjør $ 80\% $ av originalprisen.
}
\st{''Rabatten var først 30\%, så økte rabatten med 50\%. Da ble rabatten 45\%.''\vsk

\textit{Forklaring:} ''Rabatten'' er en størrelse vi regner ut ifra orignalprisen til solbrillene, men her viser vi til at \textbf{rabatten er utgangspunktet} for den kommende prosentregningen. Når prisen er 1\,000\,kr, starter vi med 300\,kr i rabatt. Videre er
\[ 300\enh{kr} \text{ økt med } 50\%=300\enh{kr}\cdot1,5=450\enh{kr} \]
og 
\[ \text{antall prosent 450 utgjør av 1\,000}=\frac{450}{100}=45 \]
Altså er den nye rabatten 45\%.
} \vsk

I de to (gule) tekstboksene over regnet vi ut den økte rabatten via originalprisen på solbrillene (1\,000\,kr). Dette er strengt tatt ikke nødvendig:
\begin{itemize}
	\item Rabatten var først 30\%, så økte rabatten med 50 prosentpoeng. Da ble rabatten 
	\[ 30\%+50\%= 80\% \]
	\item Rabatten var først 30\%, så økte rabatten med 50\%. Da ble rabatten
	\[ 30\%\cdot 1,5 =45\% \]
\end{itemize}
\reg[Prosentpoeng \label{propoeng}]{
$ a\% $ økt/redusert med $ b $ prosentpoeng $ = a\%\pm b\% $.\vsk

$ a\% $ økt/redusert med $ b\% $ $ = $ $ a\%\cdot(1\pm b\%) $
}
\info{Merk}{
Andre linje i \rref{propoeng} er egentlig identisk med \rref{vekstfaktendr}.
}
\eks{
	En dag var 5\% av elevene på en skole borte. Dagen etter var 7,5\% av elevene borte.
\abc{
\item Hvor mange prosentpoeng økte fraværet med?
\item Hvor mange prosent økte fraværet med?
}
	
	\sv
\abc{
\item $ {7,5\%-5\%=2,5\%} $, derfor har fraværet økt med 2,5 prosentpoeng. \vsk

\item Her må vi svare på hvor mye endringen, altså 2,5\%, utgjør av 5\%. Av \rref{proaavb} har vi at	
\alg{
	\text{antall prosent 2,5\% utgjør av 5\%}&=2,5\%\cdot \frac{100}{5\%} \\
	&= 50
}
Altså har fraværet økt med 50\%.
}
}
\info{Merk}{
Å i \textsl{Example 1} over stille spørsmålet
	''Hvor mange prosentpoeng økte fraværet med?'',
er det samme som å stille spørsmålet

	''Hvor mange prosent av det totale elevantallet økte fraværet med?''.
}
\newpage
\subsection{Gjentatt prosentvis endring}
Hva om vi foretar en prosentvis endring gjentatte ganger? La oss som et eksempel starte med 2000, og utføre 10\% økning 3 påfølgende ganger (se \rref{vekstfaktendr}): 
\algv{
	\text{verdi etter 1. endring}&=\quad\;\mathclap{\overbrace{2000}^{\text{originalverdi}}}\quad\cdot1,10=2\,200	\\
	\text{verdi etter 2. endring}&=\overbrace{2\,000\cdot1,10}^{\text{2\,200}}\cdot1,10=2\,420 \\
	\text{verdi etter 3. endring}&=\overbrace{2\,420\cdot1,10\cdot1,10}^{\text{2\,420}}\cdot1,10=2\,662 
}
Mellomregningen vi gjorde over kan kanskje virke unødvendig, men utnytter vi skrivemåten for potenser\footnote{Se \mb, s.101} kommer et mønster til syne:
\alg{
	\text{verdi etter 1. endring}=2\,000\cdot1,10^1=2\,200 \\		
	\text{verdi etter 2. endring}=2\,000\cdot1,10^2=2\,420 \\	
	\text{verdi etter 3. endring}=2\,000\cdot1,10^3=2\,662 
}
\reg[Gjentatt vekst eller nedgang \label{progjen}]{\vs
	\[ \text{ny verdi}=\text{originalverdi}\cdot \text{vekstfaktor}^{\text{antall endringer}} \]
}
\eks[1]{
Finn den nye verdien når 20\% økning utføres 6 påfølgende ganger med 10\,000 som originalverdi.

\sv
Vekstfaktoren er $ 1,2 $, og da er
\alg{
\text{ny verdi} &= 10\,000\cdot 1,2^6\\
&=29\,859,84 
}
}
\newpage
\eks[2]{
	Marion har kjøpt seg en ny bil til en verdi av 300\,000\enh{kr}, og hun forventer at verdien vil synke med 12\% hvert år de neste fire årene. Hva er bilen da verd om fire år?
	
	\sv
	Siden den årlige nedgangen er 12\%, blir vekstfaktoren 0,88. Starverdien er 300\,000, og tiden er 4:
	\[ 300\,000\cdot0,88^4\approx179\,908 \]
	Marion forventer altså at bilen er verdt ca. 179\,908\enh{kr} om fire år.
}

\section{Forhold}
\prbxl{0.7}{
Med \textit{forholdet} mellom to størrelser mener vi den ene størrelsen delt på den andre. Har vi for eksempel 1 rød kule og 5 blå kuler, sier vi at
}\qquad
\parbox[r][][l]{0.2\linewidth}{
\fig{bolle3}
}
\st{ \vs
	\[ \text{forholdet mellom antall røde kuler og antall blå kuler}=\frac{1}{5} \]}\regv
\prbxl{0.51}{
Forholdet kan vi også skrive som $ {1:5} $. Verdien til dette regnestykket er
	\[ 1:5=0,2 \]}\qquad
\prbxr{0.4}{Om vi skriver forholdet som brøk eller som delestykke vil avhenge litt av oppgavene vi skal løse.} \\
I denne sammenhengen kalles 0,2 \textit{forholdstallet}.\regv

\reg[Forhold]{\vs
	\[\text{forholdet mellom \textit{a} og \textit{b}}= \frac{a}{b} \]
	Verdien til brøken kalles forholdstallet.
}
\eks[1]{
	I en klasse er det 10 handballspillere og 5 fotballspillere.
	\abc{
\item Hva er forholdet mellom antall handballspillere og fotballspillere?

\item Hva er forholdet mellom antall fotballspillere og handballspillere?
} 
\sv
\abc{
\item Forholdet mellom antall fotballspillere og handballspillere er
\[ \frac{10}{5}=2 \]

\item Forholdet mellom antall handballspillere og fotballspillere er
\[  \frac{5}{10}=0,5 \]
}
}

\subsection{Målestokk}
I \mb\,(s.145\,-\,149) har vi sett på formlike trekantar. Prinsippet om at forholdet mellom samsvarende sider er det samme, kan utvides til å gjelde de fleste andre former, som f. eks. firkanter, sirkler, prismer, kuler osv. Dette er et fantastisk prinsipp som gjør at små tegninger eller figurer (modeller) kan gi oss informasjon om størrelsene til virkelige gjenstander.\regv

\reg[Målestokk \label{maalstk}]{
En målestokk er forholdet mellom en lengde på en modell av en gjenstand og den samsvarende lengden på den virkelige gjenstanden.
\[ \text{målestokk}=\frac{\text{en lengde i en modell}}{\text{den samsvarende lengden i virkeligheten}
} \]
}
\eks[1]{
På en tegning av et hus er en vegg 6\enh{cm}. I virkeligheten er denne veggen 12\enh{m}. \os 

Hva er målestokken på tegningen?

\sv
Først må vi passe på at lengdene har samme benevning\footnote{Se \refsec{regnmforbenvn}.}. Vi gjør om 12\enh{m} til antall cm:
\[ 12\enh{m}=1200\enh{cm} \]
Nå har vi at
\alg{
\text{målestokk}&=\frac{6\enh{cm}}{12\enh{cm}} \br
&= \frac{6}{12}
}
Vi bør også prøve å forkorte brøken så mye som mulig:
\[ \text{målestokk}=\frac{1}{6} \]
}

\info{Tips}{
Målestokk på kart er omtrent alltid gitt som en brøk med teller lik $ 1 $. Dette gjør at man kan lage seg disse reglene:
\begin{tcolorbox}[boxrule=0.3 mm,arc=0mm,colback=white] \alg{
		\text{lengde i virkelighet}&=\text{lengde på kart}\cdot \text{nevner til målestokk} \vn
		\text{lengde i virkelighet}&=\frac{\text{lengde på kart}}{\text{nevner til målestokk}}
}
\end{tcolorbox}
}
\newpage
\eks[2]{
	Kartet under har målestokk $ {1:25\,000} $. 
	\abc{
		\item Luftlinjen (den blå) mellom Helland og Vike er  10,4\enh{cm} på kartet. Hvor langt er det mellom Helland og Vike i virkeligheten?
		\item Tresfjordbrua er ca 1300 m i virkeligheten. Hvor lang er Tresfjordbrua på kartet?
	}
	\figp{vikves}
	\sv
	\abc{
		\item $ \text{Virkelig avstand mellom Helland og Vike}=10,4\enh{cm}\cdot 25\,000 $ \\
		$ \phantom{6Virkelig avstand mellom Helland og Vike}=260\,000\enh{cm} $
		\item $ \text{Lengde til Tresfjordbrua på kart}=\frac{1\,300\enh{m}}{25\,000}=0,0052 \enh{m}$
	}
}
\newpage
\subsection{Blandingsforhold}
I mange sammenhenger skal vi blande to sorter i riktig forhold. \\[3pt]

\prbxl{0.6}{ På en flaske med solbærsirup kan du for eksempel lese symbolet ''2 +5'', som betyr at man skal blande sirup og vann i forholdet $ {2:5} $. Heller vi 2\,dL sirup i en kanne, må vi fylle på med 5\,dL vann for å lage saften i rett forhold.}\qquad
\prbxr{0.3}{Blander du solbærsirup og vann, får du solbærsaft :-)}
\vsk

Noen ganger bryr vi oss ikke om \textsl{hvor mye} vi blander, så lenge forholdet er rikitig. For eksempel kan vi blande to fulle bøtter med solbærsirup med fem fulle bøtter vann, og fortsatt være sikre på at forholdet er riktig, selv om vi ikke vet hvor mange liter bøtta rommer. Når vi bare bryr oss om forholdet, bruker vi ordet \textit{del}. ''2 + 5'' på sirupflasken leser vi da som ''2 deler sirup på 5 deler vann''. Dette betyr at saften vår i alt inneholder $ {2+5}=7 $ deler:\vspace{3pt}
\fig{forh}
Dette betyr at 1 del utgjør $ \frac{1}{7} $ av blandingen, sirupen utgjør $ \frac{2}{7} $ av blandingen, og vannet utgjør $ \frac{5}{7} $ av blandingen.
\newpage
\reg[Deler i et forhold]{En blanding med forholdet $ {a:b} $ har til sammen $ {a+b} $ deler.
	\begin{itemize}
		\item 1 del utgjør $ \frac{1}{a+b} $ av blandingen.
		\item $ a $ utgjør $ \frac{a}{a+b} $ av blandingen.
		\item $ b $ utgjør $ \frac{b}{a+b} $ av blandingen.
	\end{itemize}
}
\eks[1]{
	Ei kanne som rommer 21\,dL er fylt med en saft der sirup og vann er blandet i forholdet $ {2:5} $. \os
	\textbf{a)} Hvor mye vann er det i kanna?\os
	\textbf{b)} Hvor mye sirup er det i kanna?
	
	\sv
	\textbf{a)} Til sammen består saften av $ {2+5=7} $ deler. Da 5 av disse er vann, må vi ha at
	\alg{
		\text{mengde vann}&=\frac{5}{7}\text{ av 21\,dL} \br
		&= \frac{5\cdot21}{7}\enh{dL}\br
		&= 15\enh{dL}
	}
	\textbf{b)} Vi kan løse denne oppgaven på samme måte som oppgave a), men det er raskere å merke seg at hvis vi har 15\,dL vann av i alt 21\,dL, må vi ha $ (21-15)\enh{dL}=6\enh{dL} $ sirup.
}
\newpage
\eks[2]{
	I et malerspann er grønn og rød maling blandet i forholdet ${ 3:7} $, og det er 5\,L av denne blandingen. Du ønsker å gjøre om forholdet til $ 3:11 $.\os
	Hvor mye rød maling må du helle oppi spannet?
	
	\sv
	I spannet har vi \y{3+7=10} deler. Siden det er 5\,L i alt, må vi ha at\vs
	\alg{
		\text{1 del}&=\frac{1}{10} \text{ av 5\,L} \br
		&= \frac{1\cdot5}{10} \text{\,L} \br
		&= 0,5 \enh{L}
	}
	Når vi har 7 deler rødmaling, men ønsker 11, må vi blande oppi 4 deler til. Da trenger vi
	\[ 4\cdot0,5 \enh{L}=2\enh{L} \]
	Vi må helle oppi 2\,L rødmaling for å få forholdet $ {3:11} $.
}
\eks[3]{I en ferdig blandet saft er forholdet mellom sirup og vann $ {3:5} $.\os
	
	Hvor mange deler saft og/eller vann må du legge til for at forholdet skal bli $ {1:4} $?
	
	\sv
	Brøken vi ønsker, $ \frac{1}{4} $, kan vi skrive om til en brøk med samme teller som brøken vi har (altså $ \frac{3}{5} $):
	\[ \frac{1}{4}=\frac{1\cdot3}{4\cdot3}=\frac{3}{12} \]
	I vårt opprinnelige forhold har vi 3 deler sirup og 5 deler vann. Skal dette gjøres om til 3 deler sirup og 12 deler vann, må vi legge til 7 deler vann.
}

\newpage

\end{document}


