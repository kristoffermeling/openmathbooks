\documentclass[english, 11 pt, class=article, crop=false]{standalone}
\usepackage[T1]{fontenc}
\usepackage[utf8]{luainputenc}
\usepackage{lmodern} % load a font with all the characters
\usepackage{geometry}
\geometry{verbose,paperwidth=16.1 cm, paperheight=24 cm, inner=2.3cm, outer=1.8 cm, bmargin=2cm, tmargin=1.8cm}
\setlength{\parindent}{0bp}
\usepackage{import}
\usepackage[subpreambles=false]{standalone}
\usepackage{amsmath}
\usepackage{amssymb}
\usepackage{esint}
\usepackage{babel}
\usepackage{tabu}
\makeatother
\makeatletter

\usepackage{titlesec}
\usepackage{ragged2e}
\RaggedRight
\raggedbottom
\frenchspacing

% Norwegian names of figures, chapters, parts and content
\addto\captionsenglish{\renewcommand{\figurename}{Figur}}
\makeatletter
\addto\captionsenglish{\renewcommand{\chaptername}{Kapittel}}
\addto\captionsenglish{\renewcommand{\partname}{Del}}

\addto\captionsenglish{\renewcommand{\contentsname}{Innhald}}

\usepackage{graphicx}
\usepackage{float}
\usepackage{subfig}
\usepackage{placeins}
\usepackage{cancel}
\usepackage{framed}
\usepackage{wrapfig}
\usepackage[subfigure]{tocloft}
\usepackage[font=footnotesize]{caption} % Figure caption
\usepackage{bm}
\usepackage[dvipsnames, table]{xcolor}
\definecolor{shadecolor}{rgb}{0.105469, 0.613281, 1}
\colorlet{shadecolor}{Emerald!15} 
\usepackage{icomma}
\makeatother
\usepackage[many]{tcolorbox}
\usepackage{multicol}
\usepackage{stackengine}

% For tabular
\addto\captionsenglish{\renewcommand{\tablename}{Tabell}}
\usepackage{array}
\usepackage{multirow}
\usepackage{longtable} %breakable table

% Ligningsreferanser
\usepackage{mathtools}
\mathtoolsset{showonlyrefs}

% index
\usepackage{imakeidx}
\makeindex[title=Indeks]

%Footnote:
\usepackage[bottom, hang, flushmargin]{footmisc}
\usepackage{perpage} 
\MakePerPage{footnote}
\addtolength{\footnotesep}{2mm}
\renewcommand{\thefootnote}{\arabic{footnote}}
\renewcommand\footnoterule{\rule{\linewidth}{0.4pt}}
\renewcommand{\thempfootnote}{\arabic{mpfootnote}}

%colors
\definecolor{c1}{cmyk}{0,0.5,1,0}
\definecolor{c2}{cmyk}{1,0.25,1,0}
\definecolor{n3}{cmyk}{1,0.,1,0}
\definecolor{neg}{cmyk}{1,0.,0.,0}

% Lister med bokstavar
\usepackage[inline]{enumitem}

\newcounter{rg}
\numberwithin{rg}{chapter}
\newcommand{\reg}[2][]{\begin{tcolorbox}[boxrule=0.3 mm,arc=0mm,colback=blue!3] {\refstepcounter{rg}\phantomsection \large \textbf{\therg \;#1} \vspace{5 pt}}\newline #2  \end{tcolorbox}\vspace{-5pt}}

\newcommand{\regg}[2]{\begin{tcolorbox}[boxrule=0.3 mm,arc=0mm,colback=blue!3] {\large \textbf{#1} \vspace{5 pt}}\newline #2  \end{tcolorbox}\vspace{-5pt}}

\newcommand\alg[1]{\begin{align} #1 \end{align}}

\newcommand\eks[2][]{\begin{tcolorbox}[boxrule=0.3 mm,arc=0mm,enhanced jigsaw,breakable,colback=green!3] {\large \textbf{Eksempel #1} \vspace{5 pt}\\} #2 \end{tcolorbox}\vspace{-5pt} }

\newcommand{\st}[1]{\begin{tcolorbox}[boxrule=0.0 mm,arc=0mm,enhanced jigsaw,breakable,colback=yellow!12]{ #1} \end{tcolorbox}}

\newcommand{\spr}[1]{\begin{tcolorbox}[boxrule=0.3 mm,arc=0mm,enhanced jigsaw,breakable,colback=yellow!7] {\large \textbf{Språkboksen} \vspace{5 pt}\\} #1 \end{tcolorbox}\vspace{-5pt} }

\newcommand{\sym}[1]{\colorbox{blue!15}{#1}}

\newcommand{\info}[2]{\begin{tcolorbox}[boxrule=0.3 mm,arc=0mm,enhanced jigsaw,breakable,colback=cyan!6] {\large \textbf{#1} \vspace{5 pt}\\} #2 \end{tcolorbox}\vspace{-5pt} }

\newcommand\algv[1]{\vspace{-11 pt}\begin{align*} #1 \end{align*}}

\newcommand{\regv}{\vspace{5pt}}
\newcommand{\mer}{\textsl{Merk}: }
\newcommand\vsk{\vspace{11pt}}
\newcommand\vs{\vspace{-11pt}}
\newcommand\vsabc{\vspace{-5pt}}
\newcommand\vsb{\vspace{-16pt}}
\newcommand\sv{\vsk \textbf{Svar:} \vspace{4 pt}\\}
\newcommand\br{\\[5 pt]}
\newcommand{\fpath}[1]{../fig/#1}
\newcommand\algvv[1]{\vs\vs\begin{align*} #1 \end{align*}}
\newcommand{\y}[1]{$ {#1} $}
\newcommand{\os}{\\[5 pt]}
\newcommand{\prbxl}[2]{
	\parbox[l][][l]{#1\linewidth}{#2
}}
\newcommand{\prbxr}[2]{\parbox[r][][l]{#1\linewidth}{
		\setlength{\abovedisplayskip}{5pt}
		\setlength{\belowdisplayskip}{5pt}	
		\setlength{\abovedisplayshortskip}{0pt}
		\setlength{\belowdisplayshortskip}{0pt} 
		\begin{shaded}
			\footnotesize	#2 \end{shaded}}}

\newcommand{\fgbxr}[2]{
	\parbox[r][][l]{#1\linewidth}{#2
}}

\renewcommand{\cfttoctitlefont}{\Large\bfseries}
\setlength{\cftaftertoctitleskip}{0 pt}
\setlength{\cftbeforetoctitleskip}{0 pt}

\newcommand{\bs}{\\[3pt]}
\newcommand{\vn}{\\[6pt]}
\newcommand{\fig}[1]{\begin{figure}
		\centering
		\includegraphics[]{\fpath{#1}}
\end{figure}}


\newcommand{\sectionbreak}{\clearpage} % New page on each section

% Section comment
\newcommand{\rmerk}[1]{
\rule{\linewidth}{1pt}
#1 \\[-4pt]
\rule{\linewidth}{1pt}
}

% Equation comments
\newcommand{\cm}[1]{\llap{\color{blue} #1}}

\newcommand\fork[2]{\begin{tcolorbox}[boxrule=0.3 mm,arc=0mm,enhanced jigsaw,breakable,colback=yellow!7] {\large \textbf{#1 (forklaring)} \vspace{5 pt}\\} #2 \end{tcolorbox}\vspace{-5pt} }


%%% Rule boxes %%%
\newcommand{\gangdestihundre}{Å gonge desimaltall med 10, 100 osv.}
\newcommand{\delmedtihundre}{Deling med 10, 100, 1\,000 osv.}
\newcommand{\ompref}{Omgjering av prefiksar}


%License
\newcommand{\lic}{\textit{Anvend matematikk for grunnskule og VGS by Sindre Sogge Heggen is licensed under CC BY-NC-SA 4.0. To view a copy of this license, visit\\ 
		\net{http://creativecommons.org/licenses/by-nc-sa/4.0/}{http://creativecommons.org/licenses/by-nc-sa/4.0/}}}

%references
\newcommand{\net}[2]{{\color{blue}\href{#1}{#2}}}
\newcommand{\hrs}[2]{\hyperref[#1]{\color{blue}\textsl{#2 \ref*{#1}}}}
\newcommand{\rref}[1]{\hrs{#1}{Regel}}
\newcommand{\refkap}[1]{\hrs{#1}{Kapittel}}
\newcommand{\refsec}[1]{\hrs{#1}{Seksjon}}
\newcommand{\refdsec}[1]{\hrs{#1}{Delseksjon}}

\newcommand{\colr}[1]{{\color{red} #1}}
\newcommand{\colb}[1]{{\color{blue} #1}}
\newcommand{\colo}[1]{{\color{orange} #1}}
\newcommand{\colc}[1]{{\color{cyan} #1}}
\definecolor{projectgreen}{cmyk}{100,0,100,0}
\newcommand{\colg}[1]{{\color{projectgreen} #1}}

\newcommand{\mb}{\net{https://sindrsh.github.io/FirstPrinciplesOfMath/}{MB}}
\newcommand{\enh}[1]{\,\textrm{#1}}

\newcommand{\metode}[2]{
\textsl{#1} \\[-8pt]
\rule{#2}{0.75pt}
}

\newcommand{\linje}{\rule{\linewidth}{1pt} }

% Opg
\newcommand{\abc}[1]{
\begin{enumerate}[label=\alph*),leftmargin=18pt]
#1
\end{enumerate}
}
\newcommand{\abcs}[2]{
	\begin{enumerate}[label=\alph*),start=#1,leftmargin=18pt]
		#2
	\end{enumerate}
}

\newcommand{\abch}[1]{
	\hspace{-2pt}	\begin{enumerate*}[label=\alph*), itemjoin=\hspace{1cm}]
		#1
	\end{enumerate*}
}

\newcommand{\abchs}[2]{
	\hspace{-2pt}	\begin{enumerate*}[label=\alph*), itemjoin=\hspace{1cm}, start=#1]
		#2
	\end{enumerate*}
}

\newcommand{\abcn}[1]{
	\begin{enumerate}[label=\arabic*),leftmargin=20pt]
		#1
	\end{enumerate}
}


\newcommand{\opgt}{
\newpage
\phantomsection \addcontentsline{toc}{section}{Oppgaver} \section*{Oppgaver for kapittel \thechapter}\vs \setcounter{section}{1}}
\newcounter{opg}
\numberwithin{opg}{section}
\newcommand{\op}[1]{\vspace{15pt} \refstepcounter{opg}\large \textbf{\color{blue}\theopg} \vspace{2 pt} \label{#1} \\}
\newcommand{\oprgn}[1]{\vspace{15pt} \refstepcounter{opg}\large \textbf{\color{blue}\theopg\;(regneark)} \vspace{2 pt} \label{#1} \\}
\newcommand{\oppr}[1]{\vspace{15pt} \refstepcounter{opg}\large \textbf{\color{blue}\theopg\;(programmering)} \vspace{2 pt} \label{#1} \\}
\newcommand{\ekspop}{\vsk\textbf{Gruble \thechapter}\vspace{2 pt} \\}
\newcommand{\nes}{\stepcounter{section}
	\setcounter{opg}{0}}
\newcommand{\opr}[1]{\vspace{3pt}\textbf{\ref{#1}}}
\newcommand{\tbs}{\vspace{5pt}}

%Vedlegg
\newcounter{vedl}
\newcommand{\vedlg}[1]{\refstepcounter{vedl}\phantomsection\section*{G.\thevedl\;#1}  \addcontentsline{toc}{section}{G.\thevedl\; #1} }
\newcommand{\nreqvd}{\refstepcounter{vedleq}\tag{\thevedl \thevedleq}}

\newcounter{vedlE}
\newcommand{\vedle}[1]{\refstepcounter{vedlE}\phantomsection\section*{E.\thevedlE\;#1}  \addcontentsline{toc}{section}{E.\thevedlE\; #1} }

\newcounter{opge}
\numberwithin{opge}{part}
\newcommand{\ope}[1]{\vspace{15pt} \refstepcounter{opge}\large \textbf{\color{blue}E\theopge} \vspace{2 pt} \label{#1} \\}

%Excel og GGB:

\newcommand{\g}[1]{\begin{center} {\tt #1} \end{center}}
\newcommand{\gv}[1]{\begin{center} \vspace{-11 pt} {\tt #1}  \end{center}}
\newcommand{\cmds}[2]{{\tt #1}\\[-3pt]
	#2}


\usepackage{datetime2}
\usepackage[]{hyperref}
\begin{document}

\section{Å finne størrelser}
Likningar, formlar og funksjonar (og utttrykk) er omgrep som dukkar i forskjellige sammenhenger, men som i bunn og grunn handlar om det same; \textsl{dei uttrykker relasjonar mellom størrelsar}. Når alle størrelsane utanom den éine er kjent, kan vi finne denne enten direkte eller indirekte.\vsk

\subsection{Å finne størrelser direkte}
Mange av regelboksane i boka inneheld ein formel. Når ein størrelse står aleine på éi side av formelen, seier vi at det er ein formel for \textsl{den} størrelsen. For eksempel inneheld \rref{maalstk} ein formel for 'målestokk'. Når dei andre størrelsane er gitt, er det snakk om å sette verdiane inn i formelen og regne ut for å finne den ukjente,  'målestokk'.  \vsk

Men ofte har vi berre ei skildring av ein situasjon, og da må vi sjølv lage formlane. Da gjeld det å først identifisere kva størrelsar som er til stades, og så finne relasjonen mellom dei.\regv
\eks[1]{
For ein taxi er det følgende kostnader:
\begin{itemize}
	\item Du må betale 50\enh{kr} uansett hvor langt du blir kjørt.
	\item I tillegg betaler du 15\enh{kr} for kvar kilometer du blir kjørt.
\end{itemize}
\abc{
\item Sett opp eit uttrykk for kor mykje taxituren kostar for kvar kilometer du blir kjørt.
\item Hva kostar ein taxitur på 17\enh{km}?
}

\sv \vsabc
\abc{
\item Her er det to ukjente størrelsar; 'kostnaden for taxituren' og 'antal kilometer køyrt'. Relasjonen mellom dei er denne:
	\[ \text{kostnaden for taxituren}=50+15\cdot\text{antal kilometer køyrt} \]
\item Vi har no at
\[ \text{kostnaden for taxituren}=50+15\cdot17= 305 \]
Taxituren kostar altså 305\enh{kr}.
}
}
\info{Tips}{
Ved å la enkeltbokstaver representere størrelsar, får ein kortare uttrykk. La $ k $ stå for 'kostnad for taxituren' og $ x $ for 'antal kilometer køyrt'. Da blir uttrykket fra \textsl{Eksempel 1} over dette:
\[ k=50+15x \]
I tillegg kan ein gjerne bruke skrivemåten for funksjonar:
\[ k(x)=50+15x \] 
}
\subsection{Å finne størrelsar indirekte}
\subsubsection{Når formlane er kjente}
\eks[1]{
	Vi har sett at strekninga $ s $ vi har køyrt, farta $ f $ vi har halde, og tida $ t $ vi har brukt kan settast i samanheng via formelen\footnote{$ \text{strekning}=\text{fart}\cdot \text{tid} $}:
	\[ s = f\cdot t \] 
	Dette er altså ein formel for $ s $. Ønsker vi i staden ein formel for $ f $, kan vi gjere om formelen ved å følge prinsippa for likningar\footnote{Sjå \mb, s. 121.}:
	\alg{
		s &= f\cdot t \br
		\frac{s}{t}&=\frac{f\cdot \bcancel{t}}{\bcancel{t}} \br
		\frac{s}{t}&=f
	}
}
\newpage
\eks[2]{
	\textit{Ohms lov} seier at strømmen $ I $ gjennom ein metallisk ledar (med konstant temeperatur) er gitt ved formelen
	\[ I = \frac{U}{R} \]
	der $ U $ er spenninga og $ R $ er resistansen.
	
	\abc{
	\item Skriv om formelen til ein formel for $ R $.	
}
Strøm målast i Ampere (A), spenning i Volt (V) og motstand i Ohm ($ \Omega $).
\abcs{2}{
\item Hvis strømmen er 2\,A og spenninga 12\,V, kva er da resistansen?
}

	
	\sv \vsabc
\abc{
	\item Vi gjer om formelen slik at $ R $ står aleine på éi side av likskapsteiknet:\vs
\alg{
	I\cdot R&=\frac{U\cdot \cancel{R}}{\cancel{R}} \br
	I\cdot R &= U \br
	\frac{\cancel{I}\cdot R}{\cancel{I}} &= \frac{U}{I}\br 
	R &= \frac{U}{I}
}
\item Vi bruker formelen vi fant i a), og får at
\alg{
	R &= \frac{U}{I} \br
	&= \frac{12}{2} \\
	&= 6
}
Resistansen er altså $ 6\,\Omega $.
}
}
\newpage
\eks[3]{
	Gitt ein temperatur $ T_C $ målt i antall grader Celsius ($ ^\circ C $). Temperaturen $ T_F $ målt i antall grader Fahrenheit ($ ^\circ F $) er da gitt ved formelen
	\[ T_F = \frac{9}{5}\cdot T_C+32 \]
\abc{
\item Skriv om formelen til ein formel for $ T_C $.
\item Vis ein temperatur er målt til 59$ ^\circ F $, kva er da temperaturen målt i $ ^\circ C $?
}
	
	\sv \vsabc
	
	\abc{
	\item Vi isolerer $ T_C $ på én side av likhetstegnet:
	\alg{
		T_F &= \frac{9}{5}\cdot T_C+32 \\
		T_F-32 &= \frac{9}{5}\cdot T_C \\
		5(T_F-32) &= \cancel{5}\cdot\frac{9}{\cancel{5}}\cdot F_C \\
		5(T_F-32) &= 9T_C \\
		\frac{5(T_F-32)}{9} &= \frac{\cancel{9}T_C}{\cancel{9}} \\
		\frac{5(T_F-32)}{9} &= T_C
	}
	\item Vi bruker formelen fra a), og finn at
	\alg{
		T_C&= \frac{5(59-32)}{9} \br
		&= \frac{5(27)}{9} \br
		&= 5\cdot 3 \\
		&= 15
	}
}
}

\subsubsection{Når formlane er ukjente}
\eks[1]{
Tenk at klassen ønsker å fare på ein klassetur som til saman kostar 11\,000\,kr. For å dekke utgiftane har de allereie skaffa 2\,000\,kr, resten skal skaffast gjennom loddsalg. For kvart lodd som selgast, tener de 25\,kr.\os

\abc{
\item Lag ei ligning for kor mange lodd klassen må selge for å få råd til klasseturen.
\item Løys likninga.
}

\sv
\abc{
\item Vi startar med å tenke oss reknestykket i ord:
\small
\[ \text{pengar allereie skaffa}+\text{antal lodd}\cdot\text{pengar per lodd}=\text{prisen på turen} \]
\normalsize
Den einaste størrelsen vi ikkje veit om er 'antal lodd'. Vi erstattar\footnote{Dette gjer vi berre fordi det da blir mindre for oss å skrive.} \textit{antal lodd} med $ x $, og sett verdiane til dei andre størrelsane inn i likninga:
\[ 2\,000+x\cdot25 = 11\,000 \]
\item \ \vs \vs
\alg{
	25x &= 11\,000-2\,000\\
	25 x &= 9\,000\\
	\frac{\cancel{25} x}{\cancel{25}} &= \frac{9\,000}{25} \\
	x &= 360
}
}
}
\newpage
\eks[2]{Ein vennegjeng ønsker å spleise på ein bil som kostar 50\,000\enh{kr}, men det er usikkert kor mange personar som skal vere med på å spleise.\os 
	\textbf{a)} Kall 'antal personar som blir med på å spleise' for $ P $ og 'utgift per person' for $ U $,  og lag ein formel for $ U $.\os
	
	\textbf{b)} Finn utgifta per person viss 20 personar blir med.
	
	\sv
	\textbf{a)} Sidan prisen på bilen skal delast på antal personar som er med i spleiselaget, må formelen bli
	\[ U = \frac{50\,000}{P} \]
	
	\textbf{b)} Vi erstattar $ P $ med 20, og får
	\alg{
		U &= \frac{50\,000}{20}\\
		&= 2\,500
	}
	Utgifta per person er altså 2\,500\enh{kr} .
}
\section{Funksjoners egenskaper}
\textit{Denne seksjonen tar utgangspunkt i at lesaren er kjed med funksjonar, sjå \mb, kapittel 9.}
\subsection{Funksjoner med samme verdi; skjæringspunkt}
\reg[Skjæringspunkt til grafer]{
	Et punkt der to funksjonar har same verdi kallast eit \textit{skjæringspunkt} til funksjonane.
}
\eks[1]{
Gitt de to funksjonane
\alg{
	f(x) &= 2x+1 \vn 
	g(x) &= x+4
}
Finn skjæringspunktet til $ f(x) $ og $ g(x) $.

\sv

Vi kan finne skjæringspunktet både ved ein \textsl{grafisk} og ein \textsl{algebraisk} metode. \vsk

\metode{Grafisk metode}{0.6\linewidth} \os

Vi teikner grafane til funksjonane inn i det same koordinatsystemet:
\fig{lig1}
Vi les av at funksjonane har same verdi når $ {x=3} $, og da har begge funksjonane verdien 7. Altså er skjæringspunktet $ (3, 7) $. \vsk
\vsk

\metode{Algebraisk metode}{0.6\linewidth} \os
At $ f(x) $ og $ g(x) $ har samme verdi gir likninga
\alg{
f(x)&=g(x) \\
2x+1 &=x+4 \\
x&=3
}
Vidare har vi at
\alg{
f(3)&=2\cdot3+1=7 \vn 
g(3)&=3+4=7
}
Altså er $ (3, 7) $ skjæringspunktet til grafane.\vsk

{\footnotesize \mer Det hadde sjølvsagt halde å berre finne éin av $ f(3) $ og $ g(3) $.}
}
\newpage
\eks[2]{
Ein klasse planlegg ein tur som krever bussreise. Dei får tilbud frå to busselskap:
\begin{itemize}
	\item \textbf{Busselskap 1} \\
	Klassen betaler 10\,000\,kr uansett, og 10\enh{kr} per km.
	\item \textbf{Busselskap 2} \\
	Klassen betaler 4\,000\,kr uansett, og 30\enh{kr} per km.
\end{itemize}
For kva lengde kjøyrt tilbyr busselskapa same pris?

\sv

Vi innfører følgande variablar:
\begin{itemize}
	\item $ x=\text{antall kilometer kjørt} $ \\
	\item $ f(x)=\text{pris for Busselskap 1} $ \\
	\item $ g(x)=\text{pris for Busselskap 2} $
\end{itemize}
Da er
\alg{
	f(x)&=10x+10\,000\vn
	g(x)&=30x+4\,000
}
Vidare løyser vi no oppgåva både med ein grafisk og ein algebraisk metode. \vsk

\metode{Grafisk metode}{0.6\linewidth} \os
Vi teikner grafane til funksjonane inn i same koordinatsystem:
\fig{lig2}
Vi les av at funksjonane har same verdi når $ {x=200} $. Dette betyr at busselskapa tilbyr same pris viss klassen skal køyre 200\enh{km}.\vsk \vsk

\metode{Algebraisk metode}{0.6\linewidth} \os
Busselskapa har same pris når
\alg{
f(x)&=g(x) \\
10x+10\,000&=30x+6\,000 \\
4\,000&=20x \\
x&=200
}
Busselskapa tilbyr altså samm pris viss klassen skal køyre 200\enh{km}.
}

\subsection{Null-, bunn- og toppunkt}
\reg[Null-, bunn- og toppunkt]{
\vs	
	\begin{itemize}
	\item \textbf{Nullpunkt} \\[-5pt]
	Ein $ x $-verdi som gir funksjonsverdi 0.
	\item \textbf{Lokalt bunnpunkt} \\[-5pt]
Punkt der funksjonen (frå venstre) går frå å synke i verdi til å stige i verdi.
\item \textbf{Lokalt toppunkt} \\[-5pt]
Punkt der funksjonen (frå venstre) går frå å stige i verdi til å synke i verdi
\item \textbf{Globalt bunnpunkt} \\[-5pt]
Punkt der funksjonen har sin lavaste verdi.
\item \textbf{Globalt toppunkt} \\[-5pt]
Punkt der funksjonen har sin høgste verdi.
\end{itemize}
}
\eks[]{
\fig{funkdroft}
}
\info{Kvifor er nullpunkt ein verdi?}{
Det kan kanskje verke litt rart at vi kallar $ x $-verdiar for nullpunkt, punkt har jo både ein $ x $-verdi og eni $ y $-verdi. Men når det er snakk om nullpunkt, er det underforstått at $ {y=0} $, og da er det tilstrekkeleg å vite $ x $-verdien for å avgjere kva punkt det er snakk om.  
}
\section{Likningssett}
Vi har så langt sett på likningar med eitt ukjend tal, men det kan også vere to eller fleire tal som er ukjende. Som regel er det slik at 
\st{\begin{itemize}
	\item er det to ukjende, trengs minst to likningar for å finne løysingar som er konstantar.
	\item er det tre ukjende, trengs minst tre likningar for å finne løysingar som er konstantar.
\end{itemize}}
Og slik fortset det. Likningane som gir oss den nødvendige informasjonen om dei ukjende, kallast eit \textit{likningssett}. I denne boka skal vi konsentrere oss om \textsl{lineære likninger med to ukjente}, som betyr at likningssettet består av uttrykk for lineære funksjonar.
\subsection{Innsettingsmetoden}
\reg[Innsettingsmetoden]{
Eit lineært likningssett som består av to ukjende, $ x $ og $ y $, kan løses ved å 
\begin{enumerate}
	\item bruke den éine likninga til å finne eit uttrykk for $ x $.
	\item sette uttrykket fra punkt 1 inn i den andre likninga, og løyse denne med hensyn på $ y $.
	\item sette løysninga for $ y $ inn i uttrykket for $ x $.
\end{enumerate}
{\footnotesize \mer I punkta over kan sjølvsagt $ x $ og $ y $ bytte roller.}
}
\newpage
\eks[1]{
Løys likningssettet, og sett prøve på svaret.
\alg{
	x-y&=5 \tag{I} \label{eks4a}\vn
	x+y&=9 \tag{II} \label{eks4b}
}
\sv

Av \eqref{eks4a} har vi at 
\algv{
x-y &= 5 \\
x&=5+y
}
Vi set dette uttrykket for $ x $ inn i \eqref{eks4b}:
\alg{
5+y+y &=9 \\
2y&=9-5 \\
2y&=4 \\
y&=2
}
Vi set løysinga for $ y $ inn i uttrykket for $ x $:
\alg{
x&=5+y \\
&=5+2 \\
&=7
}
Altså er $ x=7 $ og $ y=2 $. \vsk

Vi set prøve på svaret:
\alg{
x-y&=7-2=5 \vn
x+y &=7+2=9
} 
}
\newpage
\eks[2]{
Løs likningssettet \vs
\alg{
	7x-5y&=-8 \tag{I} \label{eks1a}\vn
	5x-2y&=4x-5 \tag{II} \label{eks1b}
}
\sv

Ved innsettingsmetoden kan ein ofte spare seg for ein del utrekning ved å velge likninga og den ukjende som gir det finaste uttrykket innleiingsvis. Vi observerer at \eqref{eks1b} gir et fint uttrykk for $ x $:
\alg{
7x-5y&=-6 \\
x&=2y-5
}
Vi set dette uttrykket for $ x $ inn i \eqref{eks1a}:
\alg{
	7x-5y&=-8\\
7(2y-5)-5y&=-8 \\
14y-35-5y&=-8 \\
9y &=27 \\
y&=3
}
Vi set løysinga for $ y $ inn i uttrykket for $ x $:
\alg{
x&=2y-5 \\
&=2\cdot 3-5\\
&=1
}
Altså er $ x=1 $ og $ y=3 $.
}
\newpage
\eks[3]{
Løys likningssettet \vs
\alg{
3x-4y&=-2 \tag{I} \label{eks2a}\vn
9y-5x&=6x+y \tag{II} \label{eks2b}
} \vs \vs

\sv

Vi velg her å bruke \eqref{eks1a} til å finne eit uttrykk for $ y $:
\alg{
3x-4y&=-2 \\
3x+2&=4y \\
\frac{3x+2}{4}&=y
}
Vi set dette uttrykket for $ y $ inn i \eqref{eks2b}:
\alg{
9y-5x&=6x+y \\	
9\cdot\frac{3x+2}{4}-5x&=6x+\frac{3x+2}{4} \\
9(3x+2)-20x&=24x+3x+2 \\
27x+18-20x &=24x+3x+2 \\
-20x&=-16 \\
x&=\frac{4}{5}
}
Vi set løysinga for $ x $ inn i uttrykket for $ y $:
\alg{
y&=\frac{3x+2}{4} \br
&=\frac{3\cdot\frac{4}{5}+2}{4} \br
&=\frac{\frac{22}{5}}{4} \\
&=\frac{11}{10}
}
Altså er $ x=\frac{4}{5} $ og $ y=\frac{11}{10} $.
}
\eks[4]{
	''Bror min er dobbelt så gamal som meg. Til saman er vi 9 år gamle. Kor gamal er eg?''.
	
	\sv
	''Bror min er dobbelt så gamal som meg.'' betyr at
	\[ \text{brors alder}=2\cdot \text{min alder} \]
	''Til saman er vi 9 år gamle.'' betyr at
	\[ \text{brors alder}+\text{min alder}=\text{9} \]
	Erstattar vi 'brors alder' med ''$2\cdot\text{min alder} $'', får vi
	\[ 2\cdot\text{min alder}+\text{min alder}=\text{9} \]
	Altså er
	\algv{
		3\cdot \text{min alder} &= 9 \\
		\frac{\cancel{3}\cdot \text{min alder}}{\cancel{3}}&= \frac{9}{3} \\
		\text{min alder} &= 3
	}
	''Eg'' er altså 3 år gammel.
}
\subsection{Grafisk metode}
\reg[Grafisk løsning av likningssett]{
Eit lineært likningssett som består av to ukjende, $ x $ og $ y $, kan løysast ved å 
	\begin{enumerate}
		\item omskrive dei to likningane til uttrykk for to linjer.
		\item finne skjæringspunktet til linjene.
\end{enumerate}
}
\newpage
\eks[1]{
Løys likningsettet\vs
\alg{
	x-y&=5 \tag{I} \label{eks3a}\vn
	x+y&=9 \tag{II} \label{eks3b}
}\vs 
\sv
Av \eqref{eks3a} har vi at \vs
\alg{
x-y&=5 \\
y&=x-5
}
Av \eqref{eks3b} har vi at \vs
\alg{
x+y&=9 \\
y &= 9-x 
}
Vi teikner desse to linjene inn i eit koordinatsystem:
\fig{ligset1}
Altså er $ x=7 $ og $ y=2 $.
}
\end{document}


