\documentclass[english, 11 pt, class=article, crop=false]{standalone}
\usepackage[T1]{fontenc}
%\renewcommand*\familydefault{\sfdefault} % For dyslexia-friendly text
\usepackage{lmodern} % load a font with all the characters
\usepackage{geometry}
\geometry{verbose,paperwidth=16.1 cm, paperheight=24 cm, inner=2.3cm, outer=1.8 cm, bmargin=2cm, tmargin=1.8cm}
\setlength{\parindent}{0bp}
\usepackage{import}
\usepackage[subpreambles=false]{standalone}
\usepackage{amsmath}
\usepackage{amssymb}
\usepackage{esint}
\usepackage{babel}
\usepackage{tabu}
\makeatother
\makeatletter

\usepackage{titlesec}
\usepackage{ragged2e}
\RaggedRight
\raggedbottom
\frenchspacing

% Norwegian names of figures, chapters, parts and content
\addto\captionsenglish{\renewcommand{\figurename}{Figur}}
\makeatletter
\addto\captionsenglish{\renewcommand{\chaptername}{Kapittel}}
\addto\captionsenglish{\renewcommand{\partname}{Del}}


\usepackage{graphicx}
\usepackage{float}
\usepackage{subfig}
\usepackage{placeins}
\usepackage{cancel}
\usepackage{framed}
\usepackage{wrapfig}
\usepackage[subfigure]{tocloft}
\usepackage[font=footnotesize,labelfont=sl]{caption} % Figure caption
\usepackage{bm}
\usepackage[dvipsnames, table]{xcolor}
\definecolor{shadecolor}{rgb}{0.105469, 0.613281, 1}
\colorlet{shadecolor}{Emerald!15} 
\usepackage{icomma}
\makeatother
\usepackage[many]{tcolorbox}
\usepackage{multicol}
\usepackage{stackengine}

\usepackage{esvect} %For vectors with capital letters

% For tabular
\usepackage{array}
\usepackage{multirow}
\usepackage{longtable} %breakable table

% Ligningsreferanser
\usepackage{mathtools}
\mathtoolsset{showonlyrefs}

% index
\usepackage{imakeidx}
\makeindex[title=Indeks]

%Footnote:
\usepackage[bottom, hang, flushmargin]{footmisc}
\usepackage{perpage} 
\MakePerPage{footnote}
\addtolength{\footnotesep}{2mm}
\renewcommand{\thefootnote}{\arabic{footnote}}
\renewcommand\footnoterule{\rule{\linewidth}{0.4pt}}
\renewcommand{\thempfootnote}{\arabic{mpfootnote}}

%colors
\definecolor{c1}{cmyk}{0,0.5,1,0}
\definecolor{c2}{cmyk}{1,0.25,1,0}
\definecolor{n3}{cmyk}{1,0.,1,0}
\definecolor{neg}{cmyk}{1,0.,0.,0}

% Lister med bokstavar
\usepackage[inline]{enumitem}

\newcounter{rg}
\numberwithin{rg}{chapter}
\newcommand{\reg}[2][]{\begin{tcolorbox}[boxrule=0.3 mm,arc=0mm,colback=blue!3] {\refstepcounter{rg}\phantomsection \large \textbf{\therg \;#1} \vspace{5 pt}}\newline #2  \end{tcolorbox}\vspace{-5pt}}

\newcommand\alg[1]{\begin{align} #1 \end{align}}

\newcommand\eks[2][]{\begin{tcolorbox}[boxrule=0.3 mm,arc=0mm,enhanced jigsaw,breakable,colback=green!3] {\large \textbf{Eksempel #1} \vspace{5 pt}\\} #2 \end{tcolorbox}\vspace{-5pt} }

\newcommand{\st}[1]{\begin{tcolorbox}[boxrule=0.0 mm,arc=0mm,enhanced jigsaw,breakable,colback=yellow!12]{ #1} \end{tcolorbox}}

\newcommand{\spr}[1]{\begin{tcolorbox}[boxrule=0.3 mm,arc=0mm,enhanced jigsaw,breakable,colback=yellow!7] {\large \textbf{Språkboksen} \vspace{5 pt}\\} #1 \end{tcolorbox}\vspace{-5pt} }

\newcommand{\sym}[1]{\colorbox{blue!15}{#1}}

\newcommand{\info}[2]{\begin{tcolorbox}[boxrule=0.3 mm,arc=0mm,enhanced jigsaw,breakable,colback=cyan!6] {\large \textbf{#1} \vspace{5 pt}\\} #2 \end{tcolorbox}\vspace{-5pt} }

\newcommand\algv[1]{\vspace{-11 pt}\begin{align*} #1 \end{align*}}

\newcommand{\regv}{\vspace{5pt}}
\newcommand{\mer}{\textsl{Merk}: }
\newcommand{\mers}[1]{{\footnotesize \mer #1}}
\newcommand\vsk{\vspace{11pt}}
\newcommand\vs{\vspace{-11pt}}
\newcommand\vsb{\vspace{-16pt}}
\newcommand\sv{\vsk \textbf{Svar} \vspace{4 pt}\\}
\newcommand\br{\\[5 pt]}
\newcommand{\figp}[1]{../fig/#1}
\newcommand\algvv[1]{\vs\vs\begin{align*} #1 \end{align*}}
\newcommand{\y}[1]{$ {#1} $}
\newcommand{\os}{\\[5 pt]}
\newcommand{\prbxl}[2]{
\parbox[l][][l]{#1\linewidth}{#2
	}}
\newcommand{\prbxr}[2]{\parbox[r][][l]{#1\linewidth}{
		\setlength{\abovedisplayskip}{5pt}
		\setlength{\belowdisplayskip}{5pt}	
		\setlength{\abovedisplayshortskip}{0pt}
		\setlength{\belowdisplayshortskip}{0pt} 
		\begin{shaded}
			\footnotesize	#2 \end{shaded}}}

\renewcommand{\cfttoctitlefont}{\Large\bfseries}
\setlength{\cftaftertoctitleskip}{0 pt}
\setlength{\cftbeforetoctitleskip}{0 pt}

\newcommand{\bs}{\\[3pt]}
\newcommand{\vn}{\\[6pt]}
\newcommand{\fig}[1]{\begin{figure}
		\centering
		\includegraphics[]{\figp{#1}}
\end{figure}}

\newcommand{\figc}[2]{\begin{figure}
		\centering
		\includegraphics[]{\figp{#1}}
		\caption{#2}
\end{figure}}

\newcommand{\sectionbreak}{\clearpage} % New page on each section

\newcommand{\nn}[1]{
\begin{equation}
	#1
\end{equation}
}

% Equation comments
\newcommand{\cm}[1]{\llap{\color{blue} #1}}

\newcommand\fork[2]{\begin{tcolorbox}[boxrule=0.3 mm,arc=0mm,enhanced jigsaw,breakable,colback=yellow!7] {\large \textbf{#1 (forklaring)} \vspace{5 pt}\\} #2 \end{tcolorbox}\vspace{-5pt} }
 
%colors
\newcommand{\colr}[1]{{\color{red} #1}}
\newcommand{\colb}[1]{{\color{blue} #1}}
\newcommand{\colo}[1]{{\color{orange} #1}}
\newcommand{\colc}[1]{{\color{cyan} #1}}
\definecolor{projectgreen}{cmyk}{100,0,100,0}
\newcommand{\colg}[1]{{\color{projectgreen} #1}}

% Methods
\newcommand{\metode}[2]{
	\textsl{#1} \\[-8pt]
	\rule{#2}{0.75pt}
}

%Opg
\newcommand{\abc}[1]{
	\begin{enumerate}[label=\alph*),leftmargin=18pt]
		#1
	\end{enumerate}
}
\newcommand{\abcs}[2]{
	\begin{enumerate}[label=\alph*),start=#1,leftmargin=18pt]
		#2
	\end{enumerate}
}
\newcommand{\abcn}[1]{
	\begin{enumerate}[label=\arabic*),leftmargin=18pt]
		#1
	\end{enumerate}
}
\newcommand{\abch}[1]{
	\hspace{-2pt}	\begin{enumerate*}[label=\alph*), itemjoin=\hspace{1cm}]
		#1
	\end{enumerate*}
}
\newcommand{\abchs}[2]{
	\hspace{-2pt}	\begin{enumerate*}[label=\alph*), itemjoin=\hspace{1cm}, start=#1]
		#2
	\end{enumerate*}
}

% Oppgaver
\newcommand{\opgt}{\phantomsection \addcontentsline{toc}{section}{Oppgaver} \section*{Oppgaver for kapittel \thechapter}\vs \setcounter{section}{1}}
\newcounter{opg}
\numberwithin{opg}{section}
\newcommand{\op}[1]{\vspace{15pt} \refstepcounter{opg}\large \textbf{\color{blue}\theopg} \vspace{2 pt} \label{#1} \\}
\newcommand{\ekspop}[1]{\vsk\textbf{Gruble \thechapter.#1}\vspace{2 pt} \\}
\newcommand{\nes}{\stepcounter{section}
	\setcounter{opg}{0}}
\newcommand{\opr}[1]{\vspace{3pt}\textbf{\ref{#1}}}
\newcommand{\oeks}[1]{\begin{tcolorbox}[boxrule=0.3 mm,arc=0mm,colback=white]
		\textit{Eksempel: } #1	  
\end{tcolorbox}}
\newcommand\opgeks[2][]{\begin{tcolorbox}[boxrule=0.1 mm,arc=0mm,enhanced jigsaw,breakable,colback=white] {\footnotesize \textbf{Eksempel #1} \\} \footnotesize #2 \end{tcolorbox}\vspace{-5pt} }
\newcommand{\rknut}{
Rekn ut.
}

%License
\newcommand{\lic}{\textit{Matematikken sine byggesteinar by Sindre Sogge Heggen is licensed under CC BY-NC-SA 4.0. To view a copy of this license, visit\\ 
		\net{http://creativecommons.org/licenses/by-nc-sa/4.0/}{http://creativecommons.org/licenses/by-nc-sa/4.0/}}}

%referances
\newcommand{\net}[2]{{\color{blue}\href{#1}{#2}}}
\newcommand{\hrs}[2]{\hyperref[#1]{\color{blue}\textsl{#2 \ref*{#1}}}}
\newcommand{\rref}[1]{\hrs{#1}{regel}}
\newcommand{\refkap}[1]{\hrs{#1}{kapittel}}
\newcommand{\refsec}[1]{\hrs{#1}{seksjon}}

\newcommand{\mb}{\net{https://sindrsh.github.io/FirstPrinciplesOfMath/}{MB}}


%line to seperate examples
\newcommand{\linje}{\rule{\linewidth}{1pt} }

\usepackage{datetime2}
%%\usepackage{sansmathfonts} for dyslexia-friendly math
\usepackage[]{hyperref}

\usepackage{pgfplots}
%\usepackage{needspace}
%\raggedbottom
%\goodbreak
%\Needspace{10\baselineskip}

\begin{document}
\section{Funksjoner \label{f1}}

Se for deg at du og klassen skal på klassetur. For hver person som er med må det betales 200 kr for overnattig, i tillegg så koster leie av buss med sjåfør til sammen 5000 kr. For å finne ut hvor mye penger klassen må samle inn, lager dere følgende regnestykke:
\begin{align*}
	\text{penger som må samles inn}&=\text{pris for overnatting per elev} \cdot \text{antall elever}+\text{bussutgifter} \\
	&= 200\cdot\text{antall elever} + 5000
\end{align*}
I mange sammenhenger kan det være litt tungvingt å bruke lange uttrykk som \textit{penger som må samles inn} og \textit{antall elever}. Det kan derfor være lurt av oss å bruke enkeltbokstaver istedenfor. La oss nå si at \textit{P} betyr penger som må samles inn, mens \textit{x} betyr antall elever. Regnestykket vårt blir da:
$$P=200x+5000$$
Siden vi uansett må betale 5000 kr i bussutgifter, er det slik at det eneste som forandrer på prisen vår \textit{P}, er antall elever \textit{x} som skal være med. Vi sier da at \textit{P} er en funksjon av $x$. For å tydeliggjøre at vi må vite \textit{x} for å finne \textit{P}, skriver vi ofte $P(x)$. $P(x)$ uttaler vi som \textit{P av x}.
$$P(x)=200x+5000$$

\section{Verdien av en funksjon}
Når vi har en funksjon som $ P(x) $ kan vi finne verdien til denne hvis vi vet hva \textit{x} er.
Så hva blir prisen dersom antall elever er 20? I uttrykket for $P(x)$, setter vi da inn at $x=20$:
$$P(20)=200\cdot20+5000=9000$$
Hvis \textit{x} istedenfor er 30 så erstatter vi alle \textit{x}-er i uttrykket for $ P(x) $ med 30:
$$P(30)=200\cdot30+5000=11000$$

\reg[Verdien av en funksjon når \textit{x} er kjent]{Dersom vi kjenner verdien til \textit{x} kan vi finne den tilhørende verdien til funksjonen $ f(x) $ ved å sette verdien til \textit{x} inn i uttrykket for $ f(x) $.
}

\eks[1]{Gitt funksjonen $ f(x)=3x+4 $. Finn  verdien til \textit{f} når $ x=8 $. \\

\textbf{Svar:} $ f(8)=3\cdot8+4=24+4=28 $
}

\eks[2]{ Gitt funksjonen $ h(x)=x^2 +3x+2 $. Finn $ h(-2) $. \\

\textbf{Svar:} $ h(-2)=(-2)^2 +3\cdot (-2) +2 =0$}

\section[Verdien av \textit{x} når $ f(x) $ er kjent]{Verdien av \textit{x} når \boldmath$ f(x) $ er kjent}
Noen ganger vet vi verdien til en funksjon. Da lurer vi ofte på hva verdien til $ x $ er. La oss si vi vet at prisen for klasseturen vi har sett på endte opp med å bli 9000 kr. Hvor mange elever var da med?

Siden vi vet at $ P(x)=9000 $ kan vi nå skrive:
\[ 9000=200x + 5000 \]
Dette er en førstegradsligning med \textit{x} som ukjent.
\begin{align*}
9000&=200x + 5000 \\
9000-5000 &=200x \\
\frac{4000}{200} &=\frac{\bcancel{200}\,x}{\bcancel{200}} \\
x &= 20
\end{align*}

Altså er 20 elever med på turen.

\reg[Verdien av \textit{x} når \boldmath$ f(x) $ er kjent]{Dersom vi kjenner verdien til funksjonen $ f(x) $, får vi en ligning for $ x $.\\
}

\eks{Gitt funksjonen $ f(x)=3x+4 $. Hva er \textit{x} når $ f(x)=7 $? \\

\textbf{Svar:} \vspace{-10 pt}\begin{align*}
7&=3x+4 \\
7-4 &= 3x \\
\frac{3}{3} &= \frac{\bcancel{3}x}{\bcancel{3}} \\
1 &= x
\end{align*}\vspace{-20 pt}}
\eks[2]{Gitt funksjonen $ h(x)=x^2 +3 $. Finn \textit{x} når $ h(x)=19 $. \\

\textbf{Svar:} \vspace{-10 pt} \begin{align*}
19&=x^2 +3 \\
19-3 &= x^2 \\
16 &= x^2 \\
\sqrt{16} &= \sqrt{x^2} \\
\pm 4 &= x
\end{align*}\vspace{-20 pt}}
\section{Tegning av grafen til en funksjon}	
Som vi har sett, vil verdien til en funksjon $ f(x) $ avhenge av hva \textit{x} er. For funksjonenen $ f(x)=x^2-3x+2 $ kan vi lage oss en tabell som viser hvilke \textit{x}-verdier vi har plukket ut og hvilke verdier av $ f(x) $ disse resulterer i. La oss plukke ut $ -3,-2,-3, 0, 1, 2, 3 \text{ og }4$ som \textit{x}-verdier. Vi må da regne ut $ f(x) $ for hver av dem:

\begin{minipage}{.75\linewidth}
\begin{align*}
f(-2) &= (-2)^2-3(-2)+2=4+6+2=11 \\
f(-1) &= (-1)^2-3(-1)+2=1+3+2=5 \\
f(0) &= 0^2-3\cdot 0+2=0-0+2=1 \\
f(1) &= 1^2-3\cdot 1+2=1-3+2=-1 \\
f(2) &= 2^2-3\cdot 2+2=4-6+2=-1 \\
f(3) &= 3^2-3\cdot 3+2=9-9+2=1 \\
f(4) &= 4^2-3\cdot 4+2=16-12+2=5
\end{align*}
\end{minipage}%
\begin{minipage}{.2\linewidth}\vspace{-70pt}
\vspace{15pt}
\begin{shaded}Husk parantes når \textit{x} er et negativt \\
tall.
\end{shaded}
\end{minipage} \\

I tabellen vi nå lager skriver vi \textit{x}-verdiene vi har plukket ut i første rad og de tilhørende verdiene av $ f(x) $ i andre rad:

\begin{center}
\begin{tabular}{ |c|c|c|c|c|c|c|c| } 
 \hline
 x & -2 & -1 & 0 & 1 & 2 & 3 & 4  \\ 
 \hline
  f(x) & 11 & 5 & 1 & -1 & -1 & 1 & 5 \\
 \hline
\end{tabular}
\end{center}

Tallene fra tabellen kan vi markere som et punkt i et koordinatsystem. Vi lager oss da en tallinje som går oppover og en tallinje som går bortover. \textit{x}-verdiene finner vi ved å lese av på den horisontale tallinjen, mens $ f(x) $-verdiene leser vi av på den vertikale tallinjen.

For å tegne inn punktene fra tabellen inn i koordinatsystemet kan vi starte der hvor tallinjene krysser og gå følgende veier:

\begin{center}
\begin{tabular}{ l l l} 
 $ x=-2$ & $ f(x)=11 $ & \qquad \text{2 til \textit{venstre}, 11 \textit{opp}} \\
  $ x=-1$ & $ f(x)=5 $ & \qquad \text{1 til \textit{venstre}, 5 \textit{opp}} \\
   $ x=0$ & $ f(x)=1 $ & \qquad \text{0 bort, 1 \textit{opp}} \\
    $ x=1$ & $ f(x)=-1 $ & \qquad \text{1 til \textit{høyre}, 1 \textit{ned}} \\
     $ x=2$ & $ f(x)=-1 $ & \qquad \text{2 til \textit{høyre}, 1 \textit{ned}} \\
      $ x=2$ & $ f(x)=1 $ & \qquad \text{2 til \textit{høyre}, 1 \textit{opp}} \\
       $ x=4$ & $ f(x)=5 $ & \qquad \text{2 til \textit{høyre}, 5 \textit{opp}} \\
\end{tabular}
\end{center}

\begin{figure}[H]
\centering
\pgfmathdeclarefunction{f}{1}{%
  \pgfmathparse{#1^2 - 3*#1+1}%
}
\pgfplotsset{scaled y ticks=false}
\begin{tikzpicture}[scale=0.5]
\begin{axis}[
%/pgf/number format/1000 sep={},
	height=15cm, width=12cm,
%    scale only axis,
	grid=both,
	axis lines = left,
	xmin=-2.5, xmax=4.5,
	ymin=-1.5, ymax=11.5,
    xlabel=$x$,ylabel=$f(x)$, 
        xtick={-3,...,4} , 
    ytick={-1,1,2,...,11},
	axis lines=middle,      
every axis x label/.style={
    at={(ticklabel* cs:1.05)},
    anchor=center,
},
every axis y label/.style={
    at={(ticklabel* cs:1.05)},
    anchor=center,
}	   
]
\addplot [
    samples=100, 
    color=blue,
]{f(x)};
\end{axis} 
\end{tikzpicture}
\end{figure}
Tenk nå at vi lager en ny tabell, men at vi finner så mange punkter at de ligger helt tett i tett inntil hverandre. Punktene vil da danne en sammenhengende strek som ser slik ut:

\begin{figure}[H]
\centering
\pgfmathdeclarefunction{f}{1}{%
  \pgfmathparse{#1^2 - 3*#1+1}%
}
\pgfplotsset{scaled y ticks=false}
\begin{tikzpicture}[scale=1]
\begin{axis}[
%/pgf/number format/1000 sep={},
	height=15cm, width=12cm,
%    scale only axis,
	axis lines = left,
	xmin=-2.5, xmax=4.5,
	ymin=-1.5, ymax=11.5,
    xlabel=$x$,ylabel=$f(x)$, 
        xtick={-3,...,4} , 
    ytick={-1,1,2,...,11},
	axis lines=middle,      
every axis x label/.style={
    at={(ticklabel* cs:1.05)},
    anchor=center,
},
every axis y label/.style={
    at={(ticklabel* cs:1.05)},
    anchor=center,
}	   
]
\addplot [
    samples=100, 
    color=blue,
]{f(x)};
\end{axis} 
\end{tikzpicture}
\end{figure}

Figuren over kaller vi grafen til funksjonen $ f(x)=x^2+3x+1 $. Vi kan nå observere at punktene vi fant når \textit{x} var $ -2,-1, 0, 1, 2, 3 \text{ og } 4 $ ligger på linja til denne grafen: \\

\begin{figure}[H]
\centering
\pgfmathdeclarefunction{f}{1}{%
  \pgfmathparse{(#1)^2 - 3*#1+1}%
}

\pgfmathsetmacro\fa{f(-2)}
\pgfmathsetmacro\fb{f(-1)}
\pgfmathsetmacro\fc{f(0)}
\pgfmathsetmacro\fd{f(1)}
\pgfmathsetmacro\fe{f(2)}
\pgfmathsetmacro\ff{f(3)}
\pgfmathsetmacro\fg{f(4)}
\fa
\pgfplotsset{scaled y ticks=false}
\begin{tikzpicture}[scale=1]
\begin{axis}[
%/pgf/number format/1000 sep={},
	height=13cm, width=13cm,
%    scale only axis,
	axis lines = left,
	xmin=-2.5, xmax=4.7,
	ymin=-1.6, ymax=11.5,
    xlabel=$x$,ylabel=$f(x)$, 
        xtick={-3,...,4} , 
    ytick={-1,1,2,...,11},
	axis lines=middle,      
every axis x label/.style={
    at={(ticklabel* cs:1.05)},
    anchor=center,
},
every axis y label/.style={
    at={(ticklabel* cs:1.05)},
    anchor=center,
}	   
]
\addplot [
    samples=100, 
    color=blue,
]{f(x)};
\addplot[only marks, black] table {
      X Y		
      -2 \fa
      -1 \fb
      0 \fc
      1 \fd
      2 \fe
      3 \ff
      4 \fg
    }
node[pos=0.0, right] {(-2,11)} 
node[pos=0.16 , left] {(-1,5)}
node[pos=0.33, right] {(0,1)}
node[pos=0.50, anchor=north east] {(1,-1)}
node[pos=0.68, anchor=north west] {(2,-1)}
node[pos=0.80, right] {(3,1)}  
node[pos=1, right] {(4,5)}  
 ;
\end{axis} 
\end{tikzpicture}
\end{figure}

Og sånn vil det være for alle valg av \textit{x} så lenge vi finner punkt ved hjelp av denne funksjonen.

\section{Lineære funksjoner}
I \hr[seksjon]{f1} så vi på en klasse som skulle på tur, og fant at prisen $ P $ per \textit{x} antall elever kunne skrives som funksjonen:
\[ P(x)=200x+5000 \]
Denne funksjonen består av to ledd: Det ene er \textit{x} ganget med en konstant (200), mens det andre er bare en konstant (5000). Når en funksjon består av to slike ledd kaller vi den for en \textit{lineær funksjon}. Dette er fordi alle funksjoner av denne typen vil ha en graf som blir ei rett linje dersom vi tegner den i et koordinatsystem. 

 Si nå at vi ikke visste funksjonsuttrykket til $ P(x) $, men bare fikk se grafen til den: 

\begin{figure}[H]
\centering
\pgfmathdeclarefunction{f}{1}{%
  \pgfmathparse{200*#1 +5000}%
}
\pgfplotsset{scaled y ticks=false}
\begin{tikzpicture}[scale=1]
\begin{axis}[	axis y line=middle, axis x line=bottom,
%/pgf/number format/1000 sep={},
	height=10cm, width=12cm,
%    scale only axis,
%	axis lines = left,
%	xmin=-2.5, xmax=4.5,
	ymin=0, ymax=8000,
%    xlabel=$x$,ylabel=$f(x)$, 
    xtick={0, ..., 14} , 
    ytick={0,1000,...,7000},    
every axis x label/.style={
    at={(ticklabel* cs:1.05)},
    anchor=center,
},
every axis y label/.style={
    at={(ticklabel* cs:1.05)},
    anchor=center,
}	   
]
\addplot [
	domain = 0:15,
    samples=100, 
    color=blue,
]{f(x)};
\end{axis} 
\end{tikzpicture}
\end{figure}

Hvis vi ønsker å finne funksjonsuttrykket kan vi startet med å finne konstanten som står aleine. Når \textit{x} er 0, så blir det ene leddet i funksjonen også lik 0, og derfor er det bare det enslige konstantleddet som står igjen i dette punktet. Når $ x=0 $ er funksjonsverdien lik 5000, dette betyr at konstanten som står aleine også er lik 5000.

\begin{figure}[H]
\centering
\pgfmathdeclarefunction{f}{1}{%
  \pgfmathparse{200*#1 +5000}%
}
\pgfplotsset{scaled y ticks=false}
\begin{tikzpicture}[scale=1]
\begin{axis}[	axis y line=middle, axis x line=bottom,
%/pgf/number format/1000 sep={},
	height=10cm, width=12cm,
%    scale only axis,
%	axis lines = left,
%	xmin=-2.5, xmax=4.5,
	ymin=0, ymax=8000,
%    xlabel=$x$,ylabel=$f(x)$, 
    xtick={0, ..., 14} , 
    ytick={0,1000,...,7000},    
every axis x label/.style={
    at={(ticklabel* cs:1.05)},
    anchor=center,
},
every axis y label/.style={
    at={(ticklabel* cs:1.05)},
    anchor=center,
}	   
]
\addplot [
	domain = 0:15,
    samples=100, 
    color=blue,
]{f(x)}
node[pos=0.0, black, inner sep=0pt, pin=-45:\color{black} den enslige konstanten] {};
\end{axis} 
\end{tikzpicture}
\end{figure}

Tallet som ganges med \textit{x} forteller oss hvor mye prisen stiger for hver elev som blir med. For å finne ut av dette velger vi oss to \textit{x}-verdier som gir punkt det er lett å lese av på figuren:

\begin{figure}[H]
\centering
\pgfmathdeclarefunction{f}{1}{%
  \pgfmathparse{200*#1 +5000}%
}
\pgfplotsset{scaled y ticks=false}
\begin{tikzpicture}[scale=1]
\begin{axis}[	axis y line=middle, axis x line=bottom,
%/pgf/number format/1000 sep={},
	height=10cm, width=12cm,
%    scale only axis,
%	axis lines = left,
%	xmin=-2.5, xmax=4.5,
	ymin=0, ymax=8000,
%    xlabel=$x$,ylabel=$f(x)$, 
    xtick={0, ..., 14} , 
    ytick={0,1000,...,7000},    
every axis x label/.style={
    at={(ticklabel* cs:1.05)},
    anchor=center,
},
every axis y label/.style={
    at={(ticklabel* cs:1.05)},
    anchor=center,
}	   
]
\addplot [
	domain = 0:15,
    samples=100, 
    color=blue,
]{f(x)};
\addplot[dashed,very thin,color=black] coordinates { (5,0) (5,6000) };
\addplot[dashed,very thin,color=black] coordinates { (10,0) (10,7000) };
\addplot[dashed,very thin,color=black] coordinates { (0,6000) (5,6000) };
\addplot[dashed,very thin,color=black] coordinates { (0,7000) (10,7000) };
\end{axis} 
\end{tikzpicture}
\end{figure}

For $ x=5 $ ser vi at prisen blir 6000. Når $ x=10 $ blir den derimot 7000. Dette betyr at prisen har steget med $ 7000-6000=1000 $ når antallet elever økte med $ 10-5=5 $. Prisøkningen per elev blir derfor:
\[ \frac{1000}{5}=200 \]
Tallet som skal ganges med \textit{x} er altså 200. 

Ut ifra det vi har funnet, kan vi nå skrive:
\[ P(x)=200x+5000 \]
Dette stemmer overens med uttrykket som vi egentlig visste fra før av. Generelt kaller vi konstanten som ganges med \textit{x} for \textit{stigningstallet}, mens konstanten som står aleine kaller vi for \textit{konstantleddet}.

\reg[Lineære funksjoner]{Når vi kan skrive en funksjon $ f(x) $ som et tall \textit{a} ganget med \textit{x}, pluss et annet tall \textit{b}, så kaller vi $ f(x) $ for en lineær funksjon:
\[ f(x)=a x + b \]

\textit{a} kaller vi for stigningstallet, mens \textit{b} kaller vi for konstantleddet. De to tallene er definert som:
\[ b=f(0) \]
\[ a=\frac{f(x_2)-f(x_1)}{x_2-x_1} \]
hvor $ x_1 \text{ og } x_2$ er to tilfeldig valgte \textit{x}-verdier og der $ x_2>x_1 $. 

\begin{figure}[H]
\centering
\pgfmathdeclarefunction{f}{1}{%
  \pgfmathparse{0.5*#1 +4}%
}
\pgfplotsset{scaled y ticks=false}
\begin{tikzpicture}[scale=1]
\begin{axis}[	axis y line=middle, axis x line=bottom,
	height=10cm, width=12cm,
    xticklabels={},
    yticklabels={},   
    extra x ticks={0,4, 10}, 
    extra x tick labels={0,$ x_1 $, $ x_2 $ },
    extra y ticks={6, 9}, 
    extra y tick labels={$ f(x_1) $, $ f(x_2) $ },    
%    xlabel style={at={(ticklabel* cs:1)},anchor=north west}, 
every axis x label/.style={
    at={(ticklabel* cs:1.05)},
    anchor=center,
},
every axis y label/.style={
    at={(ticklabel* cs:1.05)},
    anchor=center,
}	   
]
\addplot [
	domain = -1:15,
    samples=100, 
    color=blue,
]{f(x)}
node[pos=0.063, black, inner sep=0pt, pin=-45:\color{black} b] {};
\addplot[dashed,very thin,color=black] coordinates { (4,0) (4,6) };
\addplot[dashed,very thin,color=black] coordinates { (10,0) (10,9) };
\addplot[dashed,very thin,color=black] coordinates { (0,6) (4,6) };
\addplot[dashed,very thin,color=black] coordinates { (0,9) (10,9) };
\end{axis} 
\end{tikzpicture}
\end{figure}
}
\eks{
Prisen $ P $ kroner for å kjøre \textit{x} kilometer i en taxi er gitt ved funksjon:
\[ P(x)= 10x+100 \]
a) Hva må du betale når du har kjørt 0 kilometer?

b) Hvor mye stiger prisen med for hver kilometer du kjører?

c) Hva må du betale dersom taxien kjører deg 20 kilometer? \\

\textbf{Svar:} \vspace{5pt} \\
a) Når $ x=0 $ står vi igjen med bare konstantleddet til \textit{P}, som er 100. Prisen er derfor 100 kr

b) Hvor mye prisen stiger med per kilometer er stigningstallet til funksjonen. Prisen stiger derfor med 10 kr per kilometer.

c) $ P(20)=10\cdot20 + 100=300 $. Altså må vi betale 300 kr for å kjøre 20 kilometer.
}
\end{document}