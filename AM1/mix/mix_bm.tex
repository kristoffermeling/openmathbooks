\documentclass[english, 11 pt, class=article, crop=false]{standalone}
\usepackage[T1]{fontenc}
%\renewcommand*\familydefault{\sfdefault} % For dyslexia-friendly text
\usepackage{lmodern} % load a font with all the characters
\usepackage{geometry}
\geometry{verbose,paperwidth=16.1 cm, paperheight=24 cm, inner=2.3cm, outer=1.8 cm, bmargin=2cm, tmargin=1.8cm}
\setlength{\parindent}{0bp}
\usepackage{import}
\usepackage[subpreambles=false]{standalone}
\usepackage{amsmath}
\usepackage{amssymb}
\usepackage{esint}
\usepackage{babel}
\usepackage{tabu}
\makeatother
\makeatletter

\usepackage{titlesec}
\usepackage{ragged2e}
\RaggedRight
\raggedbottom
\frenchspacing

% Norwegian names of figures, chapters, parts and content
\addto\captionsenglish{\renewcommand{\figurename}{Figur}}
\makeatletter
\addto\captionsenglish{\renewcommand{\chaptername}{Kapittel}}
\addto\captionsenglish{\renewcommand{\partname}{Del}}


\usepackage{graphicx}
\usepackage{float}
\usepackage{subfig}
\usepackage{placeins}
\usepackage{cancel}
\usepackage{framed}
\usepackage{wrapfig}
\usepackage[subfigure]{tocloft}
\usepackage[font=footnotesize,labelfont=sl]{caption} % Figure caption
\usepackage{bm}
\usepackage[dvipsnames, table]{xcolor}
\definecolor{shadecolor}{rgb}{0.105469, 0.613281, 1}
\colorlet{shadecolor}{Emerald!15} 
\usepackage{icomma}
\makeatother
\usepackage[many]{tcolorbox}
\usepackage{multicol}
\usepackage{stackengine}

\usepackage{esvect} %For vectors with capital letters

% For tabular
\usepackage{array}
\usepackage{multirow}
\usepackage{longtable} %breakable table

% Ligningsreferanser
\usepackage{mathtools}
\mathtoolsset{showonlyrefs}

% index
\usepackage{imakeidx}
\makeindex[title=Indeks]

%Footnote:
\usepackage[bottom, hang, flushmargin]{footmisc}
\usepackage{perpage} 
\MakePerPage{footnote}
\addtolength{\footnotesep}{2mm}
\renewcommand{\thefootnote}{\arabic{footnote}}
\renewcommand\footnoterule{\rule{\linewidth}{0.4pt}}
\renewcommand{\thempfootnote}{\arabic{mpfootnote}}

%colors
\definecolor{c1}{cmyk}{0,0.5,1,0}
\definecolor{c2}{cmyk}{1,0.25,1,0}
\definecolor{n3}{cmyk}{1,0.,1,0}
\definecolor{neg}{cmyk}{1,0.,0.,0}

% Lister med bokstavar
\usepackage[inline]{enumitem}

\newcounter{rg}
\numberwithin{rg}{chapter}
\newcommand{\reg}[2][]{\begin{tcolorbox}[boxrule=0.3 mm,arc=0mm,colback=blue!3] {\refstepcounter{rg}\phantomsection \large \textbf{\therg \;#1} \vspace{5 pt}}\newline #2  \end{tcolorbox}\vspace{-5pt}}

\newcommand\alg[1]{\begin{align} #1 \end{align}}

\newcommand\eks[2][]{\begin{tcolorbox}[boxrule=0.3 mm,arc=0mm,enhanced jigsaw,breakable,colback=green!3] {\large \textbf{Eksempel #1} \vspace{5 pt}\\} #2 \end{tcolorbox}\vspace{-5pt} }

\newcommand{\st}[1]{\begin{tcolorbox}[boxrule=0.0 mm,arc=0mm,enhanced jigsaw,breakable,colback=yellow!12]{ #1} \end{tcolorbox}}

\newcommand{\spr}[1]{\begin{tcolorbox}[boxrule=0.3 mm,arc=0mm,enhanced jigsaw,breakable,colback=yellow!7] {\large \textbf{Språkboksen} \vspace{5 pt}\\} #1 \end{tcolorbox}\vspace{-5pt} }

\newcommand{\sym}[1]{\colorbox{blue!15}{#1}}

\newcommand{\info}[2]{\begin{tcolorbox}[boxrule=0.3 mm,arc=0mm,enhanced jigsaw,breakable,colback=cyan!6] {\large \textbf{#1} \vspace{5 pt}\\} #2 \end{tcolorbox}\vspace{-5pt} }

\newcommand\algv[1]{\vspace{-11 pt}\begin{align*} #1 \end{align*}}

\newcommand{\regv}{\vspace{5pt}}
\newcommand{\mer}{\textsl{Merk}: }
\newcommand{\mers}[1]{{\footnotesize \mer #1}}
\newcommand\vsk{\vspace{11pt}}
\newcommand\vs{\vspace{-11pt}}
\newcommand\vsb{\vspace{-16pt}}
\newcommand\sv{\vsk \textbf{Svar} \vspace{4 pt}\\}
\newcommand\br{\\[5 pt]}
\newcommand{\figp}[1]{../fig/#1}
\newcommand\algvv[1]{\vs\vs\begin{align*} #1 \end{align*}}
\newcommand{\y}[1]{$ {#1} $}
\newcommand{\os}{\\[5 pt]}
\newcommand{\prbxl}[2]{
\parbox[l][][l]{#1\linewidth}{#2
	}}
\newcommand{\prbxr}[2]{\parbox[r][][l]{#1\linewidth}{
		\setlength{\abovedisplayskip}{5pt}
		\setlength{\belowdisplayskip}{5pt}	
		\setlength{\abovedisplayshortskip}{0pt}
		\setlength{\belowdisplayshortskip}{0pt} 
		\begin{shaded}
			\footnotesize	#2 \end{shaded}}}

\renewcommand{\cfttoctitlefont}{\Large\bfseries}
\setlength{\cftaftertoctitleskip}{0 pt}
\setlength{\cftbeforetoctitleskip}{0 pt}

\newcommand{\bs}{\\[3pt]}
\newcommand{\vn}{\\[6pt]}
\newcommand{\fig}[1]{\begin{figure}
		\centering
		\includegraphics[]{\figp{#1}}
\end{figure}}

\newcommand{\figc}[2]{\begin{figure}
		\centering
		\includegraphics[]{\figp{#1}}
		\caption{#2}
\end{figure}}

\newcommand{\sectionbreak}{\clearpage} % New page on each section

\newcommand{\nn}[1]{
\begin{equation}
	#1
\end{equation}
}

% Equation comments
\newcommand{\cm}[1]{\llap{\color{blue} #1}}

\newcommand\fork[2]{\begin{tcolorbox}[boxrule=0.3 mm,arc=0mm,enhanced jigsaw,breakable,colback=yellow!7] {\large \textbf{#1 (forklaring)} \vspace{5 pt}\\} #2 \end{tcolorbox}\vspace{-5pt} }
 
%colors
\newcommand{\colr}[1]{{\color{red} #1}}
\newcommand{\colb}[1]{{\color{blue} #1}}
\newcommand{\colo}[1]{{\color{orange} #1}}
\newcommand{\colc}[1]{{\color{cyan} #1}}
\definecolor{projectgreen}{cmyk}{100,0,100,0}
\newcommand{\colg}[1]{{\color{projectgreen} #1}}

% Methods
\newcommand{\metode}[2]{
	\textsl{#1} \\[-8pt]
	\rule{#2}{0.75pt}
}

%Opg
\newcommand{\abc}[1]{
	\begin{enumerate}[label=\alph*),leftmargin=18pt]
		#1
	\end{enumerate}
}
\newcommand{\abcs}[2]{
	\begin{enumerate}[label=\alph*),start=#1,leftmargin=18pt]
		#2
	\end{enumerate}
}
\newcommand{\abcn}[1]{
	\begin{enumerate}[label=\arabic*),leftmargin=18pt]
		#1
	\end{enumerate}
}
\newcommand{\abch}[1]{
	\hspace{-2pt}	\begin{enumerate*}[label=\alph*), itemjoin=\hspace{1cm}]
		#1
	\end{enumerate*}
}
\newcommand{\abchs}[2]{
	\hspace{-2pt}	\begin{enumerate*}[label=\alph*), itemjoin=\hspace{1cm}, start=#1]
		#2
	\end{enumerate*}
}

% Oppgaver
\newcommand{\opgt}{\phantomsection \addcontentsline{toc}{section}{Oppgaver} \section*{Oppgaver for kapittel \thechapter}\vs \setcounter{section}{1}}
\newcounter{opg}
\numberwithin{opg}{section}
\newcommand{\op}[1]{\vspace{15pt} \refstepcounter{opg}\large \textbf{\color{blue}\theopg} \vspace{2 pt} \label{#1} \\}
\newcommand{\ekspop}[1]{\vsk\textbf{Gruble \thechapter.#1}\vspace{2 pt} \\}
\newcommand{\nes}{\stepcounter{section}
	\setcounter{opg}{0}}
\newcommand{\opr}[1]{\vspace{3pt}\textbf{\ref{#1}}}
\newcommand{\oeks}[1]{\begin{tcolorbox}[boxrule=0.3 mm,arc=0mm,colback=white]
		\textit{Eksempel: } #1	  
\end{tcolorbox}}
\newcommand\opgeks[2][]{\begin{tcolorbox}[boxrule=0.1 mm,arc=0mm,enhanced jigsaw,breakable,colback=white] {\footnotesize \textbf{Eksempel #1} \\} \footnotesize #2 \end{tcolorbox}\vspace{-5pt} }
\newcommand{\rknut}{
Rekn ut.
}

%License
\newcommand{\lic}{\textit{Matematikken sine byggesteinar by Sindre Sogge Heggen is licensed under CC BY-NC-SA 4.0. To view a copy of this license, visit\\ 
		\net{http://creativecommons.org/licenses/by-nc-sa/4.0/}{http://creativecommons.org/licenses/by-nc-sa/4.0/}}}

%referances
\newcommand{\net}[2]{{\color{blue}\href{#1}{#2}}}
\newcommand{\hrs}[2]{\hyperref[#1]{\color{blue}\textsl{#2 \ref*{#1}}}}
\newcommand{\rref}[1]{\hrs{#1}{regel}}
\newcommand{\refkap}[1]{\hrs{#1}{kapittel}}
\newcommand{\refsec}[1]{\hrs{#1}{seksjon}}

\newcommand{\mb}{\net{https://sindrsh.github.io/FirstPrinciplesOfMath/}{MB}}


%line to seperate examples
\newcommand{\linje}{\rule{\linewidth}{1pt} }

\usepackage{datetime2}
%%\usepackage{sansmathfonts} for dyslexia-friendly math
\usepackage[]{hyperref}


\newcommand{\note}{Merk}
\newcommand{\notesm}[1]{{\footnotesize \textsl{\note:} #1}}
\newcommand{\ekstitle}{Eksempel }
\newcommand{\sprtitle}{Språkboksen}
\newcommand{\expl}{forklaring}

\newcommand{\vedlegg}[1]{\refstepcounter{vedl}\section*{Vedlegg \thevedl: #1}  \setcounter{vedleq}{0}}

\newcommand\sv{\vsk \textbf{Svar} \vspace{4 pt}\\}

%references
\newcommand{\reftab}[1]{\hrs{#1}{tabell}}
\newcommand{\rref}[1]{\hrs{#1}{regel}}
\newcommand{\dref}[1]{\hrs{#1}{definisjon}}
\newcommand{\refkap}[1]{\hrs{#1}{kapittel}}
\newcommand{\refsec}[1]{\hrs{#1}{seksjon}}
\newcommand{\refdsec}[1]{\hrs{#1}{delseksjon}}
\newcommand{\refved}[1]{\hrs{#1}{vedlegg}}
\newcommand{\eksref}[1]{\textsl{#1}}
\newcommand\fref[2][]{\hyperref[#2]{\textsl{figur \ref*{#2}#1}}}
\newcommand{\refop}[1]{{\color{blue}Oppgave \ref{#1}}}
\newcommand{\refops}[1]{{\color{blue}oppgave \ref{#1}}}
\newcommand{\refgrubs}[1]{{\color{blue}gruble \ref{#1}}}

\newcommand{\openmathser}{\openmath\,-\,serien}

% Exercises
\newcommand{\opgt}{\newpage \phantomsection \addcontentsline{toc}{section}{Oppgaver} \section*{Oppgaver for kapittel \thechapter}\vs \setcounter{section}{1}}


% Sequences and series
\newcommand{\sumarrek}{Summen av en aritmetisk rekke}
\newcommand{\sumgerek}{Summen av en geometrisk rekke}
\newcommand{\regnregsum}{Regneregler for summetegnet}

% Trigonometry
\newcommand{\sincoskomb}{Sinus og cosinus kombinert}
\newcommand{\cosfunk}{Cosinusfunksjonen}
\newcommand{\trid}{Trigonometriske identiteter}
\newcommand{\deravtri}{Den deriverte av de trigonometriske funksjonene}
% Solutions manual
\newcommand{\selos}{Se løsningsforslag.}
\newcommand{\se}[1]{Se eksempel på side \pageref{#1}}

%Vectors
\newcommand{\parvek}{Parallelle vektorer}
\newcommand{\vekpro}{Vektorproduktet}
\newcommand{\vekproarvol}{Vektorproduktet som areal og volum}


% 3D geometries
\newcommand{\linrom}{Linje i rommet}
\newcommand{\avstplnpkt}{Avstand mellom punkt og plan}


% Integral
\newcommand{\bestminten}{Bestemt integral I}
\newcommand{\anfundteo}{Analysens fundamentalteorem}
\newcommand{\intuf}{Integralet av utvalge funksjoner}
\newcommand{\bytvar}{Bytte av variabel}
\newcommand{\intvol}{Integral som volum}
\newcommand{\andordlindif}{Andre ordens lineære differensialligninger}



\begin{document}
\section[Oppgaver med tall og situasjoner fra virkeligheten]{Oppgaver med tall og situasjoner fra virkelig-heten}	
\st{Se også oppgaver på \net{https://ektedata.uib.no/}{ekte.data.uib.no}}

\op{opggaafart}
Vanlig gåfart regnes for å være ca. 1,5\enh{m/s}. Hvor langt kommer man med denne farten
\abc{
	\item etter 25 min?
	\item etter 3 timer?
}	
	
\op{opgtorden}
\tagop{
	\#overslag \#proporsjonalitet
}
Når det er lyn og torden kan du bruke følgende metode for å finne ut omtrent hvor langt unna du er uværet:\os

\st{Start med å telle sekunder straks du ser et lyn. Stopp tellingen når du hører torden. Gang antall sekunder med 300, da har du et overslag på hvor mange meter du er unna uværet.}

Bruk internett til undersøke hastigheten til lys (lyn) og lyd (torden) i luft, og forklar hva denne metoden baserer seg på.
\newpage
\op{opgstigningsgrad}
\tagop{
\#prosent \#negative tall
}
Statens vegvesen definerer \outl{stigningsgrad} slik:
\st{
Stigningsgrad er definert som høydeforskjell dividert med horisontal avstand i vegens lengderetning.
Stigningsgraden uttrykkes vanligvis i \%. Den er positiv i stigning og negativ i fall sett i
profileringsretningen.
}
Si at stigningen på en veistrekke starter 0 meter over havet, og ender opp 8 meter over havet, og at veistrekket er 100 meter langt. Ved bratte veistrekker som dette bruker det å stå skilt som varsler om veiens stigningsgrad.
\abc{
\item Hvilken stigningsgrad vil være oppgitt i dette tilfellet?
\item Hva menes med at stigningsgraden er ''negativ i fall sett i\\ profileringsretningen''?
}

\op{opgstorstar} 
\tagop{
	\# modellering \# areal
}
Gitt et rektangel med omkrets 4, og la $ x $ være den éne sidelengden.
\abc{
	\item  Finn uttrykket til funksjonen $ A(x) $, som viser aralet til rektangelet.
	\item Hva er $ x $ når rektangelet har størst areal? Hvilken form har rektangelet da?
}
\mers{I !!amto finner du en \textit{generalisert} utgave av denne oppgaven. Med andre ord kan det vises at formen du finner i oppgave b) alltid vil være den formen et rektangel må ha for å ha et størst mulig areal i forhold til omkretsen.}

\newpage
\op{opgprogprim}
\tagop{\#programmering \#primtall}
Skriv et skript som lykkes i å finne alle primtall på intervallet\\ 1-100.

\op{opglysar}
\tagop{	\# omgjøring av enheter \# standardform \# proporsjonale størrelser }
Lysets hastighet i vakuum er tilnærmet lik $ 3\cdot10^8\enh{km/s} $.
\abc{
\item Et \outl{lysår} angir distansen et objekt vil reise hvis det beveger seg med lysets hastighet i ett år. Hvor langt er et lysår?
\item Lys bruker ca. 8 minutt på å bevege seg fra Sola til Jorda. Hvor langt er det mellom Sola og Jorda?
}


\op{opgdusj}
\tagop{
	\# omgjøring av enheter \# standardform \# proporsjonale størrelser 
}
Det har blitt populært å regne ut hva det koster å ta seg en dusj. Til et slikt reknestykke kan man gjøre følgende antakelser:
\begin{itemize}
	\item Energien som kreves er energien som må til for å varme opp vannet som gikk med til dusjingen fra 7$^\circ $ til $ 35^\circ $.
	\item For å øke temperaturen til 1 liter vann med 1$ ^\circ $, kreves det $ 4,2\cdot10^3\enh{J} $.
\end{itemize}
Ifølge \net{https://www.vg.no/spesial/2022/stromprisene/}{vg.no} er 645,26 øre/kWh den høyeste (gjennomsnittlige) strømprisen registrert i Oslo. 
\abc{
	\item Regn ut hva en dusj på 10 minutter ville kostet med denne prisen. 
	\item Bruk internett til å finne strømprisene for din region i dag. Sjekk hva en 10 minutters dusj vil koste deg.
}
\textit{\small Obs! I denne oppgaven ser vi bort ifra \outl{nettleie}. Alle strømforbrukere må betale nettleie for frakt av strøm, og jo mer strøm man forbruker, jo høyere vil nettleien være, men strømprisen vil ha mest å si for hvor mye en dusj koster.}
\newpage
\op{opgtannhjul}
\tagop{
	\#faktorisering \#primtall
}
I et system hvor to tannhjul virker sammen, er det ønskelig at en tann og et kammer møtes så sjeldent som mulig. Dette for å unngå slitasje. \vsk

\abc{
	\item Et tannhjul med 12 tenner er koblet til et tannhjul med 21 tenner, som vist i figuren under. Hvor mange omdreininger må det røde tannhjulet ta for at tann A og kammer 1 skal møtes igjen?
	\item Gjenta spørsmålet fra oppgave a), men hvor det blå tannhjulet er erstattet med et tannhjul som har 11 tenner?\footnote{Legg merke til at antall tenner og antall kammer alltid vil være like}.
	\item Hvis to tall ikke har noen annen felles faktor enn 1, sier vi at de er \outl{relativt primiske}. Hvorfor er systemer med tannhjul helst innrettet slik at antall tenner på de forskjellige tannhjulene er relativt primiske?
}
\fig{tannhjul}
\newpage
\linje
Hvis en størrelse har veldig høy verdi, kan det være lettere å \\forestille seg omfanget til størrelsen ved å sammenligne den med en konkret størrelse (i motsetning til å bruke SI-enhetene gram, meter og lignende). Oppgave \ref{opgbluewhale} - \ref{opgregnskog} handler om slike sammenligninger.\\
\linje

\op{opgbluewhale}
\tagop{\#enheter \#regning \#}
Ifølge \net{https://www.wwf.no/dyreleksikon/bl\%C3\%A5hval}{WWF} kan en blåhval veie opp til 200\enh{tonn}, og ifølge \net{https://www.geno.no/om-geno/om-norsk-rodt-fe/karakteristikk-hos-nrf/}{geno.no} kan en norsk NRF-ku veie opp til 650\enh{kg}. Hvor mange NRF-kyr tilsvarer vekten av én blåhval?

\op{opgamazonas}
\tagop{
	\#proporsjonale størrelser \#volum \#overslag
}
\outl{Vannføringen} i en elv viser til volumet vann en elv frakter per tidsenhet. I en \net{https://en.wikipedia.org/wiki/List_of_rivers_by_discharge}{Wikipedia}-artikkel om verdens største elver kan man lese følgende:\regv

\st{
	(...) The average flow rate at the mouth of the Amazon is sufficient to fill more than 83 such pools each second.\vsk
	
	\textsl{\textbf{Norsk oversettelse}\\ (...) Den gjennomsnittlige vannføringen i munningen av Amazonas-elven er nok til å fylle mer enn 83 slike per sekund.}
}

Gjengi utregningene som er brukt for å finne de to tallene i teksten over, når du vet at artikkelen har brukt følgende som utgangspunkt:
\begin{itemize}
	\item Bassenget det er snakk om er et olympisk svømmebasseng, som har har lengde 50\enh{m}, bredde 25\enh{m} og dybde 2\enh{m}.
	\item Den gjennomsnittlige vannføringen i Amazonas-elven er 209\,000\enh{m}$ ^3 $/\text{s}.
\end{itemize}
\newpage
\op{opgregnskog}
\tagop{\#enheter \#areal \#overslag \#proporsjonale størrelser}
I en nyhetssak skrevet av \net{https://edition.cnn.com/2020/06/02/world/tropical-forest-six-seconds-scli-intl/index.html}{CNN} i 2020 står følgende\footnote{Rapporten det vises til er en rapport fra \net{https://www.globalforestwatch.org/}{Global Forest Watch}}:
\st{
The world lost 3.8 million hectares of tropical primary forest in 2019 – equivalent to a football pitch every six seconds – according to a new report published Tuesday.
}
Ta utgangspunkt i tallet ''3.8 million'', undersøk størrelsene ''hectares'' og ''foobtall pitch'', og avgjør om denne påstanden er riktig.



\newpage
\op{opggodot}
\tagop{\#programmering \#potenser}
I spillmotoren \net{https://godotengine.org/}{Godot} har noen objekter egenskapen ''Layer''. Dette brukes til å bestemme hvilke objekter som skal/ikke skal kollidere med hverandre. 
\begin{figure}
	\centering
	\includegraphics[scale=0.7]{\figp{godot_layers}}
\end{figure}
Layers kan bestemmes i editoren, men det er også ønskelig å bruke kode for å endre på disse underveis i et spill. Hvert layer er representert ved en ''Bit'', som viser til en potens med 2 som grunntall og ''Bit'' som eksponent. Verdien til ''Bit'' er alltid 1 mindre enn verdien til ''Layer''. For eksempel, ''Layer 4'' har ''Bit 3'' og ''value 8'' (fordi $ 2^3=8 $). 
\abc{
\item Hvilken ''value'' har ''Layer 8''?
\item For å kode hvilke layers et objekt skal ha, brukes funksjonen \texttt{set\_collision\_layer(value)}, hvor \texttt{value} er summen av verdiene til hvert layer som ønskes valgt. \os

Hva må \texttt{value} settes til dersom du ønsker at et objekt skal ha ''Layer 1'', ''Layer 3'' og ''Layer 7''?
\item Hvis \texttt{value = 320}, hvilke layers har objektet da?
}

\newpage
\op{luftmotstand}
\tagop{
	\#algebra \#modellering \#andregradsfunksjon \\
	\#omgjøring av enheter \#proporsjonalitet
}

La $ F $ være summen av kreftene som virker i motsatt retning av en bils kjøreretning. Ifølge en rapport\footnote{\net{https://sintef.brage.unit.no/sintef-xmlui/handle/11250/2468761}{https://sintef.brage.unit.no/sintef-xmlui/handle/11250/2468761}} fra SINTEF kan\footnote{Det er er her forutsatt flatt strekke, og sett vekk ifra motstand ved akselerasjon.} $ F $ tilnærmes som \vspace{-5pt}
\[ F(v)= mgC_r+\frac{1}{2}\rho v^2 D_m\qquad,\qquad v\geq10\] \vspace{-30pt}
\begin{center}
	\begin{tabular}{c|c|c|c}
		& \textbf{betydning} & \textbf{verdi}&\textbf{enhet}  \\ \hline
		$ v $ & bilens hastighet & variabel& m/s \\
		$ m $& bilens masse\footnotemark & 1409 & kg\\
		$ g $& tyngdeakselerasjonen & 9.81 & m/$ \text{s}^2 $ \\
		$ C_r $ & koeffisient for bilens rullemotstand & 0.015\\
		$ \rho $ & tettheten til luft & 1.25 & kg/$ \text{m}^3 $ \\
		$ D_m $& koeffisient for bilens luftmotstandsareal\footnotemark &0.74
	\end{tabular}
\end{center} \vs

\footnotetext[3]{Det er tatt ugangspunkt i gjennomsnittsvekten til en norsk personbil.}
\footnotetext[4]{Verdien er hentet fra \net{en.wikipedia.org/wiki/Automobile\_drag\_coefficient\#Drag\_area}{en.wikipedia.org/wiki/Automobile\_drag\_coefficient\#Drag\_area}}

\abc{
	\item Tegn digitalt grafen til $ F $ for $ v\in [10, 35] $
	\item På intervallet gitt i oppgave a, for hvilken hastighet er det at 
	\begin{itemize}
		\item rullemotstanden gir det største bidraget til $ F $?
		\item luftmotstanden gir det største bidraget til $ F $?
	\end{itemize} 
	Oppgi svarene rundet av til nærmeste heltall og målt i km/h.
	\item Med ''energiforbruk''\footnote{Den totale energimengden en bil bruker på en kjørelengde vil være høyere enn det vi har kalt ''energiforbruket''. Som regel vil den totale energimengden som kreves for å kjøre en strekning være høyere jo høyere hastighet man har. Slik kan man anta at differansen i energiforbruk vi finner i denne oppgaven er et minimum for den reelle differansen i total energimengde.} mener vi her den energien som må til for å motvirke $ F $ over en viss kjørelengde.
	Ved konstant hastighet er energiforbruket etter kjørt lengde proporsjonal med $ F $. På norske motorveier er 90\,km/h og 110\,km/h vanlige fartsgrenser. Hvor stor økning i energiforbruk vil en økning fra 90\,km/h til 110\,km/h innebære?\os
	
	\item Lag en funksjon $ F_1 $ som gir $ F $ ut ifra bilens hastighet målt i km/h.
}
	
\newpage
\section{Praktiske oppgaver}
\st{Se også oppgaver på \net{https://www.mattelist.no/}{mattelist.no}}

\op{opgrab}
\tagop{\#prosentregning}
Bruk internett til å finne fem varer hvor du har oppgitt følgende:
\begin{itemize}
	\item En original pris og en prosentvis rabatt av denne prisen.
	\item Den nye prisen etter at rabatt er trekt ifra.
\end{itemize}
Undersøk om den nye og den originale prisen samsvarer med rabatten for disse fem varene.

\op{opgstillengde}
Utfør fem stille lengde hopp\footnote{Stille lengde btyr at man skal hoppe med samlede bein, og uten å ta springfart først.}. Finn gjennomstnittslengden og\\ medianlengden for hoppene dine.

\op{opglinjal}
Finn en 20\enh{cm} linjal og og noen små viskelær med lik størrelse. Legg linjalen på et bord slik at 10\enh{cm} av linjalen ligger utenfor kanten av bordet.
\abc{
\item Legg et viskelær på linjalmerket for 5\enh{cm}. Hvor langt ut på linjalen kan du plassere et annet viskelær før linjalen bikker over kanten?  
\item Stable 2 viskelær på linjalmerket for 5\enh{cm}. Hvor langt ut på linjalen kan du plassere et annet viskelær før linjalen bikker over kanten?  
\item Stable 8 viskelær på linjalmerket for 2\enh{cm}. Hvor langt ut på linjalen kan du plassere et annet viskelær før linjalen bikker over kanten? Prøv å svare på spørsmålet før du sjekker svaret i praksis.
}
\newpage
\op{opgstignvater}
\tagop{
\#prosent \#geometri \#algebra \#gjennomsnitt
}
\abc{
\item Beskriv en metode for hvordan du ved hjelp av en vater og et målband/linjal kan bestemme stigningen på et strekke i en bakke.
}
For en bakke som har jevn stigning kan man anslå stigningen på følgende måte
\begin{enumerate}
	\item Lag et merke 5 meter nedenfor et annet merke i bakken. (De 5 metrene måles langs bakken.)
	\item Mål tiden en fotball bruker på å rulle fra det øverste til det nederste merket. Passs på å la ballen trille uten å gi den ekstra fart ved skubbing.
	\item Regn ut gjennomsnittsfarten til ballen mellom de to \\merkene.
	\item Høgdeforskjellen $ h $ mellom de to merkene kan nå tilnærmes ved formelen
	\[ h= \frac{v^2}{19}\]
	hvor $ v $ er gjennomsnittsfarten fra punkt 3. ganget med 2.
	\item La $ l $ være den horisontale  (vannrette) avstanden mellom de to punktene. Da er
	\[ l=\sqrt{5^2-h^2} \]
	\item Den tilnærmede verdien til stigningen er gitt ved brøken $ \frac{h}{l} $.
\end{enumerate}
\abcs{2}{
\item Forklar formelen for $ l $ fra punkt 5.
\item Finn en bakke og sammenlign resultatene fra metoden med en vater og metoden med en fotball.
}

\newpage
\op{opgmaks}
\outl{Makspuls} er et mål på hvor mange hjerteslag hjertet maksimalt kan slå i løpet av et minutt. På siden \href{http://www.trening.no/utholdenhet/ny-formel-for-beregning-av-makspuls/}{\color{blue}trening.no} kan man lese dette:\os
\st{''Den tradisjonelle metoden å estimere maksimalpuls er å ta utgangspunkt i 220 og deretter trekke fra alderen.''}

\textbf{a)} Kall ''maksimalpuls'' for $ m $ og ''alder'' for $ a $ og lag en formel for $ m $ ut i fra sitatet over. \os
\textbf{b)} Bruk formelen fra a) til å regne ut makspulsen din.\vsk

På den samme siden kan vi lese at en ny og bedre metode er slik:\os
''Ta din alder og multipliser dette med 0,64. Deretter trekker du dette fra 211.''\os

\textbf{c)} Lag en formel for $ m $ ut ifra sitatet over.\os

\textbf{d)} Bruk formelen fra c) til å regne ut makspulsen din.

\vsk
For å fysisk måle makspulsen din kan du gjøre dette:
\st{
	\begin{enumerate}
		\item Hopp opp og ned i ca. 30 sekunder (så fort og så høyt du greier).
		\item Tell hjerteslag i 15 sekunder umiddelbart etter hoppingen.
	\end{enumerate}
}
\textbf{e)} Kall ''antall hjerteslag i løpet av 15 sekunder'' for $ A $ og lag en formel for $ m $.\os
\textbf{f)} Bruk formelen fra e) til å regne ut makspulsen din.\os
\textbf{g)} Sammenlign resultatene fra b), d) og f).

\newpage

\op{opgstorlkilopris}
I matbutikker er som regel både pris og kilopris oppgitt for en vare. Vekten til varen finner man på forpakningen til varen. Gå i din lokale matbutikk og velg ut fem varer. Sjekk om kiloprisen som butikken oppgir er rett for disse varene.

\op{opgtrapp}
På nettsiden \net{http://www.viivilla.no/}{viivilla.no} får vi vite at dette er formelen for å lage en perfekt trapp:
\st{''2 ganger opptrinn (trinnhøyde) pluss 1 gang inntrinn (trinndybde) bør bli 62 centimeter (med et slingringsmonn på et par centimeter).''}
\textbf{a)} Kall ''trinnhøyden'' for $ h $ og ''trinndybden'' for $ d $ og skriv opp formelen i sitatet (uten slingringsmonn).\os
\textbf{b)} Sjekk trappene på skolen/i huset ditt, er formelen oppfylt eller ikke?\os

\textbf{c)} Hvis ikke: Hva måtte trinnhøyden vært for at formelen skulle blitt oppfylt?\os

\textbf{d)} Skriv om formelen til en formel for $ h $.



\newpage
\section{Eksamensoppgaver}
\st{
\eksbm
}

\eksop{GE22}{geks22opg5}
\tagop{
\#regning \#omgjøring av enheter
}
Snorre skal kjøpe ny mobiltelefon.
Ved betaling får han to alternativer:
\begin{itemize}
	\item Alternativ 1: Betal 12 000 kr med en gang
	\item Alternativ 2: Betal 550 kr per måned i to år.
\end{itemize}
Snorre velger alternativ 2. \os

Hvor mye dyrere blir mobiltelefonen med \textsl{alternativ 2} enn med \textsl{alternativ 1}?

\eksop{GV21D1}{GV21D1opg2}
\tagop{
\#regnerekkefølge \#potenser
}
Regn ut $ 3(2+5)-3^2  $

\eksop{GV21D1}{GV21D1opg3}
\tagop{
	\# pi \#rottuttrykk \#desimaltall \#brøk
}
Sorter tallene fra størst til minst
\[ 3,1\qquad\sqrt{9}\qquad2,9\qquad\pi \qquad \frac{32}{10} \]

\eksop{GV21D1}{GV21D1opg6}
\tagop{
\#kobinasjonsregning \#logikk
}
Det er 29 bokstaver i alfabetet vårt, og tallsystemet vi vanligvis bruker, har 10 siffer.
Du skal lage en kode med seks tegn. De to første skal være bokstaver, og de fire neste skal være
siffer. En slik kode kan for eksempel være YA6505. \os

Hvilket av tallene under gir antall ulike koder det er mulig å lage? \os
\abch{
\item 290
\item 2320
\item 4\,092\,480
\item 8\,410\,000
}

\eksop{GV21D1}{GV21D1opg9}
\tagop{
\#ligninger \#formler
}
Sett $ n=120 $, og bestem verdien av $ p $ i formelen nedenfor. 
\[ n = 12p + 48 \]

\eksop{GV21D1}{GV21D1opg1} 
\tagop{
\#enheter \#regning
}
En løpebane på en friidrettsbane er 400 meter.
Cornelia løper 6 runder. Hvor langt løper Cornelia? \os

\mers{I eksamenssettet er dette den første av to deloppgaver. Den andre deloppgaven er altså utelatt her.}

\eksop{1PV22D1}{1PV22D1opg5}
\tagop{
\#programmering \#prosentregning
}
\python{python/1pv22d1opg5.py}
En elev har skrevet programkoden ovenfor.
Hva ønsker eleven å finne ut?
Forklar hva som skjer når programmet kjøres.
\newpage
\eksop{1PV22D1}{1PV22D1opg2}
\tagop{\#prosentregning \#statistikk \#tallforståelse}
\fig{1pv22d1opg2}	
Diagrammet viser antall elever ved en videregående skole de fire siste årene.\os

Når var det størst prosentvis økning i antall elever fra et år til det neste?
	


\end{document}

