\documentclass[english, 11 pt, class=article, crop=false]{standalone}
\usepackage[T1]{fontenc}
%\renewcommand*\familydefault{\sfdefault} % For dyslexia-friendly text
\usepackage{lmodern} % load a font with all the characters
\usepackage{geometry}
\geometry{verbose,paperwidth=16.1 cm, paperheight=24 cm, inner=2.3cm, outer=1.8 cm, bmargin=2cm, tmargin=1.8cm}
\setlength{\parindent}{0bp}
\usepackage{import}
\usepackage[subpreambles=false]{standalone}
\usepackage{amsmath}
\usepackage{amssymb}
\usepackage{esint}
\usepackage{babel}
\usepackage{tabu}
\makeatother
\makeatletter

\usepackage{titlesec}
\usepackage{ragged2e}
\RaggedRight
\raggedbottom
\frenchspacing

% Norwegian names of figures, chapters, parts and content
\addto\captionsenglish{\renewcommand{\figurename}{Figur}}
\makeatletter
\addto\captionsenglish{\renewcommand{\chaptername}{Kapittel}}
\addto\captionsenglish{\renewcommand{\partname}{Del}}


\usepackage{graphicx}
\usepackage{float}
\usepackage{subfig}
\usepackage{placeins}
\usepackage{cancel}
\usepackage{framed}
\usepackage{wrapfig}
\usepackage[subfigure]{tocloft}
\usepackage[font=footnotesize,labelfont=sl]{caption} % Figure caption
\usepackage{bm}
\usepackage[dvipsnames, table]{xcolor}
\definecolor{shadecolor}{rgb}{0.105469, 0.613281, 1}
\colorlet{shadecolor}{Emerald!15} 
\usepackage{icomma}
\makeatother
\usepackage[many]{tcolorbox}
\usepackage{multicol}
\usepackage{stackengine}

\usepackage{esvect} %For vectors with capital letters

% For tabular
\usepackage{array}
\usepackage{multirow}
\usepackage{longtable} %breakable table

% Ligningsreferanser
\usepackage{mathtools}
\mathtoolsset{showonlyrefs}

% index
\usepackage{imakeidx}
\makeindex[title=Indeks]

%Footnote:
\usepackage[bottom, hang, flushmargin]{footmisc}
\usepackage{perpage} 
\MakePerPage{footnote}
\addtolength{\footnotesep}{2mm}
\renewcommand{\thefootnote}{\arabic{footnote}}
\renewcommand\footnoterule{\rule{\linewidth}{0.4pt}}
\renewcommand{\thempfootnote}{\arabic{mpfootnote}}

%colors
\definecolor{c1}{cmyk}{0,0.5,1,0}
\definecolor{c2}{cmyk}{1,0.25,1,0}
\definecolor{n3}{cmyk}{1,0.,1,0}
\definecolor{neg}{cmyk}{1,0.,0.,0}

% Lister med bokstavar
\usepackage[inline]{enumitem}

\newcounter{rg}
\numberwithin{rg}{chapter}
\newcommand{\reg}[2][]{\begin{tcolorbox}[boxrule=0.3 mm,arc=0mm,colback=blue!3] {\refstepcounter{rg}\phantomsection \large \textbf{\therg \;#1} \vspace{5 pt}}\newline #2  \end{tcolorbox}\vspace{-5pt}}

\newcommand\alg[1]{\begin{align} #1 \end{align}}

\newcommand\eks[2][]{\begin{tcolorbox}[boxrule=0.3 mm,arc=0mm,enhanced jigsaw,breakable,colback=green!3] {\large \textbf{Eksempel #1} \vspace{5 pt}\\} #2 \end{tcolorbox}\vspace{-5pt} }

\newcommand{\st}[1]{\begin{tcolorbox}[boxrule=0.0 mm,arc=0mm,enhanced jigsaw,breakable,colback=yellow!12]{ #1} \end{tcolorbox}}

\newcommand{\spr}[1]{\begin{tcolorbox}[boxrule=0.3 mm,arc=0mm,enhanced jigsaw,breakable,colback=yellow!7] {\large \textbf{Språkboksen} \vspace{5 pt}\\} #1 \end{tcolorbox}\vspace{-5pt} }

\newcommand{\sym}[1]{\colorbox{blue!15}{#1}}

\newcommand{\info}[2]{\begin{tcolorbox}[boxrule=0.3 mm,arc=0mm,enhanced jigsaw,breakable,colback=cyan!6] {\large \textbf{#1} \vspace{5 pt}\\} #2 \end{tcolorbox}\vspace{-5pt} }

\newcommand\algv[1]{\vspace{-11 pt}\begin{align*} #1 \end{align*}}

\newcommand{\regv}{\vspace{5pt}}
\newcommand{\mer}{\textsl{Merk}: }
\newcommand{\mers}[1]{{\footnotesize \mer #1}}
\newcommand\vsk{\vspace{11pt}}
\newcommand\vs{\vspace{-11pt}}
\newcommand\vsb{\vspace{-16pt}}
\newcommand\sv{\vsk \textbf{Svar} \vspace{4 pt}\\}
\newcommand\br{\\[5 pt]}
\newcommand{\figp}[1]{../fig/#1}
\newcommand\algvv[1]{\vs\vs\begin{align*} #1 \end{align*}}
\newcommand{\y}[1]{$ {#1} $}
\newcommand{\os}{\\[5 pt]}
\newcommand{\prbxl}[2]{
\parbox[l][][l]{#1\linewidth}{#2
	}}
\newcommand{\prbxr}[2]{\parbox[r][][l]{#1\linewidth}{
		\setlength{\abovedisplayskip}{5pt}
		\setlength{\belowdisplayskip}{5pt}	
		\setlength{\abovedisplayshortskip}{0pt}
		\setlength{\belowdisplayshortskip}{0pt} 
		\begin{shaded}
			\footnotesize	#2 \end{shaded}}}

\renewcommand{\cfttoctitlefont}{\Large\bfseries}
\setlength{\cftaftertoctitleskip}{0 pt}
\setlength{\cftbeforetoctitleskip}{0 pt}

\newcommand{\bs}{\\[3pt]}
\newcommand{\vn}{\\[6pt]}
\newcommand{\fig}[1]{\begin{figure}
		\centering
		\includegraphics[]{\figp{#1}}
\end{figure}}

\newcommand{\figc}[2]{\begin{figure}
		\centering
		\includegraphics[]{\figp{#1}}
		\caption{#2}
\end{figure}}

\newcommand{\sectionbreak}{\clearpage} % New page on each section

\newcommand{\nn}[1]{
\begin{equation}
	#1
\end{equation}
}

% Equation comments
\newcommand{\cm}[1]{\llap{\color{blue} #1}}

\newcommand\fork[2]{\begin{tcolorbox}[boxrule=0.3 mm,arc=0mm,enhanced jigsaw,breakable,colback=yellow!7] {\large \textbf{#1 (forklaring)} \vspace{5 pt}\\} #2 \end{tcolorbox}\vspace{-5pt} }
 
%colors
\newcommand{\colr}[1]{{\color{red} #1}}
\newcommand{\colb}[1]{{\color{blue} #1}}
\newcommand{\colo}[1]{{\color{orange} #1}}
\newcommand{\colc}[1]{{\color{cyan} #1}}
\definecolor{projectgreen}{cmyk}{100,0,100,0}
\newcommand{\colg}[1]{{\color{projectgreen} #1}}

% Methods
\newcommand{\metode}[2]{
	\textsl{#1} \\[-8pt]
	\rule{#2}{0.75pt}
}

%Opg
\newcommand{\abc}[1]{
	\begin{enumerate}[label=\alph*),leftmargin=18pt]
		#1
	\end{enumerate}
}
\newcommand{\abcs}[2]{
	\begin{enumerate}[label=\alph*),start=#1,leftmargin=18pt]
		#2
	\end{enumerate}
}
\newcommand{\abcn}[1]{
	\begin{enumerate}[label=\arabic*),leftmargin=18pt]
		#1
	\end{enumerate}
}
\newcommand{\abch}[1]{
	\hspace{-2pt}	\begin{enumerate*}[label=\alph*), itemjoin=\hspace{1cm}]
		#1
	\end{enumerate*}
}
\newcommand{\abchs}[2]{
	\hspace{-2pt}	\begin{enumerate*}[label=\alph*), itemjoin=\hspace{1cm}, start=#1]
		#2
	\end{enumerate*}
}

% Oppgaver
\newcommand{\opgt}{\phantomsection \addcontentsline{toc}{section}{Oppgaver} \section*{Oppgaver for kapittel \thechapter}\vs \setcounter{section}{1}}
\newcounter{opg}
\numberwithin{opg}{section}
\newcommand{\op}[1]{\vspace{15pt} \refstepcounter{opg}\large \textbf{\color{blue}\theopg} \vspace{2 pt} \label{#1} \\}
\newcommand{\ekspop}[1]{\vsk\textbf{Gruble \thechapter.#1}\vspace{2 pt} \\}
\newcommand{\nes}{\stepcounter{section}
	\setcounter{opg}{0}}
\newcommand{\opr}[1]{\vspace{3pt}\textbf{\ref{#1}}}
\newcommand{\oeks}[1]{\begin{tcolorbox}[boxrule=0.3 mm,arc=0mm,colback=white]
		\textit{Eksempel: } #1	  
\end{tcolorbox}}
\newcommand\opgeks[2][]{\begin{tcolorbox}[boxrule=0.1 mm,arc=0mm,enhanced jigsaw,breakable,colback=white] {\footnotesize \textbf{Eksempel #1} \\} \footnotesize #2 \end{tcolorbox}\vspace{-5pt} }
\newcommand{\rknut}{
Rekn ut.
}

%License
\newcommand{\lic}{\textit{Matematikken sine byggesteinar by Sindre Sogge Heggen is licensed under CC BY-NC-SA 4.0. To view a copy of this license, visit\\ 
		\net{http://creativecommons.org/licenses/by-nc-sa/4.0/}{http://creativecommons.org/licenses/by-nc-sa/4.0/}}}

%referances
\newcommand{\net}[2]{{\color{blue}\href{#1}{#2}}}
\newcommand{\hrs}[2]{\hyperref[#1]{\color{blue}\textsl{#2 \ref*{#1}}}}
\newcommand{\rref}[1]{\hrs{#1}{regel}}
\newcommand{\refkap}[1]{\hrs{#1}{kapittel}}
\newcommand{\refsec}[1]{\hrs{#1}{seksjon}}

\newcommand{\mb}{\net{https://sindrsh.github.io/FirstPrinciplesOfMath/}{MB}}


%line to seperate examples
\newcommand{\linje}{\rule{\linewidth}{1pt} }

\usepackage{datetime2}
%%\usepackage{sansmathfonts} for dyslexia-friendly math
\usepackage[]{hyperref}


\newcommand{\note}{Merk}
\newcommand{\notesm}[1]{{\footnotesize \textsl{\note:} #1}}
\newcommand{\ekstitle}{Eksempel }
\newcommand{\sprtitle}{Språkboksen}
\newcommand{\expl}{forklaring}

\newcommand{\vedlegg}[1]{\refstepcounter{vedl}\section*{Vedlegg \thevedl: #1}  \setcounter{vedleq}{0}}

\newcommand\sv{\vsk \textbf{Svar} \vspace{4 pt}\\}

%references
\newcommand{\reftab}[1]{\hrs{#1}{tabell}}
\newcommand{\rref}[1]{\hrs{#1}{regel}}
\newcommand{\dref}[1]{\hrs{#1}{definisjon}}
\newcommand{\refkap}[1]{\hrs{#1}{kapittel}}
\newcommand{\refsec}[1]{\hrs{#1}{seksjon}}
\newcommand{\refdsec}[1]{\hrs{#1}{delseksjon}}
\newcommand{\refved}[1]{\hrs{#1}{vedlegg}}
\newcommand{\eksref}[1]{\textsl{#1}}
\newcommand\fref[2][]{\hyperref[#2]{\textsl{figur \ref*{#2}#1}}}
\newcommand{\refop}[1]{{\color{blue}Oppgave \ref{#1}}}
\newcommand{\refops}[1]{{\color{blue}oppgave \ref{#1}}}
\newcommand{\refgrubs}[1]{{\color{blue}gruble \ref{#1}}}

\newcommand{\openmathser}{\openmath\,-\,serien}

% Exercises
\newcommand{\opgt}{\newpage \phantomsection \addcontentsline{toc}{section}{Oppgaver} \section*{Oppgaver for kapittel \thechapter}\vs \setcounter{section}{1}}


% Sequences and series
\newcommand{\sumarrek}{Summen av en aritmetisk rekke}
\newcommand{\sumgerek}{Summen av en geometrisk rekke}
\newcommand{\regnregsum}{Regneregler for summetegnet}

% Trigonometry
\newcommand{\sincoskomb}{Sinus og cosinus kombinert}
\newcommand{\cosfunk}{Cosinusfunksjonen}
\newcommand{\trid}{Trigonometriske identiteter}
\newcommand{\deravtri}{Den deriverte av de trigonometriske funksjonene}
% Solutions manual
\newcommand{\selos}{Se løsningsforslag.}
\newcommand{\se}[1]{Se eksempel på side \pageref{#1}}

%Vectors
\newcommand{\parvek}{Parallelle vektorer}
\newcommand{\vekpro}{Vektorproduktet}
\newcommand{\vekproarvol}{Vektorproduktet som areal og volum}


% 3D geometries
\newcommand{\linrom}{Linje i rommet}
\newcommand{\avstplnpkt}{Avstand mellom punkt og plan}


% Integral
\newcommand{\bestminten}{Bestemt integral I}
\newcommand{\anfundteo}{Analysens fundamentalteorem}
\newcommand{\intuf}{Integralet av utvalge funksjoner}
\newcommand{\bytvar}{Bytte av variabel}
\newcommand{\intvol}{Integral som volum}
\newcommand{\andordlindif}{Andre ordens lineære differensialligninger}



\usepackage{xr}
\externaldocument{../AM_bm}

\begin{document}
\subsubsection*{Kapittel\ref{Storlogenh}}
\opr{opggjeromtilm}
\abch{
	\item 484\,000\enh{m}
	\item 91\,000\enh{m}
	\item 2\,402\,000\enh{m}
}

\opr{opggjeromtilg}
\abch{
	\item 484\,000\enh{g}
	\item 9\,100\enh{g}
	\item 240\,200\enh{g}
}

\opr{opggjeromtilL}
\abch{
	\item 48\enh{l}
	\item 91\enh{l}
	\item 240\enh{cl}
}

\opr{opggjeromblanda} \vs
\begin{multicols}{3}
	\abc{
		\item 0,0124\enh{km}
		\item 4,2\enh{m}
		\item 581,5\enh{mm}
		\item 7,4\enh{m}
		\item 15\enh{cm}
		
		\item 0,097\enh{hg}
		\item 0,00015\enh{g}
		\item 141\,900\enh{mg}
		\item 0,00031\enh{hg}
		\item 0,064039\enh{kg}
		\item 8,9\enh{l}
		\item 69\,140\enh{cl}
		\item 15000\enh{ml}
		\item 9,18\enh{l}
		\item 55\enh{ml}
	}
\end{multicols}

\opr{opgvolprisl} $ 720\enh{cm}^3 $


\opr{opgvolkjegl}
\abch{
	\item $ 32\enh{dm}^3 $
	\item $ 32\enh{l} $
}

\opr{opgvolpyrl}
\abch{
	\item $ 120\enh{cm}^3 $
	\item $ 0,12\enh{l} $
}

\nes

\opr{opgstorlbolt}
\abch{
\item Ca. $ 10,19\enh{m/s} $
\item Han startet med $ 0\enh{m/s} $ som fart, og trengte de første metrene til å akselerere.
\item Ca. $ 12,35\enh{m/s} $.
}

\nes
\opr{opgstorlbolt2} Ca.  $ 36,68\enh{m/s} $ og ca. $ 44.46\enh{m/s} $

\opr{opgstorlfinn} 
Skriv ned eksempel på et dyr, et insekt, en gjenstand eller annet som veier mellom 1-100\,mg, cg, dg, g, dag, hg og kg.


\subsubsection*{Kapittel \ref{Statistikk}}
\opr{opgstatsentrodd1}
\abch{
	\item 2
	\item 3
	\item 5 
}

\opr{opgstatsentrodd2}
\abch{
\item 8 
\item 6 
\item $ \frac{67}{11}$
}

\opr{opgstatsentrpar1}
\abch{
\item 5, 8 og 16
\item 8
\item $ 8,5 $
}

\opr{opgstatsentrpar2}
\abch{
\item 5 og 11
\item 8.5
\item 9
}

\opr{opgrikest}
\abch{
\item 3,2
\item 4185.48
\item Medianen
}


\opr{opgfrkvtb1} \selos

\opr{opgfrkvtb2} \selos

\opr{opgstatsoyl}

\opr{statsoylrgn} 
\newpage

\opr{stathvorfor}
Av de fire undersøkelsene på side \pageref{undersok}, hvorfor har vi
\abc{
	\item I undersøkelse 1 har hver verdi frekvens lik 1, og da er det unødvendig å lage en frekvenstabell. Punktene i undersøkelse 3 gir samme informasjon som en frekvenstabell. Informasjonen gitt i undersøkelse 4 er allerede gitt i form av en frekvenstabell.
	\item vist søylediagram bare for undersøkelse  2 og 3?
	\item vist sektordiagram bare for undersøkelse 2 og ?
	\item vist linjediagram bare for undersøkelse 4?
}

\opr{stathvorfor2} Spredningsmål gir bare mening for tallverdier.

\subsection*{Kapittel \label{Brok}}
\opr{finbrokdel}
\abch{
	\item  6
	\item  15
	\item  42
	\item  80
}

\opr{finnbrokdel2}
\abch{
	\item $ \frac{8}{15} $
	\item $ \frac{48}{77} $
	\item $ \frac{9}{65} $ 
}

\opr{brokfirma} 320\,000\,kr


\opr{sktilpro}
\abch{
	\item $ 78\% $ 
	\item $ 91,2\%$
	\item $ 0,7\% $
	\item $ 193,54\% $ 
}

\opr{proverdi}
\abch{
	\item 0,57 
	\item 0,981
	\item 2,19
	\item 0,003
}

\opr{sktilpro}
\abch{
	\item $ 70\% $
	\item $ 22\% $ 
	\item $ 36\% $
	\item $ 145\% $
}

\op{finnpro}
Finn \os
\abch{
	\item 100
	\item 250
	\item 63
	\item 560
	\item 30
}

\newpage
\op{proav}
\abch{
	\item 40\%
	\item 25\%
	\item ca 42,86\%
	\item ca 22,22\%
}

\opr{proundsk2}

\opr{prook} 
\abch{
	\item 44
	\item 325
	\item 1008
	\item 649
	\item 200 
}

\op{prored} \vs
\abc{
	\item 36
	\item 175
	\item 112
}
\op{bitcoin}
Du kjøper en hest for 20\,000\enh{kr}, og håper at verdien til hesten vil stige med 8\% i løpet av et år. Hvor mye er den i så fall verd da?

\op{pckjop}
Du kjøper en ny gaming-PC til 20\,000\enh{kr}, og regner med at verdien til PCen vil synke med 12\% i løpet av et år. Hvor mye er den i så fall verd da?

\op{opgbrokpp}
Si at originalprisen på en bukse er 500\enh{kr}. Først ble det gitt 20\% rabatt på denne prisen, men etter en stund ble det gitt 30\% rabatt. Avgjør hvilke av utsagnene under som er sann/ikke sann
\begin{enumerate}[label=(\roman*)]
	\item Når rabatten gikk fra å være 20\% til å være 30\%, ble originalprisen redusert med 10\%.
	\item Når rabatten gikk fra å være 20\% til å være 30\%, økte rabatten med 10\%.
	\item Når rabatten gikk fra å være 20\% til å være 30\%, økte rabatten med 10 prosentpoeng.	
\end{enumerate}

\nes

\op{finnvekstf1} \vs
\abc{
	\item Finn vekstfaktoren fra oppgave \ref{prook}a).
	\item Finn vekstfaktoren fra oppgave \ref{prook}b).
	\item Finn vekstfaktoren fra oppgave \ref{prook}c).
}

\op{finnvesktf2} \vs
\abc{
	\item Finn vekstfaktoren fra oppgave \ref{prored}a).
	\item Finn vekstfaktoren fra oppgave \ref{prored}b).
	\item Finn vekstfaktoren fra oppgave \ref{prored}c).
}

\opr{forh2}
Finn forholdet og forholdstallet mellom antall hester og griser når vi har:\os
\begin{tabular}{@{}l l l}	
	\textbf{a)} 5 hester og 2 griser. &\textbf{b)} 12 griser og 4 hester.
\end{tabular}

\newpage
\opr{forh2}


\vsk \vspace{12pt}
\begin{comment}
Oppgave om hvilket dyr som er sterkest i forhold til vekten. Skaraben er verdens sterkeste.
\end{comment}




\op{forh}
De fleste brus inneholder ca 10\enh{g} karbohydrater per 100\enh{g}. En type saftsirup inneholder 44\enh{g} karbohydrater per 100\enh{g}. Saften skal lages med 2 deler sirup og 9 deler vann. \os

Inneholder saften mer eller mindre karbohydrater per 100\,g enn en brus? \os

\mers{I denne oppgaven går vi ut ifra at både 1\,dl vann og 1\,dl saftsirup veier 100\,g.}


\end{document}

