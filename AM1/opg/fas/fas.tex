\documentclass[english, 11 pt, class=article, crop=false]{standalone}
\usepackage[T1]{fontenc}
%\renewcommand*\familydefault{\sfdefault} % For dyslexia-friendly text
\usepackage{lmodern} % load a font with all the characters
\usepackage{geometry}
\geometry{verbose,paperwidth=16.1 cm, paperheight=24 cm, inner=2.3cm, outer=1.8 cm, bmargin=2cm, tmargin=1.8cm}
\setlength{\parindent}{0bp}
\usepackage{import}
\usepackage[subpreambles=false]{standalone}
\usepackage{amsmath}
\usepackage{amssymb}
\usepackage{esint}
\usepackage{babel}
\usepackage{tabu}
\makeatother
\makeatletter

\usepackage{titlesec}
\usepackage{ragged2e}
\RaggedRight
\raggedbottom
\frenchspacing

% Norwegian names of figures, chapters, parts and content
\addto\captionsenglish{\renewcommand{\figurename}{Figur}}
\makeatletter
\addto\captionsenglish{\renewcommand{\chaptername}{Kapittel}}
\addto\captionsenglish{\renewcommand{\partname}{Del}}


\usepackage{graphicx}
\usepackage{float}
\usepackage{subfig}
\usepackage{placeins}
\usepackage{cancel}
\usepackage{framed}
\usepackage{wrapfig}
\usepackage[subfigure]{tocloft}
\usepackage[font=footnotesize,labelfont=sl]{caption} % Figure caption
\usepackage{bm}
\usepackage[dvipsnames, table]{xcolor}
\definecolor{shadecolor}{rgb}{0.105469, 0.613281, 1}
\colorlet{shadecolor}{Emerald!15} 
\usepackage{icomma}
\makeatother
\usepackage[many]{tcolorbox}
\usepackage{multicol}
\usepackage{stackengine}

\usepackage{esvect} %For vectors with capital letters

% For tabular
\usepackage{array}
\usepackage{multirow}
\usepackage{longtable} %breakable table

% Ligningsreferanser
\usepackage{mathtools}
\mathtoolsset{showonlyrefs}

% index
\usepackage{imakeidx}
\makeindex[title=Indeks]

%Footnote:
\usepackage[bottom, hang, flushmargin]{footmisc}
\usepackage{perpage} 
\MakePerPage{footnote}
\addtolength{\footnotesep}{2mm}
\renewcommand{\thefootnote}{\arabic{footnote}}
\renewcommand\footnoterule{\rule{\linewidth}{0.4pt}}
\renewcommand{\thempfootnote}{\arabic{mpfootnote}}

%colors
\definecolor{c1}{cmyk}{0,0.5,1,0}
\definecolor{c2}{cmyk}{1,0.25,1,0}
\definecolor{n3}{cmyk}{1,0.,1,0}
\definecolor{neg}{cmyk}{1,0.,0.,0}

% Lister med bokstavar
\usepackage[inline]{enumitem}

\newcounter{rg}
\numberwithin{rg}{chapter}
\newcommand{\reg}[2][]{\begin{tcolorbox}[boxrule=0.3 mm,arc=0mm,colback=blue!3] {\refstepcounter{rg}\phantomsection \large \textbf{\therg \;#1} \vspace{5 pt}}\newline #2  \end{tcolorbox}\vspace{-5pt}}

\newcommand\alg[1]{\begin{align} #1 \end{align}}

\newcommand\eks[2][]{\begin{tcolorbox}[boxrule=0.3 mm,arc=0mm,enhanced jigsaw,breakable,colback=green!3] {\large \textbf{Eksempel #1} \vspace{5 pt}\\} #2 \end{tcolorbox}\vspace{-5pt} }

\newcommand{\st}[1]{\begin{tcolorbox}[boxrule=0.0 mm,arc=0mm,enhanced jigsaw,breakable,colback=yellow!12]{ #1} \end{tcolorbox}}

\newcommand{\spr}[1]{\begin{tcolorbox}[boxrule=0.3 mm,arc=0mm,enhanced jigsaw,breakable,colback=yellow!7] {\large \textbf{Språkboksen} \vspace{5 pt}\\} #1 \end{tcolorbox}\vspace{-5pt} }

\newcommand{\sym}[1]{\colorbox{blue!15}{#1}}

\newcommand{\info}[2]{\begin{tcolorbox}[boxrule=0.3 mm,arc=0mm,enhanced jigsaw,breakable,colback=cyan!6] {\large \textbf{#1} \vspace{5 pt}\\} #2 \end{tcolorbox}\vspace{-5pt} }

\newcommand\algv[1]{\vspace{-11 pt}\begin{align*} #1 \end{align*}}

\newcommand{\regv}{\vspace{5pt}}
\newcommand{\mer}{\textsl{Merk}: }
\newcommand{\mers}[1]{{\footnotesize \mer #1}}
\newcommand\vsk{\vspace{11pt}}
\newcommand\vs{\vspace{-11pt}}
\newcommand\vsb{\vspace{-16pt}}
\newcommand\sv{\vsk \textbf{Svar} \vspace{4 pt}\\}
\newcommand\br{\\[5 pt]}
\newcommand{\figp}[1]{../fig/#1}
\newcommand\algvv[1]{\vs\vs\begin{align*} #1 \end{align*}}
\newcommand{\y}[1]{$ {#1} $}
\newcommand{\os}{\\[5 pt]}
\newcommand{\prbxl}[2]{
\parbox[l][][l]{#1\linewidth}{#2
	}}
\newcommand{\prbxr}[2]{\parbox[r][][l]{#1\linewidth}{
		\setlength{\abovedisplayskip}{5pt}
		\setlength{\belowdisplayskip}{5pt}	
		\setlength{\abovedisplayshortskip}{0pt}
		\setlength{\belowdisplayshortskip}{0pt} 
		\begin{shaded}
			\footnotesize	#2 \end{shaded}}}

\renewcommand{\cfttoctitlefont}{\Large\bfseries}
\setlength{\cftaftertoctitleskip}{0 pt}
\setlength{\cftbeforetoctitleskip}{0 pt}

\newcommand{\bs}{\\[3pt]}
\newcommand{\vn}{\\[6pt]}
\newcommand{\fig}[1]{\begin{figure}
		\centering
		\includegraphics[]{\figp{#1}}
\end{figure}}

\newcommand{\figc}[2]{\begin{figure}
		\centering
		\includegraphics[]{\figp{#1}}
		\caption{#2}
\end{figure}}

\newcommand{\sectionbreak}{\clearpage} % New page on each section

\newcommand{\nn}[1]{
\begin{equation}
	#1
\end{equation}
}

% Equation comments
\newcommand{\cm}[1]{\llap{\color{blue} #1}}

\newcommand\fork[2]{\begin{tcolorbox}[boxrule=0.3 mm,arc=0mm,enhanced jigsaw,breakable,colback=yellow!7] {\large \textbf{#1 (forklaring)} \vspace{5 pt}\\} #2 \end{tcolorbox}\vspace{-5pt} }
 
%colors
\newcommand{\colr}[1]{{\color{red} #1}}
\newcommand{\colb}[1]{{\color{blue} #1}}
\newcommand{\colo}[1]{{\color{orange} #1}}
\newcommand{\colc}[1]{{\color{cyan} #1}}
\definecolor{projectgreen}{cmyk}{100,0,100,0}
\newcommand{\colg}[1]{{\color{projectgreen} #1}}

% Methods
\newcommand{\metode}[2]{
	\textsl{#1} \\[-8pt]
	\rule{#2}{0.75pt}
}

%Opg
\newcommand{\abc}[1]{
	\begin{enumerate}[label=\alph*),leftmargin=18pt]
		#1
	\end{enumerate}
}
\newcommand{\abcs}[2]{
	\begin{enumerate}[label=\alph*),start=#1,leftmargin=18pt]
		#2
	\end{enumerate}
}
\newcommand{\abcn}[1]{
	\begin{enumerate}[label=\arabic*),leftmargin=18pt]
		#1
	\end{enumerate}
}
\newcommand{\abch}[1]{
	\hspace{-2pt}	\begin{enumerate*}[label=\alph*), itemjoin=\hspace{1cm}]
		#1
	\end{enumerate*}
}
\newcommand{\abchs}[2]{
	\hspace{-2pt}	\begin{enumerate*}[label=\alph*), itemjoin=\hspace{1cm}, start=#1]
		#2
	\end{enumerate*}
}

% Oppgaver
\newcommand{\opgt}{\phantomsection \addcontentsline{toc}{section}{Oppgaver} \section*{Oppgaver for kapittel \thechapter}\vs \setcounter{section}{1}}
\newcounter{opg}
\numberwithin{opg}{section}
\newcommand{\op}[1]{\vspace{15pt} \refstepcounter{opg}\large \textbf{\color{blue}\theopg} \vspace{2 pt} \label{#1} \\}
\newcommand{\ekspop}[1]{\vsk\textbf{Gruble \thechapter.#1}\vspace{2 pt} \\}
\newcommand{\nes}{\stepcounter{section}
	\setcounter{opg}{0}}
\newcommand{\opr}[1]{\vspace{3pt}\textbf{\ref{#1}}}
\newcommand{\oeks}[1]{\begin{tcolorbox}[boxrule=0.3 mm,arc=0mm,colback=white]
		\textit{Eksempel: } #1	  
\end{tcolorbox}}
\newcommand\opgeks[2][]{\begin{tcolorbox}[boxrule=0.1 mm,arc=0mm,enhanced jigsaw,breakable,colback=white] {\footnotesize \textbf{Eksempel #1} \\} \footnotesize #2 \end{tcolorbox}\vspace{-5pt} }
\newcommand{\rknut}{
Rekn ut.
}

%License
\newcommand{\lic}{\textit{Matematikken sine byggesteinar by Sindre Sogge Heggen is licensed under CC BY-NC-SA 4.0. To view a copy of this license, visit\\ 
		\net{http://creativecommons.org/licenses/by-nc-sa/4.0/}{http://creativecommons.org/licenses/by-nc-sa/4.0/}}}

%referances
\newcommand{\net}[2]{{\color{blue}\href{#1}{#2}}}
\newcommand{\hrs}[2]{\hyperref[#1]{\color{blue}\textsl{#2 \ref*{#1}}}}
\newcommand{\rref}[1]{\hrs{#1}{regel}}
\newcommand{\refkap}[1]{\hrs{#1}{kapittel}}
\newcommand{\refsec}[1]{\hrs{#1}{seksjon}}

\newcommand{\mb}{\net{https://sindrsh.github.io/FirstPrinciplesOfMath/}{MB}}


%line to seperate examples
\newcommand{\linje}{\rule{\linewidth}{1pt} }

\usepackage{datetime2}
%%\usepackage{sansmathfonts} for dyslexia-friendly math
\usepackage[]{hyperref}

\usepackage{xr}
\externaldocument{/home/sindre/P/P}

\begin{document}
\section*{Kapittel \ref*{Rreg}}\vs
\opr{pm} \textbf{a)} 4 \textbf{b)} $ -48 $ \textbf{c)} 90 \textbf{d)} $ -8 $

\opr{rgnrek} \textbf{a)} 31 \textbf{b)} $ -7 $ \textbf{c)} 6 \textbf{d)} $ 3 $

\section*{Kapittel \ref*{Br}}\vs
\opr{br1} \textbf{a)} 0,5 \textbf{b)} 2 \textbf{c)} 0,2 \textbf{d)} 0,75

\opr{br2} $ \frac{20}{3} $ \textbf{b)} $ -\frac{18}{5} $ 

\opr{br5} $ \frac{18}{3}$ (Aller helst bør man regne ut at $ 18:3=6 $.) 

\opr{br3} \textbf{a)} $ \frac{4}{15} $ \textbf{b)} $ -\frac{3}{30} $

\opr{br4} \textbf{a)} $ \frac{20}{27} $ \textbf{b)} $ \frac{14}{32}\; (=\frac{7}{16}) $

\opr{br6} $ \frac{8}{15} $

\opr{br7} \textbf{a)} $ \frac{7}{4} $ \textbf{b)} $ \frac{1}{3} $

\opr{br8} \textbf{a)} $ \frac{16}{24} $ \textbf{b)} $ \frac{33}{27} $

\opr{br9} \textbf{a)} $ \frac{37}{30} $ \textbf{b)} $ \frac{5}{12} $

\opr{br10} \textbf{a)} $ \frac{12}{25} $ \textbf{b)} $ \frac{24}{9}\;(=\frac{8}{3}) $

\opr{br11} \textbf{a)} $ \frac{h}{2} $ \textbf{b)} $ \frac{1}{2h} $ \textbf{c)} $ \frac{\pi a}{2b} $

\opr{br12} $ \frac{a}{2} $


\section*{Kapittel \ref*{Lig}}\vs
\opr{lig1} \textbf{a)} \y{x=13} \textbf{b)} \y{x=-5}

\opr{lig2} \textbf{a)} \y{x=20} \textbf{b)} \y{x=4}

\opr{lig3} \textbf{a)} \y{x=21} \textbf{b)} \y{x=2} \textbf{c)} \y{x=15}

\opr{lig4} \textbf{a)} \y{x=\frac{22}{3}} \textbf{b)} \y{x=6} \textbf{c)} \y{x=-15} \textbf{d)} \y{x=2}

\opr{lig6} Ola 4000 kr, Kari 8000 kr

\opr{lig7} 0,2 m

\opr{lig8} 1300 kr

\section*{Kapittel \ref*{Oko}}\vs
\opr{kr} \textbf{a)} ca 1,40\,kr \textbf{b)} ca. 1.02 \,kr \textbf{c)} ca 0.95\,kr

\opr{hvormye} 3,6\%

\opr{elsereal} I 2017 (Reallønn 2017: ca. 464\,455\,kr, Reallønn 2012: ca 436\,635\,kr)

\opr{5eksh17d2} ca 580\,008\,kr

\opr{2eksv17d1} 1200\,kr

\opr{ser} \textbf{a)} 20\,000\,kr \textbf{b)} 80\,000\,kr \textbf{c)} 1\,6000 \textbf{d)} 21\,600

\opr{anu} 5\,783\,kr

\opr{serogan}
\textbf{a)} Bilde (a) er serielån fordi avdragene er like store. Bilde (b) er annuitetslån fordi terminbeløpene (renter $ + $ avdrag) er like store.

\opr{spar} ca. 63\,000\,kr

\opr{kred} \textbf{a)} 55\,000\,kr \textbf{b} 60\,500\,kr \textbf{c)} 33\,000\,kr

\opr{4eksh2016d2} ca. 569\,000\,kr

\opr{pensj} \textbf{a)} 210\,300\,kr \textbf{b} 48\,369\,kr

\opr{miraogborge} Mira betaler 16\,400\,kr og Børge betaler 17\,850\,kr. Børge betaler mest.

\opr{borge3} Trinn 1: ca. 965\,kr, Trinn 2: ca 3699\,kr (totalt ca. 4664\,kr)

\opr{borge4} 279\,117\,kr

\opr{nora}\\
\textbf{a)}\\
\begin{tabular}{r r}
	\textbf{Inntekter} & Budsjett \\ \hline 
	Nettolønn & 23\,000 \\ \hline
	\textit{Sum} & 23\,000 \\ \hline \\
	\textbf{Utgifter} & \\ \hline
	Leia av hybel & 6\,000 \\
	Mat & 4\,500 \\
	Annet & 1\,500\\ \hline
	\textit{Sum} & 12\,000\\ \hline \\ \hline

	\textbf{Resultat} & 11\,000\\ \hline
\end{tabular}\vsk

\textbf{b)}\\
\begin{tabular}{r r r r}
	\textbf{Inntekter} & Budsjett & Regnskap & Avvik \\ \hline 
	Lønn & 23\,000 & 23\,000 & 0\\
	FLAX-gevinst & 0& 1\,000 & 1\,000\\ \hline
	Sum & 23\,000 & 24\,00 & 1\,000\\\hline 
	& \\
	\textbf{Utgifter} & \\ \hline
	Leia av hybel & 6\,000 & 6\,000 &0 \\
	Mat & 4\,500 & 5\,500 & $ \color{red}-1\,000 $\\
	Annet. & 1\,500 & 1\,800 & $ \color{red}-300 $\\ 
	FLAX-lodd & 0 & 100 & $ \color{red}-100 $ \\
	\hline
	Sum & 12\,000 & 13\,400 & $ \color{red}-1\,400 $\\ \hline
	& \\ \hline
	\textbf{Resultat} & 11\,000 & 11\,600 & $ \color{red}-400 $ \\ \hline
\end{tabular}\os
11\,600 i overskudd. Overskuddet 400 \textsl{mindre} enn budsjettert.

\section*{Kapittel \ref*{San}}
Obs! Mange av brøkene i disse oppgavene kan forkortes, men fordi vi noen ganger skal bruke svar fra en deloppgave i andre utregninger, forkorter vi ikke. \vsk

\opr{klover}
\textbf{a)} $ \frac{13}{52}$
\textbf{b)} $ \frac{26}{52} $
\textbf{c)} $ \frac{52}{52}-\frac{13}{52}=\frac{39}{52} $, $ \frac{13}{52}+\frac{13}{52}+\frac{13}{52}=\frac{39}{52} $

\opr{klovog8}
\textbf{a)} $ \frac{4}{52} $ 
\textbf{b)} $ \frac{13}{52} $
\textbf{c)} $ \frac{16}{52} $
\textbf{d)} $ \frac{36}{52} $
\end{document}

