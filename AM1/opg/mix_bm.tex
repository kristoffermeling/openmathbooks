\documentclass[english, 11 pt, class=article, crop=false]{standalone}
\usepackage[T1]{fontenc}
%\renewcommand*\familydefault{\sfdefault} % For dyslexia-friendly text
\usepackage{lmodern} % load a font with all the characters
\usepackage{geometry}
\geometry{verbose,paperwidth=16.1 cm, paperheight=24 cm, inner=2.3cm, outer=1.8 cm, bmargin=2cm, tmargin=1.8cm}
\setlength{\parindent}{0bp}
\usepackage{import}
\usepackage[subpreambles=false]{standalone}
\usepackage{amsmath}
\usepackage{amssymb}
\usepackage{esint}
\usepackage{babel}
\usepackage{tabu}
\makeatother
\makeatletter

\usepackage{titlesec}
\usepackage{ragged2e}
\RaggedRight
\raggedbottom
\frenchspacing

% Norwegian names of figures, chapters, parts and content
\addto\captionsenglish{\renewcommand{\figurename}{Figur}}
\makeatletter
\addto\captionsenglish{\renewcommand{\chaptername}{Kapittel}}
\addto\captionsenglish{\renewcommand{\partname}{Del}}


\usepackage{graphicx}
\usepackage{float}
\usepackage{subfig}
\usepackage{placeins}
\usepackage{cancel}
\usepackage{framed}
\usepackage{wrapfig}
\usepackage[subfigure]{tocloft}
\usepackage[font=footnotesize,labelfont=sl]{caption} % Figure caption
\usepackage{bm}
\usepackage[dvipsnames, table]{xcolor}
\definecolor{shadecolor}{rgb}{0.105469, 0.613281, 1}
\colorlet{shadecolor}{Emerald!15} 
\usepackage{icomma}
\makeatother
\usepackage[many]{tcolorbox}
\usepackage{multicol}
\usepackage{stackengine}

\usepackage{esvect} %For vectors with capital letters

% For tabular
\usepackage{array}
\usepackage{multirow}
\usepackage{longtable} %breakable table

% Ligningsreferanser
\usepackage{mathtools}
\mathtoolsset{showonlyrefs}

% index
\usepackage{imakeidx}
\makeindex[title=Indeks]

%Footnote:
\usepackage[bottom, hang, flushmargin]{footmisc}
\usepackage{perpage} 
\MakePerPage{footnote}
\addtolength{\footnotesep}{2mm}
\renewcommand{\thefootnote}{\arabic{footnote}}
\renewcommand\footnoterule{\rule{\linewidth}{0.4pt}}
\renewcommand{\thempfootnote}{\arabic{mpfootnote}}

%colors
\definecolor{c1}{cmyk}{0,0.5,1,0}
\definecolor{c2}{cmyk}{1,0.25,1,0}
\definecolor{n3}{cmyk}{1,0.,1,0}
\definecolor{neg}{cmyk}{1,0.,0.,0}

% Lister med bokstavar
\usepackage[inline]{enumitem}

\newcounter{rg}
\numberwithin{rg}{chapter}
\newcommand{\reg}[2][]{\begin{tcolorbox}[boxrule=0.3 mm,arc=0mm,colback=blue!3] {\refstepcounter{rg}\phantomsection \large \textbf{\therg \;#1} \vspace{5 pt}}\newline #2  \end{tcolorbox}\vspace{-5pt}}

\newcommand\alg[1]{\begin{align} #1 \end{align}}

\newcommand\eks[2][]{\begin{tcolorbox}[boxrule=0.3 mm,arc=0mm,enhanced jigsaw,breakable,colback=green!3] {\large \textbf{Eksempel #1} \vspace{5 pt}\\} #2 \end{tcolorbox}\vspace{-5pt} }

\newcommand{\st}[1]{\begin{tcolorbox}[boxrule=0.0 mm,arc=0mm,enhanced jigsaw,breakable,colback=yellow!12]{ #1} \end{tcolorbox}}

\newcommand{\spr}[1]{\begin{tcolorbox}[boxrule=0.3 mm,arc=0mm,enhanced jigsaw,breakable,colback=yellow!7] {\large \textbf{Språkboksen} \vspace{5 pt}\\} #1 \end{tcolorbox}\vspace{-5pt} }

\newcommand{\sym}[1]{\colorbox{blue!15}{#1}}

\newcommand{\info}[2]{\begin{tcolorbox}[boxrule=0.3 mm,arc=0mm,enhanced jigsaw,breakable,colback=cyan!6] {\large \textbf{#1} \vspace{5 pt}\\} #2 \end{tcolorbox}\vspace{-5pt} }

\newcommand\algv[1]{\vspace{-11 pt}\begin{align*} #1 \end{align*}}

\newcommand{\regv}{\vspace{5pt}}
\newcommand{\mer}{\textsl{Merk}: }
\newcommand{\mers}[1]{{\footnotesize \mer #1}}
\newcommand\vsk{\vspace{11pt}}
\newcommand\vs{\vspace{-11pt}}
\newcommand\vsb{\vspace{-16pt}}
\newcommand\sv{\vsk \textbf{Svar} \vspace{4 pt}\\}
\newcommand\br{\\[5 pt]}
\newcommand{\figp}[1]{../fig/#1}
\newcommand\algvv[1]{\vs\vs\begin{align*} #1 \end{align*}}
\newcommand{\y}[1]{$ {#1} $}
\newcommand{\os}{\\[5 pt]}
\newcommand{\prbxl}[2]{
\parbox[l][][l]{#1\linewidth}{#2
	}}
\newcommand{\prbxr}[2]{\parbox[r][][l]{#1\linewidth}{
		\setlength{\abovedisplayskip}{5pt}
		\setlength{\belowdisplayskip}{5pt}	
		\setlength{\abovedisplayshortskip}{0pt}
		\setlength{\belowdisplayshortskip}{0pt} 
		\begin{shaded}
			\footnotesize	#2 \end{shaded}}}

\renewcommand{\cfttoctitlefont}{\Large\bfseries}
\setlength{\cftaftertoctitleskip}{0 pt}
\setlength{\cftbeforetoctitleskip}{0 pt}

\newcommand{\bs}{\\[3pt]}
\newcommand{\vn}{\\[6pt]}
\newcommand{\fig}[1]{\begin{figure}
		\centering
		\includegraphics[]{\figp{#1}}
\end{figure}}

\newcommand{\figc}[2]{\begin{figure}
		\centering
		\includegraphics[]{\figp{#1}}
		\caption{#2}
\end{figure}}

\newcommand{\sectionbreak}{\clearpage} % New page on each section

\newcommand{\nn}[1]{
\begin{equation}
	#1
\end{equation}
}

% Equation comments
\newcommand{\cm}[1]{\llap{\color{blue} #1}}

\newcommand\fork[2]{\begin{tcolorbox}[boxrule=0.3 mm,arc=0mm,enhanced jigsaw,breakable,colback=yellow!7] {\large \textbf{#1 (forklaring)} \vspace{5 pt}\\} #2 \end{tcolorbox}\vspace{-5pt} }
 
%colors
\newcommand{\colr}[1]{{\color{red} #1}}
\newcommand{\colb}[1]{{\color{blue} #1}}
\newcommand{\colo}[1]{{\color{orange} #1}}
\newcommand{\colc}[1]{{\color{cyan} #1}}
\definecolor{projectgreen}{cmyk}{100,0,100,0}
\newcommand{\colg}[1]{{\color{projectgreen} #1}}

% Methods
\newcommand{\metode}[2]{
	\textsl{#1} \\[-8pt]
	\rule{#2}{0.75pt}
}

%Opg
\newcommand{\abc}[1]{
	\begin{enumerate}[label=\alph*),leftmargin=18pt]
		#1
	\end{enumerate}
}
\newcommand{\abcs}[2]{
	\begin{enumerate}[label=\alph*),start=#1,leftmargin=18pt]
		#2
	\end{enumerate}
}
\newcommand{\abcn}[1]{
	\begin{enumerate}[label=\arabic*),leftmargin=18pt]
		#1
	\end{enumerate}
}
\newcommand{\abch}[1]{
	\hspace{-2pt}	\begin{enumerate*}[label=\alph*), itemjoin=\hspace{1cm}]
		#1
	\end{enumerate*}
}
\newcommand{\abchs}[2]{
	\hspace{-2pt}	\begin{enumerate*}[label=\alph*), itemjoin=\hspace{1cm}, start=#1]
		#2
	\end{enumerate*}
}

% Oppgaver
\newcommand{\opgt}{\phantomsection \addcontentsline{toc}{section}{Oppgaver} \section*{Oppgaver for kapittel \thechapter}\vs \setcounter{section}{1}}
\newcounter{opg}
\numberwithin{opg}{section}
\newcommand{\op}[1]{\vspace{15pt} \refstepcounter{opg}\large \textbf{\color{blue}\theopg} \vspace{2 pt} \label{#1} \\}
\newcommand{\ekspop}[1]{\vsk\textbf{Gruble \thechapter.#1}\vspace{2 pt} \\}
\newcommand{\nes}{\stepcounter{section}
	\setcounter{opg}{0}}
\newcommand{\opr}[1]{\vspace{3pt}\textbf{\ref{#1}}}
\newcommand{\oeks}[1]{\begin{tcolorbox}[boxrule=0.3 mm,arc=0mm,colback=white]
		\textit{Eksempel: } #1	  
\end{tcolorbox}}
\newcommand\opgeks[2][]{\begin{tcolorbox}[boxrule=0.1 mm,arc=0mm,enhanced jigsaw,breakable,colback=white] {\footnotesize \textbf{Eksempel #1} \\} \footnotesize #2 \end{tcolorbox}\vspace{-5pt} }
\newcommand{\rknut}{
Rekn ut.
}

%License
\newcommand{\lic}{\textit{Matematikken sine byggesteinar by Sindre Sogge Heggen is licensed under CC BY-NC-SA 4.0. To view a copy of this license, visit\\ 
		\net{http://creativecommons.org/licenses/by-nc-sa/4.0/}{http://creativecommons.org/licenses/by-nc-sa/4.0/}}}

%referances
\newcommand{\net}[2]{{\color{blue}\href{#1}{#2}}}
\newcommand{\hrs}[2]{\hyperref[#1]{\color{blue}\textsl{#2 \ref*{#1}}}}
\newcommand{\rref}[1]{\hrs{#1}{regel}}
\newcommand{\refkap}[1]{\hrs{#1}{kapittel}}
\newcommand{\refsec}[1]{\hrs{#1}{seksjon}}

\newcommand{\mb}{\net{https://sindrsh.github.io/FirstPrinciplesOfMath/}{MB}}


%line to seperate examples
\newcommand{\linje}{\rule{\linewidth}{1pt} }

\usepackage{datetime2}
%%\usepackage{sansmathfonts} for dyslexia-friendly math
\usepackage[]{hyperref}


\newcommand{\note}{Merk}
\newcommand{\notesm}[1]{{\footnotesize \textsl{\note:} #1}}
\newcommand{\ekstitle}{Eksempel }
\newcommand{\sprtitle}{Språkboksen}
\newcommand{\expl}{forklaring}

\newcommand{\vedlegg}[1]{\refstepcounter{vedl}\section*{Vedlegg \thevedl: #1}  \setcounter{vedleq}{0}}

\newcommand\sv{\vsk \textbf{Svar} \vspace{4 pt}\\}

%references
\newcommand{\reftab}[1]{\hrs{#1}{tabell}}
\newcommand{\rref}[1]{\hrs{#1}{regel}}
\newcommand{\dref}[1]{\hrs{#1}{definisjon}}
\newcommand{\refkap}[1]{\hrs{#1}{kapittel}}
\newcommand{\refsec}[1]{\hrs{#1}{seksjon}}
\newcommand{\refdsec}[1]{\hrs{#1}{delseksjon}}
\newcommand{\refved}[1]{\hrs{#1}{vedlegg}}
\newcommand{\eksref}[1]{\textsl{#1}}
\newcommand\fref[2][]{\hyperref[#2]{\textsl{figur \ref*{#2}#1}}}
\newcommand{\refop}[1]{{\color{blue}Oppgave \ref{#1}}}
\newcommand{\refops}[1]{{\color{blue}oppgave \ref{#1}}}
\newcommand{\refgrubs}[1]{{\color{blue}gruble \ref{#1}}}

\newcommand{\openmathser}{\openmath\,-\,serien}

% Exercises
\newcommand{\opgt}{\newpage \phantomsection \addcontentsline{toc}{section}{Oppgaver} \section*{Oppgaver for kapittel \thechapter}\vs \setcounter{section}{1}}


% Sequences and series
\newcommand{\sumarrek}{Summen av en aritmetisk rekke}
\newcommand{\sumgerek}{Summen av en geometrisk rekke}
\newcommand{\regnregsum}{Regneregler for summetegnet}

% Trigonometry
\newcommand{\sincoskomb}{Sinus og cosinus kombinert}
\newcommand{\cosfunk}{Cosinusfunksjonen}
\newcommand{\trid}{Trigonometriske identiteter}
\newcommand{\deravtri}{Den deriverte av de trigonometriske funksjonene}
% Solutions manual
\newcommand{\selos}{Se løsningsforslag.}
\newcommand{\se}[1]{Se eksempel på side \pageref{#1}}

%Vectors
\newcommand{\parvek}{Parallelle vektorer}
\newcommand{\vekpro}{Vektorproduktet}
\newcommand{\vekproarvol}{Vektorproduktet som areal og volum}


% 3D geometries
\newcommand{\linrom}{Linje i rommet}
\newcommand{\avstplnpkt}{Avstand mellom punkt og plan}


% Integral
\newcommand{\bestminten}{Bestemt integral I}
\newcommand{\anfundteo}{Analysens fundamentalteorem}
\newcommand{\intuf}{Integralet av utvalge funksjoner}
\newcommand{\bytvar}{Bytte av variabel}
\newcommand{\intvol}{Integral som volum}
\newcommand{\andordlindif}{Andre ordens lineære differensialligninger}



\begin{document}

\st{Se også oppgaver på \net{https://ektedata.uib.no/}{ekte.data.uib.no}}	\newpage	

\op{opgtorden}
\tagop{
\#overslag \#proporsjonalitet
}
Når det er lyn og torden kan du bruke følgende metode for å finne ut omtrent hvor langt unna du er uværet:\os

\textsl{Start med å telle sekunder straks du ser et lyn. Stopp tellingen når du hører torden. Gang antall sekunder med 300, da har du et overslag på hvor mange meter du er unna uværet.}\os

Bruk internett til undersøke hastigheten til lys (lyn) og lyd (torden) i luft, og forklar hva denne metoden baserer seg på.

\eksop{1PV22D1}{1PV22D1opg5}
\tagop{
\#programmering \#prosentregning
}
\python{1pv22d1opg5.py}
En elev har skrevet programkoden ovenfor.
Hva ønsker eleven å finne ut?
Forklar hva som skjer når programmet kjøres.

\eksop{1PV22D1}{1PV22D1opg2}
\tagop{\#prosentregning \#statistikk \#tallforståelse}
\fig{1pv22d1opg2}	
Diagrammet viser antall elever ved en videregående skole de fire siste årene.\os

Når var det størst prosentvis økning i antall elever fra et år til det neste?
\newpage		
\op{opgdusj}
\tagop{
\# omgjøring av enheter \# standardform \\\# proporsjonale størrelser 
}
Det har det blitt populært å regne ut hva det koster å ta seg en dusj. Til et slikt reknestykke kan man gjøre følgende antakelser:
\begin{itemize}
	\item Energien som kreves er energien som må til for å varme opp vannet som gikk med til dusjingen fra 7$^\circ $ til $ 35^\circ $.
	\item For å øke temperaturen til 1 liter vann med 1$ ^\circ $, kreves det $ 4,2\cdot10^3\enh{J} $.
\end{itemize}
Ifølge \net{https://www.vg.no/spesial/2022/stromprisene/}{vg.no} var 395,41 øre/kWh den høyeste (gjennomsnittlige) strømprisen registrert i Oslo. 
\abc{
\item Regn ut hva en dusj på 10 minutter ville kostet med denne prisen. 
\item Bruk internett til å finne strømprisene for din region i dag. Sjekk hva en 10 minutters dusj vil koste deg.
}

\newpage
\op{luftmotstand}
\tagop{
	\#algebra \#modellering \#andregradsfunksjon \\
	\#omgjøring av enheter \#proporsjonalitet
}

La $ F $ være summen av kreftene som virker i motsatt retning av en bils kjøreretning. Ifølge en rapport\footnote{\net{https://sintef.brage.unit.no/sintef-xmlui/handle/11250/2468761}{https://sintef.brage.unit.no/sintef-xmlui/handle/11250/2468761}} fra SINTEF kan\footnote{Det er er her forutsatt flatt strekke, og sett vekk ifra motstand ved akselerasjon.} $ F $ tilnærmes som
\[ F(v)= mgC_r+\frac{1}{2}\rho v^2 D_m\qquad,\qquad v\geq10\]
\begin{center}
	\begin{tabular}{c|c|c|c}
		& \textbf{betydning} & \textbf{verdi}&\textbf{enhet}  \\ \hline
		$ v $ & bilens hastighet & variabel& m/s \\
		$ m $& bilens masse\footnotemark & 1409 & kg\\
		$ g $& tyngdeakselerasjonen & 9.81 & m/$ \text{s}^2 $ \\
		$ C_r $ & koeffisient for bilens rullemotstand & 0.015\\
		$ \rho $ & tettheten til luft & 1.25 & kg/$ \text{m}^3 $ \\
		$ D_m $& koeffisient for bilens luftmotstandsareal\footnotemark &0.74
	\end{tabular}
\end{center}

\footnotetext[3]{Det er tatt ugangspunkt i gjennomsnittsvekten til en norsk personbil.}
\footnotetext[4]{Verdien er hentet fra \net{en.wikipedia.org/wiki/Automobile\_drag\_coefficient\#Drag\_area}{en.wikipedia.org/wiki/Automobile\_drag\_coefficient\#Drag\_area}}

\abc{
	\item Tegn grafen til $ F $ for $ v\in [10, 35] $
	\item På intervallet gitt i oppgave a, for hvilken hastighet er det at 
	\begin{itemize}
		\item rullemotstanden gir det største bidraget til $ F $?
		\item luftmotstanden gir det største bidraget til $ F $?
	\end{itemize} 
	Oppgi svarene rundet av til nærmeste heltall og målt i km/h.\newpage
	\item Med ''energiforbruk''\footnote{Den totale energimengden en bil bruker på en kjørelengde vil være høyere enn det vi har kalt ''energiforbruket''. Som regel vil den totale energimengden som kreves for å kjøre en strekning være høyere jo høyere hastighet man har. Slik kan man anta at differansen i energiforbruk vi finner i denne oppgaven er et minimum for den reelle differansen i total energimengde.} mener vi her den energien som må til for å motvirke $ F $ over en viss kjørelengde.
	Ved konstant hastighet er energiforbruket etter kjørt lengde proporsjonal med $ F $. På norske motorveier er 90\,km/h og 110\,km/h vanlige fartsgrenser. Hvor stor økning i energiforbruk vil en økning fra 90\,km/h til 110\,km/h innebære?\os
	
	\item Lag en funksjon $ F_1 $ som gir $ F $ ut ifra bilens hastighet målt i km/h.
}
\end{document}

