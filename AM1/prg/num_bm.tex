\documentclass[english, 11 pt, class=article, crop=false]{standalone}
\usepackage[T1]{fontenc}
%\renewcommand*\familydefault{\sfdefault} % For dyslexia-friendly text
\usepackage{lmodern} % load a font with all the characters
\usepackage{geometry}
\geometry{verbose,paperwidth=16.1 cm, paperheight=24 cm, inner=2.3cm, outer=1.8 cm, bmargin=2cm, tmargin=1.8cm}
\setlength{\parindent}{0bp}
\usepackage{import}
\usepackage[subpreambles=false]{standalone}
\usepackage{amsmath}
\usepackage{amssymb}
\usepackage{esint}
\usepackage{babel}
\usepackage{tabu}
\makeatother
\makeatletter

\usepackage{titlesec}
\usepackage{ragged2e}
\RaggedRight
\raggedbottom
\frenchspacing

% Norwegian names of figures, chapters, parts and content
\addto\captionsenglish{\renewcommand{\figurename}{Figur}}
\makeatletter
\addto\captionsenglish{\renewcommand{\chaptername}{Kapittel}}
\addto\captionsenglish{\renewcommand{\partname}{Del}}


\usepackage{graphicx}
\usepackage{float}
\usepackage{subfig}
\usepackage{placeins}
\usepackage{cancel}
\usepackage{framed}
\usepackage{wrapfig}
\usepackage[subfigure]{tocloft}
\usepackage[font=footnotesize,labelfont=sl]{caption} % Figure caption
\usepackage{bm}
\usepackage[dvipsnames, table]{xcolor}
\definecolor{shadecolor}{rgb}{0.105469, 0.613281, 1}
\colorlet{shadecolor}{Emerald!15} 
\usepackage{icomma}
\makeatother
\usepackage[many]{tcolorbox}
\usepackage{multicol}
\usepackage{stackengine}

\usepackage{esvect} %For vectors with capital letters

% For tabular
\usepackage{array}
\usepackage{multirow}
\usepackage{longtable} %breakable table

% Ligningsreferanser
\usepackage{mathtools}
\mathtoolsset{showonlyrefs}

% index
\usepackage{imakeidx}
\makeindex[title=Indeks]

%Footnote:
\usepackage[bottom, hang, flushmargin]{footmisc}
\usepackage{perpage} 
\MakePerPage{footnote}
\addtolength{\footnotesep}{2mm}
\renewcommand{\thefootnote}{\arabic{footnote}}
\renewcommand\footnoterule{\rule{\linewidth}{0.4pt}}
\renewcommand{\thempfootnote}{\arabic{mpfootnote}}

%colors
\definecolor{c1}{cmyk}{0,0.5,1,0}
\definecolor{c2}{cmyk}{1,0.25,1,0}
\definecolor{n3}{cmyk}{1,0.,1,0}
\definecolor{neg}{cmyk}{1,0.,0.,0}

% Lister med bokstavar
\usepackage[inline]{enumitem}

\newcounter{rg}
\numberwithin{rg}{chapter}
\newcommand{\reg}[2][]{\begin{tcolorbox}[boxrule=0.3 mm,arc=0mm,colback=blue!3] {\refstepcounter{rg}\phantomsection \large \textbf{\therg \;#1} \vspace{5 pt}}\newline #2  \end{tcolorbox}\vspace{-5pt}}

\newcommand\alg[1]{\begin{align} #1 \end{align}}

\newcommand\eks[2][]{\begin{tcolorbox}[boxrule=0.3 mm,arc=0mm,enhanced jigsaw,breakable,colback=green!3] {\large \textbf{Eksempel #1} \vspace{5 pt}\\} #2 \end{tcolorbox}\vspace{-5pt} }

\newcommand{\st}[1]{\begin{tcolorbox}[boxrule=0.0 mm,arc=0mm,enhanced jigsaw,breakable,colback=yellow!12]{ #1} \end{tcolorbox}}

\newcommand{\spr}[1]{\begin{tcolorbox}[boxrule=0.3 mm,arc=0mm,enhanced jigsaw,breakable,colback=yellow!7] {\large \textbf{Språkboksen} \vspace{5 pt}\\} #1 \end{tcolorbox}\vspace{-5pt} }

\newcommand{\sym}[1]{\colorbox{blue!15}{#1}}

\newcommand{\info}[2]{\begin{tcolorbox}[boxrule=0.3 mm,arc=0mm,enhanced jigsaw,breakable,colback=cyan!6] {\large \textbf{#1} \vspace{5 pt}\\} #2 \end{tcolorbox}\vspace{-5pt} }

\newcommand\algv[1]{\vspace{-11 pt}\begin{align*} #1 \end{align*}}

\newcommand{\regv}{\vspace{5pt}}
\newcommand{\mer}{\textsl{Merk}: }
\newcommand{\mers}[1]{{\footnotesize \mer #1}}
\newcommand\vsk{\vspace{11pt}}
\newcommand\vs{\vspace{-11pt}}
\newcommand\vsb{\vspace{-16pt}}
\newcommand\sv{\vsk \textbf{Svar} \vspace{4 pt}\\}
\newcommand\br{\\[5 pt]}
\newcommand{\figp}[1]{../fig/#1}
\newcommand\algvv[1]{\vs\vs\begin{align*} #1 \end{align*}}
\newcommand{\y}[1]{$ {#1} $}
\newcommand{\os}{\\[5 pt]}
\newcommand{\prbxl}[2]{
\parbox[l][][l]{#1\linewidth}{#2
	}}
\newcommand{\prbxr}[2]{\parbox[r][][l]{#1\linewidth}{
		\setlength{\abovedisplayskip}{5pt}
		\setlength{\belowdisplayskip}{5pt}	
		\setlength{\abovedisplayshortskip}{0pt}
		\setlength{\belowdisplayshortskip}{0pt} 
		\begin{shaded}
			\footnotesize	#2 \end{shaded}}}

\renewcommand{\cfttoctitlefont}{\Large\bfseries}
\setlength{\cftaftertoctitleskip}{0 pt}
\setlength{\cftbeforetoctitleskip}{0 pt}

\newcommand{\bs}{\\[3pt]}
\newcommand{\vn}{\\[6pt]}
\newcommand{\fig}[1]{\begin{figure}
		\centering
		\includegraphics[]{\figp{#1}}
\end{figure}}

\newcommand{\figc}[2]{\begin{figure}
		\centering
		\includegraphics[]{\figp{#1}}
		\caption{#2}
\end{figure}}

\newcommand{\sectionbreak}{\clearpage} % New page on each section

\newcommand{\nn}[1]{
\begin{equation}
	#1
\end{equation}
}

% Equation comments
\newcommand{\cm}[1]{\llap{\color{blue} #1}}

\newcommand\fork[2]{\begin{tcolorbox}[boxrule=0.3 mm,arc=0mm,enhanced jigsaw,breakable,colback=yellow!7] {\large \textbf{#1 (forklaring)} \vspace{5 pt}\\} #2 \end{tcolorbox}\vspace{-5pt} }
 
%colors
\newcommand{\colr}[1]{{\color{red} #1}}
\newcommand{\colb}[1]{{\color{blue} #1}}
\newcommand{\colo}[1]{{\color{orange} #1}}
\newcommand{\colc}[1]{{\color{cyan} #1}}
\definecolor{projectgreen}{cmyk}{100,0,100,0}
\newcommand{\colg}[1]{{\color{projectgreen} #1}}

% Methods
\newcommand{\metode}[2]{
	\textsl{#1} \\[-8pt]
	\rule{#2}{0.75pt}
}

%Opg
\newcommand{\abc}[1]{
	\begin{enumerate}[label=\alph*),leftmargin=18pt]
		#1
	\end{enumerate}
}
\newcommand{\abcs}[2]{
	\begin{enumerate}[label=\alph*),start=#1,leftmargin=18pt]
		#2
	\end{enumerate}
}
\newcommand{\abcn}[1]{
	\begin{enumerate}[label=\arabic*),leftmargin=18pt]
		#1
	\end{enumerate}
}
\newcommand{\abch}[1]{
	\hspace{-2pt}	\begin{enumerate*}[label=\alph*), itemjoin=\hspace{1cm}]
		#1
	\end{enumerate*}
}
\newcommand{\abchs}[2]{
	\hspace{-2pt}	\begin{enumerate*}[label=\alph*), itemjoin=\hspace{1cm}, start=#1]
		#2
	\end{enumerate*}
}

% Oppgaver
\newcommand{\opgt}{\phantomsection \addcontentsline{toc}{section}{Oppgaver} \section*{Oppgaver for kapittel \thechapter}\vs \setcounter{section}{1}}
\newcounter{opg}
\numberwithin{opg}{section}
\newcommand{\op}[1]{\vspace{15pt} \refstepcounter{opg}\large \textbf{\color{blue}\theopg} \vspace{2 pt} \label{#1} \\}
\newcommand{\ekspop}[1]{\vsk\textbf{Gruble \thechapter.#1}\vspace{2 pt} \\}
\newcommand{\nes}{\stepcounter{section}
	\setcounter{opg}{0}}
\newcommand{\opr}[1]{\vspace{3pt}\textbf{\ref{#1}}}
\newcommand{\oeks}[1]{\begin{tcolorbox}[boxrule=0.3 mm,arc=0mm,colback=white]
		\textit{Eksempel: } #1	  
\end{tcolorbox}}
\newcommand\opgeks[2][]{\begin{tcolorbox}[boxrule=0.1 mm,arc=0mm,enhanced jigsaw,breakable,colback=white] {\footnotesize \textbf{Eksempel #1} \\} \footnotesize #2 \end{tcolorbox}\vspace{-5pt} }
\newcommand{\rknut}{
Rekn ut.
}

%License
\newcommand{\lic}{\textit{Matematikken sine byggesteinar by Sindre Sogge Heggen is licensed under CC BY-NC-SA 4.0. To view a copy of this license, visit\\ 
		\net{http://creativecommons.org/licenses/by-nc-sa/4.0/}{http://creativecommons.org/licenses/by-nc-sa/4.0/}}}

%referances
\newcommand{\net}[2]{{\color{blue}\href{#1}{#2}}}
\newcommand{\hrs}[2]{\hyperref[#1]{\color{blue}\textsl{#2 \ref*{#1}}}}
\newcommand{\rref}[1]{\hrs{#1}{regel}}
\newcommand{\refkap}[1]{\hrs{#1}{kapittel}}
\newcommand{\refsec}[1]{\hrs{#1}{seksjon}}

\newcommand{\mb}{\net{https://sindrsh.github.io/FirstPrinciplesOfMath/}{MB}}


%line to seperate examples
\newcommand{\linje}{\rule{\linewidth}{1pt} }

\usepackage{datetime2}
%%\usepackage{sansmathfonts} for dyslexia-friendly math
\usepackage[]{hyperref}


\newcommand{\note}{Merk}
\newcommand{\notesm}[1]{{\footnotesize \textsl{\note:} #1}}
\newcommand{\ekstitle}{Eksempel }
\newcommand{\sprtitle}{Språkboksen}
\newcommand{\expl}{forklaring}

\newcommand{\vedlegg}[1]{\refstepcounter{vedl}\section*{Vedlegg \thevedl: #1}  \setcounter{vedleq}{0}}

\newcommand\sv{\vsk \textbf{Svar} \vspace{4 pt}\\}

%references
\newcommand{\reftab}[1]{\hrs{#1}{tabell}}
\newcommand{\rref}[1]{\hrs{#1}{regel}}
\newcommand{\dref}[1]{\hrs{#1}{definisjon}}
\newcommand{\refkap}[1]{\hrs{#1}{kapittel}}
\newcommand{\refsec}[1]{\hrs{#1}{seksjon}}
\newcommand{\refdsec}[1]{\hrs{#1}{delseksjon}}
\newcommand{\refved}[1]{\hrs{#1}{vedlegg}}
\newcommand{\eksref}[1]{\textsl{#1}}
\newcommand\fref[2][]{\hyperref[#2]{\textsl{figur \ref*{#2}#1}}}
\newcommand{\refop}[1]{{\color{blue}Oppgave \ref{#1}}}
\newcommand{\refops}[1]{{\color{blue}oppgave \ref{#1}}}
\newcommand{\refgrubs}[1]{{\color{blue}gruble \ref{#1}}}

\newcommand{\openmathser}{\openmath\,-\,serien}

% Exercises
\newcommand{\opgt}{\newpage \phantomsection \addcontentsline{toc}{section}{Oppgaver} \section*{Oppgaver for kapittel \thechapter}\vs \setcounter{section}{1}}


% Sequences and series
\newcommand{\sumarrek}{Summen av en aritmetisk rekke}
\newcommand{\sumgerek}{Summen av en geometrisk rekke}
\newcommand{\regnregsum}{Regneregler for summetegnet}

% Trigonometry
\newcommand{\sincoskomb}{Sinus og cosinus kombinert}
\newcommand{\cosfunk}{Cosinusfunksjonen}
\newcommand{\trid}{Trigonometriske identiteter}
\newcommand{\deravtri}{Den deriverte av de trigonometriske funksjonene}
% Solutions manual
\newcommand{\selos}{Se løsningsforslag.}
\newcommand{\se}[1]{Se eksempel på side \pageref{#1}}

%Vectors
\newcommand{\parvek}{Parallelle vektorer}
\newcommand{\vekpro}{Vektorproduktet}
\newcommand{\vekproarvol}{Vektorproduktet som areal og volum}


% 3D geometries
\newcommand{\linrom}{Linje i rommet}
\newcommand{\avstplnpkt}{Avstand mellom punkt og plan}


% Integral
\newcommand{\bestminten}{Bestemt integral I}
\newcommand{\anfundteo}{Analysens fundamentalteorem}
\newcommand{\intuf}{Integralet av utvalge funksjoner}
\newcommand{\bytvar}{Bytte av variabel}
\newcommand{\intvol}{Integral som volum}
\newcommand{\andordlindif}{Andre ordens lineære differensialligninger}




\begin{document}
\section{Introduksjon til Python}
\pythonut{helloworld.py}{
Hello world!
}

Python deler talll inn i tre typer:
\begin{center}
	\begin{tabular}{r|l}
		\texttt{int} & reelle heltall \\
		\texttt{float} & relle tall \\
		\texttt{complex} & komplekse tall
	\end{tabular} 
\end{center}
Vi skal i denne boka konsentrere oss om \texttt{int} og \texttt{float}. Tallypene definerer vi ved å ekskludere eller inkludere punktum:
\python{intfloat.py}
\pythonut{opr.py}{
a+b =  7\\
a-b =  3\\
a*b =  10\\
a/b =  2.5\\
a**b =  25\\
a//b =  2 \\
a\%b =  1
}
\subsection*{Boolske verdier og vilkår}
De boolske verdiene er verdiene \pytype{True} og \pytype{False}. Disse vil være resultatet når vi sjekker om objekter er like eller ulike. For å sjekke dette har vi de \outl{sammenlignende operatorene}:
\begin{center}
	\begin{tabular}{c|c}
		\textbf{operator} & \textbf{betydning} \\ \hline
		\texttt{==}	& er lik \\ \rowcolor{gray!10}
		\texttt{!=} & er \textsl{ikke} lik\\
		\texttt{>} & er større enn \\ \rowcolor{gray!10}
		\texttt{>=} & er større enn, eller lik \\
		\texttt{<} & er mindre enn \\ \rowcolor{gray!10}
		\texttt{<=} & er mindre enn, eller lik \\		
	\end{tabular}
\end{center}
\pythonut{bool1.py}{
False\\
True\\
True\\
False\\
}\vsk

I tillegg til de sammenlignende operatorene kan vi bruke de \outl{logiske operatorene} \pytype{and}, \pytype{or} og \pytype{not}
\pythonut{bool2.py}{
False\\
True\\
True

}
\spr{
Sjekker som bruker de sammenlignende og de logiske operatorene, skal vi heretter kalle \outl{vilkår}.
}
\subsection*{\pymet{if, else og elif} uttrykkene}
Når vi ønsker å utføre handlinger bare \textsl{hvis} et vilkår er sant (\pytype{True}), bruker vi \pymet{if} uttrykket foran vilkåret. Koden vi skriver med innrykk under \pymet{if}-linjen, vil bare bli utført hvis vilkåret gir \pytype{True}. 
\pythonut{if.py}{
 Jepp, c er større enn b
} 
Hvis man først vil sjekke om et vilkår er sant, og så utføre handlinger hvis det \textsl{ikke} er det, kan vi bruke \pymet{else} uttrykket:
\pythonut{else.py}{
Men denne kommer, fordi vilkåret i if-linja over var False
}
\pymet{else} uttrykket tar bare hensyn til (og gir ikke mening uten) \pymet{if} uttrykket like over seg. Hvis vi vil at handlinger skal utføres \textsl{bare} hvis ingen tidligere \pymet{if} uttrykk ga noe utslag, må vi bruke\footnote{\pymet{elif} er en forkortelse for \pymet{else if}, som også kan brukes.} \pymet{elif} uttrykket. Dette er et \pymet{if} uttrykk som slår inn hvis \pymet{if} uttrykket over \textsl{ikke} ga utslag.
\pythonut{elif.py}{
Nå er vi sikre på at 1 < b < 3
}

\subsection*{Lister}
Lister kan vi bruke for å samle objekter. Objektene som er i listen kalles \outl{elementene} til listen.
\python{list1.py}
Elementene i lister er \outl{indekserte}. Første objekt har indeks 0, andre objekt har indeks 1 og så videre:
\pythonut{list1.py}{
96 \\
99 \\
98
}
Med den innebygde funksjonen \texttt{append()} kan vi legge til et objekt i enden av listen. Dette er en \outl{innebygd funksjon}\footnote{Kort fortalt betyr det at det bare er noen typer objekter som kan bruke denne funksjonen.}, som vi skriver i enden av navnet på listen, med et punktum foran.
\pythonut{list3.py}{
[] \newline
[3] \newline
[3, 7]
}
Med funksjonen \texttt{pop()} kan vi hente ut et objekt fra listen
\pythonut{list4.py}{
a = 19 \\
min\_liste = [6, 10, 15] \\
a = 10 \\
min\_liste = [6, 15]
}
\info{Forklar for deg selv}{
Hva er forskjellen på å skive \texttt{a = min\_liste[1]} og å skrive \texttt{a = min\_liste.pop(1)}?
}

\subsection*{\pymet{for} looper}
For objekter som inneholder flere elementer, kan vi bruke \pymet{for}-looper til å utføre handlinger for hvert element. Handlingene må vi skrive med et innrykk etter \pymet{for}-uttrykket:
\pythonut{for.py}{
5 \\
50 \\[12pt]

10\\
100\\[12pt]

15\\
150
}
\spr{
Å gå gjennom hvert element i (for eksempel) en liste kalles \outl{å iterere over listen}.
}\vsk

Ofte er det ønskelig å iterere over heltallene $ 0, 1, 2 $ og så videre. Til dette kan vi bruke \pymet{range()}-funksjonen:
\pythonut{forrang.py}{
0 \\
1 \\
2
}
\subsection*{\pymet{while} loop}
Hvis vi ønsker at handlinger skal utføres fram til et vilkår er sant, kan vi bruke \pymet{while} metoden:
\pythonut{while1.py}{
1\\
2\\
3\\
4\\
}
\newpage

\info{Obs!}{
	Bruk aldri innebygde funksjoner som navn på variabler.
}


\end{document}