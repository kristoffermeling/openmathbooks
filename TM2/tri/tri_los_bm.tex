\documentclass[english, 11 pt, class=article, crop=false]{standalone}
\usepackage[T1]{fontenc}
%\renewcommand*\familydefault{\sfdefault} % For dyslexia-friendly text
\usepackage{lmodern} % load a font with all the characters
\usepackage{geometry}
\geometry{verbose,paperwidth=16.1 cm, paperheight=24 cm, inner=2.3cm, outer=1.8 cm, bmargin=2cm, tmargin=1.8cm}
\setlength{\parindent}{0bp}
\usepackage{import}
\usepackage[subpreambles=false]{standalone}
\usepackage{amsmath}
\usepackage{amssymb}
\usepackage{esint}
\usepackage{babel}
\usepackage{tabu}
\makeatother
\makeatletter

\usepackage{titlesec}
\usepackage{ragged2e}
\RaggedRight
\raggedbottom
\frenchspacing

% Norwegian names of figures, chapters, parts and content
\addto\captionsenglish{\renewcommand{\figurename}{Figur}}
\makeatletter
\addto\captionsenglish{\renewcommand{\chaptername}{Kapittel}}
\addto\captionsenglish{\renewcommand{\partname}{Del}}


\usepackage{graphicx}
\usepackage{float}
\usepackage{subfig}
\usepackage{placeins}
\usepackage{cancel}
\usepackage{framed}
\usepackage{wrapfig}
\usepackage[subfigure]{tocloft}
\usepackage[font=footnotesize,labelfont=sl]{caption} % Figure caption
\usepackage{bm}
\usepackage[dvipsnames, table]{xcolor}
\definecolor{shadecolor}{rgb}{0.105469, 0.613281, 1}
\colorlet{shadecolor}{Emerald!15} 
\usepackage{icomma}
\makeatother
\usepackage[many]{tcolorbox}
\usepackage{multicol}
\usepackage{stackengine}

\usepackage{esvect} %For vectors with capital letters

% For tabular
\usepackage{array}
\usepackage{multirow}
\usepackage{longtable} %breakable table

% Ligningsreferanser
\usepackage{mathtools}
\mathtoolsset{showonlyrefs}

% index
\usepackage{imakeidx}
\makeindex[title=Indeks]

%Footnote:
\usepackage[bottom, hang, flushmargin]{footmisc}
\usepackage{perpage} 
\MakePerPage{footnote}
\addtolength{\footnotesep}{2mm}
\renewcommand{\thefootnote}{\arabic{footnote}}
\renewcommand\footnoterule{\rule{\linewidth}{0.4pt}}
\renewcommand{\thempfootnote}{\arabic{mpfootnote}}

%colors
\definecolor{c1}{cmyk}{0,0.5,1,0}
\definecolor{c2}{cmyk}{1,0.25,1,0}
\definecolor{n3}{cmyk}{1,0.,1,0}
\definecolor{neg}{cmyk}{1,0.,0.,0}

% Lister med bokstavar
\usepackage[inline]{enumitem}

\newcounter{rg}
\numberwithin{rg}{chapter}
\newcommand{\reg}[2][]{\begin{tcolorbox}[boxrule=0.3 mm,arc=0mm,colback=blue!3] {\refstepcounter{rg}\phantomsection \large \textbf{\therg \;#1} \vspace{5 pt}}\newline #2  \end{tcolorbox}\vspace{-5pt}}

\newcommand\alg[1]{\begin{align} #1 \end{align}}

\newcommand\eks[2][]{\begin{tcolorbox}[boxrule=0.3 mm,arc=0mm,enhanced jigsaw,breakable,colback=green!3] {\large \textbf{Eksempel #1} \vspace{5 pt}\\} #2 \end{tcolorbox}\vspace{-5pt} }

\newcommand{\st}[1]{\begin{tcolorbox}[boxrule=0.0 mm,arc=0mm,enhanced jigsaw,breakable,colback=yellow!12]{ #1} \end{tcolorbox}}

\newcommand{\spr}[1]{\begin{tcolorbox}[boxrule=0.3 mm,arc=0mm,enhanced jigsaw,breakable,colback=yellow!7] {\large \textbf{Språkboksen} \vspace{5 pt}\\} #1 \end{tcolorbox}\vspace{-5pt} }

\newcommand{\sym}[1]{\colorbox{blue!15}{#1}}

\newcommand{\info}[2]{\begin{tcolorbox}[boxrule=0.3 mm,arc=0mm,enhanced jigsaw,breakable,colback=cyan!6] {\large \textbf{#1} \vspace{5 pt}\\} #2 \end{tcolorbox}\vspace{-5pt} }

\newcommand\algv[1]{\vspace{-11 pt}\begin{align*} #1 \end{align*}}

\newcommand{\regv}{\vspace{5pt}}
\newcommand{\mer}{\textsl{Merk}: }
\newcommand{\mers}[1]{{\footnotesize \mer #1}}
\newcommand\vsk{\vspace{11pt}}
\newcommand\vs{\vspace{-11pt}}
\newcommand\vsb{\vspace{-16pt}}
\newcommand\sv{\vsk \textbf{Svar} \vspace{4 pt}\\}
\newcommand\br{\\[5 pt]}
\newcommand{\figp}[1]{../fig/#1}
\newcommand\algvv[1]{\vs\vs\begin{align*} #1 \end{align*}}
\newcommand{\y}[1]{$ {#1} $}
\newcommand{\os}{\\[5 pt]}
\newcommand{\prbxl}[2]{
\parbox[l][][l]{#1\linewidth}{#2
	}}
\newcommand{\prbxr}[2]{\parbox[r][][l]{#1\linewidth}{
		\setlength{\abovedisplayskip}{5pt}
		\setlength{\belowdisplayskip}{5pt}	
		\setlength{\abovedisplayshortskip}{0pt}
		\setlength{\belowdisplayshortskip}{0pt} 
		\begin{shaded}
			\footnotesize	#2 \end{shaded}}}

\renewcommand{\cfttoctitlefont}{\Large\bfseries}
\setlength{\cftaftertoctitleskip}{0 pt}
\setlength{\cftbeforetoctitleskip}{0 pt}

\newcommand{\bs}{\\[3pt]}
\newcommand{\vn}{\\[6pt]}
\newcommand{\fig}[1]{\begin{figure}
		\centering
		\includegraphics[]{\figp{#1}}
\end{figure}}

\newcommand{\figc}[2]{\begin{figure}
		\centering
		\includegraphics[]{\figp{#1}}
		\caption{#2}
\end{figure}}

\newcommand{\sectionbreak}{\clearpage} % New page on each section

\newcommand{\nn}[1]{
\begin{equation}
	#1
\end{equation}
}

% Equation comments
\newcommand{\cm}[1]{\llap{\color{blue} #1}}

\newcommand\fork[2]{\begin{tcolorbox}[boxrule=0.3 mm,arc=0mm,enhanced jigsaw,breakable,colback=yellow!7] {\large \textbf{#1 (forklaring)} \vspace{5 pt}\\} #2 \end{tcolorbox}\vspace{-5pt} }
 
%colors
\newcommand{\colr}[1]{{\color{red} #1}}
\newcommand{\colb}[1]{{\color{blue} #1}}
\newcommand{\colo}[1]{{\color{orange} #1}}
\newcommand{\colc}[1]{{\color{cyan} #1}}
\definecolor{projectgreen}{cmyk}{100,0,100,0}
\newcommand{\colg}[1]{{\color{projectgreen} #1}}

% Methods
\newcommand{\metode}[2]{
	\textsl{#1} \\[-8pt]
	\rule{#2}{0.75pt}
}

%Opg
\newcommand{\abc}[1]{
	\begin{enumerate}[label=\alph*),leftmargin=18pt]
		#1
	\end{enumerate}
}
\newcommand{\abcs}[2]{
	\begin{enumerate}[label=\alph*),start=#1,leftmargin=18pt]
		#2
	\end{enumerate}
}
\newcommand{\abcn}[1]{
	\begin{enumerate}[label=\arabic*),leftmargin=18pt]
		#1
	\end{enumerate}
}
\newcommand{\abch}[1]{
	\hspace{-2pt}	\begin{enumerate*}[label=\alph*), itemjoin=\hspace{1cm}]
		#1
	\end{enumerate*}
}
\newcommand{\abchs}[2]{
	\hspace{-2pt}	\begin{enumerate*}[label=\alph*), itemjoin=\hspace{1cm}, start=#1]
		#2
	\end{enumerate*}
}

% Oppgaver
\newcommand{\opgt}{\phantomsection \addcontentsline{toc}{section}{Oppgaver} \section*{Oppgaver for kapittel \thechapter}\vs \setcounter{section}{1}}
\newcounter{opg}
\numberwithin{opg}{section}
\newcommand{\op}[1]{\vspace{15pt} \refstepcounter{opg}\large \textbf{\color{blue}\theopg} \vspace{2 pt} \label{#1} \\}
\newcommand{\ekspop}[1]{\vsk\textbf{Gruble \thechapter.#1}\vspace{2 pt} \\}
\newcommand{\nes}{\stepcounter{section}
	\setcounter{opg}{0}}
\newcommand{\opr}[1]{\vspace{3pt}\textbf{\ref{#1}}}
\newcommand{\oeks}[1]{\begin{tcolorbox}[boxrule=0.3 mm,arc=0mm,colback=white]
		\textit{Eksempel: } #1	  
\end{tcolorbox}}
\newcommand\opgeks[2][]{\begin{tcolorbox}[boxrule=0.1 mm,arc=0mm,enhanced jigsaw,breakable,colback=white] {\footnotesize \textbf{Eksempel #1} \\} \footnotesize #2 \end{tcolorbox}\vspace{-5pt} }
\newcommand{\rknut}{
Rekn ut.
}

%License
\newcommand{\lic}{\textit{Matematikken sine byggesteinar by Sindre Sogge Heggen is licensed under CC BY-NC-SA 4.0. To view a copy of this license, visit\\ 
		\net{http://creativecommons.org/licenses/by-nc-sa/4.0/}{http://creativecommons.org/licenses/by-nc-sa/4.0/}}}

%referances
\newcommand{\net}[2]{{\color{blue}\href{#1}{#2}}}
\newcommand{\hrs}[2]{\hyperref[#1]{\color{blue}\textsl{#2 \ref*{#1}}}}
\newcommand{\rref}[1]{\hrs{#1}{regel}}
\newcommand{\refkap}[1]{\hrs{#1}{kapittel}}
\newcommand{\refsec}[1]{\hrs{#1}{seksjon}}

\newcommand{\mb}{\net{https://sindrsh.github.io/FirstPrinciplesOfMath/}{MB}}


%line to seperate examples
\newcommand{\linje}{\rule{\linewidth}{1pt} }

\usepackage{datetime2}
%%\usepackage{sansmathfonts} for dyslexia-friendly math
\usepackage[]{hyperref}


\newcommand{\note}{Merk}
\newcommand{\notesm}[1]{{\footnotesize \textsl{\note:} #1}}
\newcommand{\ekstitle}{Eksempel }
\newcommand{\sprtitle}{Språkboksen}
\newcommand{\expl}{forklaring}

\newcommand{\vedlegg}[1]{\refstepcounter{vedl}\section*{Vedlegg \thevedl: #1}  \setcounter{vedleq}{0}}

\newcommand\sv{\vsk \textbf{Svar} \vspace{4 pt}\\}

%references
\newcommand{\reftab}[1]{\hrs{#1}{tabell}}
\newcommand{\rref}[1]{\hrs{#1}{regel}}
\newcommand{\dref}[1]{\hrs{#1}{definisjon}}
\newcommand{\refkap}[1]{\hrs{#1}{kapittel}}
\newcommand{\refsec}[1]{\hrs{#1}{seksjon}}
\newcommand{\refdsec}[1]{\hrs{#1}{delseksjon}}
\newcommand{\refved}[1]{\hrs{#1}{vedlegg}}
\newcommand{\eksref}[1]{\textsl{#1}}
\newcommand\fref[2][]{\hyperref[#2]{\textsl{figur \ref*{#2}#1}}}
\newcommand{\refop}[1]{{\color{blue}Oppgave \ref{#1}}}
\newcommand{\refops}[1]{{\color{blue}oppgave \ref{#1}}}
\newcommand{\refgrubs}[1]{{\color{blue}gruble \ref{#1}}}

\newcommand{\openmathser}{\openmath\,-\,serien}

% Exercises
\newcommand{\opgt}{\newpage \phantomsection \addcontentsline{toc}{section}{Oppgaver} \section*{Oppgaver for kapittel \thechapter}\vs \setcounter{section}{1}}


% Sequences and series
\newcommand{\sumarrek}{Summen av en aritmetisk rekke}
\newcommand{\sumgerek}{Summen av en geometrisk rekke}
\newcommand{\regnregsum}{Regneregler for summetegnet}

% Trigonometry
\newcommand{\sincoskomb}{Sinus og cosinus kombinert}
\newcommand{\cosfunk}{Cosinusfunksjonen}
\newcommand{\trid}{Trigonometriske identiteter}
\newcommand{\deravtri}{Den deriverte av de trigonometriske funksjonene}
% Solutions manual
\newcommand{\selos}{Se løsningsforslag.}
\newcommand{\se}[1]{Se eksempel på side \pageref{#1}}

%Vectors
\newcommand{\parvek}{Parallelle vektorer}
\newcommand{\vekpro}{Vektorproduktet}
\newcommand{\vekproarvol}{Vektorproduktet som areal og volum}


% 3D geometries
\newcommand{\linrom}{Linje i rommet}
\newcommand{\avstplnpkt}{Avstand mellom punkt og plan}


% Integral
\newcommand{\bestminten}{Bestemt integral I}
\newcommand{\anfundteo}{Analysens fundamentalteorem}
\newcommand{\intuf}{Integralet av utvalge funksjoner}
\newcommand{\bytvar}{Bytte av variabel}
\newcommand{\intvol}{Integral som volum}
\newcommand{\andordlindif}{Andre ordens lineære differensialligninger}



\begin{document}

\textit{Så lenge ikke annet er nevnt, tas det for gitt at $ n\in \mathbb{Z} $.} \vsk

\opr{rads}\\
\textbf{a)}
\algv{
	60^\circ &= \frac{60\pi}{180}\br
	&= \frac{\pi}{3}
}
\textbf{b)}
\algv{
	15^\circ &= \frac{15\pi}{180}\br
	&= \frac{\pi}{12}
}
\opr{grad}\\
\textbf{a)}
\algv{
	\frac{11\pi}{12} &= \frac{11\pi}{12}\cdot\frac{180}{\pi}^\circ\br
	&= 165^\circ
}
\textbf{b)}
\algv{
	\frac{11\pi}{6} &= \frac{11\pi}{6}\cdot\frac{180}{\pi}^\circ\br
	&= 330^\circ
}

\opr{pytg}\\
I enhetssirkelen er $ |\cos x| $ og $ |\sin x| $ katetene i en rettvinklete trekant med en hypotenus med lengde lik 1. Av Pygagoras' setning får man da at:
\alg{
	|\cos x|^2 + |\sin x|^2 =1\\
	\cos^2 x + \sin^2 x = 1
}

\opr{tanx}\\
\textbf{a)} \algv{\tan x&= \frac{\sin x}{\cos x} \br
	&= \frac{0}{1} \\
	&= 0}
\newpage
\textbf{b)} \algv{\tan x& = \frac{\frac{1}{2}}{-\frac{\sqrt{3}}{2}} \br
	&= -\frac{1}{2}\cdot\frac{2}{\sqrt{3}} \br
	&= -\frac{1}{\sqrt{3}}}


\opr{bruk1}\\
Vi har at:
\alg{
	\cos^2\left(\frac{\pi}{6}\right) + \sin^2\left(\frac{\pi}{6}\right)  &= 1\\
	\sin^2\left(\frac{\pi}{6}\right) &= 1-\left(\frac{\sqrt{3}}{2}\right)^2 \\
	\sin^2\left(\frac{\pi}{6}\right) &= 1- \frac{3}{4} \\
	\sin\left(\frac{\pi}{6}\right) &= \pm\sqrt{\frac{1}{4}}\\
	&= \pm\frac{1}{2}
}
Siden $ \frac{\pi}{6} $ ligger i første kvadrant, må sinusverdien være positiv, og altså lik $ \frac{1}{2} $.\vsk

\opr{sin2xopg}\\
\textbf{a)} Vi legger merke til at $ {2x=x+x} $, av \eqref{suv} har vi da at:
\alg{
	\sin(2x) &= \sin(x+x) \\
	&= \sin x \cos x + \cos x \sin x \\
	=& 2 \cos x \sin x
}
\textbf{b)} 
\algv{
	2\cos \left(\frac{\pi}{12}\right)\sin \left(\frac{\pi}{12}\right)&=
	\sin\left(\frac{\pi}{6}\right)\\ &= \frac{1}{2} }

\opr{cossomsin} \\
Vi legger til og trekker ifra $ \frac{\pi}{2} $ i argumentet:
\[ \cos\left(3x-\frac{7\pi}{2}\right) = \cos\left(3x-\frac{7\pi}{2}+\frac{\pi}{2}-\frac{\pi}{2}\right) \]
Fra (\ref{cossomsi}) har vi da at
\alg{\cos\left(3x-\frac{7\pi}{2}+\frac{\pi}{2}-\frac{\pi}{2}\right) &= \sin\left(3x-\frac{7\pi}{2}+\frac{\pi}{2}\right) \\
	&= \sin(3x + 2\pi) \\
	&= \sin(3x)}

\opr{cossinsin}\\
Av (\ref{rsin}) vet vi at vi kan skrive:
\alg{f(x) = r \sin(2x + c) }
hvor $ r $ er gitt som:
\algv{
	r &= \sqrt{1^2+\sqrt{3}^2} \\
	&= 2
}
mens $ c $ er gitt ved ligningssettet: 
\algv{\cos c &=\frac{\sqrt{3}}{2} \br
	\sin c &= \frac{1}{2}}
En $ c $ som oppfyller dette er $ c = \frac{\pi}{6} $.

Altså får vi:
\[ \cos( 2x) + \sqrt 3\sin (2x) =2 \sin\left(2x+\frac{\pi}{6}\right) \]

\opr{sincos0}\\
\textbf{a)} Siden $ \acos 0 = \frac{\pi}{2} $ har vi fra (\ref{coslig}) at:
\[ x = \frac{\pi}{2}+2\pi n\quad \vee \quad -\frac{\pi}{2}+2\pi n \]
Men siden $ n $ er et vilkårlig heltall, kan vi alltids trekke ut $ 2\pi $ fra leddet $ 2\pi n $ uten at løsningen er forandret. Gjør vi dette kan vi skrive:
\alg{
	-\frac{\pi}{2} + 2\pi n &= -\frac{\pi}{2} + \pi+\pi + 2\pi n \br
	&= \frac{\pi}{2} + \pi(2n+1)
}
Vi ser da at vi kan skrive de to løsningene som $ \frac{\pi}{2} $ pluss enten et partalls antall $ \pi $ eller et oddetalls antall $ \pi $. Altså  kan vi skrive begge løsningene kombinert som $ \frac{\pi}{2} $ pluss alle heltalls antall $ \pi $:
\[ x = \frac{\pi}{2}+\pi n \]
\textbf{b)} Siden $ \asin 0 = 0 $ har vi fra (\ref{sinlig}) at:
\[ x = 2\pi n \quad\vee\quad x=\pi +2\pi n \]
Men siden $ x=\pi + 2\pi=\pi(2n+1) $, kan vi med samme argumentasjon som i opg. a) skrive:
\[ x = \pi n \]
\opr{triligns}\\
\textbf{a)} \alg{
	\cos x&=\frac{\sqrt{2}}{2} \\
	x&= \pm \frac{\pi}{4}+2\pi n
}
\textbf{b)} Av opg. \ref{sincos0} a) vet vi at:
\alg{
	\frac{\pi}{3}x &= \frac{\pi}{2}+\pi n \\
	x &= \frac{3}{2}(1+2 n)
}
Faktoriseringen av uttrykket over gjør det lettere å identifisere hvilke $ x $ som ligger i intervallet $ [0, 5] $. Dette er $ x=\frac{3}{2} $ og $ x = \frac{9}{2} $. \vsk

\textbf{c)} \algv{
	2\sin(3x) &=1 \\
	\sin(3x)&= \frac{1}{2} \\
}
Vinklene $ \frac{\pi}{6} $ og $ \frac{5\pi}{6} $ har sinusverdien $ \frac{1}{2} $, altså har vi at:
\alg{
	3x &= \frac{\pi}{6} +2\pi n \\
	x &= \frac{1}{3}\left(\frac{\pi}{6}+2\pi n\right)
}
eller at:
\alg{
	3x &= \frac{5\pi}{6} +2\pi n \\
	x &= \frac{1}{3}\left(\frac{5\pi}{6}+2\pi n\right)
}
\textbf{d)} Siden $ -\frac{\pi}{3} $ og $ -\frac{2\pi}{3} $ har sinusverdien, har vi at:
\alg{
	2x-\pi &= -\frac{\pi}{3}+2\pi n \\
	x &= \frac{1}{2}\left(\frac{2\pi}{3}+2\pi n\right) \\
	&= \frac{\pi}{3}(1+3n)
}
eller at:
\alg{
	2x-\pi &= -\frac{2\pi}{3}+2\pi n \\
	x &= \frac{1}{2}\left(\frac{\pi}{3}+2\pi n\right) \\
	&= \frac{\pi}{6}(1+6n)
}
\textbf{e)} \algv{
	2\sqrt{3}\tan \left(4x+\frac{\pi}{2}\right) &= 2 \\
	\tan \left(4x+\frac{\pi}{2}\right) &= \frac{1}{\sqrt{3}}
}
Siden $ \frac{\pi}{6} $ har $ \frac{1}{\sqrt{3}} $ som tangensverdi, har vi at:
\alg{
	4x+\frac{\pi}{2} &= \frac{\pi}{6}+\pi n\\
	x &= \frac{1}{4}\left(\pi n-\frac{\pi}{3}\right)
}
\opr{asinbcos0o}\\
\textbf{a} \algv{
	\sqrt{3} \sin x - \cos x &=0 \\
	\frac{\sqrt{3}\sin x}{\cos x} &= \frac{\cos x}{\cos x}\\
	\tan x &= \frac{1}{\sqrt{3}}\\
	x &= \frac{\pi}{6}+\pi n
}

\textbf{b)} \algv{
	\sin x + \sqrt{3}\cos x &=0 \\
	\frac{\sin x}{\cos x} &= -\sqrt{3} \br
	\tan x &= -\sqrt{3} \\
	x &= -\frac{\pi}{3}+\pi n
}

\textbf{c)} \algv{
	\cos (2x) + \sin (2x)=&0 \\
	\tan (2x) &= -1 \\
	2x &= -\frac{\pi}{4}+\pi n \\
	x &= \frac{1}{2}\left(\pi n-\frac{\pi}{4}\right)
}

\opr{kombo}\\
\textbf{a)}  Vi omskriver ligningen til sinusligningen
\alg{
	r\sin(x+c)&= \sqrt{2}
}
hvor:
\alg{r &= \sqrt{1^2+1^2} \\
	&= \sqrt{2}
}
og der $ c $ oppfyller ligningssettet:
\alg{
	\cos c &= \frac{1}{\sqrt{2}} \br
	\sin c &=  \frac{1}{\sqrt{2}}
}
$ c=\frac{\pi}{4} $ oppfyller kravene over \big($ {\frac{1}{\sqrt{2}}=\frac{2}{\sqrt{2}\sqrt{2}}=\frac{\sqrt{2}}{2}} $\,\big), altså får vi:
\alg{\sqrt{2}\sin\left(x+\frac{\pi}{4}\right) &= \sqrt{2} \br
	\sin\left(x+\frac{\pi}{4}\right)  &= 1
}
Altså må:
\alg{
	x+\frac{\pi}{4}&= \frac{\pi}{2}+2\pi n \\
	x &= \frac{\pi}{4}+2\pi n
}

\textbf{b)} Vi omskriver ligningen til sinusligningen
\alg{
	r\sin\left(\frac{x}{2\pi}+c\right)&= 1
}
hvor:
\alg{r &= \sqrt{\sqrt{3}^2+(-1)^2} \\
	&= 2
}
og der $ c $ oppfyller ligningssettet:
\alg{
	\cos c &= -\frac{1}{2} \br
	\sin c &= \frac{\sqrt{3}}{2}
}
$ c=\frac{2\pi}{3} $ oppfyller kravene over, altså får vi:
\alg{2\sin\left(\frac{x}{2\pi}+\frac{2\pi}{3}\right) &= 1 \br
	\sin\left(\frac{x}{2\pi}+\frac{2\pi}{3}\right) &= \frac{1}{2} 
}
Vi må altså enten ha at:
\alg{\frac{x}{2\pi}+\frac{2\pi}{3} &= \frac{\pi}{6}+2\pi n \br
	\frac{x}{2\pi}&=-\frac{\pi}{2}+2\pi n\br
	x &= \pi^2(4n-1) }
eller at:
\alg{\frac{x}{2\pi}+\frac{2\pi}{3} &= \frac{5\pi}{6}+2\pi n \br
	\frac{x}{2\pi}&=\frac{\pi}{6}+2\pi n\br
	x &= 2\pi\left(\frac{\pi}{6}+2\pi n\right) }

\opr{abctri}\\
\textbf{a)} Vi løser andregradsligningen mhp. $ \sin x $, og får at:
\[ \sin x = \frac{1}{2}\quad\vee\quad x= -1 \]
I tilfellet der $ {\sin x = \frac{1}{2}} $ har vi at
\[ x = \frac{\pi}{6}+2\pi n\quad\vee\quad \frac{5\pi}{6}+2\pi n \]
I tilfellet der $ {\sin x = -1} $, har vi at
\[ x = -\frac{\pi}{2}+2\pi n\]
\textbf{b)} Vi løser andregradsligningen mhp. $ \cos (3x) $, og får at:
\[ \cos (3x) = 4\quad\vee\quad x= 1 \]
Ligningen $ {\cos (3x) = 4} $ har ligningen reell løsning. Hvis $ {\cos (3x) = -1} $, har vi at
\[ x = \frac{\pi +2\pi n}{3}\]
\textbf{c)} Vi løser andregradsligningen mhp. $ \cos x $, og får at:
\[ \cos x = 4\quad\vee\quad x= 1 \]
I tilfellet av at $ {\cos x = -4} $ har ligningen ingen reell løsning. I tilfellet av at $ {\sin x = -1} $, må:
\[ x = 2\pi n\]
\textbf{d)} Vi løser andregradsligningen mhp. $ \tan(\pi x) $, og får at:
\[ \tan(\pi x) = \sqrt{3}\]
Altså må:
\[ x = \frac{1}{3}+n\]

\opr{kvadopg}\\
\textbf{a)} 
\algv{
	-\cos^2 x+15\sin^2 x &= 3(\cos^2 x + \sin^2 x) \\
	12 \sin^2 x &= 4\cos^2 x \\
	\frac{ 12 \sin^2 x }{\cos^2 x}&= \frac{4\cos^2 x }{\cos^2 x}\\
	12 \tan^2 x &= 4 \\
	\tan^2 x &= \frac{1}{3} \\
	\tan x &= \pm \frac{1}{\sqrt{3}}
}
Siden $ {\atan \frac{1}{\sqrt{3}}=\frac{\pi}{6}} $ og $ {\atan \left(-\frac{1}{\sqrt{3}}\right)=-\frac{\pi}{6}} $, får vi at:
\[ x = \pm \frac{\pi}{6}+\pi n \]

\textbf{b)} Vi starter med å mulitpliserer ligningen med $ 2 $, og får:
\alg{
	2\cos^2 \left(\frac{x}{4}\right) - 4 \sin^2 \left(\frac{x}{4}\right) &= -1 \br
	2\cos^2 \left(\frac{x}{4}\right) - 4 \sin^2 \left(\frac{x}{4}\right) &= -1\left(\cos^2\left(\frac{1}{4}\right)+\sin^2\left(\frac{1}{4}\right)\right) \br
	- 3 \sin^2 \left(\frac{x}{4}\right)&= -3\cos^2 \left(\frac{x}{4}\right) \br
	\tan^2 \left(\frac{x}{4}\right)&= 1 \br
	\tan \left(\frac{x}{4}\right) &= \pm 1
}
Siden $ \atan 1 = \frac{\pi}{4} $ og $ \atan (-1) = -\frac{\pi}{4} $, får vi at:
\alg{
	\frac{x}{4} &= \pm \frac{\pi}{4}+\pi n \\
	x&= \pm \pi + 4\pi n
}
Det er helt greit slå seg til ro med svaret over (på en eksamen vil dette gi full uttelling), men vi skal likevel foreta en operasjon som gjør at uttrykket blir enda mer kompakt:\vsk

Siden vi av leddet $ 4\pi n $ kan trekke ut så mange multiplum av $ 4\pi $ vi måtte ønske, kan vi for løsningen hvor $ \pi $ har negativt fortegn skrive:
\alg{x &= -\pi + 4\pi + 4\pi n \\
	x &= \pi +2\pi + 4\pi n \\
	&= \pi +2\pi(2n+1)
}
Videre utfører vi en enkel faktorisering av løsningen der $ \pi $ er positiv:
\[ x=\pi+ 2\pi(2n) \]
Av de to uttrykkene over innser vi at $ x $ kan skrives som $ \pi $ pluss $ 2\pi $ ganget med et vilkårlig oddetall eller et vilkårlig partall, altså alle heltall! Derfor får vi at:
\[ x = \pi +2\pi n \]

\opr{cosmaks}\\
\textbf{a)} Når $ a>0 $ må $ f $ må ha sin største verdi når cosinusverdien er lik 1, altså når:
\[ \cos(kx + c) = 1 \]
Av (\ref{coslig1}) får vi da at:
\[ kx +c = 2\pi n \]
Videre må $ f $ ha sin minste verdi når cosinusverdien er lik $ -1 $, altså når:
\[ \cos(kx + c) = -1 \]
Av (\ref{coslig_1}) får vi da at:
\[ kx +c = 2\pi n \]

\textbf{b)} Når $ a<0 $ får vi bare det omvendte av situasjonen i opg. a).\vsk

\opr{cosfunko}\\
\textbf{a)}
Perioden $ P $ finner vi av relasjonen:
\[ k = \frac{2\pi}{P}  \]
som gir:
\algv{
	P &= \frac{2\pi}{k}\br
	&= \frac{2\pi}{3}
}
\textbf{b)} Maksimumsverdien $ f_{maks} $ er amplituden addert med konstantleddet:
\alg{f_{maks}&=3+4 \\
	&= 7
}
Minimumsverdien $ f_{min} $ er konstandleddet fratrekt amplituden:
\alg{f_{min}&=4-3 \\
	&= 1
}
\textbf{c)} $ f $ må ha sitt maksimum når cosinusverdien blir $ -1 $, dette skjer når:
\alg{3x+\frac{\pi}{12} &= \pi +2\pi n \br
	3x &=  \pi-\frac{\pi}{12}+2\pi n \br 
	x &= \frac{1}{3}\left(2\pi n-\frac{\pi}{12}\right)
}
Videre må $ f $ ha sitt minimum når cosinusverdien blir $ 1 $, altså når:
\alg{3x+\frac{\pi}{12} &= 2\pi n \br
	3x &= 2\pi n -\frac{\pi}{12} \br
	x &= \frac{1}{3}\left(2\pi n-\frac{\pi}{12}\right)
}

\opr{sinmaks}\\
\textbf{a)} Når $ a>0 $ må $ f $ må ha sin største verdi når sinusverdien er lik 1, altså når:
\[ \sin(kx + c) = 1 \]
Av (\ref{sinlig1}) får vi da at:
\[ kx +c = \frac{\pi}{2}+ 2\pi n \]

Videre må $ f $ ha sin minste verdi når sinusverdien er lik $ -1 $, altså når:
\[ \sin(kx + c) = -1 \]
Av (\ref{sinlig_1}) får vi da at:
\[ kx +c = -\frac{\pi}{2}+2\pi n \]

\textbf{b)} Når $ a<0 $ får vi bare det omvendte av situasjonen i opg. a).\vsk 

\opr{sinfnpkt}\\
\textbf{a)} \algv{
	P &= \frac{2\pi}{k} \br
	&= \frac{2\pi}{\frac{\pi}{2}} \br
	&= 4
}
\textbf{b)} $ f $ har maksimumspunkter der hvor sinusverdien er lik $ -1 $, altså når:
\alg{ 
	\frac{\pi}{2} x &= -\frac{\pi}{2}+2\pi n\\
	x &= 4 n-1
}
På intervallet $ x\in[-3, 3] $ vil $ x=-1 $ og $ x=3 $ oppfylle ligningen over. $ f_{maks} $ vil i alle tilfeller være lik 3, derfor blir toppunktene $ (-1, 3) $ og $ (3, 3) $.\vsk

\textbf{c)} \alg{
	-2\sin\left(\frac{\pi}{2}x\right) +1&=0  \br
	\sin\left(\frac{\pi}{2}x\right) &= \frac{1}{2}\br
}
Av (\ref{sinlig}) kan vi enten ha at:
\alg{\frac{\pi}{2}x &= \frac{\pi}{6}+2\pi n \br
	x &= \frac{1}{3}(1+12n)}
eller at:
\alg{\frac{\pi}{2}x &= \pi-\frac{\pi}{6}+2\pi n \br
	x &= \frac{1}{3}(5+12n)}
På intervallet $ x\in[-3, 3] $ vil $ x=-\frac{7}{3} $, $ x=\frac{5}{3} $ og $ x=\frac{1}{3} $ oppfylle én av de to ligningene over, dette er altså nullpunktene til $ f $.

\textbf{d)} Vi markerer nullpunktene og toppunktene og skisserer sinuskurven som går mellom disse på intervallet $ x\in[-3, 3] $.\vsk

\opr{skissin}\\
Perioden til $ f $ tilsvarer den horisontale avstanden mellom toppunktene, som er $ 2\pi $. Amplituden er vertikalavstanden mellom likevektslinja og toppunktene, som altså er 2. Bunnpunktene ligger en halv bølgelengde unna toppunktene, og siden amplituden er 2, må bunnpunktene ha $ y $-verdien $ 1-2=-1 $.\vs
\fig{skissin}

\newpage
\opr{sincosfig}
\vspace{-10 pt}
\subimport{fig/}{cosopg2}
\abc{
	\item Vi observerer at vertikalavstanden mellom toppunkt og bunnpunkt til grafen er 6, som betyr at amplituden til funksjonen er 3. Horisontalavstanden mellom to toppunkt er $ 2 $, og da har vi av \eqref{ker2pioverP} at
	\[ k= \frac{2\pi}{2}=\pi \]
	Videre har  vi av \eqref{c} at
	\alg{
		\pi\cdot1 +c &=0 \\
		c &= -\pi
	}
	Topp- og bunnpunktene har den samme vertikalavstanden til linja $ y=2 $, som dermed er likvektslinja til funksjonen. Dermed har vi at
	\[ f(x)=3\cos(\pi x-\pi)+2 \]
	\item Sinusfunksjonen finner vi på tilsvarende måte som ved cosinusfunksjonen i a), med eneste forskjell at $ c $ er gitt av \eqref{sinc}. Eventult kan vi bruke \eqref{cossomsi}, og får da
	\alg{
		f(x)&=3\cos(\pi x-\pi)+2\\
		&=3\cos\left(\pi x-\pi +\frac{\pi}{2}-\frac{\pi}{2}\right)\\
		&=3\sin\left(\pi x -\frac{\pi}{2}\right) 
	}
}

\newpage

\opr{tanfopg}
Vi har at
\[ f(x)= a \tan (kx+c)+d=a\frac{\sin(kx+c)}{\cos(kx+c)}+d \]
Når cosinus-verdien går mot 0, vil sinus-verdien gå mot $ \pm 1 $, og dermed vil verdien gå mot $ \pm \infty $. Funksjonen har dermed en vertikal asymptote når $ \cos(k x+c) = 0 $. Av \eqref{cosligeq} har vi at da er
\[ kx+c=\pm\frac{\pi}{2}+2\pi n \]

\grubr{opgtangr}
Fordi forholdet $ \frac{a}{b}$ er det samme som $ \frac{\sin x}{\cos x} $, må det finnes et tall $ c $ som er slik at:
\alg{
	c\sin x = a \tag{I} \label{gr21}\\
	c\cos x = b \tag{II} \label{gr22}
}
Kvadrerer vi begge ligninger og legger dem sammen, får vi:
\alg{
	(c\sin x)^2+(c\cos x)^2 &= a^2 + b^2 \\
	c^2 &= a^2+b^2 \\
	c &= \sqrt{a^2+b^2}
}
Setter vi dette tilbake i \eqref{gr21} og \eqref{gr22}, finner vi at:
\alg{ \sin x &= \frac{a}{\sqrt{a^2+b^2}} \\
	\cos x &= \frac{b}{\sqrt{a^2+b^2}}
}

\grubr{opgtridet} \\
Vi setter $ \vec{u}=[a, b] $, $ \vec{v}=[c, d] $. Videre lar vi $ \alpha $ og $\beta $ være vinklene som henholdsvis $ \vec{u} $ og $ \vec{v} $ danner med horisontalaksen. Da har vi at
\alg{
\det(\vec{u}, \vec{v})&=ad-bc \\
&= bd\left(\frac{a}{b}-\frac{c}{d}\right) \\
&= bd\left(\tan \alpha-\tan \beta\right)\br
&= bd\left(\frac{\sin \alpha}{\cos \alpha}-\frac{\sin \alpha}{\cos \alpha}\right)\br \\
&= \frac{bd}{\cos \alpha \cos \beta}\left(\sin \alpha \cos \beta-\sin \beta \cos\alpha\right)
}
Av \eqref{sinuminv} i \tmto\ har vi at
\[ \sin \alpha \cos \beta-\sin \beta \cos\alpha = \sin(\alpha-\beta) \]
hvor $ \alpha-\beta=\pm \angle(\vec{u}, \vec{v}) $. I tillegg er 
\[ \frac{b}{\cos \alpha}=|\vec{u}|\qquad,\qquad\frac{d}{\cos \beta}=|\vec{v}| \]
Dermed er
\alg{
\frac{bd}{\cos \alpha \cos \beta}\left(\sin \alpha \cos \beta-\sin \beta \cos\alpha\right) = |\vec{u}||\vec{v}|\sin(\pm \angle(\vec{u}, \vec{v}))
}
Da av \eqref{-sinxsinx} i \tmto\ har vi at $ |\sin (-x)|=\sin x $, og dermed er
\[|\det(\vec{u}, \vec{v})|= ||\vec{u}||\vec{v}|\sin(\pm \angle(\vec{u}, \vec{v}))|=|\vec{u}||\vec{v}|\sin \angle(\vec{u}, \vec{v}) \]
%\det(\vec{u}, \vec{v})=|u||v|\sin\angle(\vec{u}, \vec{v})
\end{document}


