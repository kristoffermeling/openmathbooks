\documentclass[english, 11 pt, class=article, crop=false]{standalone}
\usepackage[T1]{fontenc}
%\renewcommand*\familydefault{\sfdefault} % For dyslexia-friendly text
\usepackage{lmodern} % load a font with all the characters
\usepackage{geometry}
\geometry{verbose,paperwidth=16.1 cm, paperheight=24 cm, inner=2.3cm, outer=1.8 cm, bmargin=2cm, tmargin=1.8cm}
\setlength{\parindent}{0bp}
\usepackage{import}
\usepackage[subpreambles=false]{standalone}
\usepackage{amsmath}
\usepackage{amssymb}
\usepackage{esint}
\usepackage{babel}
\usepackage{tabu}
\makeatother
\makeatletter

\usepackage{titlesec}
\usepackage{ragged2e}
\RaggedRight
\raggedbottom
\frenchspacing

% Norwegian names of figures, chapters, parts and content
\addto\captionsenglish{\renewcommand{\figurename}{Figur}}
\makeatletter
\addto\captionsenglish{\renewcommand{\chaptername}{Kapittel}}
\addto\captionsenglish{\renewcommand{\partname}{Del}}


\usepackage{graphicx}
\usepackage{float}
\usepackage{subfig}
\usepackage{placeins}
\usepackage{cancel}
\usepackage{framed}
\usepackage{wrapfig}
\usepackage[subfigure]{tocloft}
\usepackage[font=footnotesize,labelfont=sl]{caption} % Figure caption
\usepackage{bm}
\usepackage[dvipsnames, table]{xcolor}
\definecolor{shadecolor}{rgb}{0.105469, 0.613281, 1}
\colorlet{shadecolor}{Emerald!15} 
\usepackage{icomma}
\makeatother
\usepackage[many]{tcolorbox}
\usepackage{multicol}
\usepackage{stackengine}

\usepackage{esvect} %For vectors with capital letters

% For tabular
\usepackage{array}
\usepackage{multirow}
\usepackage{longtable} %breakable table

% Ligningsreferanser
\usepackage{mathtools}
\mathtoolsset{showonlyrefs}

% index
\usepackage{imakeidx}
\makeindex[title=Indeks]

%Footnote:
\usepackage[bottom, hang, flushmargin]{footmisc}
\usepackage{perpage} 
\MakePerPage{footnote}
\addtolength{\footnotesep}{2mm}
\renewcommand{\thefootnote}{\arabic{footnote}}
\renewcommand\footnoterule{\rule{\linewidth}{0.4pt}}
\renewcommand{\thempfootnote}{\arabic{mpfootnote}}

%colors
\definecolor{c1}{cmyk}{0,0.5,1,0}
\definecolor{c2}{cmyk}{1,0.25,1,0}
\definecolor{n3}{cmyk}{1,0.,1,0}
\definecolor{neg}{cmyk}{1,0.,0.,0}

% Lister med bokstavar
\usepackage[inline]{enumitem}

\newcounter{rg}
\numberwithin{rg}{chapter}
\newcommand{\reg}[2][]{\begin{tcolorbox}[boxrule=0.3 mm,arc=0mm,colback=blue!3] {\refstepcounter{rg}\phantomsection \large \textbf{\therg \;#1} \vspace{5 pt}}\newline #2  \end{tcolorbox}\vspace{-5pt}}

\newcommand\alg[1]{\begin{align} #1 \end{align}}

\newcommand\eks[2][]{\begin{tcolorbox}[boxrule=0.3 mm,arc=0mm,enhanced jigsaw,breakable,colback=green!3] {\large \textbf{Eksempel #1} \vspace{5 pt}\\} #2 \end{tcolorbox}\vspace{-5pt} }

\newcommand{\st}[1]{\begin{tcolorbox}[boxrule=0.0 mm,arc=0mm,enhanced jigsaw,breakable,colback=yellow!12]{ #1} \end{tcolorbox}}

\newcommand{\spr}[1]{\begin{tcolorbox}[boxrule=0.3 mm,arc=0mm,enhanced jigsaw,breakable,colback=yellow!7] {\large \textbf{Språkboksen} \vspace{5 pt}\\} #1 \end{tcolorbox}\vspace{-5pt} }

\newcommand{\sym}[1]{\colorbox{blue!15}{#1}}

\newcommand{\info}[2]{\begin{tcolorbox}[boxrule=0.3 mm,arc=0mm,enhanced jigsaw,breakable,colback=cyan!6] {\large \textbf{#1} \vspace{5 pt}\\} #2 \end{tcolorbox}\vspace{-5pt} }

\newcommand\algv[1]{\vspace{-11 pt}\begin{align*} #1 \end{align*}}

\newcommand{\regv}{\vspace{5pt}}
\newcommand{\mer}{\textsl{Merk}: }
\newcommand{\mers}[1]{{\footnotesize \mer #1}}
\newcommand\vsk{\vspace{11pt}}
\newcommand\vs{\vspace{-11pt}}
\newcommand\vsb{\vspace{-16pt}}
\newcommand\sv{\vsk \textbf{Svar} \vspace{4 pt}\\}
\newcommand\br{\\[5 pt]}
\newcommand{\figp}[1]{../fig/#1}
\newcommand\algvv[1]{\vs\vs\begin{align*} #1 \end{align*}}
\newcommand{\y}[1]{$ {#1} $}
\newcommand{\os}{\\[5 pt]}
\newcommand{\prbxl}[2]{
\parbox[l][][l]{#1\linewidth}{#2
	}}
\newcommand{\prbxr}[2]{\parbox[r][][l]{#1\linewidth}{
		\setlength{\abovedisplayskip}{5pt}
		\setlength{\belowdisplayskip}{5pt}	
		\setlength{\abovedisplayshortskip}{0pt}
		\setlength{\belowdisplayshortskip}{0pt} 
		\begin{shaded}
			\footnotesize	#2 \end{shaded}}}

\renewcommand{\cfttoctitlefont}{\Large\bfseries}
\setlength{\cftaftertoctitleskip}{0 pt}
\setlength{\cftbeforetoctitleskip}{0 pt}

\newcommand{\bs}{\\[3pt]}
\newcommand{\vn}{\\[6pt]}
\newcommand{\fig}[1]{\begin{figure}
		\centering
		\includegraphics[]{\figp{#1}}
\end{figure}}

\newcommand{\figc}[2]{\begin{figure}
		\centering
		\includegraphics[]{\figp{#1}}
		\caption{#2}
\end{figure}}

\newcommand{\sectionbreak}{\clearpage} % New page on each section

\newcommand{\nn}[1]{
\begin{equation}
	#1
\end{equation}
}

% Equation comments
\newcommand{\cm}[1]{\llap{\color{blue} #1}}

\newcommand\fork[2]{\begin{tcolorbox}[boxrule=0.3 mm,arc=0mm,enhanced jigsaw,breakable,colback=yellow!7] {\large \textbf{#1 (forklaring)} \vspace{5 pt}\\} #2 \end{tcolorbox}\vspace{-5pt} }
 
%colors
\newcommand{\colr}[1]{{\color{red} #1}}
\newcommand{\colb}[1]{{\color{blue} #1}}
\newcommand{\colo}[1]{{\color{orange} #1}}
\newcommand{\colc}[1]{{\color{cyan} #1}}
\definecolor{projectgreen}{cmyk}{100,0,100,0}
\newcommand{\colg}[1]{{\color{projectgreen} #1}}

% Methods
\newcommand{\metode}[2]{
	\textsl{#1} \\[-8pt]
	\rule{#2}{0.75pt}
}

%Opg
\newcommand{\abc}[1]{
	\begin{enumerate}[label=\alph*),leftmargin=18pt]
		#1
	\end{enumerate}
}
\newcommand{\abcs}[2]{
	\begin{enumerate}[label=\alph*),start=#1,leftmargin=18pt]
		#2
	\end{enumerate}
}
\newcommand{\abcn}[1]{
	\begin{enumerate}[label=\arabic*),leftmargin=18pt]
		#1
	\end{enumerate}
}
\newcommand{\abch}[1]{
	\hspace{-2pt}	\begin{enumerate*}[label=\alph*), itemjoin=\hspace{1cm}]
		#1
	\end{enumerate*}
}
\newcommand{\abchs}[2]{
	\hspace{-2pt}	\begin{enumerate*}[label=\alph*), itemjoin=\hspace{1cm}, start=#1]
		#2
	\end{enumerate*}
}

% Oppgaver
\newcommand{\opgt}{\phantomsection \addcontentsline{toc}{section}{Oppgaver} \section*{Oppgaver for kapittel \thechapter}\vs \setcounter{section}{1}}
\newcounter{opg}
\numberwithin{opg}{section}
\newcommand{\op}[1]{\vspace{15pt} \refstepcounter{opg}\large \textbf{\color{blue}\theopg} \vspace{2 pt} \label{#1} \\}
\newcommand{\ekspop}[1]{\vsk\textbf{Gruble \thechapter.#1}\vspace{2 pt} \\}
\newcommand{\nes}{\stepcounter{section}
	\setcounter{opg}{0}}
\newcommand{\opr}[1]{\vspace{3pt}\textbf{\ref{#1}}}
\newcommand{\oeks}[1]{\begin{tcolorbox}[boxrule=0.3 mm,arc=0mm,colback=white]
		\textit{Eksempel: } #1	  
\end{tcolorbox}}
\newcommand\opgeks[2][]{\begin{tcolorbox}[boxrule=0.1 mm,arc=0mm,enhanced jigsaw,breakable,colback=white] {\footnotesize \textbf{Eksempel #1} \\} \footnotesize #2 \end{tcolorbox}\vspace{-5pt} }
\newcommand{\rknut}{
Rekn ut.
}

%License
\newcommand{\lic}{\textit{Matematikken sine byggesteinar by Sindre Sogge Heggen is licensed under CC BY-NC-SA 4.0. To view a copy of this license, visit\\ 
		\net{http://creativecommons.org/licenses/by-nc-sa/4.0/}{http://creativecommons.org/licenses/by-nc-sa/4.0/}}}

%referances
\newcommand{\net}[2]{{\color{blue}\href{#1}{#2}}}
\newcommand{\hrs}[2]{\hyperref[#1]{\color{blue}\textsl{#2 \ref*{#1}}}}
\newcommand{\rref}[1]{\hrs{#1}{regel}}
\newcommand{\refkap}[1]{\hrs{#1}{kapittel}}
\newcommand{\refsec}[1]{\hrs{#1}{seksjon}}

\newcommand{\mb}{\net{https://sindrsh.github.io/FirstPrinciplesOfMath/}{MB}}


%line to seperate examples
\newcommand{\linje}{\rule{\linewidth}{1pt} }

\usepackage{datetime2}
%%\usepackage{sansmathfonts} for dyslexia-friendly math
\usepackage[]{hyperref}

\begin{document}

\opgt

\setcounter{section}{1}	
\op{leno}
Finn lengden av vektorene:\os
\begin{tabular}{@{}l l}
\textbf{a)} $ [-2, 1, 5] $ & \quad\textbf{b)} $ [\sqrt{3}, 2,  \sqrt{2}] $
\end{tabular}	
	
\op{closest}
Hvilket av punktene $ {B=(3, -2, 1)} $ og $ {C=(0, 5, 6) }$ ligger nærmest punktet $ {A=(1, -1, -2)} $?

\op{lenfakt}
Gitt vektoren
\[ \vec{u}=[ad, bd, cd]\]
\textbf{a)} Vis at
\[ |\vec{u}|=d\sqrt{a^2 + b^2 + c^2} \]
når $ d>0 $.\os

\textbf{b)} Forklar at
\[ |\vec{u}|=|d|\sqrt{a^2 + b^2 + c^2} \]
når $ d<0 $.

\nes
\op{skalfaktor}
Gitt vektorene
\[ \vec{u}=[ad, bd, cd] \text{ og } \vec{v}=[eh, fh, gh] \]
Vis at
\[ \vec{u}\cdot\vec{v}=dh(ae+bf+cg) \] \vs

\op{skalproo}
Finn skalarproduktet av vektorene:\os

\textbf{a)} $ \vec{a}=[2, 4, 6] $ og $ \vec{b}=[-5, 0, -1] $\os 

\textbf{b)} $ \vec{a}=[-9, 1, 5] $ og $\vec{b}= [-2, 1, -2] $\os

\textbf{c)} $ \vec{a}=\left[\frac{1}{5}, \frac{3}{5},- \frac{1}{5}\right] $ og $ \vec{b}=[512, -128, 64] $. \textsl{Tips:} Bruk resultatet fra opg. \ref{skalfaktor}.
\newpage
\op{skalpro2o}
Finn skalarproduktet av $ \vec{a} $ og $ \vec{b} $, som utspenner vinkelen $ \theta $, når du vet at\os

\textbf{a)} $ |\vec{a}|=5 $, $ |\vec{b}|= 2$ og $ \theta = 60^\circ $\os

\textbf{b)} $ |\vec{a}|=5 $, $ |\vec{b}|= 2$ og $ \theta = 150^\circ $

\op{finntheta} 
Finn vinkelen mellom $ \vec{a} $ og $ \vec{b} $ når\os

\textbf{a)} $ \vec{a}=[ 5 ,-5,  2]$ og $ \vec{b}=[ 3 ,-4 , 5] $\os

\textbf{b)} $ \vec{a}=[ 2 ,-1,  -3]$ og $ \vec{b}=[ -1 ,-3 , -2] $\os

\textbf{c)} $ \vec{a}=[ -1 ,-2,  2]$ og $ \vec{b}=[ -3 , 5 , -4] $

\op{forkort}
Forkort uttrykkene når du vet at $ |\vec{a}|=1 $, $ |\vec{b}|=2 $, $ |\vec{c}|=5 $, $ \vec{a}\cdot\vec{b}=0 $ og $ \vec{b}\cdot\vec{c}=0 $.\os

\textbf{a)} $ \vec{b}\cdot(\vec{a}+\vec{c}) + 3(\vec{a}+\vec{b})^2 $ \os

\textbf{b)} $ (\vec{a}+ \vec{b}+\vec{c})^2$

\nes
\op{orto}
Sjekk om $ \vec{a} $ og $ \vec{b} $ er ortogonale når\os

\textbf{a)} $ \vec{a}=[2, 4, -2] $ og $ \vec{b}=[3, 1, 1] $\os

\textbf{b)} $ \vec{a}=[-18, 12, 9] $ og $ \vec{b}=[1, -2, 1] $\os

\textbf{c)} $ \vec{a}=[5, 5, -1] $ og $ \vec{b}=[5, -4, 5] $

\op{torto}
Gitt vektoren
\[ \vec{u}=[-5, -1, 6] \]
Finn $ t $ slik at $ \vec{u}\perp \vec{v} $ når\os

\textbf{a)} $ \vec{v}=[t, 3t, 2] $\os

\textbf{b)} $ \vec{v}=[t, t^2, 1] $

\op{sjekkpar}
Sjekk om $ \vec{a}\parallel\vec{b} $ når\os

\textbf{a)} $ \vec{a}=[8, 4, -2] $ og $ \vec{b}=[4, 2, 4] $\os

\textbf{b)} $ \vec{a}=[-3, 5, 2] $ og $ \vec{b}=\left[-\frac{9}{7}, \frac{15}{7}, \frac{6}{7}\right] $ 
\newpage
\op{finntpar}
Gitt vektoren 
\[ \vec{a}=[-3, 1, 8] \]
Om mulig, finn $ t $ slik at $ \vec{a}\parallel\vec{b} $ når\os

\textbf{a)} $ \vec{b}=[t+3, 1-t, -16] $\os

\textbf{b)} $ \vec{b}=[t^2+2, t, -(5t^2+3)] $

\op{finnsogt}
Finn $ s $ og $ t $ slik at $\vec{u}=[4, 6+s, -(s+t)] $ og $ \vec{v}=\left[\frac{12}{7}, \frac{2t-9s}{7}, \frac{3s-t}{7}\right] $ er parallelle. \os

\nes
\op{vis22det}
Vis at
\[  \left|\begin{matrix}
ae & be \\
cf & df
\end{matrix}\right|=ef\left|\begin{matrix}
a & b \\
c & d
\end{matrix}\right| \]

\begin{comment}
\op{arparo}
Finn aralet til parallellogrammet utspent av (\textsl{Tips:} Bruk resultatet fra opg. \ref{vis22det}):

\textbf{a)} $ [-2, 7] $ og $ [12, 8] $

\textbf{b)} $ [-2, 4] $ og $ [24, -16] $\\
\end{comment}

\op{abparvekpro0}
Vis at hvis $ \vec{u}||\vec{v} $, så er $ \vec{u}\times\vec{v}=0 $

\op{lagrangesid}
For to vektorer $ \vec{u} $ og $ \vec{v} $ er \textit{Lagranges identitet} gitt som
\[ |\vec{u}\times\vec{v}|^2=|\vec{u}|^2|\vec{v}|^2-(\vec{u}\cdot\vec{v}\,)^2 \]
Bruk identiteten og definisjonen av skalarproduktet til å vise at
\[ |\vec{u}\times\vec{v}|=|\vec{u}||\vec{v}|\sin \angle(\vec{u}, \vec{v})  \]\vs

\op{tetraro}
Et tetraeted er utspent av vektorene $ \vec{a}=[2, -2, 1],\; \vec{b}=[3, -3, 1] $ og $ \vec{c}=[2, -3, 2] $, hvor $ \vec{a} $ og $ \vec{b} $ utspenner grunnflaten.\os

\textbf{a)} Vis at arealet av grunnflaten er $ \sqrt{2} $.\os

\textbf{b)} Vis at volumet av tetraetedet er $ \frac{1}{6} $.
\newpage
\op{parfinnh}
Et parallellepidet er utspent av vektorene $ \vec{a},\; \vec{b} $ og $ \vec{c} $. Vi har at ${|a|=3}$,  $|{\vec{b}|=4}$, $ {\vec{a}\cdot \vec{b}=0}  $, og at grunnflaten er utspent av $ \vec{a} $ og $ \vec{b} $. \os

\textbf{a)} Finn lengden av diagonalen til grunnflaten.\os

La $ \theta $ være vinkelen mellom $ {\vec{a}\times\vec{b}} $ og $ \vec{c} $ og la $ {\theta\in[0^\circ, 90^\circ]} $.\os

\textbf{b)} Lag en tegning og forklar hvorfor høyden $ h $ i parallellepipedet er gitt som
\[ h= |\vec{c}|\cos \theta \]
\textbf{d)} Forklar hvorfor volumet $ V $ av parallellepidetet kan skrives som
\[ V= |\vec{a}\times\vec{b}||c|\cos \theta\]\vs

\op{kryserlikdet}
Gitt vektorene $ \vec{u}=[a, b, c] $, $ \vec{v}=[d, e, f] $ og $ \vec{w}=[g, h, i] $. Vis at
\[ \vec{u}\times\vec{v}\cdot\vec{w}= \vec{w}\times\vec{u}\cdot\vec{v}\]
Tre pyramider er utspent av vektorene $ \vec{u}=[a, b, c] $, $ \vec{v}=[d, e, f] $ og $ \vec{w}=[g, h, i] $. Grunnflatene til pyramidene er henholdsvis utspent av $ \vec{u} $ og $ \vec{v} $, $ \vec{u} $ og $ \vec{w} $ og $ \vec{v} $ og $ \vec{w} $. Hva er uttrykket til volumet av pyramidene?


\end{document}