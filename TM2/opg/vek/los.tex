\documentclass[english, 11 pt, class=article, crop=false]{standalone}
\usepackage[T1]{fontenc}
%\renewcommand*\familydefault{\sfdefault} % For dyslexia-friendly text
\usepackage{lmodern} % load a font with all the characters
\usepackage{geometry}
\geometry{verbose,paperwidth=16.1 cm, paperheight=24 cm, inner=2.3cm, outer=1.8 cm, bmargin=2cm, tmargin=1.8cm}
\setlength{\parindent}{0bp}
\usepackage{import}
\usepackage[subpreambles=false]{standalone}
\usepackage{amsmath}
\usepackage{amssymb}
\usepackage{esint}
\usepackage{babel}
\usepackage{tabu}
\makeatother
\makeatletter

\usepackage{titlesec}
\usepackage{ragged2e}
\RaggedRight
\raggedbottom
\frenchspacing

% Norwegian names of figures, chapters, parts and content
\addto\captionsenglish{\renewcommand{\figurename}{Figur}}
\makeatletter
\addto\captionsenglish{\renewcommand{\chaptername}{Kapittel}}
\addto\captionsenglish{\renewcommand{\partname}{Del}}


\usepackage{graphicx}
\usepackage{float}
\usepackage{subfig}
\usepackage{placeins}
\usepackage{cancel}
\usepackage{framed}
\usepackage{wrapfig}
\usepackage[subfigure]{tocloft}
\usepackage[font=footnotesize,labelfont=sl]{caption} % Figure caption
\usepackage{bm}
\usepackage[dvipsnames, table]{xcolor}
\definecolor{shadecolor}{rgb}{0.105469, 0.613281, 1}
\colorlet{shadecolor}{Emerald!15} 
\usepackage{icomma}
\makeatother
\usepackage[many]{tcolorbox}
\usepackage{multicol}
\usepackage{stackengine}

\usepackage{esvect} %For vectors with capital letters

% For tabular
\usepackage{array}
\usepackage{multirow}
\usepackage{longtable} %breakable table

% Ligningsreferanser
\usepackage{mathtools}
\mathtoolsset{showonlyrefs}

% index
\usepackage{imakeidx}
\makeindex[title=Indeks]

%Footnote:
\usepackage[bottom, hang, flushmargin]{footmisc}
\usepackage{perpage} 
\MakePerPage{footnote}
\addtolength{\footnotesep}{2mm}
\renewcommand{\thefootnote}{\arabic{footnote}}
\renewcommand\footnoterule{\rule{\linewidth}{0.4pt}}
\renewcommand{\thempfootnote}{\arabic{mpfootnote}}

%colors
\definecolor{c1}{cmyk}{0,0.5,1,0}
\definecolor{c2}{cmyk}{1,0.25,1,0}
\definecolor{n3}{cmyk}{1,0.,1,0}
\definecolor{neg}{cmyk}{1,0.,0.,0}

% Lister med bokstavar
\usepackage[inline]{enumitem}

\newcounter{rg}
\numberwithin{rg}{chapter}
\newcommand{\reg}[2][]{\begin{tcolorbox}[boxrule=0.3 mm,arc=0mm,colback=blue!3] {\refstepcounter{rg}\phantomsection \large \textbf{\therg \;#1} \vspace{5 pt}}\newline #2  \end{tcolorbox}\vspace{-5pt}}

\newcommand\alg[1]{\begin{align} #1 \end{align}}

\newcommand\eks[2][]{\begin{tcolorbox}[boxrule=0.3 mm,arc=0mm,enhanced jigsaw,breakable,colback=green!3] {\large \textbf{Eksempel #1} \vspace{5 pt}\\} #2 \end{tcolorbox}\vspace{-5pt} }

\newcommand{\st}[1]{\begin{tcolorbox}[boxrule=0.0 mm,arc=0mm,enhanced jigsaw,breakable,colback=yellow!12]{ #1} \end{tcolorbox}}

\newcommand{\spr}[1]{\begin{tcolorbox}[boxrule=0.3 mm,arc=0mm,enhanced jigsaw,breakable,colback=yellow!7] {\large \textbf{Språkboksen} \vspace{5 pt}\\} #1 \end{tcolorbox}\vspace{-5pt} }

\newcommand{\sym}[1]{\colorbox{blue!15}{#1}}

\newcommand{\info}[2]{\begin{tcolorbox}[boxrule=0.3 mm,arc=0mm,enhanced jigsaw,breakable,colback=cyan!6] {\large \textbf{#1} \vspace{5 pt}\\} #2 \end{tcolorbox}\vspace{-5pt} }

\newcommand\algv[1]{\vspace{-11 pt}\begin{align*} #1 \end{align*}}

\newcommand{\regv}{\vspace{5pt}}
\newcommand{\mer}{\textsl{Merk}: }
\newcommand{\mers}[1]{{\footnotesize \mer #1}}
\newcommand\vsk{\vspace{11pt}}
\newcommand\vs{\vspace{-11pt}}
\newcommand\vsb{\vspace{-16pt}}
\newcommand\sv{\vsk \textbf{Svar} \vspace{4 pt}\\}
\newcommand\br{\\[5 pt]}
\newcommand{\figp}[1]{../fig/#1}
\newcommand\algvv[1]{\vs\vs\begin{align*} #1 \end{align*}}
\newcommand{\y}[1]{$ {#1} $}
\newcommand{\os}{\\[5 pt]}
\newcommand{\prbxl}[2]{
\parbox[l][][l]{#1\linewidth}{#2
	}}
\newcommand{\prbxr}[2]{\parbox[r][][l]{#1\linewidth}{
		\setlength{\abovedisplayskip}{5pt}
		\setlength{\belowdisplayskip}{5pt}	
		\setlength{\abovedisplayshortskip}{0pt}
		\setlength{\belowdisplayshortskip}{0pt} 
		\begin{shaded}
			\footnotesize	#2 \end{shaded}}}

\renewcommand{\cfttoctitlefont}{\Large\bfseries}
\setlength{\cftaftertoctitleskip}{0 pt}
\setlength{\cftbeforetoctitleskip}{0 pt}

\newcommand{\bs}{\\[3pt]}
\newcommand{\vn}{\\[6pt]}
\newcommand{\fig}[1]{\begin{figure}
		\centering
		\includegraphics[]{\figp{#1}}
\end{figure}}

\newcommand{\figc}[2]{\begin{figure}
		\centering
		\includegraphics[]{\figp{#1}}
		\caption{#2}
\end{figure}}

\newcommand{\sectionbreak}{\clearpage} % New page on each section

\newcommand{\nn}[1]{
\begin{equation}
	#1
\end{equation}
}

% Equation comments
\newcommand{\cm}[1]{\llap{\color{blue} #1}}

\newcommand\fork[2]{\begin{tcolorbox}[boxrule=0.3 mm,arc=0mm,enhanced jigsaw,breakable,colback=yellow!7] {\large \textbf{#1 (forklaring)} \vspace{5 pt}\\} #2 \end{tcolorbox}\vspace{-5pt} }
 
%colors
\newcommand{\colr}[1]{{\color{red} #1}}
\newcommand{\colb}[1]{{\color{blue} #1}}
\newcommand{\colo}[1]{{\color{orange} #1}}
\newcommand{\colc}[1]{{\color{cyan} #1}}
\definecolor{projectgreen}{cmyk}{100,0,100,0}
\newcommand{\colg}[1]{{\color{projectgreen} #1}}

% Methods
\newcommand{\metode}[2]{
	\textsl{#1} \\[-8pt]
	\rule{#2}{0.75pt}
}

%Opg
\newcommand{\abc}[1]{
	\begin{enumerate}[label=\alph*),leftmargin=18pt]
		#1
	\end{enumerate}
}
\newcommand{\abcs}[2]{
	\begin{enumerate}[label=\alph*),start=#1,leftmargin=18pt]
		#2
	\end{enumerate}
}
\newcommand{\abcn}[1]{
	\begin{enumerate}[label=\arabic*),leftmargin=18pt]
		#1
	\end{enumerate}
}
\newcommand{\abch}[1]{
	\hspace{-2pt}	\begin{enumerate*}[label=\alph*), itemjoin=\hspace{1cm}]
		#1
	\end{enumerate*}
}
\newcommand{\abchs}[2]{
	\hspace{-2pt}	\begin{enumerate*}[label=\alph*), itemjoin=\hspace{1cm}, start=#1]
		#2
	\end{enumerate*}
}

% Oppgaver
\newcommand{\opgt}{\phantomsection \addcontentsline{toc}{section}{Oppgaver} \section*{Oppgaver for kapittel \thechapter}\vs \setcounter{section}{1}}
\newcounter{opg}
\numberwithin{opg}{section}
\newcommand{\op}[1]{\vspace{15pt} \refstepcounter{opg}\large \textbf{\color{blue}\theopg} \vspace{2 pt} \label{#1} \\}
\newcommand{\ekspop}[1]{\vsk\textbf{Gruble \thechapter.#1}\vspace{2 pt} \\}
\newcommand{\nes}{\stepcounter{section}
	\setcounter{opg}{0}}
\newcommand{\opr}[1]{\vspace{3pt}\textbf{\ref{#1}}}
\newcommand{\oeks}[1]{\begin{tcolorbox}[boxrule=0.3 mm,arc=0mm,colback=white]
		\textit{Eksempel: } #1	  
\end{tcolorbox}}
\newcommand\opgeks[2][]{\begin{tcolorbox}[boxrule=0.1 mm,arc=0mm,enhanced jigsaw,breakable,colback=white] {\footnotesize \textbf{Eksempel #1} \\} \footnotesize #2 \end{tcolorbox}\vspace{-5pt} }
\newcommand{\rknut}{
Rekn ut.
}

%License
\newcommand{\lic}{\textit{Matematikken sine byggesteinar by Sindre Sogge Heggen is licensed under CC BY-NC-SA 4.0. To view a copy of this license, visit\\ 
		\net{http://creativecommons.org/licenses/by-nc-sa/4.0/}{http://creativecommons.org/licenses/by-nc-sa/4.0/}}}

%referances
\newcommand{\net}[2]{{\color{blue}\href{#1}{#2}}}
\newcommand{\hrs}[2]{\hyperref[#1]{\color{blue}\textsl{#2 \ref*{#1}}}}
\newcommand{\rref}[1]{\hrs{#1}{regel}}
\newcommand{\refkap}[1]{\hrs{#1}{kapittel}}
\newcommand{\refsec}[1]{\hrs{#1}{seksjon}}

\newcommand{\mb}{\net{https://sindrsh.github.io/FirstPrinciplesOfMath/}{MB}}


%line to seperate examples
\newcommand{\linje}{\rule{\linewidth}{1pt} }

\usepackage{datetime2}
%%\usepackage{sansmathfonts} for dyslexia-friendly math
\usepackage[]{hyperref}


\newcommand{\note}{Merk}
\newcommand{\notesm}[1]{{\footnotesize \textsl{\note:} #1}}
\newcommand{\ekstitle}{Eksempel }
\newcommand{\sprtitle}{Språkboksen}
\newcommand{\expl}{forklaring}

\newcommand{\vedlegg}[1]{\refstepcounter{vedl}\section*{Vedlegg \thevedl: #1}  \setcounter{vedleq}{0}}

\newcommand\sv{\vsk \textbf{Svar} \vspace{4 pt}\\}

%references
\newcommand{\reftab}[1]{\hrs{#1}{tabell}}
\newcommand{\rref}[1]{\hrs{#1}{regel}}
\newcommand{\dref}[1]{\hrs{#1}{definisjon}}
\newcommand{\refkap}[1]{\hrs{#1}{kapittel}}
\newcommand{\refsec}[1]{\hrs{#1}{seksjon}}
\newcommand{\refdsec}[1]{\hrs{#1}{delseksjon}}
\newcommand{\refved}[1]{\hrs{#1}{vedlegg}}
\newcommand{\eksref}[1]{\textsl{#1}}
\newcommand\fref[2][]{\hyperref[#2]{\textsl{figur \ref*{#2}#1}}}
\newcommand{\refop}[1]{{\color{blue}Oppgave \ref{#1}}}
\newcommand{\refops}[1]{{\color{blue}oppgave \ref{#1}}}
\newcommand{\refgrubs}[1]{{\color{blue}gruble \ref{#1}}}

\newcommand{\openmathser}{\openmath\,-\,serien}

% Exercises
\newcommand{\opgt}{\newpage \phantomsection \addcontentsline{toc}{section}{Oppgaver} \section*{Oppgaver for kapittel \thechapter}\vs \setcounter{section}{1}}


% Sequences and series
\newcommand{\sumarrek}{Summen av en aritmetisk rekke}
\newcommand{\sumgerek}{Summen av en geometrisk rekke}
\newcommand{\regnregsum}{Regneregler for summetegnet}

% Trigonometry
\newcommand{\sincoskomb}{Sinus og cosinus kombinert}
\newcommand{\cosfunk}{Cosinusfunksjonen}
\newcommand{\trid}{Trigonometriske identiteter}
\newcommand{\deravtri}{Den deriverte av de trigonometriske funksjonene}
% Solutions manual
\newcommand{\selos}{Se løsningsforslag.}
\newcommand{\se}[1]{Se eksempel på side \pageref{#1}}

%Vectors
\newcommand{\parvek}{Parallelle vektorer}
\newcommand{\vekpro}{Vektorproduktet}
\newcommand{\vekproarvol}{Vektorproduktet som areal og volum}


% 3D geometries
\newcommand{\linrom}{Linje i rommet}
\newcommand{\avstplnpkt}{Avstand mellom punkt og plan}


% Integral
\newcommand{\bestminten}{Bestemt integral I}
\newcommand{\anfundteo}{Analysens fundamentalteorem}
\newcommand{\intuf}{Integralet av utvalge funksjoner}
\newcommand{\bytvar}{Bytte av variabel}
\newcommand{\intvol}{Integral som volum}
\newcommand{\andordlindif}{Andre ordens lineære differensialligninger}



\begin{document}

\opr{leno} \se{arg1}

\opr{closest}
\algv{\vv{AB} &= [3-1 , 2-(-1) , 1-(-2)]\\
&= [2, 3, 3] \\
\left|\vv{AB}\right|&= \sqrt{20}\\
&\\
\vv{AC} &= [0-1 , 5-(-1) , 6-(-2)]\\
&= [-1, 6, 8] \\
\left|\vv{AC}\right|&= \sqrt{101}
}
Siden $ {\sqrt{101}>\sqrt{20}} $ er $ B $ nærmest $ A $.\vsk

\opr{lenfakt}\\
\textbf{a)}
\algv{
	|\vec{u}|&=\sqrt{(ad)^2 + (bd)^2 + (bd)^2}\\
	&= \sqrt{a^2 d^2 + b^2 d^2 + c^2 d^2}\\
	&= \sqrt{d^2(a^2+b^2+c^2)}\\
	&= d\sqrt{a^2 + b^2 + c^2}
}
\textbf{b)} Som i opg. a) kan vi også her skrive
\[ |\vec{u}|=\sqrt{d^2(a^2+b^2+c^2)}\]
men siden $ d^2 $ er et positivt tall, mens $ d $ er negativ, har vi at:
\[ d\neq \sqrt{d^2} \]
istedenfor er:
\[ |d|= \sqrt{d^2} \]
derfor kan vi skrive:
\[ |\vec{u}|=|d|\sqrt{a^2 + b^2 + c^2} \]
\opr{skalfaktor}
\algv{
\vec{u}\cdot\vec{v} &= [ad, bd, cd]\cdot[eh, fh, gh]\\
&= adeh+bdfh+cdgh \\
&= dh(ae+ bf+ cg)
}

\opr{skalproo}\\
\textbf{a)} \se{arg1}\\
\textbf{b)} \se{arg1}\\
\textbf{c)}
\algv{
\vec{a}\cdot\vec{b}&= \left[\dfrac{1}{5}, \dfrac{3}{5}, \dfrac{1}{5}\right]\cdot\vec{b}=[512, -128, 64] \br
&= \frac{1}{5}[1, 3, -1]\cdot 64[8, -2, 1]\br
&= \frac{64}{5}(8-6-1) \br
&= \frac{64}{5}
}

\opr{skalpro2o} \\
\textbf{a)} \se{arg1}
\textbf{b)} \se{arg1}

\opr{finntheta} 
Finn vinkelen mellom $ \vec{a} $ og $ \vec{b} $ når:

\textbf{a)} $ \vec{a}=[ 5 ,-5,  2]$ og $ \vec{b}=[ 3 ,-4 , 5] $
\algv{
|\vec{a}| &= \sqrt{5^2 +(-5)^2 +2^2} \\
&= \sqrt{54} \\
&= \sqrt{9\cdot6} \\
&= 3\sqrt 6\\
& \\
|\vec{b}| &= \sqrt{3^2+(-4)^2 + 5^2} \\
&= \sqrt{50}\\
&= \sqrt{25\cdot2}\\
&= 5\sqrt{2}
}
\algv{
	\vec{a}\cdot\vec{b} &= [ 5 ,-5,  2]\cdot\vec{b}=[ 3 ,-4 , 5] \\
	&=15+20+10\\
&= 45 }
\algv{\cos \theta &= \frac{\vec{a}\cdot\vec{b}}{|\vec{a}||\vec{b}|}\br
&= \frac{45}{3\sqrt{6}\cdot5\sqrt{2}} \br
&= \frac{3}{2\sqrt{3}}\br
&= \frac{3\sqrt{3}}{2\sqrt{3}\sqrt{3}} \br
&= \frac{\sqrt{3}}{2}
}
Dette betyr at $ \theta=30^\circ $.

\opr{forkort} \\
\textbf{a)} \se{}

\textbf{b)} 
\algv{
(\vec{a}+ \vec{b}+\vec{c})^2 &= (\vec{a}+ \vec{b}+\vec{c})\cdot(\vec{a}+ \vec{b}+\vec{c}) \\
&= \vec{a}^{\,2}+\vec{a}\cdot\vec{b}+\vec{a}\cdot\vec{c}+\vec{b}\cdot\vec{a}+\vec{b}^{\,2}+\vec{b}\cdot\vec{c}+\vec{c}\cdot\vec{a}+\vec{c}\cdot\vec{b}+\vec{c}^{\,2} \\
&= 1^2+0+ \vec{a}\cdot\vec{c}+0+2^2+0+\vec{c}\cdot\vec{a}+0+5^2 \\
&= 2(15+\vec{a}\cdot\vec{c})
}

\opr{orto} \\
\textbf{a)} \se{arg1}\\
\textbf{b)} \se{arg1} \\
\textbf{b)} \se{arg1}

\opr{torto}\\
\textbf{a)} \se{arg1}

\textbf{b)} Vi krever at:
\alg{\vec{u}\cdot\vec{v} &= 0 \\ 
[-5, -1, 6] \cdot[t, t^2, 1]&= 0 \\
-5t -t^2 +6 &= 0
}
Siden $ (-2)\cdot(-3) = 6 $ og $ -2+(-3)=-5 $ kan vi skrive at:
\[ (t-2)(t-3)=0 \]
Kravet er dermed oppfylt hvis $ t\in\lbrace2, 3\rbrace $. 

\opr{sjekkpar}
\textbf{a)} Vi regner fort ut at forholdet mellom både førstekomponentene og andrekomponenten er 2, men at forholdet mellom tredjekomponentene er $ -\frac{1}{2} $. Vektorene er derfor ikke parallelle.

\textbf{b)} Vi observerer at:
\[ \vec{b}=\frac{3}{7}[-3, 5, 2] \]
dermed er $ \vec{b} $ et multiplum av $ \vec{a} $ og da er $ \vec{a}||\vec{b} $.

\opr{finntpar}\\
\textbf{a)} Vi bruker forholdet mellom første- og tredjekomponententene for å sette opp en ligning for $ t $:
\alg{-\frac{t+3}{3}&=-\frac{16}{8} \br
t+3 &= 6 \\
t &= 3
}
Forholdet mellom andrekomponentene blir da:
\alg{
\frac{1-3}{1}&= -2
}
Forholdet er $ -2 $ for alle komponentene når $ t=3 $ og da er $ \vec{a}||\vec{b} $.

\textbf{b)} Også her bruker vi første- og tredjekomponententene for å sette opp en ligning for $ t $, fordi vi da får isolert det kvadratiske leddet:
\alg{
	-\frac{t^2+2}{3}&= -\frac{(5t^2+3)}{8} \\
	8t^2 + 16 &= 15 t^2 + 9 \\
	7 x^2 &= 7 \\
	x &= \pm 1
}
Når $ t=1 $ er forholdet mellom både førstkomponentene og tredjekomponentene lik
\[ -\frac{1^2+2}{3}=-1 \]
Og forholdet mellom andrekomponentene er:
\[ \frac{1}{1}=1 \]
For $ t=1 $ er altså $ \vec{a} $ og $ \vec{b} $ ikke parallelle. Vi ser derimot fort at forholdet mellom hver av komponentene blir $ -1 $ når $ t=-1 $, for dette valget av $ t $ er derfor $ \vec{a}||\vec{b} $.

\opr{finnsogt}
$\vec{u}=[4, 6+s, -(s+t)] $ og $ \vec{v}=\left[\frac{12}{7}, \frac{2t-9s}{7}, \frac{3s-t}{7}\right] $
Vi starter med å observere at:
\[ \vec{v}=\frac{1}{7}[12, 2t-9s, 3s-t] \]
Vi definerer $ \vec{w}= [12, 2t-9s, 3s-t] $. Skal vi ha at $ \vec{u}||\vec{v} $, må vi også hat at $ \vec{u}||\vec{w} $. Siden forholdet mellom førstekomponentene til $ \vec{u} $ og $ \vec{w} $ er 3, krever vi at $ \vec{u}=3\vec{w} $. Da kan vi sette opp følgende ligningssystem:
\alg{
2t-9s &= 3(6+s) \tag{I}\label{3I}\br
3s-t &= -3(s+t) \tag{II}\label{3II}
}
Av \ref{3II} får vi at:
\alg{
3s-t &= -3s-3t \\
2t &= -6s \\
t &= -3s 
}
Setter vi $ t=-3s $ inn i (\ref{3II}) får vi:
\alg{
2(-3s) -9s &= 18+3s\\
-6s -9s &=  18 +3s \\
-18s &= 18 \\
s&=-1
}
Altså er $ \vec{u} $ og $ \vec{v} $ parallelle hvis $ s=-1 $ og $ t=-3s=3 $. 

\opr{vis22det}
\alg{
\left|\begin{matrix}
	ae & be \\
	cf & df
\end{matrix}\right|&= aedf-becf \\
&= ef(ad-bc) \\
&= ef\left|\begin{matrix}
	a & b \\
	c & d
\end{matrix}\right|
i}

\opr{arparo}

\textbf{b)} Arelet er gitt som tallverdien til $ \det(\vec{a}, \vec{b}) $:
\alg{
\det(\vec{a}, \vec{b}) &= \left|\begin{matrix}
	-2 & 4 \\
	24 & -16
\end{matrix}\right|\\	&= 2\cdot8 \left|\begin{matrix}
-1 & 2 \\
3 & -2
\end{matrix}\right| \\
&= 16((-1)\cdot(-2)-2\cdot3) \\
&= 16\cdot(-4) \\
&= -64
}
Arealet er altså 64.

\opr{abparvekpro0}\\
Hvis $ \vec{u}||\vec{v} $ betyr dette at hvis vi skriver $ \vec{u}=[a, b, c] $, så kan vi skrive $ \vec{v}=d[a, b, c] $. Vi får da at:
\alg{
\vec{u}\times\vec{v}&= \left|\begin{matrix}
	\vec{e}_x & \vec{e}_y & \vec{e}_z \\
	a & b & c \\
	da & db & dc
\end{matrix}\right| \\
&= d\vec{e}_x \left|\begin{matrix}
	b & c \\
	b & c
\end{matrix}\right|-d\vec{e}_y \left|\begin{matrix}
a & c \\
a & c
\end{matrix}\right|+d\vec{e}_x \left|\begin{matrix}
a & b \\
a & b
\end{matrix}\right| \\
&= 0
}
Resultatet fra \ref{vis22det} er her brukt i andre linje for å forenkle regningen av $ 2\times2 $ determinantene.

\opr{tetraro}\\
\textbf{a)} Arealet til grunnflaten tilsvarer lengden av vektoren $ \vec{a}\times\vec{b} $:
\alg{
	\vec{a}\times\vec{b}  &= 
\left|\begin{matrix}
	\vec{e}_x & \vec{e}_y & \vec{e}_z \\
	2 & -2 & 1 \\
	3 & -3 & 1
\end{matrix}\right| \\
&= [(-2)\cdot1-1\cdot(-3), -(2\cdot1-1\cdot3), 2\cdot(-3)-(-2)\cdot3] \\
&= [-2+3, -(2-3), -6+6] \\
&= [1, 1, 0] \\
& \\
|\vec{a}\times\vec{b} |&= \sqrt{1^2+1^2} \\
&= \sqrt{2}
}
\textbf{b)} Av (??) vet vi at volumet $ V $ er gitt som:
\[ V = \frac{1}{6}|\vec{a}\times\vec{b} \cdot \vec{c}| \]
Vi har at:
\algv{
\vec{a}\times\vec{b} \cdot \vec{c} &= [1, 1, 0]\cdot[2, -3, 2] \\
&= 2-3
&= -1
}
Og dermed er $ V=\frac{1}{6} $.

\opr{parfinnh}\\
\textbf{a)} Diagonalen til grunnflaten kan uttrykkes som vektoren $ \vec{a}+\vec{b} $, og lengden blir da (husk at $ |\vec{u}|^2 = \vec{u}^{\,2} $):
\alg{
|\vec{a}+\vec{b}| &= \sqrt{\left(\vec{a}+\vec{b}\right)^2} \\
&= \sqrt{\vec{a}^{\,2}+\vec{a}\cdot\vec{b}+\vec{b}^{\,2}} \\
&= \sqrt{3^2 +0 +4^2} \\
&= \sqrt{25} \\
&= 5
}
\end{document}