\documentclass[english, 11 pt, class=article, crop=false]{standalone}
\usepackage[T1]{fontenc}
%\renewcommand*\familydefault{\sfdefault} % For dyslexia-friendly text
\usepackage{lmodern} % load a font with all the characters
\usepackage{geometry}
\geometry{verbose,paperwidth=16.1 cm, paperheight=24 cm, inner=2.3cm, outer=1.8 cm, bmargin=2cm, tmargin=1.8cm}
\setlength{\parindent}{0bp}
\usepackage{import}
\usepackage[subpreambles=false]{standalone}
\usepackage{amsmath}
\usepackage{amssymb}
\usepackage{esint}
\usepackage{babel}
\usepackage{tabu}
\makeatother
\makeatletter

\usepackage{titlesec}
\usepackage{ragged2e}
\RaggedRight
\raggedbottom
\frenchspacing

% Norwegian names of figures, chapters, parts and content
\addto\captionsenglish{\renewcommand{\figurename}{Figur}}
\makeatletter
\addto\captionsenglish{\renewcommand{\chaptername}{Kapittel}}
\addto\captionsenglish{\renewcommand{\partname}{Del}}


\usepackage{graphicx}
\usepackage{float}
\usepackage{subfig}
\usepackage{placeins}
\usepackage{cancel}
\usepackage{framed}
\usepackage{wrapfig}
\usepackage[subfigure]{tocloft}
\usepackage[font=footnotesize,labelfont=sl]{caption} % Figure caption
\usepackage{bm}
\usepackage[dvipsnames, table]{xcolor}
\definecolor{shadecolor}{rgb}{0.105469, 0.613281, 1}
\colorlet{shadecolor}{Emerald!15} 
\usepackage{icomma}
\makeatother
\usepackage[many]{tcolorbox}
\usepackage{multicol}
\usepackage{stackengine}

\usepackage{esvect} %For vectors with capital letters

% For tabular
\usepackage{array}
\usepackage{multirow}
\usepackage{longtable} %breakable table

% Ligningsreferanser
\usepackage{mathtools}
\mathtoolsset{showonlyrefs}

% index
\usepackage{imakeidx}
\makeindex[title=Indeks]

%Footnote:
\usepackage[bottom, hang, flushmargin]{footmisc}
\usepackage{perpage} 
\MakePerPage{footnote}
\addtolength{\footnotesep}{2mm}
\renewcommand{\thefootnote}{\arabic{footnote}}
\renewcommand\footnoterule{\rule{\linewidth}{0.4pt}}
\renewcommand{\thempfootnote}{\arabic{mpfootnote}}

%colors
\definecolor{c1}{cmyk}{0,0.5,1,0}
\definecolor{c2}{cmyk}{1,0.25,1,0}
\definecolor{n3}{cmyk}{1,0.,1,0}
\definecolor{neg}{cmyk}{1,0.,0.,0}

% Lister med bokstavar
\usepackage[inline]{enumitem}

\newcounter{rg}
\numberwithin{rg}{chapter}
\newcommand{\reg}[2][]{\begin{tcolorbox}[boxrule=0.3 mm,arc=0mm,colback=blue!3] {\refstepcounter{rg}\phantomsection \large \textbf{\therg \;#1} \vspace{5 pt}}\newline #2  \end{tcolorbox}\vspace{-5pt}}

\newcommand\alg[1]{\begin{align} #1 \end{align}}

\newcommand\eks[2][]{\begin{tcolorbox}[boxrule=0.3 mm,arc=0mm,enhanced jigsaw,breakable,colback=green!3] {\large \textbf{Eksempel #1} \vspace{5 pt}\\} #2 \end{tcolorbox}\vspace{-5pt} }

\newcommand{\st}[1]{\begin{tcolorbox}[boxrule=0.0 mm,arc=0mm,enhanced jigsaw,breakable,colback=yellow!12]{ #1} \end{tcolorbox}}

\newcommand{\spr}[1]{\begin{tcolorbox}[boxrule=0.3 mm,arc=0mm,enhanced jigsaw,breakable,colback=yellow!7] {\large \textbf{Språkboksen} \vspace{5 pt}\\} #1 \end{tcolorbox}\vspace{-5pt} }

\newcommand{\sym}[1]{\colorbox{blue!15}{#1}}

\newcommand{\info}[2]{\begin{tcolorbox}[boxrule=0.3 mm,arc=0mm,enhanced jigsaw,breakable,colback=cyan!6] {\large \textbf{#1} \vspace{5 pt}\\} #2 \end{tcolorbox}\vspace{-5pt} }

\newcommand\algv[1]{\vspace{-11 pt}\begin{align*} #1 \end{align*}}

\newcommand{\regv}{\vspace{5pt}}
\newcommand{\mer}{\textsl{Merk}: }
\newcommand{\mers}[1]{{\footnotesize \mer #1}}
\newcommand\vsk{\vspace{11pt}}
\newcommand\vs{\vspace{-11pt}}
\newcommand\vsb{\vspace{-16pt}}
\newcommand\sv{\vsk \textbf{Svar} \vspace{4 pt}\\}
\newcommand\br{\\[5 pt]}
\newcommand{\figp}[1]{../fig/#1}
\newcommand\algvv[1]{\vs\vs\begin{align*} #1 \end{align*}}
\newcommand{\y}[1]{$ {#1} $}
\newcommand{\os}{\\[5 pt]}
\newcommand{\prbxl}[2]{
\parbox[l][][l]{#1\linewidth}{#2
	}}
\newcommand{\prbxr}[2]{\parbox[r][][l]{#1\linewidth}{
		\setlength{\abovedisplayskip}{5pt}
		\setlength{\belowdisplayskip}{5pt}	
		\setlength{\abovedisplayshortskip}{0pt}
		\setlength{\belowdisplayshortskip}{0pt} 
		\begin{shaded}
			\footnotesize	#2 \end{shaded}}}

\renewcommand{\cfttoctitlefont}{\Large\bfseries}
\setlength{\cftaftertoctitleskip}{0 pt}
\setlength{\cftbeforetoctitleskip}{0 pt}

\newcommand{\bs}{\\[3pt]}
\newcommand{\vn}{\\[6pt]}
\newcommand{\fig}[1]{\begin{figure}
		\centering
		\includegraphics[]{\figp{#1}}
\end{figure}}

\newcommand{\figc}[2]{\begin{figure}
		\centering
		\includegraphics[]{\figp{#1}}
		\caption{#2}
\end{figure}}

\newcommand{\sectionbreak}{\clearpage} % New page on each section

\newcommand{\nn}[1]{
\begin{equation}
	#1
\end{equation}
}

% Equation comments
\newcommand{\cm}[1]{\llap{\color{blue} #1}}

\newcommand\fork[2]{\begin{tcolorbox}[boxrule=0.3 mm,arc=0mm,enhanced jigsaw,breakable,colback=yellow!7] {\large \textbf{#1 (forklaring)} \vspace{5 pt}\\} #2 \end{tcolorbox}\vspace{-5pt} }
 
%colors
\newcommand{\colr}[1]{{\color{red} #1}}
\newcommand{\colb}[1]{{\color{blue} #1}}
\newcommand{\colo}[1]{{\color{orange} #1}}
\newcommand{\colc}[1]{{\color{cyan} #1}}
\definecolor{projectgreen}{cmyk}{100,0,100,0}
\newcommand{\colg}[1]{{\color{projectgreen} #1}}

% Methods
\newcommand{\metode}[2]{
	\textsl{#1} \\[-8pt]
	\rule{#2}{0.75pt}
}

%Opg
\newcommand{\abc}[1]{
	\begin{enumerate}[label=\alph*),leftmargin=18pt]
		#1
	\end{enumerate}
}
\newcommand{\abcs}[2]{
	\begin{enumerate}[label=\alph*),start=#1,leftmargin=18pt]
		#2
	\end{enumerate}
}
\newcommand{\abcn}[1]{
	\begin{enumerate}[label=\arabic*),leftmargin=18pt]
		#1
	\end{enumerate}
}
\newcommand{\abch}[1]{
	\hspace{-2pt}	\begin{enumerate*}[label=\alph*), itemjoin=\hspace{1cm}]
		#1
	\end{enumerate*}
}
\newcommand{\abchs}[2]{
	\hspace{-2pt}	\begin{enumerate*}[label=\alph*), itemjoin=\hspace{1cm}, start=#1]
		#2
	\end{enumerate*}
}

% Oppgaver
\newcommand{\opgt}{\phantomsection \addcontentsline{toc}{section}{Oppgaver} \section*{Oppgaver for kapittel \thechapter}\vs \setcounter{section}{1}}
\newcounter{opg}
\numberwithin{opg}{section}
\newcommand{\op}[1]{\vspace{15pt} \refstepcounter{opg}\large \textbf{\color{blue}\theopg} \vspace{2 pt} \label{#1} \\}
\newcommand{\ekspop}[1]{\vsk\textbf{Gruble \thechapter.#1}\vspace{2 pt} \\}
\newcommand{\nes}{\stepcounter{section}
	\setcounter{opg}{0}}
\newcommand{\opr}[1]{\vspace{3pt}\textbf{\ref{#1}}}
\newcommand{\oeks}[1]{\begin{tcolorbox}[boxrule=0.3 mm,arc=0mm,colback=white]
		\textit{Eksempel: } #1	  
\end{tcolorbox}}
\newcommand\opgeks[2][]{\begin{tcolorbox}[boxrule=0.1 mm,arc=0mm,enhanced jigsaw,breakable,colback=white] {\footnotesize \textbf{Eksempel #1} \\} \footnotesize #2 \end{tcolorbox}\vspace{-5pt} }
\newcommand{\rknut}{
Rekn ut.
}

%License
\newcommand{\lic}{\textit{Matematikken sine byggesteinar by Sindre Sogge Heggen is licensed under CC BY-NC-SA 4.0. To view a copy of this license, visit\\ 
		\net{http://creativecommons.org/licenses/by-nc-sa/4.0/}{http://creativecommons.org/licenses/by-nc-sa/4.0/}}}

%referances
\newcommand{\net}[2]{{\color{blue}\href{#1}{#2}}}
\newcommand{\hrs}[2]{\hyperref[#1]{\color{blue}\textsl{#2 \ref*{#1}}}}
\newcommand{\rref}[1]{\hrs{#1}{regel}}
\newcommand{\refkap}[1]{\hrs{#1}{kapittel}}
\newcommand{\refsec}[1]{\hrs{#1}{seksjon}}

\newcommand{\mb}{\net{https://sindrsh.github.io/FirstPrinciplesOfMath/}{MB}}


%line to seperate examples
\newcommand{\linje}{\rule{\linewidth}{1pt} }

\usepackage{datetime2}
%%\usepackage{sansmathfonts} for dyslexia-friendly math
\usepackage[]{hyperref}


\newcommand{\note}{Merk}
\newcommand{\notesm}[1]{{\footnotesize \textsl{\note:} #1}}
\newcommand{\ekstitle}{Eksempel }
\newcommand{\sprtitle}{Språkboksen}
\newcommand{\expl}{forklaring}

\newcommand{\vedlegg}[1]{\refstepcounter{vedl}\section*{Vedlegg \thevedl: #1}  \setcounter{vedleq}{0}}

\newcommand\sv{\vsk \textbf{Svar} \vspace{4 pt}\\}

%references
\newcommand{\reftab}[1]{\hrs{#1}{tabell}}
\newcommand{\rref}[1]{\hrs{#1}{regel}}
\newcommand{\dref}[1]{\hrs{#1}{definisjon}}
\newcommand{\refkap}[1]{\hrs{#1}{kapittel}}
\newcommand{\refsec}[1]{\hrs{#1}{seksjon}}
\newcommand{\refdsec}[1]{\hrs{#1}{delseksjon}}
\newcommand{\refved}[1]{\hrs{#1}{vedlegg}}
\newcommand{\eksref}[1]{\textsl{#1}}
\newcommand\fref[2][]{\hyperref[#2]{\textsl{figur \ref*{#2}#1}}}
\newcommand{\refop}[1]{{\color{blue}Oppgave \ref{#1}}}
\newcommand{\refops}[1]{{\color{blue}oppgave \ref{#1}}}
\newcommand{\refgrubs}[1]{{\color{blue}gruble \ref{#1}}}

\newcommand{\openmathser}{\openmath\,-\,serien}

% Exercises
\newcommand{\opgt}{\newpage \phantomsection \addcontentsline{toc}{section}{Oppgaver} \section*{Oppgaver for kapittel \thechapter}\vs \setcounter{section}{1}}


% Sequences and series
\newcommand{\sumarrek}{Summen av en aritmetisk rekke}
\newcommand{\sumgerek}{Summen av en geometrisk rekke}
\newcommand{\regnregsum}{Regneregler for summetegnet}

% Trigonometry
\newcommand{\sincoskomb}{Sinus og cosinus kombinert}
\newcommand{\cosfunk}{Cosinusfunksjonen}
\newcommand{\trid}{Trigonometriske identiteter}
\newcommand{\deravtri}{Den deriverte av de trigonometriske funksjonene}
% Solutions manual
\newcommand{\selos}{Se løsningsforslag.}
\newcommand{\se}[1]{Se eksempel på side \pageref{#1}}

%Vectors
\newcommand{\parvek}{Parallelle vektorer}
\newcommand{\vekpro}{Vektorproduktet}
\newcommand{\vekproarvol}{Vektorproduktet som areal og volum}


% 3D geometries
\newcommand{\linrom}{Linje i rommet}
\newcommand{\avstplnpkt}{Avstand mellom punkt og plan}


% Integral
\newcommand{\bestminten}{Bestemt integral I}
\newcommand{\anfundteo}{Analysens fundamentalteorem}
\newcommand{\intuf}{Integralet av utvalge funksjoner}
\newcommand{\bytvar}{Bytte av variabel}
\newcommand{\intvol}{Integral som volum}
\newcommand{\andordlindif}{Andre ordens lineære differensialligninger}



\begin{document}
\opr{parlinjeo} \se{lirop}

\opr{krysslinj}
Vi bruker kravet for $ x $ og $ z $-koordinaten for å sette opp et ligningssystem:
\alg{
-3-2t &= -7-3s \tag{I} \label{krysI} \\
1-t &= s \tag{II} \label{krysII}
}
Av (\ref{krysII}) har vi et uttrykk for $ s $. Setter vi dette inn i (\ref{krysI}) får vi:
\alg{
-3-2t&=-7+3(1-t) \\
-3-2t&=-7+3-3t \\
t &=-1
}
Altså er $ t=-1 $ og $ s=2 $. For disse verdiene gir begge parameteriseringene punktet $ A=(-1, 1, 2) $.

\opr{finnparplan} \se{plpare}

\opr{finnparplan2} \se{plpare2}

\opr{finnplan} \se{plroe1}

\opr{finnplan2}\\
\textbf{a)} Av parameteriseringen ser vi at to retningsvektorer må være $ [2, 3, 0] $ og $ [0, 2, -1] $. 

\textbf{b)} En normalvektor for planet er gitt ved vektorproduktet av retningsvektorene:
\alg{
\left|\begin{matrix}
	\vec{e}_x & \vec{e}_y & \vec{e}_z \\
	2 & 3 & 0 \\
	0 & 2 & -1
\end{matrix}\right| &= \vec{e}_x \left|\begin{matrix}
3 & 0 \\
2 & -1 
\end{matrix}\right|-\vec{e}_y \left|\begin{matrix}
2 & 0 \\
0 & -1 
\end{matrix}\right|+\vec{e}_z \left|\begin{matrix}
2 & 3 \\
0 & 2 
\end{matrix}\right| \\
&= \vec{e}_x (-3-0)-\vec{e}_x(-2-0)+\vec{e}_z(4-0) \\
&= [-3, -2, 4]  
}
Av parameteriseringen ser vi at $ (-4, 2, 1) $ er et punkt i planet, derfor kan vi skrive:
\alg{
-3(x-(-4))+2(y-2)+4(z-1)&= 0\\
-3x-12-2y-4+4z-4&= 0\\
-3x -2y+4z -20&=0 
}

\opr{finnplan4} \\
Vi krever at:\vs
\alg{
	(-2, 1, 1)\cdot[3t, 5, t] &= 0 \\
	-6t+5+t &= 0 \\
	t &= 1
}
Altså er $ [3, 5, 1] $ en retningsvektor for planet. Ligningen til planet blir da:
\alg{
3(x-(-2))+5(y-1)+(z-1) &= 0 \\
3x+5y + z &= 0
}

\opr{kuleopg} \se{kulee}

\opr{kuleopg2}\\
\textbf{a)} Vi starter med å skrive de fullstendige kvadratene:
\alg{
x^2-6x &= (x^2-3)^2-3^2 \\
y^2+2y &= (y+1)^2-1^2 \\
z^2-10z &= (z-5)^2-5^2
}
Vi kan derfor skrive:
\alg{
(x^2-3)^2+(y+1)^2 +(z-5)^2 -14-9-1-25&= 0 \\
(x^2-3)^2+(y+1)^2 +(z-5)^2 &= 49 \\
(x^2-3)^2+(y+1)^2 +(z-5)^2 &= 7^2 
}
Altså har kula sentrom i $ S=(3, -1, 5) $ og radius $ r=7 $.

\textbf{b)} Vi setter koordinatene til $ A $ faktoriserte venstresiden av kuleligningen og får:
\alg{
(4-3)^2+(1+1)^2 +(6-5)^2 &= 6
}
Ligningen over representerer den kvadrerte avstanden mellom $ S $ og $ A $, siden $ 6<49 $ må $ A $ ligge inni kula.

For $ B $ får vi:
\[ (-6-3)^2+(-4+1)^2 +(1-5)^2 = 106 \]
Siden $ 106>49 $ ligger $ B $ utenfor kula.

\opr{toparllin}\\
\textbf{a)} Av ligningen ser vi at $ [3, -2, 1] $ er en normalvektor.

\textbf{b)} I ligningen for $ \beta $ ser vi at hvis $ x=y=0 $, så må også $ z=0 $. $ \beta $ inneholder derfor origo.

\textbf{c)} Avstanden $ h $ mellom $ \alpha $ og $ \beta $ må tilsvare avstande mellom $ \alpha $ og et punkt i $ \beta $. Vi bruker svaret fra b) og får:
\alg{
h &= \frac{|3\cdot0-2\cdot2+1\cdot0+12|}{|[3, -2, 1]|} \br
&= \frac{12}{\sqrt{9+4+1}} \br
&= \frac{12}{\sqrt{14}}
}

\opr{kuleopg3}\\
\textbf{a)} For å finne $ S $ skriver vi de fullstendige kvadratene:
\alg{
x^2-6x &= (x-3)^2-3^2 \\
y^2+4y &= (y+2)^2-2^2 \\
z^2 &= z^2
}
Kuleligningen kan vi derfor skrive som:
\alg{
(x-3)^2+(y+2)^2+z^2-23-3^2-2^2 = 0 \\
(x-3)^2+(y+2)^2+z^2 &= 36 \\
(x-3)^2+(y+2)^2+z^2 &= 6^2
}
Altså har vi $ S=(3, -2, 0) $.

\textbf{b)} Av ligningen til planet ser vi at $ [2, -1, -2] $ er en normalvektor for $ \alpha $. Dette må være en retningsvektor for linja som går gjennom $ A $ og $ S $, som dermed kan parameteriseres ved:
\[ l: \left\lbrace{
	\begin{array}{l}
	x=3+ 2t  \\
	y=-2-t   \\
	z=-2t
	\end{array}
}\right.  \]

\textbf{c)} \textit{Løsningsmetode 1:}
Linja og kuleflata skjærer der parameteriseringen til linja oppfyler kuleligningen:
\alg{
((3+2t)-3)^2+((-2-t)+2)^2+(-2t-0)^2 &= 36 \\
(2t)^2+(-t)^2+(-2t)^2 &= 36 \\
9t^2 &= 36 \\
t^2 &= 4 \\
t &= \pm 2
}
For $ t=-2 $ gir parameteriseringen punktet $ (-1, 0, 4) $ mens for $ t=2 $ får vi punktet $ (7, -4, -4) $.

\textit{Løsningsmetode 2:} Vi har funnet at $ [2, -1, -2] $ er en retningsvektor for linja gjennom $ A $ og $ S $, denne vektoren har lengde 3. Vi kan derfor lage oss en retningsvektor med lengde $ 1 $ ved å skrive $ \frac{1}{3}[2, -1, 2] $. Siden avstanden mellom $ S $ og de to punktene vi søker er lik radiusen 6, må de være gitt ved uttrykket
\alg{
S\pm 6\cdot\frac{1}{2}[2, -1, -2] &= S\pm2[2, -1, -2]
}
Regner man ut dette får man (selvølgelig) samme svar som for \textit{Løsningsmetode 1}.

\textbf{d)} \se{plpare3} eller bruk lignende resonnement som \textit{Løsningsmetode 2} i opg. c).

\textbf{e)} Radiusen $ R $ til sirkelen, radiusen $ r $ til kula og linjestykket $ AS $ utgjør en rettvinklet trekant. Av Pytagoras' setning har vi da at:
\alg{
R^2 &= r^2-|\vv{AS}|^2 \\
 &= 6^2-3^2 \\
&= 36-9 \\
&= 27 \\
R &= \pm\sqrt{27} 
}
$ R $ har altså lengden $ \sqrt{27} $.
\end{document}