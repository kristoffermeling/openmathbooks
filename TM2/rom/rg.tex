\newcommand{\plro}{\reg[Parameteriseringen av et plan i rommet]{
		Et plan $ \alpha $ som med inneholder punktet $ {P=(x_0, y_0, z_0)} $ og to ikke-parallelle vektorer $ {\vec{u}=[a_1, b_1, c_1]} $ og ${ \vec{v}=[a_2, b_2, c_3]} $ kan parameteriseres ved
			\[\alpha: \left\lbrace{
		\begin{array}{lll}
		x= x_0 + a_1s+ a_2t   \\
		y= y_0 + b_1s+ b_2t    \\
		z= z_0 + c_1s+ c_2t 
		\end{array}
	}\right. \]	
	hvor $ s, t \in \mathbb{R}$. 
	}}
\newcommand{\plroe}{}
\newcommand{\liro}{}
\newcommand{\liroe}{
\newcommand{\plaro}{\reg[Ligningen til et plan i rommet]{
		Et plan med normalvektor $ {n=[a, b, c]} $ kan uttrykkes ved ligningen
		\begin{equation}\label{planlig}
		a(x-x_0) + b(y-y_0) + c(z-z_0)=0
		\end{equation}
		hvor $ A=(x_0, y_0, z_0) $ er et vilkårlig punkt i planet. \vsk\\
		
		Eventuelt kan man skrive
		\begin{equation}\label{planlig2}
		ax + by + zc + d=0
		\end{equation}
		hvor $ -(ax_0 + by_0 +cz_0)=d $.
	}}	
\newcommand{\plaroe}{\eks[1]{\label{plroe1}
		Et plan er utspent av vektorene $ {\vec{u}=[1, -2, 2] }$ og $ \vec{v}=[-3, \\3, 1] $ og inneholder punktet $ {A=(-3, 3, 4)} $. Finn en ligning for planet.
		
		\sv
		En normalvektor til planet kan vi finne ved
		\footnotesize
		\alg{\vec{u}\times\vec{v} &= \begin{vmatrix}
				\vec{e}_1 & \vec{e}_2 & \vec{e}_3 \\
				1 & -2 & 2 \\
				-3 & 3 & 1
			\end{vmatrix} \\
			&= (-2\cdot1-3\cdot2)\vec{e}_1 -(1\cdot1-(-3)\cdot2)\vec{e}_2 + (1\cdot3-(-2)\cdot(-3))\vec{e}_3 \\
			&= [-8, -7, -3] \\
			&= -[8, 7, 3]
		} \normalsize
		Vi har nå en normalvektor og et punkt i planet, og får dermed ligningen
		\alg{
			8(x+3)+7(y-3)+3(z-4) &=0 \\
			8x+24+7y-21+3z-12 &= 0 \\
			8x+7y+3z-9=0
		}\vs
	}}	
\newcommand{\kule}{}	
\newcommand{\kulee}{}
\newcommand{\avpli}{}
\newcommand{\avplp}{}
\newcommand{\limtpl}{}
\newcommand{\vtolin}{}
\newcommand{\vtopl}{}
\newcommand{\vplli}{}