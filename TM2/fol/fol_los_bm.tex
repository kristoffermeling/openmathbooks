\documentclass[english, 11 pt, class=article, crop=false]{standalone}
\usepackage[T1]{fontenc}
%\renewcommand*\familydefault{\sfdefault} % For dyslexia-friendly text
\usepackage{lmodern} % load a font with all the characters
\usepackage{geometry}
\geometry{verbose,paperwidth=16.1 cm, paperheight=24 cm, inner=2.3cm, outer=1.8 cm, bmargin=2cm, tmargin=1.8cm}
\setlength{\parindent}{0bp}
\usepackage{import}
\usepackage[subpreambles=false]{standalone}
\usepackage{amsmath}
\usepackage{amssymb}
\usepackage{esint}
\usepackage{babel}
\usepackage{tabu}
\makeatother
\makeatletter

\usepackage{titlesec}
\usepackage{ragged2e}
\RaggedRight
\raggedbottom
\frenchspacing

% Norwegian names of figures, chapters, parts and content
\addto\captionsenglish{\renewcommand{\figurename}{Figur}}
\makeatletter
\addto\captionsenglish{\renewcommand{\chaptername}{Kapittel}}
\addto\captionsenglish{\renewcommand{\partname}{Del}}


\usepackage{graphicx}
\usepackage{float}
\usepackage{subfig}
\usepackage{placeins}
\usepackage{cancel}
\usepackage{framed}
\usepackage{wrapfig}
\usepackage[subfigure]{tocloft}
\usepackage[font=footnotesize,labelfont=sl]{caption} % Figure caption
\usepackage{bm}
\usepackage[dvipsnames, table]{xcolor}
\definecolor{shadecolor}{rgb}{0.105469, 0.613281, 1}
\colorlet{shadecolor}{Emerald!15} 
\usepackage{icomma}
\makeatother
\usepackage[many]{tcolorbox}
\usepackage{multicol}
\usepackage{stackengine}

\usepackage{esvect} %For vectors with capital letters

% For tabular
\usepackage{array}
\usepackage{multirow}
\usepackage{longtable} %breakable table

% Ligningsreferanser
\usepackage{mathtools}
\mathtoolsset{showonlyrefs}

% index
\usepackage{imakeidx}
\makeindex[title=Indeks]

%Footnote:
\usepackage[bottom, hang, flushmargin]{footmisc}
\usepackage{perpage} 
\MakePerPage{footnote}
\addtolength{\footnotesep}{2mm}
\renewcommand{\thefootnote}{\arabic{footnote}}
\renewcommand\footnoterule{\rule{\linewidth}{0.4pt}}
\renewcommand{\thempfootnote}{\arabic{mpfootnote}}

%colors
\definecolor{c1}{cmyk}{0,0.5,1,0}
\definecolor{c2}{cmyk}{1,0.25,1,0}
\definecolor{n3}{cmyk}{1,0.,1,0}
\definecolor{neg}{cmyk}{1,0.,0.,0}

% Lister med bokstavar
\usepackage[inline]{enumitem}

\newcounter{rg}
\numberwithin{rg}{chapter}
\newcommand{\reg}[2][]{\begin{tcolorbox}[boxrule=0.3 mm,arc=0mm,colback=blue!3] {\refstepcounter{rg}\phantomsection \large \textbf{\therg \;#1} \vspace{5 pt}}\newline #2  \end{tcolorbox}\vspace{-5pt}}

\newcommand\alg[1]{\begin{align} #1 \end{align}}

\newcommand\eks[2][]{\begin{tcolorbox}[boxrule=0.3 mm,arc=0mm,enhanced jigsaw,breakable,colback=green!3] {\large \textbf{Eksempel #1} \vspace{5 pt}\\} #2 \end{tcolorbox}\vspace{-5pt} }

\newcommand{\st}[1]{\begin{tcolorbox}[boxrule=0.0 mm,arc=0mm,enhanced jigsaw,breakable,colback=yellow!12]{ #1} \end{tcolorbox}}

\newcommand{\spr}[1]{\begin{tcolorbox}[boxrule=0.3 mm,arc=0mm,enhanced jigsaw,breakable,colback=yellow!7] {\large \textbf{Språkboksen} \vspace{5 pt}\\} #1 \end{tcolorbox}\vspace{-5pt} }

\newcommand{\sym}[1]{\colorbox{blue!15}{#1}}

\newcommand{\info}[2]{\begin{tcolorbox}[boxrule=0.3 mm,arc=0mm,enhanced jigsaw,breakable,colback=cyan!6] {\large \textbf{#1} \vspace{5 pt}\\} #2 \end{tcolorbox}\vspace{-5pt} }

\newcommand\algv[1]{\vspace{-11 pt}\begin{align*} #1 \end{align*}}

\newcommand{\regv}{\vspace{5pt}}
\newcommand{\mer}{\textsl{Merk}: }
\newcommand{\mers}[1]{{\footnotesize \mer #1}}
\newcommand\vsk{\vspace{11pt}}
\newcommand\vs{\vspace{-11pt}}
\newcommand\vsb{\vspace{-16pt}}
\newcommand\sv{\vsk \textbf{Svar} \vspace{4 pt}\\}
\newcommand\br{\\[5 pt]}
\newcommand{\figp}[1]{../fig/#1}
\newcommand\algvv[1]{\vs\vs\begin{align*} #1 \end{align*}}
\newcommand{\y}[1]{$ {#1} $}
\newcommand{\os}{\\[5 pt]}
\newcommand{\prbxl}[2]{
\parbox[l][][l]{#1\linewidth}{#2
	}}
\newcommand{\prbxr}[2]{\parbox[r][][l]{#1\linewidth}{
		\setlength{\abovedisplayskip}{5pt}
		\setlength{\belowdisplayskip}{5pt}	
		\setlength{\abovedisplayshortskip}{0pt}
		\setlength{\belowdisplayshortskip}{0pt} 
		\begin{shaded}
			\footnotesize	#2 \end{shaded}}}

\renewcommand{\cfttoctitlefont}{\Large\bfseries}
\setlength{\cftaftertoctitleskip}{0 pt}
\setlength{\cftbeforetoctitleskip}{0 pt}

\newcommand{\bs}{\\[3pt]}
\newcommand{\vn}{\\[6pt]}
\newcommand{\fig}[1]{\begin{figure}
		\centering
		\includegraphics[]{\figp{#1}}
\end{figure}}

\newcommand{\figc}[2]{\begin{figure}
		\centering
		\includegraphics[]{\figp{#1}}
		\caption{#2}
\end{figure}}

\newcommand{\sectionbreak}{\clearpage} % New page on each section

\newcommand{\nn}[1]{
\begin{equation}
	#1
\end{equation}
}

% Equation comments
\newcommand{\cm}[1]{\llap{\color{blue} #1}}

\newcommand\fork[2]{\begin{tcolorbox}[boxrule=0.3 mm,arc=0mm,enhanced jigsaw,breakable,colback=yellow!7] {\large \textbf{#1 (forklaring)} \vspace{5 pt}\\} #2 \end{tcolorbox}\vspace{-5pt} }
 
%colors
\newcommand{\colr}[1]{{\color{red} #1}}
\newcommand{\colb}[1]{{\color{blue} #1}}
\newcommand{\colo}[1]{{\color{orange} #1}}
\newcommand{\colc}[1]{{\color{cyan} #1}}
\definecolor{projectgreen}{cmyk}{100,0,100,0}
\newcommand{\colg}[1]{{\color{projectgreen} #1}}

% Methods
\newcommand{\metode}[2]{
	\textsl{#1} \\[-8pt]
	\rule{#2}{0.75pt}
}

%Opg
\newcommand{\abc}[1]{
	\begin{enumerate}[label=\alph*),leftmargin=18pt]
		#1
	\end{enumerate}
}
\newcommand{\abcs}[2]{
	\begin{enumerate}[label=\alph*),start=#1,leftmargin=18pt]
		#2
	\end{enumerate}
}
\newcommand{\abcn}[1]{
	\begin{enumerate}[label=\arabic*),leftmargin=18pt]
		#1
	\end{enumerate}
}
\newcommand{\abch}[1]{
	\hspace{-2pt}	\begin{enumerate*}[label=\alph*), itemjoin=\hspace{1cm}]
		#1
	\end{enumerate*}
}
\newcommand{\abchs}[2]{
	\hspace{-2pt}	\begin{enumerate*}[label=\alph*), itemjoin=\hspace{1cm}, start=#1]
		#2
	\end{enumerate*}
}

% Oppgaver
\newcommand{\opgt}{\phantomsection \addcontentsline{toc}{section}{Oppgaver} \section*{Oppgaver for kapittel \thechapter}\vs \setcounter{section}{1}}
\newcounter{opg}
\numberwithin{opg}{section}
\newcommand{\op}[1]{\vspace{15pt} \refstepcounter{opg}\large \textbf{\color{blue}\theopg} \vspace{2 pt} \label{#1} \\}
\newcommand{\ekspop}[1]{\vsk\textbf{Gruble \thechapter.#1}\vspace{2 pt} \\}
\newcommand{\nes}{\stepcounter{section}
	\setcounter{opg}{0}}
\newcommand{\opr}[1]{\vspace{3pt}\textbf{\ref{#1}}}
\newcommand{\oeks}[1]{\begin{tcolorbox}[boxrule=0.3 mm,arc=0mm,colback=white]
		\textit{Eksempel: } #1	  
\end{tcolorbox}}
\newcommand\opgeks[2][]{\begin{tcolorbox}[boxrule=0.1 mm,arc=0mm,enhanced jigsaw,breakable,colback=white] {\footnotesize \textbf{Eksempel #1} \\} \footnotesize #2 \end{tcolorbox}\vspace{-5pt} }
\newcommand{\rknut}{
Rekn ut.
}

%License
\newcommand{\lic}{\textit{Matematikken sine byggesteinar by Sindre Sogge Heggen is licensed under CC BY-NC-SA 4.0. To view a copy of this license, visit\\ 
		\net{http://creativecommons.org/licenses/by-nc-sa/4.0/}{http://creativecommons.org/licenses/by-nc-sa/4.0/}}}

%referances
\newcommand{\net}[2]{{\color{blue}\href{#1}{#2}}}
\newcommand{\hrs}[2]{\hyperref[#1]{\color{blue}\textsl{#2 \ref*{#1}}}}
\newcommand{\rref}[1]{\hrs{#1}{regel}}
\newcommand{\refkap}[1]{\hrs{#1}{kapittel}}
\newcommand{\refsec}[1]{\hrs{#1}{seksjon}}

\newcommand{\mb}{\net{https://sindrsh.github.io/FirstPrinciplesOfMath/}{MB}}


%line to seperate examples
\newcommand{\linje}{\rule{\linewidth}{1pt} }

\usepackage{datetime2}
%%\usepackage{sansmathfonts} for dyslexia-friendly math
\usepackage[]{hyperref}


\newcommand{\note}{Merk}
\newcommand{\notesm}[1]{{\footnotesize \textsl{\note:} #1}}
\newcommand{\ekstitle}{Eksempel }
\newcommand{\sprtitle}{Språkboksen}
\newcommand{\expl}{forklaring}

\newcommand{\vedlegg}[1]{\refstepcounter{vedl}\section*{Vedlegg \thevedl: #1}  \setcounter{vedleq}{0}}

\newcommand\sv{\vsk \textbf{Svar} \vspace{4 pt}\\}

%references
\newcommand{\reftab}[1]{\hrs{#1}{tabell}}
\newcommand{\rref}[1]{\hrs{#1}{regel}}
\newcommand{\dref}[1]{\hrs{#1}{definisjon}}
\newcommand{\refkap}[1]{\hrs{#1}{kapittel}}
\newcommand{\refsec}[1]{\hrs{#1}{seksjon}}
\newcommand{\refdsec}[1]{\hrs{#1}{delseksjon}}
\newcommand{\refved}[1]{\hrs{#1}{vedlegg}}
\newcommand{\eksref}[1]{\textsl{#1}}
\newcommand\fref[2][]{\hyperref[#2]{\textsl{figur \ref*{#2}#1}}}
\newcommand{\refop}[1]{{\color{blue}Oppgave \ref{#1}}}
\newcommand{\refops}[1]{{\color{blue}oppgave \ref{#1}}}
\newcommand{\refgrubs}[1]{{\color{blue}gruble \ref{#1}}}

\newcommand{\openmathser}{\openmath\,-\,serien}

% Exercises
\newcommand{\opgt}{\newpage \phantomsection \addcontentsline{toc}{section}{Oppgaver} \section*{Oppgaver for kapittel \thechapter}\vs \setcounter{section}{1}}


% Sequences and series
\newcommand{\sumarrek}{Summen av en aritmetisk rekke}
\newcommand{\sumgerek}{Summen av en geometrisk rekke}
\newcommand{\regnregsum}{Regneregler for summetegnet}

% Trigonometry
\newcommand{\sincoskomb}{Sinus og cosinus kombinert}
\newcommand{\cosfunk}{Cosinusfunksjonen}
\newcommand{\trid}{Trigonometriske identiteter}
\newcommand{\deravtri}{Den deriverte av de trigonometriske funksjonene}
% Solutions manual
\newcommand{\selos}{Se løsningsforslag.}
\newcommand{\se}[1]{Se eksempel på side \pageref{#1}}

%Vectors
\newcommand{\parvek}{Parallelle vektorer}
\newcommand{\vekpro}{Vektorproduktet}
\newcommand{\vekproarvol}{Vektorproduktet som areal og volum}


% 3D geometries
\newcommand{\linrom}{Linje i rommet}
\newcommand{\avstplnpkt}{Avstand mellom punkt og plan}


% Integral
\newcommand{\bestminten}{Bestemt integral I}
\newcommand{\anfundteo}{Analysens fundamentalteorem}
\newcommand{\intuf}{Integralet av utvalge funksjoner}
\newcommand{\bytvar}{Bytte av variabel}
\newcommand{\intvol}{Integral som volum}
\newcommand{\andordlindif}{Andre ordens lineære differensialligninger}



\begin{document}
\textbf{a)}
Vi bruker den eksplisitte fomelen for en aritmetisk følge, og får:
\alg{
	a_4 &= a_1 + d(i-1) \\
	30 &= 3 +d(4-1) \\
	27 &= 3d \\
	9 &= d
}
Altså er
\[ a_i=3+9(i-1) \]
\textbf{c)}
Vi observerer at:
\alg{
	a_5-a_3 &= a_1+d(5-1)-(a_1+d(3-1)) \\
	a_5-a_3&= 2d \\
	26-14 &= 2d \\
	6 &= d
}
Videre har vi at:\vs
\alg{
	a_3 &= a_1+2d \\
	14 &= a_1 +12 \\
	2 &= a_1
}
Altså er
\[ a_i=2+6(i-1) \]

\opr{eksgeo}\\
\textbf{a)}
Vi har at:
\[ k=\frac{a_2}{a_1}= \frac{1}{3}\]
Dermed er det eksplisitte uttrykket gitt som:
\alg{
a_n &= \frac{1}{2}\cdot\left(\frac{1}{3}\right)^{i-1} \\
&= \frac{1}{2}\cdot\frac{1}{3^{i-1}} \\
&= \frac{1}{2}\cdot3^{1-i}
}
\textbf{b)}
Vi vet at:
\algv{a_1\cdot k^{4-1}&= a_4 \\
5 \cdot k^3 &= 40 \\
k^3 &= 8 \\
k &= 2
}
Altså får vi:
\[ a_n = 5\cdot2^{i-1} \]
\newpage
\opr{sum10ar}\\
\textbf{a)} Vi observerer at rekka er en aritmetisk rekke med $ a_1= 7$ og $ d= 6$. For å finne summen trenger vi verdien til $ a_{10} $:
\alg{a_{10} &= 7+6(10-1) \\
&= 61}
Summen $ S_{10} $ blir da:
\alg{
S_{10} &= 10\cdot\frac{7+61}{2}\\
&= 340
}
\textbf{b)} Se \textsl{a}.

\opr{ar435} \\
Rekken er aritmetisk med \y{a_1=8} og \y{d=3}. Vi har at:
\alg{
n\frac{8+(8+3(n-1))}{2} &= 435 \\
3n^2 +13n-870 &= 0 
}
Vi bruker \textit{abc}-formelen og får at $ {n\in\lbrace15, -\frac{58}{3}\rbrace} $, hvorav $ {n=15} $ er eneste mulige svar. 

\opr{opgarekfraeks}
Dette er den aritmetiske rekken fra eksempelet på side \pageref{arrekeks}. I formelen for $ S_n $ setter vi inn det eksplisitte uttrykket for $ a_n $, og får at	
\alg{ S_n &= n\frac{a_1+a_1+d(n-1)}{2} \\
	2\cdot903 &= n(3+3+4(n-1))\\
	0 &= 6n+4n^2-4n - 2\cdot903 \\
	0 &= 2n^2+n-903
}
Denne ligningen har løsningene $ {n\in \big\lbrace21, -\frac{43}{2} \big\rbrace}$. Vi søker et positivt heltall, derfor er $ {n=21} $ eneste mulige løsning.

\opr{opggerekfraeks}
Dette er den geometriske rekken fra eksempelet på side \pageref{gerekeks}. Vi lar $ n $ være antall ledd, og får at
\alg{
	3\cdot\frac{1-2^n}{1-2} &= 93 \\
	2^n -1 &= \frac{93}{3} \br
	2^n &= 31+1 \\
	2^n &= 2^5 \\
	n &= 5
}

\opr{viseks3} \\
\vs
\algv{
3\cdot9\cdot27\cdot\ldots \cdot 3^n &= 3^1\cdot3^2\cdot3^3\ldots\cdot3^n \\
&= 3^{1+2+\ldots+ n} \\
&=3^{n\frac{1+n}{2}} \\
&= 3^{\frac{1}{2}n(n+1)}
}
Som er det vi skulle vise.

\opr{geon} \\
Rekken er geometrisk, med $ {a_1=3} $ og $ {k=4} $. For å finne summen må vi vite hvor mange ledd rekken består av:
\alg{
3 \cdot 4 ^{n-1}=768 \\
4^{n-1} &= 256 \\
4^{n-1} &= 4^4\\
n-1 &= 4 \\
n &= 5
}

\opr{geoa12} \\
\textbf{a)}
Summen $ S_n $ er gitt som:
\algv{
S_n &= 2\cdot\frac{1-3^k}{1-3} \\
&= 2\cdot\frac{1-3^k}{-2} \\
&= 3^k-1
}
\textbf{b)} \algv{S_3&=3^3-1 \\
&= 26}

\textbf{c)} \algv{
3^n-1 &= 728\\
3^n &= 729\\
3^n &= 3^6\\
n &= 6
}


\opr{opggerekuendeks}
Dette er gen geometriske rekken fra eksempelet på side \pageref{gerekuendeks}.
\abc{
\item Hvis rekka har en endelig sum $ {S_\infty=\frac{3}{2}} $, er
\alg{
	\frac{x}{x-1}=\frac{3}{2} \\
	2x &= 3(x-1) \\
	x &=3
}
Summen av rekka er altså $ \frac{3}{2} $ når $ {x=3} $. \vsk

\item Skal summen bli $ -1 $, må $ x $ oppfylle følgende ligning:
\alg{
	\frac{x}{x-1}&= -1 \\
	x &= -(x-1) \\
	x &= \frac{1}{2}
}
Men $ {x=\frac{1}{2} }$ oppfyller ikke kravet om at rekka er konvergent (den er divergent) for dette valget av $ x $. Altså er det ingen verdier for $ x $ som oppfyller ligningen.
}

\newpage
\opr{1over4}\\
\textbf{a)} Dette er en uendelig geometrisk rekke med $ k=\frac{1}{4} $. Siden $ |k|<1 $ er rekka konvergent.

\textbf{b)} Siden rekka er uendelig geometrisk og konvergent, har rekka en endelig sum $ S_\infty $ gitt ved:
\alg{
S_\infty&=\frac{a_1}{1-k} \br
&= \frac{4}{\frac{3}{4}}\\
&= \frac{16}{3}
}

\begin{comment}
	\opr{stav}\\
Dette blir en uendelig geometrisk rekke på formen:
\[ 1+\frac{1}{10}+\frac{1}{100}+... \]
Siden $ k=\frac{1}{10} $ er $ |k|<1 $ og derfor er rekka konvergent. Summen $ S_\infty $ er da gitt som:
\alg{
S_\infty&=\frac{a_1}{1-k} \\
&= \frac{1}{1-\frac{1}{10}} \\
&= \frac{1}{\frac{9}{10}} \br
&= \frac{10}{9}
}
Altså blir lengden $ \frac{10}{9} $ meter.
\end{comment}
\opr{099er1} \\
\textbf{a)} $ {\frac{9}{10}+\frac{9}{10^2}+\frac{9}{10^3}+\ldots} $. Dette er en geometrisk rekke med $ {a_1 = \frac{9}{10}} $ og $ k=10^{-1} $.
\textbf{b)} Fordi $ {|k|<1 }$ er rekken konvergent. Den uendelige summen er derfor gitt som:
\algv{
S_\infty &= \frac{\frac{9}{10}}{1-\frac{1}{10}} \\
&= \frac{\frac{9}{10}}{\frac{9}{10}} \\
&= 1
}
Summen av rekken blir 1, altså er \y{0.999...=1} (!).

\opr{geokonv} \\
\textbf{a)} Vi observerer at ${ k=x-2} $. Skal rekka konvergere må altså $ {|x-2|<1} $. Skal dette være sant må vi ha at:
\alg{
-1 &< x-2 \\
1 &< x
}
og videre at:
\algv{
x-2 &< 1 \\
x < 3
}
Derfor må vi ha at $ {1<x<3} $.
\newpage
\textbf{b)} \algv{
\frac{\frac{1}{3}}{1-(x-2)}&=\frac{2}{9} \br
\frac{1}{3(3-x)} &= \frac{2}{9} \br
\frac{2}{18-6x} &= \frac{2}{9} \br
18-6x &= 9 \\
x &= \frac{3}{2}
}
$  {x=\frac{3}{2}} $ ligger i konvergensområdet, og er derfor et gyldig svar.

\textbf{c)} \algv{
\frac{\frac{1}{3}}{1-(x-2)}&=\frac{1}{6} \br
\frac{1}{3(3-x)} &= \frac{1}{6} \br
3(3-x) &= 6 \\
x &= 1
}
Men $  {x=1} $ ligger ikke i konvergensområdet, og er derfor ikke et gyldig svar. $ S_n=\frac{1}{6} $ har derfor ingen løsning.

\opr{ind}\\
\textbf{a)} Vi sjekker påstanden for $ {n=1}$:
\alg{1& = \frac{1(1+1)}{2} \\
1&= 1}
Påstanden er sann for $ {n=1} $, vi går derfor videre til å sjekke påstanden for $ {n=k+1} $. Når vi antar at formelen stemmer fram til ledd, $ k $ får vi:
\algv{
1+2+3+\ldots+(k+1) &= \frac{(k+1)(k+1+1)}{2} \br
\frac{k(k+1)}{2}+k+1 &= \frac{(k+1)(k+2)}{2} \br
\frac{k(k+1)+2(k+1)}{2} &= \\
\frac{(k+1)(k+2)}{2} &= \frac{(k+1)(k+2)}{2}
}
Dermed har vi vist det vi skulle.

\textbf{b)} Vi sjekker påstanden for $ {n=1} $:
\alg{1 &= 2^n-1 \\
1 &= 1
}
Påstanden er sann for $ n=1 $, vi går derfor videre til å sjekke påstanden for $ n=k+1 $. Når vi antar at formelen stemmer fram til ledd $ k $, får vi:
\algv{
1+2 +2^2 +...+ 2^{k+1-1}&= 2^{k+1}-1 \\
2^k-1+ 2^k &= \\
2\cdot 2^k-1 &= \\
2^{k+1}-1 &= 2^{k+1}-1
}
Dermed har vi vist det vi skulle.

\textbf{c)} 
Vi sjekker påstanden for $ {n=1}$:
\alg{4 &= \frac{4}{3}(4^1-1) \\
4&= \frac{4}{3}\cdot3\\
4&=4}
Påstanden er sann for $ {n=1} $, vi går derfor videre til å sjekke påstanden for $ {n=k+1} $. Når vi antar at formelen stemmer fram til ledd, $ k $ får vi:
\alg{
4+4^2+4^3+\ldots+4^{k+1}&=  \frac{4}{3}(4^{k+1}-1) \\
\frac{4}{3}(4^k-1)+4^{k+1} &=  \\
\frac{4^{k+1}-1+3\cdot 4^{k+1}}{3} &= \\
\frac{4}{3}(4^{k+1}-1)&=\frac{4}{3}(4^{k+1}-1)
}
Dermed har vi vist det vi skulle.

\textbf{d)} Vi sjekker påstanden for $ n=1 $:
\alg{1 &= \frac{1(2\cdot1+1)(1+1)}{6} \\
&= \frac{6}{6} \\
1&= 1
}
Påstanden er sann for $ n=1 $, vi går derfor videre til å sjekke påstanden for $ n=k+1 $. Når vi antar at formelen stemmer fram til ledd $ k $ får vi:
\alg{
1^2 + 2^2 + 3^3...+ (k+1)^2 &= \frac{(k+1)(2(k+1)+1)((k+1)+1)}{6} \br
\frac{k(2k+1)(k+1)}{6} + (k+1)^2 &= \frac{(k+1)(2k+3)(k+2)}{6} \br
\frac{k(2k+1)(k+1)+6(k+1)^2}{6} &= \br
\frac{(k+1)(k(2k+1)+6(k+1)}{6}&= \br
\frac{(k+1)(k(2k+1)+2k+4k+6)}{6} &= \br
\frac{(k+1)(k(2k+3)+4k+6)}{6} &= \br
\frac{(k+1)(k(2k+3)+2(2k+3)}{6} &= \br
\frac{(k+1)(k+2)(2k+3)}{6} &= \frac{(k+1)(2k+3)(k+2)}{6} 
}
Og dermed har vi vist det vi skulle. 

\textsl{Merk}: Faktorisering er en treningsak, men observer hvordan vi i overgangen mellom linje 5 og 6 framkalte leddet $ 2k+3 $. Hvis man ikke kommer i mål med ren faktorisering, kan man selvfølgelig etter linje 4 vise at $ k(2k+1)+6(k+1)=(2k+3)(k+2) $ ved å skrive ut uttrykkene på begge sider.

\opr{div3} \\
Vi sjekker påstanden for $ {n=1}$:
\alg{1(1^2+2) &= 1\cdot3
}
Påstanden er sann for $ {n=1} $, vi går derfor videre til å sjekke påstanden for $ {n=k+1} $. Når vi antar at formelen stemmer for $ n=k $, får vi:
\alg{
(k+1)((k+1)^2+2)&= (k+1)(k^2+2k+3) \\
&= (k+1)(k(k+2)+3)
}
Antakelsen vår sier at $ k(k+2) $ er delelig med 3, noe tallet 3 også er. Faktoren $ (k(k+2)+3) $ er derfor delelig med 3, mens $ (k+1) $ er et heltall. Uttrykket i ligningen over er derfor delelig med 3.
 
\opr{factorials} \\
\textbf{a)} Vi sjekker påstanden for $ {n=1}$:
\alg{
\frac{(2\cdot 1)!}{(2\cdot1-1)!} &= 2^1\cdot1!\br
\frac{1\cdot2}{1} &=  2 \\
2 &= 2
}
Påstanden er sann for $ {n=1} $, vi går derfor videre til å sjekke påstanden for $ {n=k+1} $. Når vi antar at formelen stemmer fram til ledd $ k $, får vi:
\alg{
\frac{1\cdot2}{1}\cdot\frac{1\cdot2\cdot3\cdot4}{1\cdot2\cdot3}\cdot
\ldots\cdot \frac{(2(k+1))!}{(2(k+1)-1)!}&=2^{k+1} (k+1)! \br
2^k k! \frac{(2(k+1))!}{(2k+1)!} &= \br
2^k k! \frac{(2k+1)!(2k+2)}{(2k+1)!} &= \br
2^{k+1}k!(k+1) &= \br
2^{k+1}(k+1)! &= 2^{k+1}(k+1)!
}
Dermed har vi vist det vi skulle.

\textbf{b)} Venstresiden kan enklere skrives som:
\[ 2\cdot4\cdot6\cdot\ldots\cdot2(k+1) \]
For $ {n=1} $:
\algv{
	2 &= 2^1\cdot1!\\
	2 &= 2
}
For $ {n=k+1} $:
\algv{
	2\cdot4\cdot6\cdot\ldots\cdot2(k+1)&=2^{k+1} (k+1)! \br
	2^k k!\cdot2(k+1) &=  \\
	2^{k+1}(k+1)!&= 2^{k+1} (k+1)!
}
\newpage
\grubr{opgsumnkvad}
\textbf{a)} Summen av de $ n $ første oddetallene tilsvarer $ n^2 $ (se f. eks \ref{parodd}b), derfor kan vi skrive kvadratene som summer av oddetall.

\textbf{b)} Vi får $ n $ enere, $ {n-1} $ treere, $ {n-2} $ femmere og så videre. Den isolerte $ n $-en på høyresiden representerer de $ n $ enerene, mens summen
representerer bidragene fra alle de andre oddetallene (skriv opp hvis du syns det er vanskelig å se). 

\textbf{c)} \vs
\alg{
\sum\limits_{i=1}^n i^2 &= n+\sum\limits_{i=1}^n (n-i)(2i+1) \\
\sum\limits_{i=1}^n i^2 &= n+\sum\limits_{i=1}^n (2in+n-2i^2-i ) \\
\sum\limits_{i=1}^n i^2+\sum\limits_{i=1}^n 2i^2 &= n+\sum\limits_{i=1}^n ((2n-1)i+n ) \\
\sum\limits_{i=1}^n 3i^2 &= n+n^2+(2n-1)\frac{n(n+1)}{2} \\
\sum\limits_{i=1}^n i^2 &= \frac{2n(1+n)+(2n-1)n(n+1)}{6} \\
&= \frac{(2n+(2n-1)n)(n+1)}{6} \\
&= \frac{n(2+(2n-1))(n+1)}{6} \\
&= \frac{n(2n+1)(n+1)}{6}
}

\newpage
\grubr{r2h23d1opg3}\\
\abc{
\item Da summen av den uendelige rekka er 8, og $ a_1=4 $, har vi av \eqref{infsum} at
\[ 8=\frac{4}{1-k} \] 
Altså er $ k=\frac{1}{2} $, og dermed har vi av \eqref{sumg} at
\alg{
	S_4 = 4\cdot\frac{1-\left(\frac{1}{2}\right)^4}{1-\frac{1}{2}}=\frac{15}{2}
}
\item Da $ a_i = a_1+d(i-1) $, har vi at
\[ a_1 +a_4 + a_7 = a_1 + (a_1+3d) + (a_1+6d)=3a_1+9d=3(a_1+3d)=3a_4 \]
Altså er
\algv{
	3a_4 &= 114 \\
	a_4 &= 38
}
}


\grubr{opgfolgeoarrek}\\
Vi starter med å skrive opp noen ledd i følgen:
\alg{
a_2 &= ka_1 + d\\
a_3 &= k(ka_1 + d) + d = k^2a_1+d(1+k) \\
a_4 &= k(k^2a_1+d(k+1))=k^3a_1+d(1+k+k^2)
}
Ut ifra dette finner vi at det første leddet i $ a_n $ kan skrives som
\[ ka_1^{n-1} \]
Det andre lettet i $ a_n $ gjenkjenner vi som en geometrisk rekke med $ n-1 $ ledd, med sum lik
\[ d\frac{1-k^{n-1}}{1-k} \]
Altså er
\[ a_n = k^{n-1}a_1+d\frac{1-k^{n-1}}{1-k} \]

\end{document}