\documentclass[english, 11 pt, class=article, crop=false]{standalone}
\usepackage[T1]{fontenc}
%\renewcommand*\familydefault{\sfdefault} % For dyslexia-friendly text
\usepackage{lmodern} % load a font with all the characters
\usepackage{geometry}
\geometry{verbose,paperwidth=16.1 cm, paperheight=24 cm, inner=2.3cm, outer=1.8 cm, bmargin=2cm, tmargin=1.8cm}
\setlength{\parindent}{0bp}
\usepackage{import}
\usepackage[subpreambles=false]{standalone}
\usepackage{amsmath}
\usepackage{amssymb}
\usepackage{esint}
\usepackage{babel}
\usepackage{tabu}
\makeatother
\makeatletter

\usepackage{titlesec}
\usepackage{ragged2e}
\RaggedRight
\raggedbottom
\frenchspacing

% Norwegian names of figures, chapters, parts and content
\addto\captionsenglish{\renewcommand{\figurename}{Figur}}
\makeatletter
\addto\captionsenglish{\renewcommand{\chaptername}{Kapittel}}
\addto\captionsenglish{\renewcommand{\partname}{Del}}


\usepackage{graphicx}
\usepackage{float}
\usepackage{subfig}
\usepackage{placeins}
\usepackage{cancel}
\usepackage{framed}
\usepackage{wrapfig}
\usepackage[subfigure]{tocloft}
\usepackage[font=footnotesize,labelfont=sl]{caption} % Figure caption
\usepackage{bm}
\usepackage[dvipsnames, table]{xcolor}
\definecolor{shadecolor}{rgb}{0.105469, 0.613281, 1}
\colorlet{shadecolor}{Emerald!15} 
\usepackage{icomma}
\makeatother
\usepackage[many]{tcolorbox}
\usepackage{multicol}
\usepackage{stackengine}

\usepackage{esvect} %For vectors with capital letters

% For tabular
\usepackage{array}
\usepackage{multirow}
\usepackage{longtable} %breakable table

% Ligningsreferanser
\usepackage{mathtools}
\mathtoolsset{showonlyrefs}

% index
\usepackage{imakeidx}
\makeindex[title=Indeks]

%Footnote:
\usepackage[bottom, hang, flushmargin]{footmisc}
\usepackage{perpage} 
\MakePerPage{footnote}
\addtolength{\footnotesep}{2mm}
\renewcommand{\thefootnote}{\arabic{footnote}}
\renewcommand\footnoterule{\rule{\linewidth}{0.4pt}}
\renewcommand{\thempfootnote}{\arabic{mpfootnote}}

%colors
\definecolor{c1}{cmyk}{0,0.5,1,0}
\definecolor{c2}{cmyk}{1,0.25,1,0}
\definecolor{n3}{cmyk}{1,0.,1,0}
\definecolor{neg}{cmyk}{1,0.,0.,0}

% Lister med bokstavar
\usepackage[inline]{enumitem}

\newcounter{rg}
\numberwithin{rg}{chapter}
\newcommand{\reg}[2][]{\begin{tcolorbox}[boxrule=0.3 mm,arc=0mm,colback=blue!3] {\refstepcounter{rg}\phantomsection \large \textbf{\therg \;#1} \vspace{5 pt}}\newline #2  \end{tcolorbox}\vspace{-5pt}}

\newcommand\alg[1]{\begin{align} #1 \end{align}}

\newcommand\eks[2][]{\begin{tcolorbox}[boxrule=0.3 mm,arc=0mm,enhanced jigsaw,breakable,colback=green!3] {\large \textbf{Eksempel #1} \vspace{5 pt}\\} #2 \end{tcolorbox}\vspace{-5pt} }

\newcommand{\st}[1]{\begin{tcolorbox}[boxrule=0.0 mm,arc=0mm,enhanced jigsaw,breakable,colback=yellow!12]{ #1} \end{tcolorbox}}

\newcommand{\spr}[1]{\begin{tcolorbox}[boxrule=0.3 mm,arc=0mm,enhanced jigsaw,breakable,colback=yellow!7] {\large \textbf{Språkboksen} \vspace{5 pt}\\} #1 \end{tcolorbox}\vspace{-5pt} }

\newcommand{\sym}[1]{\colorbox{blue!15}{#1}}

\newcommand{\info}[2]{\begin{tcolorbox}[boxrule=0.3 mm,arc=0mm,enhanced jigsaw,breakable,colback=cyan!6] {\large \textbf{#1} \vspace{5 pt}\\} #2 \end{tcolorbox}\vspace{-5pt} }

\newcommand\algv[1]{\vspace{-11 pt}\begin{align*} #1 \end{align*}}

\newcommand{\regv}{\vspace{5pt}}
\newcommand{\mer}{\textsl{Merk}: }
\newcommand{\mers}[1]{{\footnotesize \mer #1}}
\newcommand\vsk{\vspace{11pt}}
\newcommand\vs{\vspace{-11pt}}
\newcommand\vsb{\vspace{-16pt}}
\newcommand\sv{\vsk \textbf{Svar} \vspace{4 pt}\\}
\newcommand\br{\\[5 pt]}
\newcommand{\figp}[1]{../fig/#1}
\newcommand\algvv[1]{\vs\vs\begin{align*} #1 \end{align*}}
\newcommand{\y}[1]{$ {#1} $}
\newcommand{\os}{\\[5 pt]}
\newcommand{\prbxl}[2]{
\parbox[l][][l]{#1\linewidth}{#2
	}}
\newcommand{\prbxr}[2]{\parbox[r][][l]{#1\linewidth}{
		\setlength{\abovedisplayskip}{5pt}
		\setlength{\belowdisplayskip}{5pt}	
		\setlength{\abovedisplayshortskip}{0pt}
		\setlength{\belowdisplayshortskip}{0pt} 
		\begin{shaded}
			\footnotesize	#2 \end{shaded}}}

\renewcommand{\cfttoctitlefont}{\Large\bfseries}
\setlength{\cftaftertoctitleskip}{0 pt}
\setlength{\cftbeforetoctitleskip}{0 pt}

\newcommand{\bs}{\\[3pt]}
\newcommand{\vn}{\\[6pt]}
\newcommand{\fig}[1]{\begin{figure}
		\centering
		\includegraphics[]{\figp{#1}}
\end{figure}}

\newcommand{\figc}[2]{\begin{figure}
		\centering
		\includegraphics[]{\figp{#1}}
		\caption{#2}
\end{figure}}

\newcommand{\sectionbreak}{\clearpage} % New page on each section

\newcommand{\nn}[1]{
\begin{equation}
	#1
\end{equation}
}

% Equation comments
\newcommand{\cm}[1]{\llap{\color{blue} #1}}

\newcommand\fork[2]{\begin{tcolorbox}[boxrule=0.3 mm,arc=0mm,enhanced jigsaw,breakable,colback=yellow!7] {\large \textbf{#1 (forklaring)} \vspace{5 pt}\\} #2 \end{tcolorbox}\vspace{-5pt} }
 
%colors
\newcommand{\colr}[1]{{\color{red} #1}}
\newcommand{\colb}[1]{{\color{blue} #1}}
\newcommand{\colo}[1]{{\color{orange} #1}}
\newcommand{\colc}[1]{{\color{cyan} #1}}
\definecolor{projectgreen}{cmyk}{100,0,100,0}
\newcommand{\colg}[1]{{\color{projectgreen} #1}}

% Methods
\newcommand{\metode}[2]{
	\textsl{#1} \\[-8pt]
	\rule{#2}{0.75pt}
}

%Opg
\newcommand{\abc}[1]{
	\begin{enumerate}[label=\alph*),leftmargin=18pt]
		#1
	\end{enumerate}
}
\newcommand{\abcs}[2]{
	\begin{enumerate}[label=\alph*),start=#1,leftmargin=18pt]
		#2
	\end{enumerate}
}
\newcommand{\abcn}[1]{
	\begin{enumerate}[label=\arabic*),leftmargin=18pt]
		#1
	\end{enumerate}
}
\newcommand{\abch}[1]{
	\hspace{-2pt}	\begin{enumerate*}[label=\alph*), itemjoin=\hspace{1cm}]
		#1
	\end{enumerate*}
}
\newcommand{\abchs}[2]{
	\hspace{-2pt}	\begin{enumerate*}[label=\alph*), itemjoin=\hspace{1cm}, start=#1]
		#2
	\end{enumerate*}
}

% Oppgaver
\newcommand{\opgt}{\phantomsection \addcontentsline{toc}{section}{Oppgaver} \section*{Oppgaver for kapittel \thechapter}\vs \setcounter{section}{1}}
\newcounter{opg}
\numberwithin{opg}{section}
\newcommand{\op}[1]{\vspace{15pt} \refstepcounter{opg}\large \textbf{\color{blue}\theopg} \vspace{2 pt} \label{#1} \\}
\newcommand{\ekspop}[1]{\vsk\textbf{Gruble \thechapter.#1}\vspace{2 pt} \\}
\newcommand{\nes}{\stepcounter{section}
	\setcounter{opg}{0}}
\newcommand{\opr}[1]{\vspace{3pt}\textbf{\ref{#1}}}
\newcommand{\oeks}[1]{\begin{tcolorbox}[boxrule=0.3 mm,arc=0mm,colback=white]
		\textit{Eksempel: } #1	  
\end{tcolorbox}}
\newcommand\opgeks[2][]{\begin{tcolorbox}[boxrule=0.1 mm,arc=0mm,enhanced jigsaw,breakable,colback=white] {\footnotesize \textbf{Eksempel #1} \\} \footnotesize #2 \end{tcolorbox}\vspace{-5pt} }
\newcommand{\rknut}{
Rekn ut.
}

%License
\newcommand{\lic}{\textit{Matematikken sine byggesteinar by Sindre Sogge Heggen is licensed under CC BY-NC-SA 4.0. To view a copy of this license, visit\\ 
		\net{http://creativecommons.org/licenses/by-nc-sa/4.0/}{http://creativecommons.org/licenses/by-nc-sa/4.0/}}}

%referances
\newcommand{\net}[2]{{\color{blue}\href{#1}{#2}}}
\newcommand{\hrs}[2]{\hyperref[#1]{\color{blue}\textsl{#2 \ref*{#1}}}}
\newcommand{\rref}[1]{\hrs{#1}{regel}}
\newcommand{\refkap}[1]{\hrs{#1}{kapittel}}
\newcommand{\refsec}[1]{\hrs{#1}{seksjon}}

\newcommand{\mb}{\net{https://sindrsh.github.io/FirstPrinciplesOfMath/}{MB}}


%line to seperate examples
\newcommand{\linje}{\rule{\linewidth}{1pt} }

\usepackage{datetime2}
%%\usepackage{sansmathfonts} for dyslexia-friendly math
\usepackage[]{hyperref}


\newcommand{\note}{Merk}
\newcommand{\notesm}[1]{{\footnotesize \textsl{\note:} #1}}
\newcommand{\ekstitle}{Eksempel }
\newcommand{\sprtitle}{Språkboksen}
\newcommand{\expl}{forklaring}

\newcommand{\vedlegg}[1]{\refstepcounter{vedl}\section*{Vedlegg \thevedl: #1}  \setcounter{vedleq}{0}}

\newcommand\sv{\vsk \textbf{Svar} \vspace{4 pt}\\}

%references
\newcommand{\reftab}[1]{\hrs{#1}{tabell}}
\newcommand{\rref}[1]{\hrs{#1}{regel}}
\newcommand{\dref}[1]{\hrs{#1}{definisjon}}
\newcommand{\refkap}[1]{\hrs{#1}{kapittel}}
\newcommand{\refsec}[1]{\hrs{#1}{seksjon}}
\newcommand{\refdsec}[1]{\hrs{#1}{delseksjon}}
\newcommand{\refved}[1]{\hrs{#1}{vedlegg}}
\newcommand{\eksref}[1]{\textsl{#1}}
\newcommand\fref[2][]{\hyperref[#2]{\textsl{figur \ref*{#2}#1}}}
\newcommand{\refop}[1]{{\color{blue}Oppgave \ref{#1}}}
\newcommand{\refops}[1]{{\color{blue}oppgave \ref{#1}}}
\newcommand{\refgrubs}[1]{{\color{blue}gruble \ref{#1}}}

\newcommand{\openmathser}{\openmath\,-\,serien}

% Exercises
\newcommand{\opgt}{\newpage \phantomsection \addcontentsline{toc}{section}{Oppgaver} \section*{Oppgaver for kapittel \thechapter}\vs \setcounter{section}{1}}


% Sequences and series
\newcommand{\sumarrek}{Summen av en aritmetisk rekke}
\newcommand{\sumgerek}{Summen av en geometrisk rekke}
\newcommand{\regnregsum}{Regneregler for summetegnet}

% Trigonometry
\newcommand{\sincoskomb}{Sinus og cosinus kombinert}
\newcommand{\cosfunk}{Cosinusfunksjonen}
\newcommand{\trid}{Trigonometriske identiteter}
\newcommand{\deravtri}{Den deriverte av de trigonometriske funksjonene}
% Solutions manual
\newcommand{\selos}{Se løsningsforslag.}
\newcommand{\se}[1]{Se eksempel på side \pageref{#1}}

%Vectors
\newcommand{\parvek}{Parallelle vektorer}
\newcommand{\vekpro}{Vektorproduktet}
\newcommand{\vekproarvol}{Vektorproduktet som areal og volum}


% 3D geometries
\newcommand{\linrom}{Linje i rommet}
\newcommand{\avstplnpkt}{Avstand mellom punkt og plan}


% Integral
\newcommand{\bestminten}{Bestemt integral I}
\newcommand{\anfundteo}{Analysens fundamentalteorem}
\newcommand{\intuf}{Integralet av utvalge funksjoner}
\newcommand{\bytvar}{Bytte av variabel}
\newcommand{\intvol}{Integral som volum}
\newcommand{\andordlindif}{Andre ordens lineære differensialligninger}



\begin{document}
\opgt
\setcounter{section}{1}	

\op{odpar}
\textbf{a)} Skriv opp de tre første partallene. Lag en rekursiv og en eksplisitt formel for det $ i $-te partallet.\os
	
\textbf{b)} Skriv opp de tre første oddetallene. Lag en eksplisitt formel for det $ i $-te oddetallet. 

\op{eksar}
Finn det eksplisitte uttrykket til  den aritmetiske følgen når du vet at\os
\begin{tabular}{@{}l l}	
	\textbf{a)} $ a_1=3 $ og $ a_4 = 30 $ \os 
	\textbf{b)} $ a_1 = 5 $ og $ a_{11} = -25 $ \os
	\textbf{c)} $ a_3 =14 $ og $ a_5=26 $ 
\end{tabular}\os

\op{eksgeo}
Finn det eksplisitte uttrykket til den geometriske følgen når du vet at\os
\begin{tabular}{@{}l l}	
	\textbf{a)} $ a_1=\frac{1}{2} $ og $ a_2 = \frac{1}{6} $ \os 
	\textbf{b)} $ a_1 = 5 $ og $ a_4 = 40 $
\end{tabular} \os

\begin{comment}
	\op
Bruk formelen fra oppgave \textsl{\ref{sumkvad}a} og det eksplisitte uttrykket for en aritmetisk følge til å forklare at summen av en aritmetisk rekke kan skrives som:
\[na_1 +\frac{dn(n-1)}{2} \]


\op
a) Bruk (\ref{sumg}) og vis at summen $ S_\infty $ når $ n\to\infty $ blir:
\[ S_\infty=a_1\frac{1}{1-k} \]
for $ |k|<1 $.
\end{comment}
\nes

\op{parodd}
\textbf{a)} Bruk figuren under til å forklare at summen $ S_n $ av de $ n $ første naturlige tallene er gitt ved
\[S_n=\frac{n(n+1)}{2}  \]
\fig{sum}
\textbf{b)} Skriv opp summen av det første, de to første og de tre første  oddetallene. Bruk en lignende figur som i oppgave a) til å vise at summen $ S_n $ av de $ n $ første oddetallene er
\[ S_n = n^2 \] 

\op{opgnegsum}
$ -1-2-3 $ er en rekke. Skriv om rekka slik at den blir uttrykt ved ledd som adderes med hverandre. 

\op{sum10ar}
Finn $ S_{10} $ for rekkene:\os
\begin{tabular}{@{}l l}	
	\textbf{a)} $ 7+13+19+25+\ldots $\quad
	\textbf{b)} $ 1+9+17+25+\ldots $
\end{tabular} \os

\op{ar435}
Gitt rekken 
\[ 8+11+14+\ldots \]
For hvilken $ n $ er summen av rekken lik 435?

\op{opgarekfraeks}
Gitt den uendelige rekken
\[ 3+7+11+... \] 
For hvilken $ n $ er summen av rekken lik 903?

\op{opggerekfraeks}
Gitt den uendelige rekken
\[ 3+6+12+24+... \]	
For hvor mange element er summen av rekken lik 93?

\op{viseks3}
Bruk summen av en aritmetisk rekke til å vise at ligningen gitt i \hyperref[prodind]{\textsl{Eksempel 3}} på s. \pageref{prodind} er sann.

\op{geon}
Gitt rekken
\[ 3+12+48+\ldots+768 \]
Finn summen av rekken. 
\newpage
\op{geoa12}
En geometrisk rekke har $ {a_1 = 2} $ og $ {k=3} $.\os 

\textbf{a)} Vis at summen $ S_n $ kan skrives som:
\[ S_n = 3^n-1 \]
\textbf{b)} Regn ut summen for de tre første leddene.\os

\textbf{c)} For hvilken $ n $ er $ S_n=728 $?



\begin{comment}
\textbf{c)} Hvis du fortsetter å spare slik, og medregner innskudd samme måned, når vil du ha 24200 kr på konto? 
\end{comment}

\op{1over4}
Gitt den uendelige rekken
\[ 4+1+\frac{1}{4}+\ldots \]
\textbf{a)} Forklar hvorfor rekken er konvergent.\os

\textbf{b)} Finn summen av den uendelige rekken.

\op{opggerekuendeks}
Gitt den uendelige rekka 
\[ 1+\frac{1}{x}+\frac{1}{x^2}+.... \]
\abc{
\item  For hvilken $ x $ er summen av rekka lik $ \frac{3}{2} $?\os

\item For hvilken $ x $ er summen av rekka lik $ -1 $?
}

\begin{comment}
	\op{stav} 
	Tenk at uendelig mange personer skal sette sammen en stav. Første person legger på en meter, neste person legger på 0.1 m, neste legger på 0.01 m osv. Hvor lang blir staven?
\end{comment}
\newpage
\op{099er1}
\textbf{a)} Skriv det uendelige desimaltallet 0.999... som en uendelig geometrisk rekke.\os

\textbf{b)} Forklar hvorfor rekken er konvergent og bruk dette faktumet til å finne summen av rekken. 

\op{geokonv}
Gitt den uendelige rekken
\[\frac{1}{3} +\frac{1}{3}(x-2)+ \frac{1}{3}(x-2)^2+\ldots\]
\textbf{a)} For hvilke $ x $ er rekken konvergent? \os

\textbf{b)} For hvilken $ x $ er $ S_n = \frac{2}{9} $?\os

\textbf{c)} For hvilken $ x $ er $ S_n= \frac{1}{6}$? \os

\nes
\op{ind}
Vis ved induksjon at for alle $ n\in \mathbb{N} $ er\\[10pt]
\begin{tabular}{@{}l l}	
	\textbf{a)} $ 1+2+3+\ldots+n = \dfrac{n(n+1)}{2} $ \\[10pt]
	\textbf{b)} $ 1+2 +2^2 +\ldots+ 2^{n-1}= 2^n-1$\\[10pt]
	\textbf{c)} $ 4+4^2+4^3+\ldots+4^n = \dfrac{4}{3}(4^n-1) $ \\[10pt]
	\textbf{d)} $ 1^2 + 2^2+3^2+ \ldots+ n^2 = \dfrac{n(2n+1)(n+1)}{6} $ 
\end{tabular} 

\op{div3}
Vis ved induksjon at $ n(n^2+2) $ er delelig med 3 for alle $ n\in\mathbb{N} $.

\op{factorials}
\textbf{a)}
Vis ved induksjon at:
\[\frac{1\cdot2}{1}\cdot\frac{1\cdot2\cdot3\cdot4}{1\cdot2\cdot3}\cdot
\ldots\cdot\frac{(2n)!}{(2n-1)!}=2^n n! \]
\textsl{Hint}: \y{(2(k+1))!=(2k+1)!(2k+2)}.\os

\textbf{b)} Hvordan kan venstresiden i a) skrives enklere? Utfør induksjonsbeviset på nytt etter forenklingen.
\newpage
\grubop{opgsumnkvad}
Måletl med denne oppgaven er å, uten bruk av induksjon, vise at summen av $ n $ kvadrater er gitt ved følgende formel:
\[ \sum\limits_{i=1}^n i^2 = \frac{n(2n+1)(n+1)}{6} \tag{I}\label{sumkvad}\]
\textbf{a)} Forklar hvorfor vi kan skrive
\[ 1^2 + 2^2 + 3^2+\ldots = 1 + (1+3) + (1+3+5)+ \ldots  \]
\textsl{Hint}: se opg. \ref{parodd} b).\os

\textbf{b)} Ut ifra det du fant i a), forklar at
\[ \sum\limits_{i=1}^n i^2 = n+\sum\limits_{i=1}^n (n-i)(2i+1)  \]
\textbf{c)} Skriv ut alle kjente summer fra b) og løs ligningen med hensyn på $ \sum\limits_{i=1}^n i^2 $, du skal da komme fram til (\ref{sumkvad}).


\end{document}

