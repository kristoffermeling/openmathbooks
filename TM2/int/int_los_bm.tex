\documentclass[english, 11 pt, class=article, crop=false]{standalone}
\usepackage[T1]{fontenc}
%\renewcommand*\familydefault{\sfdefault} % For dyslexia-friendly text
\usepackage{lmodern} % load a font with all the characters
\usepackage{geometry}
\geometry{verbose,paperwidth=16.1 cm, paperheight=24 cm, inner=2.3cm, outer=1.8 cm, bmargin=2cm, tmargin=1.8cm}
\setlength{\parindent}{0bp}
\usepackage{import}
\usepackage[subpreambles=false]{standalone}
\usepackage{amsmath}
\usepackage{amssymb}
\usepackage{esint}
\usepackage{babel}
\usepackage{tabu}
\makeatother
\makeatletter

\usepackage{titlesec}
\usepackage{ragged2e}
\RaggedRight
\raggedbottom
\frenchspacing

% Norwegian names of figures, chapters, parts and content
\addto\captionsenglish{\renewcommand{\figurename}{Figur}}
\makeatletter
\addto\captionsenglish{\renewcommand{\chaptername}{Kapittel}}
\addto\captionsenglish{\renewcommand{\partname}{Del}}


\usepackage{graphicx}
\usepackage{float}
\usepackage{subfig}
\usepackage{placeins}
\usepackage{cancel}
\usepackage{framed}
\usepackage{wrapfig}
\usepackage[subfigure]{tocloft}
\usepackage[font=footnotesize,labelfont=sl]{caption} % Figure caption
\usepackage{bm}
\usepackage[dvipsnames, table]{xcolor}
\definecolor{shadecolor}{rgb}{0.105469, 0.613281, 1}
\colorlet{shadecolor}{Emerald!15} 
\usepackage{icomma}
\makeatother
\usepackage[many]{tcolorbox}
\usepackage{multicol}
\usepackage{stackengine}

\usepackage{esvect} %For vectors with capital letters

% For tabular
\usepackage{array}
\usepackage{multirow}
\usepackage{longtable} %breakable table

% Ligningsreferanser
\usepackage{mathtools}
\mathtoolsset{showonlyrefs}

% index
\usepackage{imakeidx}
\makeindex[title=Indeks]

%Footnote:
\usepackage[bottom, hang, flushmargin]{footmisc}
\usepackage{perpage} 
\MakePerPage{footnote}
\addtolength{\footnotesep}{2mm}
\renewcommand{\thefootnote}{\arabic{footnote}}
\renewcommand\footnoterule{\rule{\linewidth}{0.4pt}}
\renewcommand{\thempfootnote}{\arabic{mpfootnote}}

%colors
\definecolor{c1}{cmyk}{0,0.5,1,0}
\definecolor{c2}{cmyk}{1,0.25,1,0}
\definecolor{n3}{cmyk}{1,0.,1,0}
\definecolor{neg}{cmyk}{1,0.,0.,0}

% Lister med bokstavar
\usepackage[inline]{enumitem}

\newcounter{rg}
\numberwithin{rg}{chapter}
\newcommand{\reg}[2][]{\begin{tcolorbox}[boxrule=0.3 mm,arc=0mm,colback=blue!3] {\refstepcounter{rg}\phantomsection \large \textbf{\therg \;#1} \vspace{5 pt}}\newline #2  \end{tcolorbox}\vspace{-5pt}}

\newcommand\alg[1]{\begin{align} #1 \end{align}}

\newcommand\eks[2][]{\begin{tcolorbox}[boxrule=0.3 mm,arc=0mm,enhanced jigsaw,breakable,colback=green!3] {\large \textbf{Eksempel #1} \vspace{5 pt}\\} #2 \end{tcolorbox}\vspace{-5pt} }

\newcommand{\st}[1]{\begin{tcolorbox}[boxrule=0.0 mm,arc=0mm,enhanced jigsaw,breakable,colback=yellow!12]{ #1} \end{tcolorbox}}

\newcommand{\spr}[1]{\begin{tcolorbox}[boxrule=0.3 mm,arc=0mm,enhanced jigsaw,breakable,colback=yellow!7] {\large \textbf{Språkboksen} \vspace{5 pt}\\} #1 \end{tcolorbox}\vspace{-5pt} }

\newcommand{\sym}[1]{\colorbox{blue!15}{#1}}

\newcommand{\info}[2]{\begin{tcolorbox}[boxrule=0.3 mm,arc=0mm,enhanced jigsaw,breakable,colback=cyan!6] {\large \textbf{#1} \vspace{5 pt}\\} #2 \end{tcolorbox}\vspace{-5pt} }

\newcommand\algv[1]{\vspace{-11 pt}\begin{align*} #1 \end{align*}}

\newcommand{\regv}{\vspace{5pt}}
\newcommand{\mer}{\textsl{Merk}: }
\newcommand{\mers}[1]{{\footnotesize \mer #1}}
\newcommand\vsk{\vspace{11pt}}
\newcommand\vs{\vspace{-11pt}}
\newcommand\vsb{\vspace{-16pt}}
\newcommand\sv{\vsk \textbf{Svar} \vspace{4 pt}\\}
\newcommand\br{\\[5 pt]}
\newcommand{\figp}[1]{../fig/#1}
\newcommand\algvv[1]{\vs\vs\begin{align*} #1 \end{align*}}
\newcommand{\y}[1]{$ {#1} $}
\newcommand{\os}{\\[5 pt]}
\newcommand{\prbxl}[2]{
\parbox[l][][l]{#1\linewidth}{#2
	}}
\newcommand{\prbxr}[2]{\parbox[r][][l]{#1\linewidth}{
		\setlength{\abovedisplayskip}{5pt}
		\setlength{\belowdisplayskip}{5pt}	
		\setlength{\abovedisplayshortskip}{0pt}
		\setlength{\belowdisplayshortskip}{0pt} 
		\begin{shaded}
			\footnotesize	#2 \end{shaded}}}

\renewcommand{\cfttoctitlefont}{\Large\bfseries}
\setlength{\cftaftertoctitleskip}{0 pt}
\setlength{\cftbeforetoctitleskip}{0 pt}

\newcommand{\bs}{\\[3pt]}
\newcommand{\vn}{\\[6pt]}
\newcommand{\fig}[1]{\begin{figure}
		\centering
		\includegraphics[]{\figp{#1}}
\end{figure}}

\newcommand{\figc}[2]{\begin{figure}
		\centering
		\includegraphics[]{\figp{#1}}
		\caption{#2}
\end{figure}}

\newcommand{\sectionbreak}{\clearpage} % New page on each section

\newcommand{\nn}[1]{
\begin{equation}
	#1
\end{equation}
}

% Equation comments
\newcommand{\cm}[1]{\llap{\color{blue} #1}}

\newcommand\fork[2]{\begin{tcolorbox}[boxrule=0.3 mm,arc=0mm,enhanced jigsaw,breakable,colback=yellow!7] {\large \textbf{#1 (forklaring)} \vspace{5 pt}\\} #2 \end{tcolorbox}\vspace{-5pt} }
 
%colors
\newcommand{\colr}[1]{{\color{red} #1}}
\newcommand{\colb}[1]{{\color{blue} #1}}
\newcommand{\colo}[1]{{\color{orange} #1}}
\newcommand{\colc}[1]{{\color{cyan} #1}}
\definecolor{projectgreen}{cmyk}{100,0,100,0}
\newcommand{\colg}[1]{{\color{projectgreen} #1}}

% Methods
\newcommand{\metode}[2]{
	\textsl{#1} \\[-8pt]
	\rule{#2}{0.75pt}
}

%Opg
\newcommand{\abc}[1]{
	\begin{enumerate}[label=\alph*),leftmargin=18pt]
		#1
	\end{enumerate}
}
\newcommand{\abcs}[2]{
	\begin{enumerate}[label=\alph*),start=#1,leftmargin=18pt]
		#2
	\end{enumerate}
}
\newcommand{\abcn}[1]{
	\begin{enumerate}[label=\arabic*),leftmargin=18pt]
		#1
	\end{enumerate}
}
\newcommand{\abch}[1]{
	\hspace{-2pt}	\begin{enumerate*}[label=\alph*), itemjoin=\hspace{1cm}]
		#1
	\end{enumerate*}
}
\newcommand{\abchs}[2]{
	\hspace{-2pt}	\begin{enumerate*}[label=\alph*), itemjoin=\hspace{1cm}, start=#1]
		#2
	\end{enumerate*}
}

% Oppgaver
\newcommand{\opgt}{\phantomsection \addcontentsline{toc}{section}{Oppgaver} \section*{Oppgaver for kapittel \thechapter}\vs \setcounter{section}{1}}
\newcounter{opg}
\numberwithin{opg}{section}
\newcommand{\op}[1]{\vspace{15pt} \refstepcounter{opg}\large \textbf{\color{blue}\theopg} \vspace{2 pt} \label{#1} \\}
\newcommand{\ekspop}[1]{\vsk\textbf{Gruble \thechapter.#1}\vspace{2 pt} \\}
\newcommand{\nes}{\stepcounter{section}
	\setcounter{opg}{0}}
\newcommand{\opr}[1]{\vspace{3pt}\textbf{\ref{#1}}}
\newcommand{\oeks}[1]{\begin{tcolorbox}[boxrule=0.3 mm,arc=0mm,colback=white]
		\textit{Eksempel: } #1	  
\end{tcolorbox}}
\newcommand\opgeks[2][]{\begin{tcolorbox}[boxrule=0.1 mm,arc=0mm,enhanced jigsaw,breakable,colback=white] {\footnotesize \textbf{Eksempel #1} \\} \footnotesize #2 \end{tcolorbox}\vspace{-5pt} }
\newcommand{\rknut}{
Rekn ut.
}

%License
\newcommand{\lic}{\textit{Matematikken sine byggesteinar by Sindre Sogge Heggen is licensed under CC BY-NC-SA 4.0. To view a copy of this license, visit\\ 
		\net{http://creativecommons.org/licenses/by-nc-sa/4.0/}{http://creativecommons.org/licenses/by-nc-sa/4.0/}}}

%referances
\newcommand{\net}[2]{{\color{blue}\href{#1}{#2}}}
\newcommand{\hrs}[2]{\hyperref[#1]{\color{blue}\textsl{#2 \ref*{#1}}}}
\newcommand{\rref}[1]{\hrs{#1}{regel}}
\newcommand{\refkap}[1]{\hrs{#1}{kapittel}}
\newcommand{\refsec}[1]{\hrs{#1}{seksjon}}

\newcommand{\mb}{\net{https://sindrsh.github.io/FirstPrinciplesOfMath/}{MB}}


%line to seperate examples
\newcommand{\linje}{\rule{\linewidth}{1pt} }

\usepackage{datetime2}
%%\usepackage{sansmathfonts} for dyslexia-friendly math
\usepackage[]{hyperref}


\newcommand{\note}{Merk}
\newcommand{\notesm}[1]{{\footnotesize \textsl{\note:} #1}}
\newcommand{\ekstitle}{Eksempel }
\newcommand{\sprtitle}{Språkboksen}
\newcommand{\expl}{forklaring}

\newcommand{\vedlegg}[1]{\refstepcounter{vedl}\section*{Vedlegg \thevedl: #1}  \setcounter{vedleq}{0}}

\newcommand\sv{\vsk \textbf{Svar} \vspace{4 pt}\\}

%references
\newcommand{\reftab}[1]{\hrs{#1}{tabell}}
\newcommand{\rref}[1]{\hrs{#1}{regel}}
\newcommand{\dref}[1]{\hrs{#1}{definisjon}}
\newcommand{\refkap}[1]{\hrs{#1}{kapittel}}
\newcommand{\refsec}[1]{\hrs{#1}{seksjon}}
\newcommand{\refdsec}[1]{\hrs{#1}{delseksjon}}
\newcommand{\refved}[1]{\hrs{#1}{vedlegg}}
\newcommand{\eksref}[1]{\textsl{#1}}
\newcommand\fref[2][]{\hyperref[#2]{\textsl{figur \ref*{#2}#1}}}
\newcommand{\refop}[1]{{\color{blue}Oppgave \ref{#1}}}
\newcommand{\refops}[1]{{\color{blue}oppgave \ref{#1}}}
\newcommand{\refgrubs}[1]{{\color{blue}gruble \ref{#1}}}

\newcommand{\openmathser}{\openmath\,-\,serien}

% Exercises
\newcommand{\opgt}{\newpage \phantomsection \addcontentsline{toc}{section}{Oppgaver} \section*{Oppgaver for kapittel \thechapter}\vs \setcounter{section}{1}}


% Sequences and series
\newcommand{\sumarrek}{Summen av en aritmetisk rekke}
\newcommand{\sumgerek}{Summen av en geometrisk rekke}
\newcommand{\regnregsum}{Regneregler for summetegnet}

% Trigonometry
\newcommand{\sincoskomb}{Sinus og cosinus kombinert}
\newcommand{\cosfunk}{Cosinusfunksjonen}
\newcommand{\trid}{Trigonometriske identiteter}
\newcommand{\deravtri}{Den deriverte av de trigonometriske funksjonene}
% Solutions manual
\newcommand{\selos}{Se løsningsforslag.}
\newcommand{\se}[1]{Se eksempel på side \pageref{#1}}

%Vectors
\newcommand{\parvek}{Parallelle vektorer}
\newcommand{\vekpro}{Vektorproduktet}
\newcommand{\vekproarvol}{Vektorproduktet som areal og volum}


% 3D geometries
\newcommand{\linrom}{Linje i rommet}
\newcommand{\avstplnpkt}{Avstand mellom punkt og plan}


% Integral
\newcommand{\bestminten}{Bestemt integral I}
\newcommand{\anfundteo}{Analysens fundamentalteorem}
\newcommand{\intuf}{Integralet av utvalge funksjoner}
\newcommand{\bytvar}{Bytte av variabel}
\newcommand{\intvol}{Integral som volum}
\newcommand{\andordlindif}{Andre ordens lineære differensialligninger}




\begin{document}
\opr{antiderint} \textbf{a)} $ f'(x)=20x^4 $ \textbf{b)}  $ f(2)-f(0)=128 $ \vsk

\opr{Fogf} $ F(4)-F(1)=8 $ \vsk

\opr{ecos2x}\\
\textbf{a)} Vi bruker kjerneregelen to ganger. Først setter vi $ u(x) = \cos^2 x $ og $ g(u)=e^{u} $. Deretter setter vi $ h(x)=\cos x $ og $ i(h)=h^2 $. Vi får da at:
\alg{
u'(x)&= i'(h)h'(x) \\
&= 2h \cdot(-\sin x) \\
&= -2\cos x \sin x 
}
Videre har vi da at:
\alg{
f'(x)&= g'(u)u'(x) \\
&= e^u \cdot( -2\cos x \sin x ) \\
&= -2\cos x \sin x\, e^{\cos^2 x}
}
\textbf{b)} Av (\ref{sin2x}) har vi at $ 2\cos x \sin x = \sin(2x) $, og derfor kan vi skrive:
\alg{
\int -\sin (2x)\, e^{\cos^2 x}\,dx &= \int -2\cos x \sin x\, e^{\cos^2 x}\,dx \\
&= \int f'(x)\,dx \\
&= f(x)+C \\
&= e^{\cos^2 x}+ C
}
Overgangen mellom andre og tredje linje følger av definisjonen av det ubestemte integralet. \vsk

\opr{visantider}\\
\textbf{a)} Vi må vise at $ \left(x^2 e^x \right)'$ tilsvarer uttrykket i integranden.
\alg{\left(x^2 e^x \right)'&= 2xe^x+x^2e^x \\
	&= xe^x(2+x)
}
\\
\textbf{b)} Vi må vise at $ \left(e^{\cos x+x^2}\right)' $ tilsvarer uttrykket i integranden:
\alg{
\left(e^{\cos x+x^2}\right)' &= e^{\cos x+x^2}\cdot(\cos x+x^2)' \\
&= e^{\cos x+x^2}(-\sin x + 2x) \\
&= -e^{\cos x+x^2}(\sin x - 2x) 
}

\opr{avcos} 
Av (\ref{ker2pioverP}) vet vi at perioden $ \cos x $ er $ 2\pi $. Dette betyr at hvis vi for en konstant $ c $ har at $ a=c $, så er $ b=a+2\pi $. Integralet blir da:
\alg{
	\int\limits_{c}^{c+2\pi} (\cos x + k)\,dx &= \Big[\sin x+kx\Big]_c^{c+2\pi} \\
	&=\Big[\sin(c+2\pi)+k(c+2\pi)-(\sin c+kc)\Big] \\
	&= 2k\pi
}
Mellom andre og tredje linje har vi brukt at $ \sin(c+2\pi)=\sin c $. Gjennomsnittet kan altså skrives som
\[ \frac{1}{(c+2\pi)-c}\cdot 2k\pi = k \]

\opr{bytvaropg}
\abc{
\item  Vi setter $ u=x^2 $ og $ g(u)=e^u $. Siden $ u'=2x $ får vi:
\alg{\int xe^{x^2}\,dx &=\frac{1}{2}\int 2xe^{x^2} \\
	&= \frac{1}{2}\int u'e^{u}\,dx \\
	&= \frac{1}{2}\int e^u\,du\\
	&= \frac{1}{2}e^u + C \\
	&= \frac{1}{2}e^{x^2}+C
}
\item  Vi starter med å finne det ubestemte integralet ved å bruke bytte av variabel. Vi setter $ u=2x^2-3$ og $ g(u)=e^u $, siden $ u'=4x $ får vi:
\alg{
	\int\limits 8xe^{2x^2-3}\,dx &=2\int 4xe^{2x^2-3}\,dx \\
	&= 2\int u'e^u\,dx \\
	&= 2 \int e^u \, du \\
	&= 2e^u + C
}
Det bestemte integralet blir derfor:
\alg{
	\left[2e^{2x^2-3}\right]_1^2&= 2\left[e^{2\cdot2^2-3}-e^{2\cdot 1-3}\right]\\
	&= 2\left[e^5-e^{-1}\right]
}
\item  Vi setter $ {u=\cos x} $ og $ {g(u)=u} $. Siden $ u'=-\sin x $, får vi:
\alg{
	\int \frac{\sin x}{\cos x} \, dx &= -\int \frac{u'}{u}\, dx \\ 
	&= -\int \frac{1}{u}\, du \\
	&= -\ln u + C \\
	&= -\ln(\cos x) + C \\
}
\item  Vi setter $ u=\cos x $ og $ g(u)=\frac{1}{u^3} $. Siden $ u'=-\sin x $, får vi:
\alg{
	\int\frac{\sin x}{\cos^3 x} \, dx  &= \int \frac{-u'}{u^3} \,dx \\
	&= -\int u^{-3}\, dx \\
	&= \frac{1}{2}u^{-2}+ C
}
Siden $ u(0)=\cos 0=1 $ og $ u(\frac{\pi}{3})=\cos\left(\frac{\pi}{3}\right)=2^{-1} $ blir det bestemte integralet:
\alg{\left[\frac{1}{2}u^{-2}\right]_{1}^{2^{-1}}
	&= \frac{1}{2}\left[(2^{-1})^{-2}-1^{-2}\right] \\
	&= \frac{1}{2}[4-1] \\
	&= \frac{3}{2}
}
\item Vi setter $ u= 2x^2+5x $ og $ g(u)=\frac{1}{u} $. Da er
\alg{
\int \frac{4x+5}{2x^2+5x}\,dx&=\int \frac{u'}{u}\,dx \br
&=\int \frac{1}{u}\,du \br
&= \ln u + C  \\
&= \ln(2x^2+5x)+ C
}
\item  Vi setter $ u=3x^2+4x+3 $ og $ g(u)=\frac{1}{u} $, og får da:
\alg{
	\int \frac{3x+2}{3x^2 + 4x+3}\,dx &= \frac{1}{2}\int \frac{6x+4}{3x^2+4x+3}\,dx \\
	&= \frac{1}{2}\int \frac{u'}{u}\,dx \\
	&= \frac{1}{2}\int u^{-1}\,du \\
	&= \frac{1}{2}\ln u \\
	&= \frac{1}{2}\ln \left(3x^2+4x+3\right)
}

\opr{trigint} \\
Av (\ref{1}) og (\ref{sin2x}) kan vi skrive:
\[ \int \sin (2x) e^{1-\cos^2 x}\,dx=\int 2\sin x \cos x\, e^{\sin^2 x} \]
Vi setter så $ u=\sin x $ og $ g(u)=2ue^{u^2} $. Siden $ u'=\cos x $ kan vi skrive:
\alg{
	\int 2\sin x \cos x\, e^{\sin^2 x}\,dx &= 2\int uu'e^{u^2}\,dx \\
	&= 2\int u e^{u^2}\,dx
}
Vi setter nå $ v=u^2 $ og $ h(v)=e^v $. Siden $ v'=2u $ får vi:
\alg{
	2\int u e^{u^2}\,dx &= \int v'e^v\,dx \\
	&= \int e^v \,dx \\
	&= e^v + C\\
	&= e^{u^2}+C \\
	&= e^{\sin^2 x}+ C
}
}

\opr{delvisintopg}\\

\textbf{b)} Vi setter $ u=\ln x $ og $ v'=x^{\frac{1}{2}} $ og får da at $ u'=x^{-1}$ og $ v = \frac{2}{3}x^{\frac{3}{2}} $:
\alg{
\int \sqrt{x}\ln x\,dx &=\int uv'\,dx \\
&= uv -\int u'v\,dx\\
&= \ln x \cdot \frac{2}{3}x^{\frac{3}{2}} - \int x^{-1}\cdot\frac{2}{3}x^{\frac{3}{2}} \\
&= \frac{2}{3}x^{\frac{3}{2}}\ln x - \frac{2}{3}\int x^{\frac{1}{2}} \\
&= \frac{2}{3}x^{\frac{3}{2}}\ln x - \frac{2}{3}\cdot\frac{2}{3}x^{\frac{3}{2}}+C \\
&= \frac{2}{9} x^{\frac{3}{2}} (3 \ln x - 2)+C
}

\textbf{c)}
Vi setter $ u = \ln x $ og $ v'= x^{-2} $, og får da at $ u'=x^{-1}$ og $ v=-x^{-1} $.
\alg{
\int \ln x\,x^{-2}\,dx &=\int uv'\,dx \\
&= uv -\int u'v\,dx\\
&= \ln x\, (-x^{-1})-\int x^{-1}(-x)^{-1}\,dx \\
&= -\frac{\ln x}{x}+\int x^{-2}\,dx \\
&=  -\frac{\ln x}{x}-\frac{1}{x}+C \\
&= -\frac{1}{x}(\ln x+1)+ C
}
Det bestemte integralet blir da:
\alg{
\left[-\frac{1}{x}(\ln |x|+1)\right]_1^e &=
-\left[\frac{1}{e}(\ln e+1)-\frac{1}{1}(\ln 1+1)\right] \\
&= -\left[\frac{2}{e}-1\right] \\
&= 1-\frac{2}{e}
}
\newpage
\opr{delvisintopg2}
Vi setter $ u=\sin x $ og $ v'=\sin x $, og får da at $ u'=\cos x $ og $ v=-\cos x $:
\alg{
\int \sin^2 x\,dx &= \sin x(-\cos x)-\int \cos x(-\cos x) \\
 &= -\sin x \cos x+\int \cos^2 x\,dx \\
 &= -\sin x \cos x+\int (1-\sin^2 x)\,dx \\
2\int \sin^2 x\,dx &= -\sin x \cos x+\int 1\,dx \\
\int \sin^2 x\,dx &= \frac{1}{2}(x-\sin x \cos x)+C
}

\opr{delbropsopg}
\textbf{a)}
Vi starter med å faktorisere nevneren i integranden:
\[ x^2-5x+6=(x-2)(x-3) \]
Altså kan vi skrive:
\alg{
\frac{13-4x}{x^2-5x+6}&= \frac{A}{x-2}+\frac{B}{x-3} \br
13-4x&=A(x-3)+B(x-2)
}
Når $ x=3 $ får vi:
\algv{
13-4\cdot3 &= B(3-2) \\
1&=B 
}
Og når $ x=2 $ får vi:
\algv{12-4\cdot2&= A(2-3) \\
5 &= -A \\
-5 &= A}
Det ubestemte integralet vi ønsker å løse kan derfor skrives som:
\alg{
\int \left(\frac{1}{x-3}-\frac{5}{x-2}\right)\,dx 
&= \ln (x-3)-5\ln(x-2)+ C
}
Det bestemte integralet blir da:
\alg{
\Big[\ln |x-3|-5\ln|x-2|\Big]_4^5 &= \ln |5-3|-5\ln|5-2|-(\ln |4-3|-5\ln|4-2|) \\
&= \ln 2-5\ln 3-\ln 1+5\ln 2 \\
&= 6\ln 2-5\ln 3
}

\opr{delbrogpoldiv}
\se{delbre2}

\opr{gerfminusdb}
\abc{
\item Ut ifra figuren ser vi at
\alg{
&\int\limits_{a}^b g\,dx = A+(b-a)k-B	\vn
&\int\limits_{a}^b f\,dx = A-B = \int\limits_{a}^bg\,dx -(b-a)k
}
\item Vi har at
\algv{
\int\limits_{a}^b g\,dx &= \int\limits_{a}^b f+k\,dx \\
&= \int\limits_{a}^b f\,dx + [kx]_a^b \\
&= \int\limits_{a}^b f\,dx +(b-a)k\\
\int\limits_{a}^b g\,dx-(b-a)k &= \int\limits_{a}^b f\,dx
}
}

\newpage
\opr{kulevolopg}\\
\textbf{a)}
Tverrsnittet langs $ x $-aksen blir en sirkel med høyde $ \sqrt{r^2-x^2} $. Tverrsnittsarealet blir derfor
\alg{
A(x)&= \pi \sqrt{r^2-x^2}^{\,2} \\
&= \pi(r^2-x^2)
}
\textbf{b)} 
\algv{
V &= \pi\int\limits_{-r}^r A(x)\,dx \\
&= \pi \int\limits_{-r}^r \left(r^2-x^2\right)\,dx \\
&= \pi\left[xr^2-\frac{1}{3}x^3\right]_{-r}^r \\
&= \frac{\pi}{3}\left(3r r^2-r^3-(3(-r)r^2-(-r)^3)\right)\\
&= \frac{\pi}{3}(3r^3-r^3+3r^3-r^3) \\
&= \frac{4\pi}{3}r^3
}

\opr{omdropg}\\
Volumet $ V $ er gitt ved ligningen:
\alg{
V&=\pi\int\limits_0^1 f^2\,dx \\
&= \pi\int\limits_0^1 \left(e^x\right)^2\,dx \\
&= \pi\int\limits_0^1 e^{2x}\,dx
}
Vi setter $ u=2x $ og $ g(u)=e^u $, da blir $ u'=2 $:
\alg{
\int\limits e^{2x}\,dx &= \frac{1}{2}\int\limits 2e^{2x}\,dx\\
 &= \frac{1}{2}\int u'e^u\,du \\
 &= \frac{1}{2}e^u+C
}
Siden $ u(0)=0 $ og $ u(1)=2 $ blir det bestemte integralet
\alg{
\left[\frac{1}{2}e^u\right]_0^2 &=\frac{1}{2}\left[e^2-e^0\right]\\
&= \frac{1}{2}\left[e^2-1\right]
}
Altså er 
\[ V = \frac{\pi}{2}(e^2-1) \]

\newpage
\grubr{opgintsin2x}\\
Vi har at
\[ \int \sin^2 x\, dx = \int \sin x \cdot \sin x \, dx\]
Vi setter $ u=\sin x $ og $ v'=\sin x$. Da er
\[ u'=\cos x\qquad\qquad v=-\cos x \]
Altså har vi at
\alg{
\int \sin x \cdot \sin x \, dx&= -\sin x \cos x +\int \cos^2 x \, dx \\
2\int \sin^2 x \, dx&= -\sin x \cos x +\int \cos^2 x \, dx + \int \sin^2 x \,dx
}
Ettersom $ \cos^2 x+ \sin^2 x =1 $, følger det at
\alg{
2\int \sin^2 x \, dx&= -\sin x \cos x +\int 1\, dx \\
2\int \sin^2 x \, dx&= -\sin x \cos x + x \\
\int \sin^2 x \, dx&= \frac{1}{2}\left(x-\sin x \cos x\right)
}

\newpage
\grubr{r2v23d1opg2}
\abc{
\item Vi har at $ f(x)=\tan x = \frac{\sin x}{\cos x} $. Da $ (\sin x)'=\cos x $ og $ (\cos x)'=-\sin x $ har vi av divisjonsregelen ved derivasjon (se \tmen) at
\alg{
	f'(x)=\frac{\cos^2 x+\sin^2 x}{\cos^2 x}= 1+\tan^2 x
}
\item Vi setter $ u=\tan x $. Av oppgave a) har vi da at
\alg{
\int \frac{1+\tan^2 x}{\tan x}\, dx = \int \frac{u'}{u}\, dx = \int \frac{1}{u}\,dx = \ln|u| + C = \ln|\tan x|
}
}


\grubr{r2h23d1opg1} \\
Vi har at
\alg{
\int_{-1}^{1}x^3+2x\,dx = \left[\frac{x^4}{4}+x^2\right]_{-1}^1=\frac{1}{4}+1-\left(\frac{1}{4}-1\right)=0
}
Svaret forteller at arealet avgrenset av $ f(x)=x^3+2x $ og $ x $-aksen for $f\geq0 $ er like stort som arealet avgrenset av $ f $ og $ x $-aksen for $ f\leq0 $ på intervallet $ x\in[-1, 1] $. \vsk

\grubr{r2h23d1opg2}\\
$ f $ og $ g $ skjærer hverandre når
\alg{
\sin x &= \cos x \\
\tan x &= 1
}
Da $ \atan 1 = \frac{\pi}{4} $, skjærer $ f $ og $ g $ hverandre når
$ x=\frac{\pi}{4}+\pi n $ for $ n\in\mathbb{Z} $.
De to skjæringspunktene i figuren må dermed være $ x\in[-\frac{3}{4}\pi, \frac{\pi}{4}] $. Da $ f>g $ på dette intervallet, er arealet til det fargede området gitt som
\alg{
\int\limits_{-\frac{3}{4}\pi}^{\frac{\pi}{4}} \cos x-\sin x \,dx &= \bigg[\sin x+\cos x\bigg]_{-\frac{3}{4}\pi}^{\frac{\pi}{4}}\\
&=\sin \left(\frac{\pi}{4}\right)+\cos\left(\frac{\pi}{4}\right)-\left(\sin\left(-\frac{3}{4}\pi\right)+\cos\left(-\frac{3}{4}\pi\right)\right) \br
&= \frac{\sqrt{2}}{2}+\frac{\sqrt{2}}{2}-\left(-\frac{\sqrt{2}}{2}-\frac{\sqrt{2}}{2}\right)\\
&=2\sqrt{2}
}

\newpage
\grubr{opgfunklen}
\fig{opgfunklen_los}
For $ i, n\in \mathbb{N} $ setter vi $ {x_{i+1}=x_i+\Delta x} $, hvor $ \lim\limits_{n\to \infty} {\Delta x = 0} $. Da er \\$ {f(x_i+1)-f(x_i)=f(x_i+\Delta)} $. Avstanden mellom $ P_{i+1} $ og $ P_{i} $ er da gitt som
\[ \sqrt{(\Delta x)^2+[f(x_i+\Delta x)-f(x_i)]^2}=\Delta x\sqrt{1+\left(\frac{f(x_i+\Delta x)-f(\Delta x)}{(\Delta x)^2}\right)^2} \]
I tilfellet der $ {\Delta x\to 0} $ gjenkjenner vi brøken som $ \left[f'(x_i)\right]^2 $, og dermed er
\[ \lim\limits_{n\to \infty}\sum\limits_{i=1}^n \left|P_{i+1}-P_i\right|=\lim\limits_{n\to \infty}\sum\limits_{i=1}^n\Delta x\sqrt{1+[f'(x_i)]^2} \]
Av \eqref{bint} har vi da at
\[ \lim\limits_{n\to \infty}\sum\limits_{i=1}^n \left|P_{i+1}-P_i\right|=\int\limits_{a}^b \sqrt{1+g^2}\,dx \]

\grubr{opgintxkvadmdef}\\
Vi har at
\alg{
\int x^2 \,dx &= \lim\limits_{n\to \infty} \sum\limits_{i=1}^n (a+(i-1)\Delta x)^2 \Delta x \\
&=\lim\limits_{n\to \infty} \sum\limits_{i=1}^n \left(a^2\Delta x + 2a(i-1)(\Delta x)^2 + (i-1)^2(\Delta x)^3\right)
}
Vi har at
\nn{
\lim\limits_{n\to \infty} \sum\limits_{i=1}^n a^2\Delta x= a^2\frac{b-a}{n}n=a^2b-a^3 
}
Ved å bruke summen av en aritmetisk rekke får vi at
\nn{
\lim\limits_{n\to \infty} \sum\limits_{i=1}^n 2a(i-1)(\Delta x)^2 =2a\frac{(n-1)n}{2}\frac{(b-a)^2}{n^2}=a(b-a)^2
}
Ved å bruke bruke \eqref{sumkvad} finner vi at
\nn{
\lim\limits_{n\to \infty} \sum\limits_{i=1}^n (i-1)^2(\Delta x)^3 = \frac{(n-1)(2(n-1)+1)n}{6}\frac{(b-a)^3}{n^3} = \frac{1}{3}(b-a)^3
}
Dermed er
\nn{
\int x^2 \,dx = a^2b-a^3+a(b-a)^2+\frac{1}{3}(b-a)^3 = \frac{1}{3}(b^3-a^3)
}

\newpage
a) Det er rett skrevet at $ x_i=a+(a-i)\Delta x $. Dette gjør at differansen mellom to naboliggende $ x_i $-verdier er $ \Delta x $, og det er denne differansen man ganger med $ f(x_i) $.\vsk

b) Det stemmer at kjerneregelen gir
\[ f'(x)=g'(u)u'(x) \]
I teksten du viser til gir kjerneregelen
\[ F'(x)=G'(u)u'(x) \]
Og da vi har definert $ F'(x)=f(x) $ og $ G'(u)=g'(u) $, kan vi skrive
\[ f(x)=g(u)u \]


\end{document}