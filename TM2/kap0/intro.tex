\documentclass[english, 11 pt, class=article, crop=false]{standalone}
\usepackage[T1]{fontenc}
%\renewcommand*\familydefault{\sfdefault} % For dyslexia-friendly text
\usepackage{lmodern} % load a font with all the characters
\usepackage{geometry}
\geometry{verbose,paperwidth=16.1 cm, paperheight=24 cm, inner=2.3cm, outer=1.8 cm, bmargin=2cm, tmargin=1.8cm}
\setlength{\parindent}{0bp}
\usepackage{import}
\usepackage[subpreambles=false]{standalone}
\usepackage{amsmath}
\usepackage{amssymb}
\usepackage{esint}
\usepackage{babel}
\usepackage{tabu}
\makeatother
\makeatletter

\usepackage{titlesec}
\usepackage{ragged2e}
\RaggedRight
\raggedbottom
\frenchspacing

% Norwegian names of figures, chapters, parts and content
\addto\captionsenglish{\renewcommand{\figurename}{Figur}}
\makeatletter
\addto\captionsenglish{\renewcommand{\chaptername}{Kapittel}}
\addto\captionsenglish{\renewcommand{\partname}{Del}}


\usepackage{graphicx}
\usepackage{float}
\usepackage{subfig}
\usepackage{placeins}
\usepackage{cancel}
\usepackage{framed}
\usepackage{wrapfig}
\usepackage[subfigure]{tocloft}
\usepackage[font=footnotesize,labelfont=sl]{caption} % Figure caption
\usepackage{bm}
\usepackage[dvipsnames, table]{xcolor}
\definecolor{shadecolor}{rgb}{0.105469, 0.613281, 1}
\colorlet{shadecolor}{Emerald!15} 
\usepackage{icomma}
\makeatother
\usepackage[many]{tcolorbox}
\usepackage{multicol}
\usepackage{stackengine}

\usepackage{esvect} %For vectors with capital letters

% For tabular
\usepackage{array}
\usepackage{multirow}
\usepackage{longtable} %breakable table

% Ligningsreferanser
\usepackage{mathtools}
\mathtoolsset{showonlyrefs}

% index
\usepackage{imakeidx}
\makeindex[title=Indeks]

%Footnote:
\usepackage[bottom, hang, flushmargin]{footmisc}
\usepackage{perpage} 
\MakePerPage{footnote}
\addtolength{\footnotesep}{2mm}
\renewcommand{\thefootnote}{\arabic{footnote}}
\renewcommand\footnoterule{\rule{\linewidth}{0.4pt}}
\renewcommand{\thempfootnote}{\arabic{mpfootnote}}

%colors
\definecolor{c1}{cmyk}{0,0.5,1,0}
\definecolor{c2}{cmyk}{1,0.25,1,0}
\definecolor{n3}{cmyk}{1,0.,1,0}
\definecolor{neg}{cmyk}{1,0.,0.,0}

% Lister med bokstavar
\usepackage[inline]{enumitem}

\newcounter{rg}
\numberwithin{rg}{chapter}
\newcommand{\reg}[2][]{\begin{tcolorbox}[boxrule=0.3 mm,arc=0mm,colback=blue!3] {\refstepcounter{rg}\phantomsection \large \textbf{\therg \;#1} \vspace{5 pt}}\newline #2  \end{tcolorbox}\vspace{-5pt}}

\newcommand\alg[1]{\begin{align} #1 \end{align}}

\newcommand\eks[2][]{\begin{tcolorbox}[boxrule=0.3 mm,arc=0mm,enhanced jigsaw,breakable,colback=green!3] {\large \textbf{Eksempel #1} \vspace{5 pt}\\} #2 \end{tcolorbox}\vspace{-5pt} }

\newcommand{\st}[1]{\begin{tcolorbox}[boxrule=0.0 mm,arc=0mm,enhanced jigsaw,breakable,colback=yellow!12]{ #1} \end{tcolorbox}}

\newcommand{\spr}[1]{\begin{tcolorbox}[boxrule=0.3 mm,arc=0mm,enhanced jigsaw,breakable,colback=yellow!7] {\large \textbf{Språkboksen} \vspace{5 pt}\\} #1 \end{tcolorbox}\vspace{-5pt} }

\newcommand{\sym}[1]{\colorbox{blue!15}{#1}}

\newcommand{\info}[2]{\begin{tcolorbox}[boxrule=0.3 mm,arc=0mm,enhanced jigsaw,breakable,colback=cyan!6] {\large \textbf{#1} \vspace{5 pt}\\} #2 \end{tcolorbox}\vspace{-5pt} }

\newcommand\algv[1]{\vspace{-11 pt}\begin{align*} #1 \end{align*}}

\newcommand{\regv}{\vspace{5pt}}
\newcommand{\mer}{\textsl{Merk}: }
\newcommand{\mers}[1]{{\footnotesize \mer #1}}
\newcommand\vsk{\vspace{11pt}}
\newcommand\vs{\vspace{-11pt}}
\newcommand\vsb{\vspace{-16pt}}
\newcommand\sv{\vsk \textbf{Svar} \vspace{4 pt}\\}
\newcommand\br{\\[5 pt]}
\newcommand{\figp}[1]{../fig/#1}
\newcommand\algvv[1]{\vs\vs\begin{align*} #1 \end{align*}}
\newcommand{\y}[1]{$ {#1} $}
\newcommand{\os}{\\[5 pt]}
\newcommand{\prbxl}[2]{
\parbox[l][][l]{#1\linewidth}{#2
	}}
\newcommand{\prbxr}[2]{\parbox[r][][l]{#1\linewidth}{
		\setlength{\abovedisplayskip}{5pt}
		\setlength{\belowdisplayskip}{5pt}	
		\setlength{\abovedisplayshortskip}{0pt}
		\setlength{\belowdisplayshortskip}{0pt} 
		\begin{shaded}
			\footnotesize	#2 \end{shaded}}}

\renewcommand{\cfttoctitlefont}{\Large\bfseries}
\setlength{\cftaftertoctitleskip}{0 pt}
\setlength{\cftbeforetoctitleskip}{0 pt}

\newcommand{\bs}{\\[3pt]}
\newcommand{\vn}{\\[6pt]}
\newcommand{\fig}[1]{\begin{figure}
		\centering
		\includegraphics[]{\figp{#1}}
\end{figure}}

\newcommand{\figc}[2]{\begin{figure}
		\centering
		\includegraphics[]{\figp{#1}}
		\caption{#2}
\end{figure}}

\newcommand{\sectionbreak}{\clearpage} % New page on each section

\newcommand{\nn}[1]{
\begin{equation}
	#1
\end{equation}
}

% Equation comments
\newcommand{\cm}[1]{\llap{\color{blue} #1}}

\newcommand\fork[2]{\begin{tcolorbox}[boxrule=0.3 mm,arc=0mm,enhanced jigsaw,breakable,colback=yellow!7] {\large \textbf{#1 (forklaring)} \vspace{5 pt}\\} #2 \end{tcolorbox}\vspace{-5pt} }
 
%colors
\newcommand{\colr}[1]{{\color{red} #1}}
\newcommand{\colb}[1]{{\color{blue} #1}}
\newcommand{\colo}[1]{{\color{orange} #1}}
\newcommand{\colc}[1]{{\color{cyan} #1}}
\definecolor{projectgreen}{cmyk}{100,0,100,0}
\newcommand{\colg}[1]{{\color{projectgreen} #1}}

% Methods
\newcommand{\metode}[2]{
	\textsl{#1} \\[-8pt]
	\rule{#2}{0.75pt}
}

%Opg
\newcommand{\abc}[1]{
	\begin{enumerate}[label=\alph*),leftmargin=18pt]
		#1
	\end{enumerate}
}
\newcommand{\abcs}[2]{
	\begin{enumerate}[label=\alph*),start=#1,leftmargin=18pt]
		#2
	\end{enumerate}
}
\newcommand{\abcn}[1]{
	\begin{enumerate}[label=\arabic*),leftmargin=18pt]
		#1
	\end{enumerate}
}
\newcommand{\abch}[1]{
	\hspace{-2pt}	\begin{enumerate*}[label=\alph*), itemjoin=\hspace{1cm}]
		#1
	\end{enumerate*}
}
\newcommand{\abchs}[2]{
	\hspace{-2pt}	\begin{enumerate*}[label=\alph*), itemjoin=\hspace{1cm}, start=#1]
		#2
	\end{enumerate*}
}

% Oppgaver
\newcommand{\opgt}{\phantomsection \addcontentsline{toc}{section}{Oppgaver} \section*{Oppgaver for kapittel \thechapter}\vs \setcounter{section}{1}}
\newcounter{opg}
\numberwithin{opg}{section}
\newcommand{\op}[1]{\vspace{15pt} \refstepcounter{opg}\large \textbf{\color{blue}\theopg} \vspace{2 pt} \label{#1} \\}
\newcommand{\ekspop}[1]{\vsk\textbf{Gruble \thechapter.#1}\vspace{2 pt} \\}
\newcommand{\nes}{\stepcounter{section}
	\setcounter{opg}{0}}
\newcommand{\opr}[1]{\vspace{3pt}\textbf{\ref{#1}}}
\newcommand{\oeks}[1]{\begin{tcolorbox}[boxrule=0.3 mm,arc=0mm,colback=white]
		\textit{Eksempel: } #1	  
\end{tcolorbox}}
\newcommand\opgeks[2][]{\begin{tcolorbox}[boxrule=0.1 mm,arc=0mm,enhanced jigsaw,breakable,colback=white] {\footnotesize \textbf{Eksempel #1} \\} \footnotesize #2 \end{tcolorbox}\vspace{-5pt} }
\newcommand{\rknut}{
Rekn ut.
}

%License
\newcommand{\lic}{\textit{Matematikken sine byggesteinar by Sindre Sogge Heggen is licensed under CC BY-NC-SA 4.0. To view a copy of this license, visit\\ 
		\net{http://creativecommons.org/licenses/by-nc-sa/4.0/}{http://creativecommons.org/licenses/by-nc-sa/4.0/}}}

%referances
\newcommand{\net}[2]{{\color{blue}\href{#1}{#2}}}
\newcommand{\hrs}[2]{\hyperref[#1]{\color{blue}\textsl{#2 \ref*{#1}}}}
\newcommand{\rref}[1]{\hrs{#1}{regel}}
\newcommand{\refkap}[1]{\hrs{#1}{kapittel}}
\newcommand{\refsec}[1]{\hrs{#1}{seksjon}}

\newcommand{\mb}{\net{https://sindrsh.github.io/FirstPrinciplesOfMath/}{MB}}


%line to seperate examples
\newcommand{\linje}{\rule{\linewidth}{1pt} }

\usepackage{datetime2}
%%\usepackage{sansmathfonts} for dyslexia-friendly math
\usepackage[]{hyperref}

\begin{document}
	
\section*{Om boka}
Denne bokas hovedmål er å virke som lærebok i faget \textsl{Matematikk R2}. Temaene i boka dekker derfor kompetansemålene til faget per 2017, bestemt av \textsl{Utdanningsdirektoratet} (\url{www.udir.no/kl06/MAT3-01/Hele/Kompetansemaal/matematikk-r2}).\vsk

Boka er delt inn i to deler, én teoridel og én GeoGebra-del. GeoGebra-delen kan lastes ned gratis fra nettsiden \url{forkalkulus.netlify.com}, som også er hjemmeside for denne boka. Hovedårsaken til en slik inndeling er at GeoGebra hyppig oppdateres. Ved å la læreteksten for GeoGebra være nettbasert, kan det sørges for at informasjonen som blir gitt alltid er tilpasset den nyeste versjonen av programvaren.\vsk

\textbf{Teoridelen}\\
En sentral del i skolematematikken er å ha en brei oversikt over ligninger som kan anvendes under visse vilkår, disse ligningene kaller vi gjerne regneregler. I de fleste lærebøker på markedet vil man erfare at noen forklaringer for regneregler er tatt med, mens andre er fullstendig utelatt. Etter forfatterens mening er dette med på å holde i live den uheldige myten om at ''matematikk er et sett med regler som må læres'' og at man ofte ''må akspetere at sånn er det bare''. Med denne holdningen undertrykker man kansje det vakreste av alt med matematikk, nemlig at (nesten) enhver sannhet bygger på en annen $ - $ alt som \textsl{kan} forklares \textsl{bør} derfor forklares.\vsk

Samtidig er læreplanen for R2 såpass omfattende at skolenes tilmålte tid til faget gjør det vanskelig å gå i dybden av hvert eneste tema. Som et et kompromiss mellom grundighet og tidspress er derfor teoridelen strukturert på følgende måte: Der hvor forfatteren mener at begrunnelesen for en regneregel er nødvendig for høy måloppnåelse i faget, er en forklaring\footnote{Å \textsl{forklare} reglene istedenfor å \textsl{bevise} dem er et bevisst valg. Et bevis stiller sterke matematiske krav som ofte må defineres både på forhånd og underveis i en utledning av en ligning, noe som kan føre til at forståelsen av hovedpoenget drukner i smådetaljer. Noen av forklaringene vil likevel være gyldige som bevis.} tatt med i forkant. Hvis en regneregel derimot presenteres direkte, vil man finne en forklaring for denne i seksjonen \textsl{Forklaringer} i samme kapittel, underforstått at dette er for den spesielt interesserte.\vsk

Teksten består av sju kapitler som er delt inn i seksjoner og delseksjoner. Alle oppgaver tilhørende hvert kapittel er satt av til siste seksjon, fasit finner du bakerst i boka (løsningsforslag ligger gratis tilgjengelig på hjemmesiden). Hver såkalt regneregel dukker opp i en blå tekstboks, som oftest etterfulgt av ett eller flere eksempler.\vsk

Rimelig unikt for denne boka, i skolesammenheng, er bruken av nummererte ligninger. Alle ligninger som blir brukt ved senere anledninger blir referert til ved et unikt nummer. Dette gjør at omskrivinger og resulter ikke kommer ''ut av det blå'', og at leseren enkelt kan finne tilbake til aktuelle ligninger. Ved digital lesning er også hyperreferanser aktivert. Dette betyr at du kan nå refererte ligninger, figurer, lenker, kapitler, seksjoner og delseksjoner ved et enkelt pekertrykk.\vsk

\textbf{GeoGebra-delen}\\
Fra og med våren 2015 har det vært spesifikke krav på eksamen i R2 om bruk av digital graftegner og CAS (Computer Algebra System). Eksamenskandidaten står fritt til å velge selv hvilket digitalt hjelpemiddel han/hun vil bruke, men på de fleste norske skoler er det GeoGebra som blir undervist. \vsk

\textit{Før kalkulus; GeoGebra i R2} tilbyr en omfattende oversikt over de mest sentrale funksjonalitetene i GeoGebra, sett fra et R2 perspektiv. Teksten følger de samme kapitlene som teoridelen og inneholder eksempler og øvingsoppgaver med løsningsforslag.
\newpage
\begin{tcolorbox}[boxrule=0.0 mm,arc=0mm,enhanced jigsaw,breakable,colback=yellow!12]
Kjære leser. \vsk

Denne boka er i utgangspunktet gratis å bruke, men jeg håper du forstår hvor mye tid og ressurser jeg har brukt på å lage den. Jeg ønsker å fortsette arbeidet med å lage lærebøker som er med på å gjøre matematikk lett tilgjengeleg for alle, men det kan bli vanskelig med mindre arbeidet gir en viss inntekt. Hvis du ender opp med å like boka, håper jeg derfor du kan donere 50\,kr via Vipps til 90559730 eller via \net{https://www.paypal.com/donate?hosted_button_id=M5PNVC6J7PBYY}{PayPal}. Vær vennlig å markere donasjonen med ''Før kalkulus'' ved bruk av Vipps. På forhand takk! \vsk

Boka blir oppdatert så snart som råd etter at skrivefeil og lignende blir oppdaget. Jeg vil derfor råde alle til å laste ned en ny versjon i ny og ne ved å følge \net{https://github.com/sindrsh/FirstPrinciplesOfMath/blob/master/G_bm.pdf}{denne linken}.\vsk

Nynorskversjonen av boka finner du \net{https://github.com/sindrsh/FirstPrinciplesOfMath/blob/master/G.pdf}{her}.\vsk

For spørsmål, ta kontakt på mail: \tt{sindre.heggen@gmail.com} 
 \end{tcolorbox}
\end{document}